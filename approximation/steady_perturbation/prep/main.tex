\documentclass{report}
\input{../../../head.tex}
\begin{document}
  摂動が無い状態のSch\"odinger方程式,
  \begin{equation}
    \hat{H}^{(0)}\ket{n^{(0)}}=E_0^{(0)}\ket{n^{(0)}}
  \end{equation}
  が厳密に解くことができるとする.
  ここに摂動$\hat{V}$を加わったこと\footnote{摂動の例: 光,電場}を考えると,摂動Hamiltonianを$\hat{V}$として,
  \begin{equation}
    \qty(\hat{H}^{(0)} + \hat{V})\ket{n} = E_n\ket{n}\label{perturbation-origin}
  \end{equation}
  とかける.摂動の大きさを表すパラメータを$\lambda$として\refe{perturbation-origin}を
  \begin{align}
    \hat{H} = \hat{H}^{(0)} + \lambda\hat{V}\label{perturbation-using-lambda}
  \end{align}
  とする.$\lambda\to 0$ならば明らかに,
  \begin{equation}
    \begin{cases}
      \ket{n}\to\ket{n^{(0)}}\\
      E_n\to E_n^{(0)}
    \end{cases}
  \end{equation}
  である.
  ここで,\refe{perturbation-using-lambda}の解が,%\footnote{$\hat{H},n,E$の肩の()を今後は省略する.}
  \begin{equation}
    \begin{cases}
      \ket{n} &= \ket{n^{(0)}} + \lambda\ket{n^{(1)}} + \lambda^2\ket{n^{(2)}} + \cdots \label{n-en-expantion} \\
      E_n &= E_n^{(0)} + \lambda E_n^{(1)} + \lambda^2 E_n^{(2)} + \cdots 
    \end{cases}
  \end{equation}
  と書けたとする.
  $\ket{n^{(1)}}$,$\ket{n^{(2)}}$,$E_n^{(1)}$,$E_n^{(2)}$を考える.
  規格化条件として
  \begin{equation}
    \braket{n^{(0)}}{n} = 1
  \end{equation}
  を定める.
  \refe{n-en-expantion}を\refe{perturbation-using-lambda}に代入して,$\lambda$の次数ごとにまとめると,
  \begin{align}
    &(E_n^{(0)} - \hat{H}^{(0)})\ket{n^{(0)}} = 0 \label{0-order} \\ 
    &(E_n^{(0)} - \hat{H}^{(0)})\ket{n^{(1)}} + E_n^{(1)}\ket{n^{(0)}} = \hat{V}\ket{n^{(0)}} \label{1st-perturbation} \\
    &(E_n^{(0)} - \hat{H}^{(0)})\ket{n^{(2)}} + E_n^{(1)}\ket{n^{(1)}} + E_n^{(2)}\ket{n^{(0)}} = \hat{V}\ket{n^{(1)}} \label{2nd-perturbation}\\
  \end{align}
  を得る.
\end{document}
