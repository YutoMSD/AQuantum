\documentclass{standalone}
\input{../../../head.tex}
\begin{document}
  次の式は厳密に解くことができるとする.
  \begin{equation}
    \hat{H}^{(0)}\ket{n^{(0)}}=E_0^{(0)}\ket{n^{(0)}}
  \end{equation}
  ここに摂動$\hat{V}$を加える\footnote{摂動の例: 光,電場}.
  \begin{equation}
    \qty(\hat{H}^{(0)}+\hat{V})\ket{n}=E_n\ket{n}
  \end{equation}
  $\hat{H}=\hat{H}^{(0)}+\lambda\hat{V}$とする.$\lambda\to0$ならば,
  \begin{equation}
    \begin{cases}
      \ket{n}\to\ket{n^{(0)}}\\
      E_n\to E_n^{(0)}
    \end{cases}
  \end{equation}
  である.
  ここで,$\hat{H}=\hat{H}^{(0)}+\lambda\hat{V}$の解を次のようにおく.\footnote{$\hat{H},n,E$の肩の()を今後は省略する.}
  \begin{equation}
    \begin{cases}
      \ket{n}=\ket{n^0}+\lambda\ket{n^1}+\lambda^2\ket{n^2}+\cdots\\
      E_n=E_n^0+\lambda+E_n^1+\lambda^2+E_n^2+\cdots
    \end{cases}
  \end{equation}
  $\ket{n^1},\ket{n^2},E_n^1,E_n^2$を求める.
  ここで,規格化条件として
  \begin{equation}
    \braket{n^0}{n}=1
  \end{equation}
  を定める.
  以上の$\hat{H},\ket{n},E_n$を用いて,Schrödinger方程式を立て,整理すると,
  $\lambda^0$,$\lambda^1$,$\lambda^2$の係数としてそれぞれ
  \begin{align}
    &(E_n^0-\hat{H}^0)\ket{n^0}=0\\
    \label{1st}
    &(E_n^0-\hat{H}^0)\ket{n^1}+E_n^1\ket{n^0}=\hat{V}\ket{n^0}\\
    \label{2nd}
    &(E_n^0-\hat{H}^0)\ket{n^2}+E_n^1\ket{n^1}+E_n^2\ket{n^0}=\hat{V}\ket{n^1}\\
  \end{align}
  を得る.
\end{document}
