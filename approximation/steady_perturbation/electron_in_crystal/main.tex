\documentclass{report}
\input{../../../head.tex}
\begin{document}
  \begin{myex}{バンドギャップ(2023年度期末試験第1問)}{}
    長さ$L$の無限井戸型ポテンシャル中の1次元自由電子のエネルギー固有値$\epsilon_0(k)$と,そのときの波動関数$\phi_k(x)$は,
    \begin{align}
      \epsilon_0(k) &= \frac{\hbar^2k^2}{2m}\ \qty(k = \frac{2\pi}{L}N,\ N\in\mathbb{Z}) \\ 
      \phi_k(x) &= \braket{x}{k} = \frac{1}{\sqrt{L}}\e^{\i kx} 
    \end{align}
    である.
    無限井戸型ポテンシャルに$V(x + a) = V(x)$を満たすポテンシャル$V(x)$が加わったときを考える.
    $V(x)$は,
    \begin{align}
      V(x)& = 2V\cos\qty(\frac{2\pi}{a}x)\\
      & = V(\e^{\i\frac{2\pi}{a}x} + \e^{-\i\frac{2\pi}{a}x})\\
      & = V(\e^{\i gx} + \e^{-\i gx})
    \end{align}
    と書けるとする.
    ただし$g$は,
    \begin{align}
      g\coloneqq \frac{2\pi}{a}\label{wave-number-g-def}
    \end{align}
    である.
    \par
    以下の問いに答えよ.
    \begin{enumerate}
      \item 結晶中の周期ポテンシャルによりバンドギャップができることを2次までの摂動論を用いて説明せよ.
      \item 縮退のある場合の摂動論を用いてバンドギャップエネルギーの大きさを見積もれ.また,Brillouinゾーン端近傍で近似した波動関数は,どのような関数に比例するか.
      \item バンドギャップとBraggの回折条件との関係について議論せよ.
    \end{enumerate}
    \tcblower
    \begin{enumerate}
      \item バンドギャップの成り立ち \par
        2次の摂動によるエネルギー補正は,
        \begin{align}
          E_n = E_n^{(0)} + \mel**{n^{(0)}}{\hat{V}}{n^{(0)}} + \sum_{m\neq n}\frac{\abs{\mel**{m^{(0)}}{\hat{V}}{n^{(0)}}}^2}{E_n^{(0)} - E_m^{(0)}} \label{2nd-order-perturbation-eigenvalue-re}
        \end{align}
        と書ける.\refe{2nd-order-perturbation-eigenvalue-re}に対して離散Fourier変換を行うことで,\refe{2nd-order-perturbation-eigenvalue-re}の状態と
        エネルギーのラベリングを$n$から$k$に変更する.
        $V_{k'k} \coloneqq \mel**{k'}{\hat{V}}{k}$とすると,
        \begin{align}
          E(k) = \epsilon^{(0)}(k) + V_{kk} + \sum_{k'\neq k}\frac{\abs{V_{k'k}}^2}{\epsilon^{(0)}(k) - \epsilon^{(0)}(k')}
        \end{align}
        と書ける.
        摂動によるエネルギーは.
        \begin{align}
          V_{k'k} &= \frac{V}{L}\int_{-L/2}^{L/2}\phi_{k'}^{*}(x)\hat{V}(x)\phi_k(x)\dd{x} \\
          &= V\qty[\frac{\sin\qty(\frac{q_+L}{2})}{\frac{q_+L}{2}} + \frac{\sin\qty(\frac{q_-L}{2})}{\frac{q_-L}{2}}] 
        \end{align}
        と計算される.ただし$q_+$と$q_-$を,
        \begin{align}
          q_+ &\coloneqq -k' + g + k \\ 
          q_- &\coloneqq -k' - g + k
        \end{align}
        と定義した.摂動によるエネルギーは$\r{sinc}$関数の形になっているので,$L\to\infty$では規格化されたデルタ関数$\tilde{\delta}(x_1, x_2)$と解釈できる.
        よって,
        \begin{align}
          V_{k'k} = V\qty(\tilde{\delta}(q_+, 0) + \tilde{\delta}(q_-, 0)) = 
          \begin{dcases}
            V & q_+ = 0\ \r{or}\ q_- = 0\\
            0 & \r{otherwise}
          \end{dcases}
        \end{align}
        である.なお,$q_+ = q_- = 0$となるのは$g = 0$であるが,$g$の定義である\refe{wave-number-g-def}よりありえない.
        摂動によって補正したエネルギーは,
        \begin{align}
          E(k) &= \epsilon^{(0)}(k) + \frac{V^2}{\epsilon^{(0)}(k) - \epsilon^{(0)}(k + g)} + \frac{V^2}{\epsilon^{(0)}(k) - \epsilon^{(0)}(k - g)} \\ 
          &= \epsilon^{(0)}(k) + \frac{V^2}{\epsilon^{(0)}(k) - \epsilon^{(0)}\qty(k + \frac{2\pi}{a})} + \frac{V^2}{\epsilon^{(0)}(k) - \epsilon^{(0)}\qty(k - \frac{2\pi}{a})}\label{2nd-order-perturbation-eigenvalue-in-electron}
        \end{align}
        となる.
        定義より,$q_+ = 0 \Leftrightarrow k' = k + g$,$q_- = 0 \Leftrightarrow k' = k - g$であることに注意する.
        \refe{2nd-order-perturbation-eigenvalue-in-electron}の振る舞いを第1 Brillouinゾーンの内側と外側で確認する.
        ポテンシャルの対称性から右側のみを計算すればよい.
        \begin{enumerate}
          \item 第1 Brillouinゾーン内側$\qty(k < \frac{\pi}{a})$の振る舞い\par
            $\epsilon^{(0)}(k)$は放物線なので,
            \begin{align}
              \begin{dcases}
                \epsilon^{(0)}(k) \ll \epsilon^{(0)}(k + g) \\
                \epsilon^{(0)}(k) < \epsilon^{(0)}(k - g)
              \end{dcases}
            \end{align}
            が成立する.
            よって,\refe{2nd-order-perturbation-eigenvalue-in-electron}の第2項が0に,第3項が負になるので,$E(k) < \epsilon^{(0)}(k)$が成り立つ.
            つまり,摂動が加わった後のエネルギーは加わる前のエネルギーより小さくなる.
          \item 第1 Brillouinゾーン外側$\qty(k > \frac{\pi}{a})$の振る舞い\par
            第1 Brillouinゾーン内側のときと同様に考えると,
            \begin{align}
              \begin{dcases}
                \epsilon^{(0)}(k) \ll \epsilon^{(0)}(k + g)\\
                \epsilon^{(0)}(k) > \epsilon^{(0)}(k - g)
              \end{dcases}
            \end{align}
            が成立する.
            よって,\refe{2nd-order-perturbation-eigenvalue-in-electron}の第2項が0に,第3項が正になるので,$E(k) > \epsilon^{(0)}(k)$が成り立つ.
            つまり,摂動が加わった後のエネルギーは加わる前のエネルギーより大きくなる.
        \end{enumerate}
        以上の議論により,結晶中の周期ポテンシャルによりバンドギャップが形成されることがわかった.
      \item バンドギャップエネルギーの見積もりと波動関数\par
        \refe{2nd-order-perturbation-eigenvalue-in-electron}に$k = \pm\frac{\pi}{a}$を代入すると発散してしまう.
        以下では2重縮退があるときの摂動を考えバンドギャップエネルギー$\Delta E$を求める.$k_+$と$k_-$を,
        \begin{align}
          k_+ &\coloneqq \frac{\pi}{a} \\ 
          k_- &\coloneqq -\frac{\pi}{a}
        \end{align}
        と定義する.
        簡単な計算により,$V_{k_+k_+} = V_{k_-k_-} = 0$,$V_{k_+k_-} = V_{k_-k_+} = V$である.
        \refe{1st-order-energy-with-degeneracy}より,1次摂動によるエネルギーは,
        \begin{align}
          E_n^{(1)} &= \frac{1}{2}\qty[\qty(V_{k_+k_+} + V_{k_-k_-})\pm\sqrt{(V_{k_+k_+} - V_{k_-k_-})^2 + 4\abs{V_{k_+k_-}}^2}] \\ 
          &= \pm V
        \end{align}
        である.2つのエネルギー補正の差がバンドギャップエネルギー$\Delta E$と解釈できるので,
        \begin{align}
          \Delta E = 2V
        \end{align}
        を得る.\par
        \refe{degeneracy-n0-def}の$\alpha$と$\beta$は\refe{secular-eq}の解であるから,$\sqrt{\alpha^2 + \beta^2} = 1$なる規格化条件を課すと,
        \begin{align}
          \mqty(\alpha \\ \beta) = 
          \begin{dcases}
            \frac{1}{\sqrt{2}}\mqty(1 \\ 1) & E_n^{(1)} = V \\ 
            \frac{1}{\sqrt{2}}\mqty(1 \\ -1) & E_n^{(1)} = -V
          \end{dcases}
        \end{align}
        となる.よって,
        $\Delta E = \pm V$に対応する波動関数はBrillouinゾーン端で,
        \begin{align}
          \psi_{+}& = \frac{1}{\sqrt{2}}\qty(\phi_{\pi/a} + \phi_{-\pi/a})\propto\cos\qty(\frac{\pi}{a}x)\label{degeneracy-wave-function-positive}\\
          \psi_{-}& = \frac{1}{\sqrt{2}}\qty(\phi_{\pi/a} - \phi_{-\pi/a})\propto\sin\qty(\frac{\pi}{a}x)\label{degeneracy-wave-function-negative}
        \end{align}
        であり,定在波が生じる.
      \item バンドギャップとBragg反射\par
        バンドギャップの起源はBragg反射である.Bragg反射は,
        \begin{align}
          2a\sin\theta = \lambda
        \end{align}
        を満たす.
        今回の場合は1次元なので$\theta = \pi/2$であり,波数は$k = 2\pi/\lambda$である.よって,Bragg条件は,
        \begin{align}
          k = \frac{\pi}{a}\label{1d-bragg-k-condition}
        \end{align}
        と書き換えられる.つまり,\refe{1d-bragg-k-condition}を満たす波数のみが反射し定在波をつくる.$V(x) = 2V\cos\qty(\frac{2\pi}{a}x)$であったから,
        \refe{degeneracy-wave-function-positive}で表される波動関数はポテンシャルが最小となる波数で確率振幅が最大となる.
        \refe{degeneracy-wave-function-negative}で表される波動関数はポテンシャルが最大となる波数で確率振幅が最大となる.
        よって,エネルギーは$\psi_+$が$\psi_-$より低くなる.これによりバンドギャップが生じる.
    \end{enumerate}
  \end{myex}
\end{document}
