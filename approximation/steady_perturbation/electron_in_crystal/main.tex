\documentclass{report}
\input{../../../head.tex}
\begin{document}
  \begin{myex}{バンドギャップ}{}
    長さ$L$の無限井戸型ポテンシャル中の1次元自由電子の波動関数$\phi_k(x)$とエネルギー固有値$\epsilon_0(k)$は,
    \begin{align}
      \phi_k(x) &= \braket{x}{k} = \frac{1}{\sqrt{L}}\e^{\i kx} \\ 
      \epsilon_0(k) &= \frac{\hbar^2k^2}{2m}\ \qty(k = \frac{2\pi}{L}N,\ N\in\mathbb{N})
    \end{align}
    である.
    無限井戸型ポテンシャルに$V(x + a) = V(x)$を満たすポテンシャル$V(x)$が加わったときを考える.
    $V(x)$は,
    \begin{align}
      V(x)& = 2V\cos\qty(\frac{2\pi}{a}x)\\
      & = V(\e^{\i\frac{2\pi}{a}x} + \e^{-\i\frac{2\pi}{a}x})\\
      & = V(\e^{\i gx} + \e^{-\i gx})
    \end{align}
    と書けるとする.
    ただし$g$は,
    \begin{align}
      g\coloneqq \frac{2\pi}{a}
    \end{align}
    である.
    \par
    以下の問いに答えよ.
    \begin{enumerate}
      \item 結晶中の周期ポテンシャルによりバンドギャップができることを2次までの摂動論を用いて説明せよ.
      \item 縮退のある場合の摂動論を用いてバンドギャップの大きさを見積もれ.
      \item バンドギャップとBraggの回折条件との関係について議論せよ.
    \end{enumerate}
    \tcblower
    \begin{enumerate}
      \item バンドギャップの成り立ち \par
        2次の摂動によるエネルギー補正は,
        \begin{align}
          E_n = E_n^{(0)} + \mel**{n^{(0)}}{\hat{V}}{n^{(0)}} + \sum_{m\neq n}\frac{\abs{\mel**{m^{(0)}}{\hat{V}}{n^{(0)}}}^2}{E_n^{(0)} - E_m^{(0)}} \label{2nd-order-perturbation-eigenvalue-re}
        \end{align}
        と書ける.\refe{2nd-order-perturbation-eigenvalue-re}に対して離散Fourier変換を行うことで,\refe{2nd-order-perturbation-eigenvalue-re}の状態と
        エネルギーのラベリングを$n$から$k$に変更する.
        $V_{k'k} \coloneqq \mel**{k'}{\hat{V}}{k}$とすると,
        \begin{align}
          E(k) = \epsilon^{(0)}(k) + V_{kk} + \sum_{k'\neq k}\frac{\abs{V_{k'k}}^2}{\epsilon^{(0)}(k) - \epsilon^{(0)}(k')}
        \end{align}
        と書ける.
        摂動によるエネルギーは.
        \begin{align}
          V_{k'k} &= \frac{V}{L}\int_{-L/2}^{L/2}\phi_{k'}^{*}(x)\hat{V}(x)\phi_k(x)\dd{x} \\
          &= V\qty[\frac{\sin\qty(\frac{qL}{2})}{\frac{qL}{2}} + \frac{\sin\qty(\frac{q'L}{2})}{\frac{q'L}{2}}] 
        \end{align}
        と計算される.ただし$q$と$q'$を,
        \begin{align}
          q &\coloneqq -k' + g + k \\ 
          q' &\coloneqq -k' - g + k
        \end{align}
        と定義した.摂動によるエネルギーは$\r{sinc}$関数の形になっているので,$L\to\infty$では規格化されたデルタ関数$\tilde{\delta}_(x_1, x_2)$と解釈できる.
        よって,
        \begin{align}
          V_{k'k} = V\qty(\tilde{\delta}(q, 0) + \tilde{\delta}(q', 0)) = 
          \begin{dcases}
            V &  k' = k + g\ \r{or}\ k' = k - g\\
            0 & \r{otherwise}
          \end{dcases}
        \end{align}
        である.したがってエネルギーは
        \begin{align}
          E(k) = \epsilon^{(0)}(k) + \frac{V^2}{\epsilon(k) - \epsilon(k + g)} + \frac{V^2}{\epsilon(k) - \epsilon(k - g)}\label{2nd-order-perturbation-eigenvalue-in-electron}
        \end{align}
        となる.\refe{2nd-order-perturbation-eigenvalue-in-electron}の振る舞いを第1Brillouinゾーンの内側と外側で確認する.
        ポテンシャルの対称性から右側のみを計算すればよい.
        \begin{enumerate}
          \item 第1 Brillouinゾーン内側$\qty(k = k_1 < \frac{\pi}{a})$の振る舞い\par
          $\epsilon(k)$は放物線なので,
          \begin{align}
            \begin{dcases}
              \epsilon(k_1) \ll \epsilon^{(0)}(k_1 + g) \\
              \epsilon(k_1) < \epsilon^{(0)}(k_1 - g)
            \end{dcases}
          \end{align}
          が成りたつ.よって,$E(k) < \epsilon^{(0)}(k)$が成り立ち,摂動が加わった後のエネルギーは加わる前のエネルギーより小さくなる.
          \item 第1 Brillouinゾーン外側$\qty(k = k_2 > \frac{\pi  }{a})$の振る舞い\par
            第1 Brillouinゾーン内側のときと同様に考えると,
            \begin{align}
              \begin{dcases}
                \epsilon(k_2) \ll \epsilon^{(0)}(k_2 + g)\\
                \epsilon(k_2) > \epsilon^{(0)}(k_2 - g)
              \end{dcases}
            \end{align}
            を得る.よって,$E(k) > \epsilon^{(0)}(k)$が成り立ち,摂動が加わった後のエネルギーは加わる前のエネルギーより大きくなる.
          \end{enumerate}
          以上の議論により,結晶中の周期ポテンシャルによりバンドギャップが形成されることがわかった.
      \item バンドギャップの大きさの見積もり\par
        \refe{2nd-order-perturbation-eigenvalue-in-electron}に$k = \pm\frac{\pi}{a}$を代入すると発散してしまう.
        以下では2重縮退があるときの摂動を考えバンドギャップ$\Delta E$を求める.
        $k_+ \coloneqq \frac{\pi}{a}$,$k_- = - \frac{\pi}{a}$とする.$V_{kk} = 0,V_{ab} = V$なので,2次摂動によるエネルギーは
        \begin{align}
          E& = \pm\frac{1}{2}\sqrt{4|V|^2}\\
          & = \pm V
        \end{align}
        である.よって,
        \begin{align}
          \Delta E = 2V
        \end{align}
        を得る.また,$E = \pm V$に対応する波動関数はそれぞれ
        \begin{align}
          \psi_{ + }& = \phi_{k = \pi/a} + \phi_{k = -\pi/a}\sim\cos\frac{\pi}{a}x\\
          \psi_{-}& = \phi_{k = \pi/a}-\phi_{k = -\pi/a}\sim\sin\frac{\pi}{a}x
        \end{align}
        であり,定在波が生じる.\\
    バンドギャップの起源はBragg反射である.Bragg反射は以下の式を満たす.
    \begin{align}
      2a\sin\theta = \lambda
    \end{align}
    今回の場合は1次元なので$\theta = \pi/2$であり,波数は$k = 2\pi/\lambda$である.よって,Bragg条件は
    \begin{align}
      k = \frac{\pi}{a}
    \end{align}
    と書き換えられる.上の条件を満たす波数のみが反射し,定在波をつくる.sinで表される波動関数はポテンシャルが最小となる波数で確率振幅が最大となる.
    cosで表される波動関数はポテンシャルが最大となる波数で確率振幅が最大となる.よって,sinの方はエネルギーが低くなり,cosの方は高くなる.これによりバンドギャップが生じる.
  \end{enumerate}
  \end{myex}
\end{document}
