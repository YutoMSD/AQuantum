\documentclass{report}
\input{../../../head.tex}
\begin{document}
  \begin{myexc}{Griffith Example7.1}{}
    [0,a]無限井戸型ポテンシャルに次の摂動が加わったときの1次摂動によるエネルギーを求めよ.
    \begin{align}
      V_1(x)&=
      \begin{dcases}
      V_0,\ 0\leq x\leq a\\
      0,\ \mathrm{otherwise}
      \end{dcases}\\
      V_2(x)&=
      \begin{dcases}
      V_0,\ 0\leq x \leq a/2\\
      0,\ \mathrm{otherwise}
      \end{dcases}
    \end{align}
    \tcblower
    $V_1(x)$の場合
    \begin{align}
      E_n^1(x)=\bra{\psi_n^{(0)}}V_0\ket{\psi_n^{(0)}}=V_0
    \end{align}
    $V_2(x)$の場合
    \begin{align}
      E_n^1=\frac{2V_0}{a}\int_{0}^{a/2}\sin^2\qty(\frac{n\pi}{a}x)dx=V_0/2
    \end{align}
  \end{myexc}
  \begin{myexc}{Griffith Example 7.3}{}
    2次元調和振動子$\hat{H^0}=\frac{\hat{p}^2}{2m}+\frac{1}{2}m\omega^2(\hat{x}^2+\hat{y}^2)$の第1励起状態は縮退している.
    \begin{align}
      \psi^0_a&=\psi_0(x)\psi_1(y)=\sqrt{\frac{2}{\pi}}\frac{m\omega}{\hbar}y\exp\qty(\frac{m\omega}{2\hbar}(x^2+y^2))\\
      \psi^0_a&=\psi_1(x)\psi_0(y)=\sqrt{\frac{2}{\pi}}\frac{m\omega}{\hbar}x\exp\qty(\frac{m\omega}{2\hbar}(x^2+y^2))
    \end{align}
    ここに摂動$\hat{H'}=\epsilon m\omega^2xy$を加える.
    \begin{align}
      \begin{pmatrix}
        \omega_{aa}-E_n^1&\omega_{ab}\\
        \omega_{ba}&\omega_{bb}-E_n^1
      \end{pmatrix}
      \begin{pmatrix}
        \alpha\\
        \beta
      \end{pmatrix}
      =
      \begin{pmatrix}
        0\\0
      \end{pmatrix}
    \end{align}
    を用いて摂動を加えた後の固有関数及び摂動による補正エネルギーを求めよ.
    \tcblower
    \begin{align}
      W_{aa}&=\int\int\psi_a^0\hat{H'}\psi_a^0dxdy\\
      &=\epsilon m\omega^2\int|\psi_0(x)|^2xdx\int|\psi_1(x)|^2ydy\\
      &=0=W_{bb}
    \end{align}
    \begin{align}
      W_{ab}&=\qty[\int\psi_0(x)\epsilon m\omega^2x\psi_1(x)dx]^2\\
      &=\epsilon\frac{\hbar\omega}{2}\qty[\int\psi_0(x)(\hat{a}_{+}+\hat{a}_{-})\psi_1(x)dx]^2\\
      &=\epsilon\frac{\hbar\omega}{2}\qty[\int\psi_0(x)\psi_0(x)dx]^2\\
      &=\epsilon\frac{\hbar\omega}{2}
    \end{align}
    よって,摂動による補正エネルギーは$E_1=\pm\epsilon\frac{\hbar\omega}{2}$,固有関数は
    \begin{align}
      \psi_{\pm}^0=\frac{1}{\sqrt{2}}\qty(\psi_b^0\pm\psi_a^0)
    \end{align}
    である.
  \end{myexc}
\end{document}
