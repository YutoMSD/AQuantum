\documentclass{report}
\input{../../../head.tex}
\begin{document}
  ここでは,2次摂動を用いた例題としてStark効果\footnote{Johanes Stark(1874-1957)}\footnote{電場によるエネルギー準位の変化をStark効果という.}を考えよう.
  \begin{myex}{量子閉じ込めStark効果}{stark}
    定常状態ののハミルトニアン$\hat{H}^{(0)}$に,電場による摂動$\hat{V}$を加えたハミルトニアン$\hat{H}$を考える.
    ただし,定常状態のポテンシャルは,長さ$L$の無限井戸型ポテンシャル$\hat{U}$である.
    $\hat{U}$,$\hat{V}$,$\hat{H}^{(0)}$,$\hat{H}$は,
    \begin{align}
      U(x) &\coloneqq
      \begin{dcases}
        0 & \ \abs{x}\leq L/2\\
        \infty & \text{otherwise}
      \end{dcases} \\ 
      V(x) &\coloneqq -e\phi(x) = eEx\ (e>0) \\ 
      \hat{H}^{(0)} &\coloneqq -\frac{\hbar^2}{2m}\dv[2]{x} + \hat{U}(x) \\ 
      \hat{H} &\coloneqq \hat{H}^{(0)} + \hat{V}(x) \\ 
    \end{align}
    と定義される.
    また,$\hat{H}^{(0)}$の固有エネルギーとそれに属する固有関数は,
    \begin{align}
      E_n^{(0)} &= \frac{\hbar^2}{2m}\qty(\frac{\pi}{L})^2n^2,\ (n = 1, 2, 3,\cdots)\\
      \phi_n(x)&=
      \begin{dcases}
        \sqrt{\frac{2}{L}}\cos\qty(\frac{n\pi}{L}x) & n: \r{odd}\\
        \sqrt{\frac{2}{L}}\sin\qty(\frac{n\pi}{L}x) & n: \r{even}
      \end{dcases}
    \end{align}
    のようになっている.
    このとき,2次の摂動まで用いて$\hat{V}$の影響による$n = 1$のエネルギー補正を計算せよ.
    \tcblower
    1次摂動によるエネルギー補正は奇関数の積分になるため0である.\footnote{もし0でないならば,電場をかける向きによりエネルギーが変わることを意味するが,これは対称性より不合理である.}.\\
    2次摂動によるエネルギー補正は,
    \begin{align}
      E_{1}^{(2)} &= \sum_{m\neq 1}\frac{\abs{V_{m1}}^2}{E_1^{(0)} - E_m^{(0)}}\\
      V_{m1} &= eE\int\phi_m^*x\phi_{1}\dd{x}\\
      &
      \begin{dcases}
      = 0 & n: \r{odd}\\
      \neq 0 & n: \r{even}
      \end{dcases}\\
      E_{1}^{(2)}&=\frac{\abs{V_{21}}^2}{E_1^{(0)} - E_2^{(0)}} + \frac{\abs{V_{41}}^2}{E_1^{(0)} - E_4^{(0)}} + \cdots\\
      &\approx \frac{\abs{V_{21}}^2}{E_1^{(0)} - E_2^{(0)}}\\
      &=-\frac{256}{234\pi^4}\frac{(eEL)^2}{E_1^{(0)}}
    \end{align}
    と計算できて,2次の摂動を考えるとエネルギーは低下することがわかる.
  \end{myex}
\end{document}
