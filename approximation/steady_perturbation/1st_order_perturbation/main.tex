\documentclass{standalone}
\input{../../../head.tex}
\begin{document}
  式(\ref{1st})の両辺に$\bra{n^1}$を作用すると
  \begin{equation}
    \bra{n^0}E_n^1\ket{n^0}=\bra{n^0}\hat{V}\ket{n^0}
  \end{equation}
  を得る.よって,1次摂動によるエネルギー補正は
  \begin{itembox}[l]{1次摂動によるエネルギー補正}
  \begin{equation}
    E_n^1=\bra{n^0}\hat{V}\ket{n^0}
  \end{equation}
  \end{itembox}
  である.
  \begin{myex}{}{}ヘリウム原子の基底エネルギー
    \begin{equation}
      \hat{H}=\hat{H}^0+\frac{e^2}{4\pi\epsilon_0r_{12}}\equiv\hat{H}^0+\hat{V}
    \end{equation}
    $\hat{H}^0$の基底エネルギーは
    \begin{equation}
      \psi^0=\frac{Z^3}{\pi a_0^3}\mathrm{e}^{-Z(r_1+r_2)/a_0}
    \end{equation}
    である.よって,$\hat{V}$による1次のエネルギー補正は以下のように計算できる.
    \begin{align}
      E^1&=\bra{\psi^0}\hat{V}\ket{\psi^0}\\
      &=\int{\psi^0}^{*}\frac{e^2}{4\pi\epsilon_0r_{12}}\psi^0d\bm{r}_1\bm{r}_2\\
      &=\frac{5}{4}Z\ \mathrm{Ry}
    \end{align}
    よって,基底エネルギー
    \begin{align}
      E_0&=E^0+E^1\\
      &=-8\ \mathrm{Ry}+\frac{5}{4}\times{2}\ \mathrm{Ry}\\
      &=-74.8\ \mathrm{eV}
    \end{align}
    を得る\footnote{測定値は$-78.6\ \mathrm{eV}$}.
  \end{myex}
  次に固有ベクトルの補正を求める.式(\ref{1st})の両辺に$\bra{m^0}\ (m\ne n)$を作用する.
  \begin{align}
    (E_n^0-\hat{H}^0)\bra{m^0}\ket{n^1}+0=\bra{m^0}\hat{V}\ket{n^0}\\
    \braket{m^0}{n^1}=\frac{\bra{m^0}\hat{V}\ket{n^0}}{E_n^0-E_m^0}\\
  \end{align}
  ただし,エネルギー縮退は無く,$E_n^0-E_m^0$とする.以上より,
  \begin{equation}
    \ket{n^1}=\sum_{m}\ket{m^0}\braket{m^0}{n^1}
  \end{equation}
  \begin{screen}
  \begin{equation}
    \ket{n^1}=\sum_{m\ne n}\frac{\bra{m^0}\hat{V}\ket{n^0}}{E_n^0-E_m^0}\ket{m^0}
  \end{equation}
  \end{screen}
  を得る.
\end{document}
