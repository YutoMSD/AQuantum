\documentclass{report}
\input{../../../head.tex}
\begin{document}
  \refe{1st}の両辺に$\bra{n^1}$を作用すると
  \begin{align}
    \bra{n^0}E_n^1\ket{n^0} = \bra{n^0}\hat{V}\ket{n^0}
  \end{align}
  を得る.よって,1次摂動によるエネルギー補正は
  \begin{itembox}[l]{1次摂動によるエネルギー補正}
    \begin{align}
      E_n^1 = \bra{n^0}\hat{V}\ket{n^0}
    \end{align}
  \end{itembox}
  である.
  \begin{myex}{ヘリウム原子の基底エネルギー}{}
    \begin{align}
      \hat{H} = \hat{H}^0 + \frac{e^2}{4\pi\epsilon_0r_{12}}\equiv\hat{H}^0 + \hat{V}
    \end{align}
    $\hat{H}^0$の基底エネルギーは
    \begin{align}
      \psi^0 = \frac{Z^3}{\pi a_0^3}\e^{-Z(r_1+r_2)/a_0}
    \end{align}
    である.よって,$\hat{V}$による1次のエネルギー補正は以下のように計算できる.
    \begin{align}
      E^1 &= \bra{\psi^0}\hat{V}\ket{\psi^0}\\
      &= \int{\psi^0}^{*}\frac{e^2}{4\pi\epsilon_0r_{12}}\psi^0\dd{\bm{r}_1}\dd{\bm{r}_2} \\
      &= \frac{5}{4}Z\ \mathrm{Ry}
    \end{align}
    よって,基底エネルギー
    \begin{align}
      E_0&=E^0+E^1\\
      &= -8\ \mathrm{Ry}+\frac{5}{4}\times{2}\ \mathrm{Ry}\\
      &= -74.8\ \mathrm{eV}
    \end{align}
    を得る\footnote{測定値は$-78.6\ \mathrm{eV}$}.
  \end{myex}
  次に固有ベクトルの補正を求める.\refe{1st}の両辺に$\bra{m^0}\ (m\ne n)$を作用する.
  \begin{align}
    (E_n^0-\hat{H}^0)\bra{m^0}\ket{n^1}+0=\bra{m^0}\hat{V}\ket{n^0}\\
    \braket{m^0}{n^1}=\frac{\bra{m^0}\hat{V}\ket{n^0}}{E_n^0-E_m^0}\\
  \end{align}
  ただし,エネルギー縮退は無く,$E_n^0-E_m^0$とする.以上より,
  \begin{align}
    \ket{n^1} = \sum_{m}\ket{m^0}\braket{m^0}{n^1}
  \end{align}
    \begin{screen}
      \begin{align}
        \ket{n^1} = \sum_{m\ne n}\frac{\bra{m^0}\hat{V}\ket{n^0}}{E_n^0 - E_m^0}\ket{m^0}
      \end{align}
    \end{screen}
  を得る.
\end{document}
