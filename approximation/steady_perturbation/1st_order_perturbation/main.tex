\documentclass{report}
\input{../../../head.tex}
\begin{document}
  まずエネルギー補正$E_n^{(1)}$について考える.
  \refe{1st-perturbation}の両辺に$\bra{n^{(0)}}$を作用すると,
  \begin{align}
    \mel**{n^{(0)}}{\qty(E_n^{(0)} - \hat{H}^{(0)})}{n^{(1)}} + \mel**{n^{(1)}}{E_n^{(1)}}{n^{(0)}} &= \mel{n^{(0)}}{\hat{V}}{n^{(0)}} \\ 
    \Leftrightarrow E_n^{(0)}\braket{n^{(0)}}{n^{(1)}} - E_n^{(0)}\braket{n^{(0)}}{n^{(1)}} + E_n^{(1)}\braket{n^{(1)}}{n^{(0)}} &= \mel{n^{(0)}}{\hat{V}}{n^{(0)}} \\ 
    \Leftrightarrow 0 + E_n^{(1)}\braket{n^{(0)}} &= \ev**{\hat{V}}{n^{(0)}} \\ 
    \Leftrightarrow E_n^{(1)} &= \ev**{\hat{V}}{n^{(0)}}
  \end{align}
  を得る.よって,1次摂動によるエネルギー補正は,
  \begin{itembox}[l]{1次摂動によるエネルギー補正}
    \begin{align}
      E_n^{(1)} = \ev**{\hat{V}}{n^{(0)}}
    \end{align}
  \end{itembox}
  である.
  \par
  次に固有ベクトル$\ket{n^{(1)}}$の補正を求める.
  \refe{1st-perturbation}の両辺に$\bra{m^{(0)}}$を左から作用すると,
  \begin{align}
    \mel**{m^{(0)}}{\qty(E_n^{(0)} - \hat{H}^{(0)})}{n^{(1)}} + \mel**{m^{(0)}}{E_n^{(1)}}{n^{(0)}} &= \mel**{m^{(0)}}{\hat{V}}{n^{(0)}} \\ 
    E_n^{(0)}\braket{m^{(0)}}{n^{(1)}} - E_m^{(0)}\braket{m^{(0)}}{n^{(1)}} + 0 &= \mel**{m^{(0)}}{\hat{V}}{n^{(0)}}\\
    \qty(E_n^{(0)} - E_m^{(0)})\braket{m^{(0)}}{n^{(1)}} &= \mel**{m^{(0)}}{\hat{V}}{n^{(0)}} \label{1st-perturbation-prep}
  \end{align}
  となる.\refe{1st-perturbation-prep}に$\bra{m^{(0)}}$をかけて,$m$に関して和を取れば,
  \begin{align}
    \sum_{m}\qty(E_n^{(0)} - E_m^{(0)})\ket{m^{(0)}}\braket{m^{(0)}}{n^{(0)}} &= \sum_{m}\mel**{m^{(0)}}{\hat{V}}{n^{(0)}}\ket{m^{(0)}} \\ 
    \Leftrightarrow \sum_{m}\qty(E_n^{(0)} - E_m^{(0)})\ket{n^{(0)}} &= \sum_{m}\mel**{m^{(0)}}{\hat{V}}{n^{(0)}}\ket{m^{(0)}} \\ 
    \Leftrightarrow \ket{n^{(0)}}\sum_{m \neq n}\qty(E_n^{(0)} - E_m^{(0)}) &= \sum_{m \neq n}\mel**{m^{(0)}}{\hat{V}}{n^{(0)}}\ket{m^{(0)}} \\
    \Leftrightarrow \ket{n^{(0)}} &= \sum_{m\neq n}\cfrac{\mel**{m^{(0)}}{\hat{V}}{n^{(0)}}}{E_n^{(0)} - E_m^{(0)}}\ket{m^{(0)}}
  \end{align}
  となる.
  途中の式変形でエネルギー縮退がないので,
  \begin{align}
    E_n^{(0)} - E_m^{(0)} 
    \begin{dcases}
      = 0 & n = m \label{no-degeneracy} \\ 
      \neq 0 & n \neq m
    \end{dcases}
  \end{align}
  とした.
  また,Hermite演算子である$\hat{H}^{(0)}$の固有ベクトルに関する完全性より,
  \begin{align}
    I = \sum_{m}\ketbra{m^{(0)}}{m^{(0)}}\label{completeness}
  \end{align}
  を用いた.
  \begin{itembox}[l]{1次摂動による固有ベクトル補正}
    \begin{align}
      \ket{n^{(1)}} = \sum_{m\neq n}\cfrac{\mel**{m^{(0)}}{\hat{V}}{n^{(0)}}}{E_n^{(0)} - E_m^{(0)}}\ket{m^{(0)}}\label{1st-order-eigenvector}
    \end{align}
  \end{itembox}
  を得る.
  \refe{1st-order-eigenvector}において,$\ket{n^{(1)}}$と$\ket{n^{(0)}}$は直交することに注意する.
\end{document}
