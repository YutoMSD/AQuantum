\documentclass{standalone}
\input{../../../head.tex}
\begin{document}
  1次摂動の式
  \begin{align}
    \label{1ji}
    (E_n^0-\hat{H}^0)\ket{n^1}+E_n^1\ket{n^0}=\hat{V}\ket{n^0}\\
    \ket{n^1}=\sum_{m\ne n}\ket{m^0}\frac{\bra{m^0}\hat{V}\ket{n^0}}{E_n^0-E_m^0}
  \end{align}
  これは$E_n^0=E_m^0$となる$m\ne n$が存在すると発散してしまう.そのため,発散する項は別で扱う必要がある.
  以下のような2重縮退がある場合を考える.
  \begin{align}
    \hat{H}^0\ket{n_a^0}&=E_n^0\ket{n_a^0}\\
    \hat{H}^0\ket{n_b^0}&=E_n^0\ket{n_b^0}
  \end{align}
  ただし,$\braket{n_i^0}{n_j^0}=\delta_{ij}$とする.$\ket{n_a^0}$と$\ket{n_b^0}$は同じ固有値をもつため,これらの線形結合
  $\ket{n_0}=\alpha\ket{n_a^0}+\beta\ket{n_b^0}$も解となる.\\
  まず,式(\ref{1ji})の両辺に左から$\bra{n_a^0}$を作用する.
  \begin{equation}
    \bra{n_a^0}(E_n^0-\hat{H}^0)\ket{n^1}+\bra{n_a^0}E_n^1\ket{n^0}=\bra{n_a}\hat{V}\ket{n^0}
  \end{equation}
  第1項は$E_n^0-E_n^0$より0.ここで,$\bra{n_i^0}\hat{V}\ket{n_j^0}=\omega_{ij}$とおけば
  \begin{equation}
    \alpha E_n^1=\alpha\omega_{aa}+\beta\omega_{ab}
  \end{equation}
  を得る.式(\ref{1ji})の両辺に左から$\bra{n_b^0}$を作用することも考えることにより,合わせて
  \begin{equation}
    \begin{cases}
      \alpha\omega_{aa}+\beta\omega_{ab}=\alpha E_n^1\\
      \alpha\omega_{ba}+\beta\omega_{bb}=\beta E_n^1
    \end{cases}
  \end{equation}
  を得る.これは行列を用いて以下のように書き直される.
  \begin{equation}
    \label{gyouretu}
    \begin{pmatrix}
      \omega_{aa}-E_n^1&\omega_{ab}\\
      \omega_{ba}&\omega_{bb}-E_n^1
    \end{pmatrix}
    \begin{pmatrix}
      \alpha\\
      \beta
    \end{pmatrix}
    =
    \begin{pmatrix}
      0\\0
    \end{pmatrix}
  \end{equation}
  $(\alpha,\beta)=(0,0)$以外の解を持つには行列式が0となればいいので,
  \begin{equation}
    \begin{vmatrix}
      \omega_{aa}-E_n^1&\omega_{ab}\\
      \omega_{ba}&\omega_{bb}-E_n^1
    \end{vmatrix}
    =0
  \end{equation}
  である.よって,1次の摂動エネルギーとして
  \begin{screen}
  \begin{equation}
    \label{syukutai}
    E_n^1=\frac{1}{2}\qty[(\omega_{aa}+\omega_{bb}\pm\sqrt{(\omega_{aa}-\omega_{bb})^2+4|\omega_{ab}|^2})]
  \end{equation}
  \end{screen}
  を得る.縮退が解けてエネルギーが2つに分かれている.
  \begin{myex}{}{}$\omega_{aa}=\omega_{bb}=0,\omega_{ab}=\omega_{ba}=\omega$の場合
  式(\ref{gyouretu})は
  \begin{equation}
    \begin{pmatrix}
      -E_n^1&\omega\\
      \omega&-E_n^1
    \end{pmatrix}
    \begin{pmatrix}
      \alpha\\
      \beta
    \end{pmatrix}
    =\begin{pmatrix}
      0\\0
    \end{pmatrix}
  \end{equation}
  となる.よって,
  \begin{equation}
    E_n^1=
    \begin{cases}
      +\omega\ (\alpha=1,\beta=1)\\
      -\omega\ (\alpha=1,\beta=-1)
    \end{cases}
  \end{equation}
  である.これは摂動を加える前に縮退していた2つの状態$\ket{n^0}=\ket{n_a^0}\pm\ket{n_b^0}$の縮退が解け,エネルギー$E_n^0+\omega$をもつ状態$\ket{n_a^0}+\ket{n_b^0}$
  とエネルギー$E_n^0-\omega$をもつ状態$\ket{n_a^0}-\ket{n_b^0}$に分かれたことを意味している.
  \end{myex}
\end{document}
