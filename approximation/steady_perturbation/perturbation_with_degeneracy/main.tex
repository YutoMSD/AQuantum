\documentclass{report}
\input{../../../head.tex}
\begin{document}
  1次摂動のエネルギー補正を求めるときに縮退が無いという条件である\refe{no-degeneracy}を用いて,
  \begin{align}
    \qty(E_n^{(0)} - \hat{H}^{(0)})\ket{n^{(1)}} + E_n^{(1)}\ket{n^{(0)}} &= \hat{V}\ket{n^{(0)}} \label{1st-order-perturbation-re-before} \\
    \ket{n^{(1)}} &= \sum_{m\neq n}\ket{m^{(0)}}\frac{\mel**{m^{(0)}}{\hat{V}}{n^{(0)}}}{E_n^{(0)} - E_m^{(0)}}\label{1st-order-perturbation-re-after}
  \end{align}
  と計算していた.
  では,エネルギー縮退が存在するとき,どのように計算すればよいだろうか.
  $\hat{H}^{(0)}$の固有値$E_n^{(0)}$に,異なる2つの固有ベクトル$\ket{n_a^{(0)}}$,$\ket{n_b^{(0)}}$が属すると状況を考える.
  すなわち,
  \begin{align}
    \hat{H}^{(0)}\ket{n_a^{(0)}} &= E_n^{(0)}\ket{n_a^{(0)}} \\
    \hat{H}^{(0)}\ket{n_b^{(0)}} &= E_n^{(0)}\ket{n_b^{(0)}}
  \end{align}
  が成立するときである.
  ただし,Hermite演算子の固有ベクトルは規格直交化できるので,$\braket{n_i^{(0)}}{n_j^{(0)}} = \delta_{ij}$とする.
  $\ket{n_a^{(0)}}$と$\ket{n_b^{(0)}}$は同じ固有値をもつため,これらの線形結合
  $\ket{n^{(0)}} = \alpha\ket{n_a^{(0)}} + \beta\ket{n_b^{(0)}}$も,固有値$E_n^{(0)}$に属する固有ベクトルである.\\
  まず,\refe{1st-order-perturbation-re-before}の両辺に左から$\bra{n_a^{(0)}}$を作用すると,
  \begin{align}
    \mel**{n_a^{(0)}}{\qty(E_n^{(0)} - \hat{H}^{(0)})}{n^{(1)}} + \mel**{n_a^{(0)}}{E_n^{(1)}}{n^{(0)}} &= \mel**{n_a^{(0)}}{\hat{V}}{n^{(0)}} \\ 
    \Leftrightarrow E_n^{(0)}\braket{n_a^{(0)}}{n^{(1)}} - E_n^{(0)}\braket{n_a^{(0)}}{n^{(1)}} + E_n^{(1)}\braket{n_a^{(0)}}{n^{(0)}} &= \mel**{n_a^{(0)}}{\hat{V}}{n^{(0)}} \\ 
    \Leftrightarrow \alpha E_n^{(1)} &= \mel**{n_a^{(0)}}{\hat{V}}{n^{(0)}}\label{degeneracy-ket-alpha}
  \end{align}
  となる.
  $V_{ij} \coloneqq \mel**{n^{(0)}_i}{\hat{V}}{n^{(0)}_j}$とすれば\refe{degeneracy-ket-alpha}は,
  \begin{align}
    \alpha E_n^{(1)} = \alpha V_{aa} + \beta V_{ab}
  \end{align}
  と書ける.同様に\refe{1st-order-perturbation-re-before}の両辺に左から$\bra{n_b^{(0)}}$を作用させたときも考えれば,
  \begin{align}
    \begin{dcases}
      \alpha V_{aa} + \beta V_{ab} = \alpha E_n^{(1)}\\
      \alpha V_{ba} + \beta V_{bb} = \beta E_n^{(1)}
    \end{dcases}
  \end{align}
  を得る.これは行列を用いて,
  \begin{align}
    \mqty(
      V_{aa} - E_n^{(1)} &  V_{ab} \\
      V_{ba} & V_{bb} - E_n^{(1)})
    \mqty(\alpha\\ \beta)
    =
    \mqty(0\\ 0)\label{secular-eq}
  \end{align}
  のように書き直される.
  行列の部分がHermite行列になっているので,\refe{secular-eq}は永年方程式である.
  永年方程式が非自明な解を持つ条件は,
  \begin{align}
    \mqty|
      V_{aa} - E_n^{(1)} &  V_{ab}\\
      V_{ba} & V_{bb} - E_n^{(1)}|
    = 0
  \end{align}
  である.よって1次の摂動エネルギーとして,
  \begin{itembox}[l]{縮退がある場合の摂動論による1次エネルギー補正}
    \begin{align}
      E_n^{(1)} = \frac{1}{2}\qty[( V_{aa}+ V_{bb}\pm\sqrt{( V_{aa}- V_{bb})^2 + 4| V_{ab}|^2})]\label{1st-order-energy-with-degeneracy}
    \end{align}
  \end{itembox}
  を得る.\refe{1st-order-energy-with-degeneracy}をみると,定常状態では1種類であったエネルギーが,
  縮退が解けて2つに分かれている.
  \begin{myex}{}{}
    $ V_{aa} =  V_{bb} = 0$,$ V_{ab} =  V_{ba} =  V$の場合
    \refe{secular-eq}は,
    \begin{align}
      \mqty
        -E_n^{(1)} &  V\\
        V & -E_n^{(1)})
      \mqty(\alpha\\\beta)
      = \mqty(0\\ 0)
    \end{align}
    となる.よって1次エネルギー補正は,
    \begin{align}
      E_n^{(1)}=
      \begin{dcases}
        + V & \alpha = 1, \beta = 1\\
        - V & \alpha = 1, \beta = -1
      \end{dcases}
    \end{align}
    と求まる.
    これは摂動を加える前に縮退していた2つの状態$\ket{n^{(0)}} = \ket{n_a^{(0)}}\pm\ket{n_b^{(0)}}$の縮退が解け,
    エネルギー$E_n^{(0)} +  V$をもつ状態$\ket{n_a^{(0)}} + \ket{n_b^{(0)}}$とエネルギー$E_n^{(0)} -  V$をもつ状態$\ket{n_a^{(0)}} - \ket{n_b^{(0)}}$に
    分かれたことを意味している.
  \end{myex}
\end{document}
