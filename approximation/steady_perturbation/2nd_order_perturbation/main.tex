\documentclass{report}
\input{../../../head.tex}
\begin{document}
  \refe{2nd-perturbation}の両辺に$\bra{n^{(0)}}$を作用すると,
  \begin{align}
    \mel**{n^{(0)}}{\qty(E_n^{(0)} - \hat{H}^{(0)})}{n^{(2)}} + \mel**{n^{(0)}}{E_n^{(1)}}{n^{(1)}} + \mel**{n^{(0)}}{E_n^{(2)}}{n^{(0)}} &= \mel**{n^{(0)}}{\hat{V}}{n^{(1)}} \\ 
    E_n^{(0)}\braket{n^{(0)}}{n^{(2)}} - E_n^{(0)}\braket{n^{(0)}}{n^{(2)}} + E_n^{(1)}\braket{n^{(0)}}{n^{(1)}} + E_n^{(2)}\braket{n^{(0)}}{n^{(0)}} &= \mel**{n^{(0)}}{\hat{V}}{n^{(1)}}\\
    E_n^{(2)} &= \mel{n^{(0)}}{\hat{V}}{n^{(1)}}\label{2nd-order-tmp}
  \end{align}
  となる.
  途中,\refe{1st-order-eigenvector}より,$\ket{n^{(1)}}$と$\ket{n^{(0)}}$は直交することを用いた.
  \refe{1st-order-eigenvector}を\refe{2nd-order-tmp}に代入すると,
  \begin{align}
    E_n^{(2)} &= \mel{n^{(0)}}{\hat{V}}{n^{(1)}} \\ 
    &= \sum_{m\neq n}\cfrac{\mel**{m^{(0)}}{\hat{V}}{n^{(0)}}}{E_n^{(0)} - E_m^{(0)}}\mel{n^{(0)}}{\hat{V}}{m^{(0)}} \\ 
    &= \sum_{m\neq n}\cfrac{\abs{\mel**{m^{(0)}}{\hat{V}}{n^{(0)}}}^2}{E_n^{(0)} - E_m^{(0)}}
  \end{align}
  と計算できて,2次摂動によるエネルギー補正を得る.
  \begin{itembox}[l]{2次摂動によるエネルギー補正}
    \begin{align}
      E_n^{(2)} = \sum_{m\neq n}\cfrac{\abs{\mel**{m^{(0)}}{\hat{V}}{n^{(0)}}}^2}{E_n^{(0)} - E_m^{(0)}}\label{2nd-order-perturbation-eigenvalue}
    \end{align}
  \end{itembox}
  また,基底状態のエネルギーは$E_{0}^{(0)} < E_m^{(0)}$である.
  つまり,常に\refe{2nd-order-perturbation-eigenvalue}の和の部分の分母は負であるため,
  \textbf{基底状態のエネルギーは2次摂動により必ず下がる}.
\end{document}
