\documentclass{report}
\input{../../../head.tex}
\begin{document}
  再びヘリウム原子のエネルギーを考えてみよう.今度は,変分法ではなく,1次の摂動を用いる.
  \begin{myex}{ヘリウム原子の基底エネルギー(1次摂動)}{}
    水素陽原子のHamiltonianを$\hat{H}_0$とし,電子-電子相互作用を水素陽原子に対する摂動$\hat{V}$と解釈してヘリウム原子の基底エネルギーを計算する.
    \begin{align}
      \hat{H} = \hat{H}^0 + \frac{e^2}{4\pi\epsilon_0r_{12}}\equiv\hat{H}^0 + \hat{V}
    \end{align}
    $\hat{H}^0$の基底エネルギーは,
    \begin{align}
      \psi^0 = \frac{Z^3}{\pi a_0^3}\e^{-Z(r_1+r_2)/a_0}
    \end{align}
    である.よって,$\hat{V}$による1次のエネルギー補正は以下のように計算できる.
    \begin{align}
      E^1 &= \bra{\psi^0}\hat{V}\ket{\psi^0}\\
      &= \int{\psi^0}^{*}\frac{e^2}{4\pi\epsilon_0r_{12}}\psi^0\dd{\bm{r}_1}\dd{\bm{r}_2} \\
      &= \frac{5}{4}Z\ \mathrm{Ry}
    \end{align}
    よって,基底エネルギー
    \begin{align}
      E_0&=E^0+E^1\\
      &= -8\ \mathrm{Ry}+\frac{5}{4}\times{2}\ \mathrm{Ry}\\
      &= -74.8\ \mathrm{eV}
    \end{align}
    を得る\footnote{測定値は$-78.6\ \mathrm{eV}$}.
  \end{myex}
\end{document}
