\documentclass{report}
\input{../../../head.tex}
\begin{document}
  再びヘリウム原子のエネルギーを考えてみよう.今度は,変分法ではなく,1次の摂動を用いる.
  \begin{myex}{ヘリウム原子の基底エネルギー(1次摂動)}{}
    水素陽原子のハミルトニアンを$\hat{H}_0$とし,電子-電子相互作用を水素陽原子に対する摂動$\hat{V}$と解釈してヘリウム原子の基底エネルギーを計算せよ.
    ただし,
    \begin{align}
      \hat{H} &\coloneqq \hat{H}^{(0)} + \hat{V} \\ 
      \hat{H}^{(0)} &\coloneqq -\cfrac{\hbar^2}{2m}\grad_{1}^{2} -\cfrac{\hbar^2}{2m}\grad_{2}^{2} - \cfrac{2e^2}{4\pi \epsilon_0 r_1}-\cfrac{2e^2}{4\pi \epsilon_0 r_2} \\ 
      \hat{V} &\coloneqq \frac{e^2}{4\pi\epsilon_0r_{12}} \\ 
    \end{align}
    である.
    \tcblower
    $\hat{H}^{(0)}$の基底状態の固有関数は,
    \begin{align}
      \psi^{(0)} = \frac{Z^3}{\pi a_0^3}\exp\qty[-\frac{Z(r_1+r_2)}{a_0}]
    \end{align}
    である.よって,$\hat{V}$による1次のエネルギー補正は以下のように計算できる.
    \begin{align}
      E^{(1)} &= \mel**{\psi^{(0)}}{\hat{V}}{\psi^{(0)}} \\
      &= \int{\psi^{(0)}}^{*}\frac{e^2}{4\pi\epsilon_0r_{12}}\psi^{(0)}\dd{\bm{r}_1}\dd{\bm{r}_2} \\
      &= \frac{5}{4}Z\ \r{Ry}
    \end{align}
    よって,基底エネルギー
    \begin{align}
      E_0 &= E^{(0)} + E^{(1)}\\
      &= -8\ \r{Ry} + \frac{5}{4}\times{2}\ \r{Ry}\\
      &= -74.8\ \r{eV}
    \end{align}
    を得る\footnote{測定値は$-78.6\ \mathrm{eV}$}.
  \end{myex}
\end{document}
