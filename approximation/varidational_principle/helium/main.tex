\documentclass{report}
\input{../../../head.tex}
\begin{document}
  本節では,変分法の威力を確認するために,ヘリウム原子の基底エネルギーを考える.
  ヘリウム原子において,$m/M \to 0$であり,原子核が動かない(原子核の運動エネルギーが無視できる)とする.
  これをBorn-Oppenheimer近似という.
  ヘリウム原子は電荷$2e$の原子核と電荷$-e$の電子を2つもつので,ハミルトニアンは,
  \begin{align}
    \hat{H} = -\cfrac{\hbar^2}{2m}\grad_{1}^{2} -\cfrac{\hbar^2}{2m}\grad_{2}^{2} - \cfrac{2e^2}{4\pi \epsilon_0 r_1} - \cfrac{2e^2}{4\pi \epsilon_0 r_2} + \cfrac{e^2}{4\pi \epsilon_0 r_{12}}
  \end{align}
  である.
  第1項から第4項は水素陽原子のハミルトニアン $\hat{H}^0$であり厳密に解くことが出来ることを利用して,第5項を無視して考えたときと,試行関数を定めて変分法を用いたときを比較する.
  なお,実験によりヘリウム原子の基底エネルギーは$-78.6\ \r{eV}$と求まっている.
  \begin{figure}[htbp]
    \centering
    \includegraphics[width=0.5\columnwidth]{fig/helium.pdf}
    \caption{ヘリウム原子の構造}\label{helium-atom}
  \end{figure}
  \begin{myex}{ヘリウム原子の基底エネルギー(荒い近似)}{rough-helium}
    計算を行うと,ヘリウムの原子番号を$Z$として$\hat{H}^0$の基底波動関数,
    \begin{align}
      \psi = \cfrac{Z^3}{\pi a_0^3}\exp\qty(-Z\frac{r_1 + r_2}{a_0})\label{helium}
    \end{align}
    と$\hat{H}^0$の基底エネルギー,
    \begin{align}
      E = -8\ \r{Ry}\approx -108.8\ \r{eV}\label{helium-rough-min}
    \end{align}
    が求まる\footnote{
      $a_0 = \cfrac{4\pi\epsilon_0\hbar^2}{me^2} \approx 5.29\times 10^{-11}\ \r{m}$: Bohr半径
    }\footnote{
      $Z = 2$
    }\footnote{
      $\r{Ry} = \cfrac{\hbar^2}{2m\omega^2} \approx 13.6 \ \r{eV}$: Rydberg定数
    }.
  \end{myex}
  \begin{myex}{ヘリウム原子の基底エネルギー(変分法)}{}
    \exref{rough-helium}の結果とヘリウム原子の基底エネルギーの測定結果は$-78.6\ \r{eV}$と大きく異なっているため,相互作用の項を取り入れた近似を考える.
    \refe{helium}を試行関数$\psi(Z)$とする.
    $\psi(Z)$を用いてエネルギーを計算する.
    \begin{align}
      E(Z) &= \cfrac{\displaystyle\int\psi^{*}\hat{H}\psi \dd{\bm{r}_1}\dd{\bm{r}_2}}{\displaystyle\int\psi^{*}\psi \dd{\bm{r}_1}\dd{\bm{r}_2}} \\
      &= -2\qty(4Z - Z^2 - \cfrac{5}{8}Z)\ \r{Ry}\label{helium-energy-function} 
    \end{align}
    となる.\refe{helium-energy-function}が最小となるような$Z$を$Z_0$とすると$Z_0 = 27/16$であったので,
    \begin{align}
      E(Z) \geq E\qty(Z_0) = -77.5\ \r{eV}\label{helium-energy-varidation-min}
    \end{align}
    となった.\refe{helium-energy-varidation-min}と\refe{helium-rough-min}を比べると,変分法による近似の方が真の基底エネルギ$-78.6\ \r{eV}$に近い値が得られた\footnote{
      $Z_0 < 2$は遮蔽効果により有効電荷が$2e$より小さくなったことを意味する.
    }\footnote{
      積分の計算はDavid J. Griffith, \textit{Introduction to Quantum Mechanics}, pp. 333-334にある.
    }.
  \end{myex}
\end{document}