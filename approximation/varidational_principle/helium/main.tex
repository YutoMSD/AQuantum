\documentclass{report}
\input{../../../head.tex}
\begin{document}
  本節では,変分法の威力を確認するために,ヘリウム原子の基底エネルギーを考える.
  ヘリウム原子において,$\frac{m}{M}\to 0$であり,原子核が動かない(原子核の運動エネルギーが無視できる)とする.ヘリウム原子は電荷$2e$の原子核と電荷$-e$の電子を2つもつので,Hamiltonianは,
    \begin{align}
      \hat{H} = -\cfrac{\hbar^2}{2m}\qty(\grad_{1}^{2} + \grad_{2}^{2}) - \cfrac{2e^2}{4\pi \epsilon_0 r_1}-\cfrac{2e^2}{4\pi \epsilon_0 r_2} + \cfrac{e^2}{4\pi \epsilon_0 r_{12}}
    \end{align}
    である.
    第1項から第4項は水素様原子のHamiltonian $\hat{H}^0$であり厳密に解くことが出来ることを利用して,第5項を無視して考えたときと,試行関数を定めて変分法を用いたときを比較する.
    なお,実験によりヘリウム原子の基底エネルギーは$-78.6\ \r{eV}$と求まっている.
  \begin{myex}{ヘリウム原子の基底エネルギー(荒い近似)}{rough-helium}
    厳密な計算の結果$\hat{H}^0$の基底波動関数,
    \begin{align}
      \psi = \cfrac{Z^3}{\pi a_0^3}\exp\qty(-Z\frac{r_1 + r_2}{a_0})\label{helium}
    \end{align}
    $\hat{H}^0$の基底エネルギー,
    \begin{align}
      E = -8\ \r{Ry}\approx -108.8\ \r{eV}
    \end{align}
    が求まる\footnote{
      $a_0 = \cfrac{4\pi\epsilon_0\hbar^2}{me^2} \approx 5.29\times 10^{-11}\ \r{m}$: Bohr半径
    }\footnote{
      $Z = 2$
    }\footnote{
      $\r{Ry} = \cfrac{\hbar^2}{2m\omega^2} \approx 13.6 \ \r{eV}$: Rydberg定数
    }.
  \end{myex}
  \begin{myex}{ヘリウム原子の基底エネルギー(変分法)}{}
    \exref{rough-helium}ではヘリウム原子の基底エネルギーの測定結果は$-78.6\ \r{eV}$と大きく異なっているため,相互作用の項を取り入れた近似が必要である.
    そこで,\refe{helium}を試行関数$\psi(Z)$とする.
    $\psi(Z)$を用いてエネルギーを計算する.
    \begin{align}
      E(Z) &= \cfrac{\int\psi^{*}\hat{H}\psi \dd{\bm{r}_1}\dd{\bm{r}_2}}{\int\psi^{*}\psi \dd{\bm{r}_1}\dd{\bm{r}_2}} \\
      &= -2\qty(4Z - Z^2 - \cfrac{5}{8}Z)\ \mathrm{Ry}\\ 
      &\geq E\qty(Z_0),\ Z_0 = \cfrac{27}{16}\\
      &=-77.5\ \r{eV}
    \end{align}
    真の基底エネルギー$-78.6\ \r{eV}$に近い値が得られた\footnote{
      $Z_0 < 2$は遮蔽効果により有効電荷が$2e$より小さくなったことを意味する.
    }\footnote{
      積分の計算はDavid J. Griffith, \textit{Introduction to Quantum Mechanics}, pp. 333-334にある.
    }.
  \end{myex}
\end{document}