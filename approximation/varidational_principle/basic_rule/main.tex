\documentclass{report}
\input{../../../head.tex}
\begin{document}
  変分法は,
  \begin{screen}
    \begin{enumerate}
      \item \textbf{試行関数}$\ket{\psi}$をたくさん用意し,
      \item それぞれのエネルギー$E(\psi)$を計算し,
      \item その中で最小の$E(\psi)$を$E_0$の近似解とする
    \end{enumerate}
  \end{screen}
  近似法である.
  \begin{figure}[H]
    \centering
    \includegraphics[width=0.6\columnwidth]{fig/variational_pmethod.pdf}
    \caption{変分法}
    \label{variational-image}
  \end{figure}
  まず,変分法の基本原理を説明する.
  \begin{itembox}[l]{変分法の基本原理}
    任意の状態ベクトル$\ket{\psi}$に対して$\ket{\psi}$でのエネルギー関数$E(\psi)$について,
    \begin{align}
      E(\psi) = \cfrac{\mel**{\psi}{\hat{H}}{\psi}}{\braket{\psi}} \geq E_0 \label{minimum}
    \end{align}
    なる不等式が成り立つ.
    ただし$E_0$は$\hat{H}$の固有エネルギーの中で最低のものである.
  \end{itembox}
  \begin{proof}
    任意の状態ベクトル$\ket{\psi}$をHilbert空間の基底$\ket{k}$を用いて,
    \begin{align}
      \ket{\psi} = \sum_{k} c_k\ket{k} \label{basis_expansion}
    \end{align}
    と展開する.
    また,
    \begin{align}
      \hat{H}\ket{k} = E_k\ket{k}
    \end{align}
    とする.
    \refe{basis_expansion}を\refe{minimum}に用いると,
    \begin{align}
      E(\psi) &= \cfrac{\mel**{\psi}{\hat{H}}{\psi}}{\braket{\psi}} \\ 
      &= \cfrac{\displaystyle \sum_{k, k'}\mel**{k}{c_k^*\hat{H}c_{k'}}{k'}}{\displaystyle \sum_{k, k'}\mel**{k}{c_k^*c_{k'}}{k'}} \\ 
      &= \cfrac{\displaystyle \sum_{k, k'}c_n^*c_{n'}E_{k'}\braket{k}{k'}}{\displaystyle \sum_{k, k'}c_k^*c_{k'}\braket{k}{k'}} \\ 
      &= E_k \geq E_0
    \end{align}
    となる.
    つまり,あらゆる状態ベクトル$\ket{\psi}$のエネルギーは基底エネルギー$E_0$以上である.
  \end{proof}
  \begin{myex}{}{}
    ポテンシャル$V(x) = \lambda x^4$中に粒子がある系を考える.
    この系のハミルトニアンは,
    \begin{align}
      \hat{H} = -\cfrac{\hbar^2}{2m}\dv[2]{x} + \lambda x^4
    \end{align}
    である.予想される基底状態が満たすべき条件は,
    \begin{itemize}
      \item $x = 0$で存在確率が最大
      \item $\abs{x} \to \infty$で存在確率が0
      \item 節がない\footnote{節があると微係数が大きい点が存在し,これは運動エネルギーを大きくしてしまう.}
    \end{itemize}
    である.この条件と変分法を用いて,基底エネルギーの近似値を求めよ.
    試行関数として,
    \begin{align}
      \psi(x, \alpha) = \exp\qty(-\frac{\alpha x^2}{2})
    \end{align}
    なる$\psi(x, \alpha)$を用いよ.
    ただし,$\alpha > 0$である.
    \tcblower
    エネルギー関数を計算する.
    見通しをよくするために,
    \begin{align}
      I_0 &\coloneqq \int_{-\infty}^{\infty}\e^{-\alpha x^2} \dd{x} \\ 
      I_1 &\coloneqq \int_{-\infty}^{\infty}x^2\e^{-\alpha x^2} \dd{x} \\ 
      I_2 &\coloneqq \int_{-\infty}^{\infty}x^4\e^{-\alpha x^2} \dd{x} 
    \end{align}
    とすると,
    \begin{align}
      E(\alpha) &= \cfrac{\displaystyle\int_{-\infty}^{\infty} \psi^{*}\hat{H}\psi \dd{x}}{\displaystyle\int_{-\infty}^{\infty} \psi^{*}\psi \dd{x}} \\ 
      &= \frac{\displaystyle \cfrac{\hbar^2}{2m}\alpha\int_{-\infty}^{\infty}\e^{-\alpha x^2}\dd{x} -\cfrac{\hbar^2\alpha^2}{2m}\int_{-\infty}^{\infty}x^2\e^{-\alpha x^2} \dd{x} + \lambda\int_{-\infty}^{\infty}x^4\e^{-\alpha x^2}\dd{x}}{\displaystyle\int_{-\infty}^{\infty}\e^{-\alpha x^2}\dd{x}} \\
      &= \frac{\cfrac{\hbar^2}{2m}\alpha I_0 - \cfrac{\hbar^2\alpha^2}{2m}I_1 + \lambda I_2}{I_0}\label{E1}
    \end{align}
    である.$I_1$は,
    \begin{align}
      I_1 = \int_{-\infty}^{\infty}x^2\e^{-\alpha x^2}\dd{x} &= \int_{-\infty}^{\infty}x\dv{x}\qty(-\frac{\e^{-\alpha x^2}}{2\alpha}) \dd{x} \\ 
      &= -\frac{1}{2\alpha}\qty(\qty[x\e^{-\alpha x^2}]_{-\infty}^{\infty} - \int_{-\infty}^{\infty}\e^{-\alpha x^2}\dd{x}) \\ 
      &= \frac{1}{2\alpha} \int_{-\infty}^{\infty}\e^{-\alpha x^2}\dd{x} \\ 
      &= \frac{1}{2\alpha} I_0\label{E1-I1}
    \end{align}
    と計算できる.$I_2$は,
    \begin{align}
      I_2 = \int_{-\infty}^{\infty}x^4\e^{-\alpha x^2} \dd{x} &= \int_{-\infty}^{\infty}x^3\dv{x}\qty(-\frac{\e^{-\alpha x^2}}{2\alpha}) \dd{x} \\ 
      &= -\frac{1}{2\alpha}\qty(\qty[x^3\e^{-\alpha x^2}]_{-\infty}^{\infty} - 3\int_{-\infty}^{\infty}x^2\e^{-\alpha x^2}\dd{x}) \\ 
      &= \frac{3}{2\alpha}I_1 \\ 
      &= \frac{3}{4\alpha^2}I_0
    \end{align}
    であるから,\refe{E1}は,
    \begin{align}
      E(\alpha) &= \frac{\cfrac{\hbar^2}{2m}\alpha I_0 - \cfrac{\hbar^2\alpha^2}{2m}\cfrac{1}{2\alpha}I_0 + \lambda\cfrac{3}{4\alpha} I_0}{I_0} \\ 
      &= \frac{\hbar^2\alpha}{4m} + \frac{3\lambda}{4\alpha^2}
    \end{align}
    となる.
    第1項は運動エネルギーを,第2項はポテンシャルエネルギーを,それぞれ表している\footnote{
      ポテンシャルエネルギーの項は$\alpha$が大きくなるほど小さくなる.
      これは,波動関数が狭まり$x=0$での存在確率が大きくなるためである.
      一方,運動エネルギーの項は$\alpha$が大きくなるほど大きくなる.
      これは,不確定性関係$\Delta x\Delta p \geq \cfrac{\hbar}{2}$より,運動量のばらつきが大きくなるためである.
    }.
    \refe{E1}の最小値が基底エネルギー$E_0$の近似解である.よって,$\dv{\alpha}E(\alpha_0) = 0$となる$\alpha_0$を\refe{E1}に代入することで近似解,
    \begin{align}
      E(\alpha_0) = \cfrac{3}{8}\qty(\cfrac{6\hbar^4\lambda}{m^2})^{1/3}
    \end{align}
    を得る.
  \end{myex}
\end{document}