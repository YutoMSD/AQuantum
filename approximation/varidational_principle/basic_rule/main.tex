\documentclass{standalone}
\input{../../../head.tex}
\begin{document}
  \begin{myprop}{変分法の基本原理}{}
    任意の状態ベクトル$\ket{\psi}$に対して以下の不等式が成り立つ.
    \begin{align}
      E(\psi) = \cfrac{\bra{\psi}\hat{H}\ket{\psi}}{\bra{\psi}\ket{\psi}} \ge E_0 \label{ne}
    \end{align}
    \tcblower
    \begin{proof}
      任意の状態ベクトル$\ket{\psi}$を
      \begin{align}
        \ket{\psi} = \sum_{k} c_k\ket{k} \label{sv}
      \end{align}
      と展開する.左から$\bra{k'}$を作用させると
      \begin{align}
        \bra{k'}\ket{\psi} = \sum_{k}c_k\bra{k'}\ket{k} = \sum_{k}c_k\delta_{k',k} = c_{k'}
      \end{align}
      を得る.これは任意の$k$に対して成り立つので\refe{sv}は以下のように変形できる.
      \begin{align}
        \ket{\psi}&=\sum_{k}\bra{k}\ket{\psi}\ket{k}\\
        &=\sum_{k}\ket{k}\bra{k}\ket{\psi}
      \end{align}
      これを用いて\refe{ne}の分母を以下のように変形する.
      \begin{align}
        \bra{\psi}\hat{H}\ket{\psi}&=\bra{\psi}\hat{H}\sum_{k}\ket{k}\bra{k}\ket{\psi}\\
        &=\sum_{k}\bra{\psi}\hat{H}\ket{k}\bra{k}\ket{\psi}\\
        &=\sum_{k}E_k\bra{\psi}\ket{k}\bra{k}\ket{\psi}\\
        &=\sum_{k}E_k \abs{\bra{k}\ket{\psi}}^2
      \end{align}
      また,
      \begin{align}
        \braket{\psi}{\psi}=\sum_{k}\abs{\braket{k}{\psi}}^2
      \end{align}
      であるので,
      \begin{align}
        E(\psi) &= \cfrac{\bra{\psi}\hat{H}\ket{\psi}}{\braket{\psi}{\psi} } \\ 
        &= \cfrac{\sum_{k}E_k\abs{\braket{k}{\psi}}^2}{\sum_{k}\abs{\braket{k}{\psi}}^2} \\ 
        &\geq \cfrac{\sum_{k}E_0\abs{\braket{k}{\psi}}^2}{\sum_{k}\abs{\braket{k}{\psi}}^2} = E_0
      \end{align}
      が示される.
    \end{proof}
  \end{myprop}
  \refe{ne}よりあらゆる状態ベクトル$\ket{\psi}$のエネルギーは基底エネルギー$E_0$以上である.
  変分法は,
  \begin{screen}
    \begin{enumerate}
      \item \textbf{試行関数}$\ket{\psi}$をたくさん用意し,
      \item それぞれのエネルギー$E(\psi)$を計算し,
      \item その中で最小の$E(\psi)$を$E_0$の近似解とする
    \end{enumerate}
  \end{screen}
  近似法である.
  \begin{myex}{}{}
    ポテンシャル$V(x)=\lambda x^4$中に粒子がある系を考える.
    この系のHamiltonianは
    \begin{align}
      \hat{H} = -\cfrac{\hbar^2}{2m}\cfrac{d^2}{dx^2} + \lambda x^4
    \end{align}
    である.予想される基底状態が満たすべき条件は
    \begin{itemize}
      \item $x = 0$で存在確率が最大
      \item $\abs{x} \to \infty$で存在確率が0
      \item 節がない\footnote{節があると微係数が大きい点が存在し,これは運動エネルギーを大きくしてしまう.}
    \end{itemize}
    である.この条件と変分法を用いて,エネルギーの近似値を求めよ.
    \tcblower
    試行関数として$\psi(x,\alpha) = \e^{-\tfrac{\alpha x^2}{2}}$,$\alpha > 0$を考える.
    \refe{ne}の右辺を計算すると,
    \begin{align}
      E(\alpha) &= \cfrac{\int \psi^{*}\hat{H}\psi \dd{x}}{\int \psi^{*}\psi \dd{x}} \label{E1} \\ 
      &= \cfrac{\hbar^2 \alpha}{4m} + \cfrac{3\lambda}{4\alpha^2}
    \end{align}
    を得る.
    第1項は運動エネルギーを,第2項はポテンシャルエネルギーを,それぞれ表している\footnote{
      ポテンシャルエネルギーの項は$\alpha$が大きくなるほど小さくなる.
      これは,波動関数が狭まり$x=0$での存在確率が大きくなるためである.
      一方,運動エネルギーの項は$\alpha$が大きくなるほど大きくなる.
      これは,不確定性関係$\Delta x\Delta p \ge \cfrac{\hbar}{2}$より,運動量のばらつきが大きくなるためである.
    }.
    \refe{E1}の最小値が基底エネルギー$E_0$の近似解である.よって,$\dv{\alpha}E(\alpha_0) = 0$となる$\alpha_0$を\refe{E1}に代入することで近似解,
    \begin{align}
      E(\alpha_0) = \cfrac{3}{8}\qty(\cfrac{6\hbar^4\lambda}{m^2})^{1/3}
    \end{align}
    を得る.
  \end{myex}
\end{document}