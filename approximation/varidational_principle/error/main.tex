\documentclass{report}
\input{../../../head.tex}
\begin{document}
  真の基底状態$\ket{E_0}$に第1励起状態$\ket{E_1}$を10\%含んだ試行関数$\ket{\psi} = \ket{E_0} + \cfrac{1}{10}\ket{E_1}$を使ってエネルギーを計算する.
  \begin{align}
    E(\psi) &= \frac{\mel**{\psi}{\hat{H}}{\psi}}{\braket{\psi}} \\
    &= \cfrac{\mel**{E_0}{\hat{H}}{E_0} + \cfrac{1}{100}\me**l{E_1}{\hat{H}}{E_1}}{1 + \cfrac{1}{100}} \\
    &= \cfrac{E_0 + 0.01E_1}{1.01}\\
    &\approx 0.99E_0 + 0.01E_1
  \end{align}
  試行関数で10\%含まれていた誤差がエネルギーでは1\%に収まっている.
  \begin{myex}{}{}
    無限井戸型ポテンシャル$[-a, a]$を考える.
    この問題を厳密に解けば$n$番目のエネルギー準位は,
    \begin{align}
      E_n = \cfrac{\hbar^2}{2m}\qty(\cfrac{n\pi}{2a})^2
    \end{align}
    と計算できるが,ここでは変分法を用いて近似解を求める.
    予想される試行関数の条件は
    \begin{itemize}
      \item $\psi(a) = \psi(-a) = 0$
      \item 節がない
    \end{itemize}
    である.よって今回は
    \begin{align}
      \psi(x) = a^2 - x^2
    \end{align}
    を採用する.この試行関数を用いたときの基底エネルギーを見積もれ.
    \tcblower
    \begin{align}
      E(\psi) &= \cfrac{\displaystyle \int_{-a}^{a}\qty(a^2 - x^2)\qty(-\frac{\hbar^2}{2m}\dv[2]{x})\qty(a^2 - x^2)\dd{x}}{\displaystyle \int_{-a}^{a}\qty(a^2 - x^2)^2\dd{x}} \\
      &= \cfrac{10}{\pi^2}E_1 \\
      &\approx 1.01E_1
    \end{align}
    真の基底エネルギー$E_1$に近い値が得られた\footnote{このくらいの計算が期末試験に出たことがある.}.
  \end{myex}
\end{document}