\documentclass{report}
\input{../../../head.tex}
\begin{document}
  真の基底状態$\ket{E_0}$に第1励起状態$\ket{E_1}$を$10\%$含んだ試行関数$\ket{\psi}=\ket{E_0}+\frac{1}{10}\ket{E_1}$を使ってエネルギーを計算する.
  \begin{align}
    E(\psi)&=\frac{\bra{\psi}\hat{H}\ket{\psi}}{\braket{\psi}{\psi}}\\
    &=\frac{\bra{E_0}\hat{H}\ket{E_0}+1/100\bra{E_1}\hat{H}\ket{E_1}}{1+1/100}\\
    &=\frac{E_0+1/100E_1}{1.01}\\
    &\approx 0.99E_0+0.01E_1
  \end{align}
  試行関数で10\%含まれていた誤差がエネルギーでは1\%に収まっている.
  \begin{myex}{}{}
    無限井戸型ポテンシャル$[-a,a]$
    厳密に解くことができるが,ここでは変分法を用いて近似解を求める.
    予想される試行関数の条件は
    \begin{itemize}
      \item $\psi(x=\pm a)=0$
      \item 節がない
    \end{itemize}
    である.よって今回は
    \begin{equation}
      \psi(x)=a^2-x^2
    \end{equation}
    を採用する.
    \begin{align}
      E(\psi)&=\frac{\int_{-a}^{a}\qty(a^2-x^2)\qty(-\frac{\hbar^2}{2m}\frac{d^2}{dx^2})\qty(a^2-x^2)dx}{\int_{-a}^{a}\qty(a^2-x^2)^2dx}\\
      &=\frac{10}{\pi^2}E_0\\
      &\approx 1.01E_0
    \end{align}
    真の基底エネルギー$E_0$に近い値が得られた\footnote{このくらいの計算が期末試験に出たことがある.}.
  \end{myex}
\end{document}