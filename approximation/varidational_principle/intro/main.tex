\documentclass{report}
\input{../../../head.tex}
\begin{document}
  \begin{align}
    \hat{H}\ket{k} = E_k\ket{k}\label{SE}
  \end{align}
  変分法(variational principle)とはHamiltonianの基底エネルギー$E_0$の近似法である
  \footnote{
    近似法には摂動法と変分法がある.
    摂動法はHamiltonianが厳密に解ける項$\hat{H}^0$と摂動項$\hat{\delta}$を用いて,$\hat{H}=\hat{H}^0+\hat{\delta}$と表され,摂動項が小さいときのみ有効である.
    これに対し,変分法はどんなときでも有効である.
  }.変分法は\refe{SE}において$\hat{H}$の一般の固有値を求めることが困難であるとき.基底エネルギーのみを求めるにときに用いられる.
  量子系において,基底エネルギーは系の特徴の1つであるため,それが分かることだけでも,十分な議論となる場合があるのだ.
\end{document}
