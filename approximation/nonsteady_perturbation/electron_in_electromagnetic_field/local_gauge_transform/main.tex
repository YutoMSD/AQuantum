\documentclass{report}
\input{../../../../head.tex}
\begin{document}
  波動関数$\psi(\bm{r})$を次のように変換する.
  \begin{equation}
    \psi(\bm{r})\to\psi'(\bm{r})=\mathrm{e}^{i\alpha(\bm{r})}\psi(\bm{r})
  \end{equation}
  この場合,運動量がGauge不変でなくなってしまう.
  例えば波動関数の微分を計算すると
  \begin{equation}
    \nabla\psi'(\bm{r})=i(\nabla\alpha(\bm{r})\mathrm{e}^{i\alpha(\bm{r})})\psi(\bm{r})+\mathrm{e}^{i\alpha(\bm{r})}\nabla\psi(\bm{r})=\mathrm{e}^{i\alpha(\bm(r))}(\nabla+i\nabla\alpha(\bm{r}))\psi(\bm{r}) 
  \end{equation}
  余分な項が加わってしまう.よって,
  \begin{equation}
    \begin{cases}
      i\hbar\frac{\partial}{\partial t}\psi'\ne -\frac{\hbar^2}{2m}\nabla^2\psi'\\
      \bra{\psi}\hat{A}\ket{\psi}\ne\bra{\psi'}\hat{A}\ket{\psi'}
    \end{cases}
  \end{equation}
  である.したがって,局所Gauge変換に対して物理は不変ではない.\\
  \textbf{局所Gauge不変性を基本原理とする物理を再構築する.}

  まずは,微分を次の共変微分として再定義する.
  \begin{equation}
    \bm{D}=\nabla+\i\frac{e}{\hbar}\bm{A}
  \end{equation}
  ただし,$\psi$と$\bm{A}$はGauge変換により
  \begin{align}
    \psi&\to\psi'=\e^{\i\alpha{\bm{r}}}\psi\\
    \bm{A}&\to\bm{A}'=\bm{A}-\frac{\hbar}{e}\nabla\alpha(\bm{r})
  \end{align}
  このように微分を定義すると,
  \begin{equation}
    \bm{D}'\psi'(\bm{r})=\e^{\i\alpha(\bm{r})}\bm{D}\psi(\bm{r})
  \end{equation}
  つまり,局所Gauge変換は波動関数の微分を
  \begin{equation}
    \bm{D}\psi(\bm{r})\to\e^{\i\alpha(\bm{r})}\bm{D}\psi(\bm{r})
  \end{equation}
  と変換することがわかる.これは大域的Gauge変換による$\nabla\psi(\bm{r})\to\e^{\i\alpha}\nabla\psi(\bm{r})$と同じ形をしている.

  次に,$\alpha(\bm{r})$に時間依存性を持たせ,$alpha(\bm{r},t)$とする.つまり波動関数を
  \begin{equation}
    \psi\to\psi'=\e^{\i\alpha(\bm{r},t)}\psi
  \end{equation}
  と変換する.このとき,時間についての偏微分を
  \begin{equation}
    D_t=\frac{\partial}{\partial t} - \i\frac{e}{\hbar}\phi
  \end{equation}
  と定義する.ただし,
  \begin{equation}
    \phi \to \phi' + \frac{\hbar}{e}\frac{\partial}{\partial t}\alpha(\bm{r},t)
  \end{equation}
  である.以上で定義した共変微分と時間微分を用いるとSchrödinger方程式は
  \begin{align}
    \i\hbar D'_t\psi' &= -\frac{\hbar^2}{2m}\bm{D}'^2\psi'\\
    \i\hbar\frac{\partial}{\partial t}\psi&= -\frac{\hbar^2}{2m}\qty(\nabla + \i\frac{e}{\hbar}\bm{A})^2\psi - e\phi\psi\\
    &= \qty[-\frac{\hbar^2}{2m}\qty(\frac{\i}{\hbar}\bm{p} + \i\frac{e}{\hbar}\bm{A})^2 - e\phi]\psi
  \end{align}
  と変換される.
  よって,
  \begin{itembox}[l]{局所Gauge変換に対して不変なSchrödinger方程式}
    \begin{equation}
      \i\hbar\frac{\partial}{\partial t}\psi=\qty[\frac{1}{2m}\qty(\bm{p} + e\bm{A})^2 - e\phi]\psi
    \end{equation}
  \end{itembox}
  を得る.

  以上の流れをまとめると,局所Gauge不変性を要請した.それによりGauge場$\bm{A}$が導入された.よって,電磁場の起源は局所Gauge不変性であるといえる.
\end{document}