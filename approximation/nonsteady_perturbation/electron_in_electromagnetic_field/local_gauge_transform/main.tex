\documentclass{report}
\input{../../../head.tex}
\begin{document}
  波動関数$\psi(\bm{r})$を次のように変換する.
  \begin{equation}
    \psi(\bm{r})\to\psi'(\bm{r})=\mathrm{e}^{i\alpha(\bm{r})}\psi(\bm{r})
  \end{equation}
  この場合,運動量がGauge不変でなくなってしまう.
  例えば波動関数の微分を計算すると
  \begin{equation}
    \nabla\psi'(\bm{r})=i(\nabla\alpha(\bm{r})\mathrm{e}^{i\alpha(\bm{r})})\psi(\bm{r})+\mathrm{e}^{i\alpha(\bm{r})}\nabla\psi(\bm{r})=\mathrm{e}^{i\alpha(\bm(r))}(\nabla+i\nabla\alpha(\bm{r}))\psi(\bm{r}) 
  \end{equation}
  余分な項が加わってしまう.よって,
  \begin{equation}
    \begin{cases}
      i\hbar\frac{\partial}{\partial t}\psi'\ne -\frac{\hbar^2}{2m}\nabla^2\psi'\\
      \bra{\psi}\hat{A}\ket{\psi}\ne\bra{\psi'}\hat{A}\ket{\psi'}
    \end{cases}
  \end{equation}
  である.したがって,局所Gauge変換に対して物理は不変ではない.\\
  \textbf{局所Gauge不変性を基本原理とする物理を再構築する.}
\end{document}
