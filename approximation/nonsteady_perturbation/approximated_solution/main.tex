\documentclass{report}
\input{../../../head.tex}
\begin{document}
  非定常摂動量子系の時間発展は,
  \begin{align}
    \begin{dcases}
      \ket{\psi(t)}_{\r{I}} = \sum_{n}c_n(t)\ket{n} \label{non-steady-quantum-system-re} \\
      \forall \i\ \hbar\dv{t}c_m(t) = \sum_{n}V_{mn}(t)\e^{i\omega_{mn}t}c_n(t)
    \end{dcases}
  \end{align}
  で表されるのであった.
  一般に\refe{non-steady-quantum-system-re}を解くことはできないので,近似解を得ることを考える.
  $\hat{V}(t) \to \lambda\hat{V}(t)$として$c_m(t)$をべき級数展開すると,
  \begin{align}
    c_m(t) = c_m^{(0)}(t) + \lambda c_m^{(1)}(t) + \lambda^2c_m^{(2)}(t) + \cdots\label{cnt-taylor-expantion}
  \end{align}
  となる.
  \refe{cnt-taylor-expantion}を\refe{non-steady-quantum-system-re}の第2式に代入すると,
  \begin{align}
    \forall m\  \i\hbar\dv{t}\qty(c_m^{(0)}(t) + \lambda c_m^{(1)}(t) + \lambda^2c_m^{(2)}(t) + \cdots) = \sum_{n}\lambda V_{mn}(t)\e^{i\omega_{mn}t}\qty(c_m^{(0)}(t) + \lambda c_m^{(1)}(t) + \lambda^2c_m^{(2)}(t) + \cdots)
  \end{align}
  となるから,
  \begin{align}
    \begin{dcases}
      \lambda^0 :& \forall m\ \i\hbar\dv{t}c_m^{(0)}(t) = 0 \label{non-steady-quantum-system-re-expantion} \\ 
      \lambda^1 :& \forall m\ \i\hbar\dv{t}c_m^{(1)}(t) = \sum_{n}V_{mn}(t)\e^{\i\omega_{mn}t}c_n^{(0)}(t)
    \end{dcases}
  \end{align}
  \refe{non-steady-quantum-system-re-expantion}の$\lambda^0$の項の結果より,
  \begin{align}
    \forall m\ c_m^{(0)}(t) = \r{const.}\label{non-steady-quantum-system-re-const}
  \end{align}
  である.
  \par
  $t = t_0$から摂動$\hat{V}(t)$を加え始めたときを考える.
  $t = t_0$で系の量子状態が$\ket{i}$であったとする.
  このとき,
  \begin{align}
    c_m^{(0)}(t_0) = \delta_m^i
  \end{align}
  である.\refe{non-steady-quantum-system-re-const}より,0次の係数$c_m(t)$は時間変化しないので,
  \begin{align}
    c_m^{(0)}(t) = \delta_m^i
  \end{align}
  となる.
  この系の状態は初期状態$\ket{i}$に依ることがわかったので,
  これからは$c_m(t) \to c_{m, i}(t)$,$c_m^{(0)}(t) \to c_{m, i}^{(0)}(t)$,$c_m^{(1)}(t) \to c_{m, i}^{(1)}(t)$のように書き替える.
  \refe{non-steady-quantum-system-re-expantion}の$\lambda^1$の係数より,
  \begin{align}
    \i\hbar\dv{t}c_{m, i}^{(1)}(t)& = \sum_{n}V_{mn}(t)\e^{\i\omega_{mn}t}c_{n, i}^{(0)}(t)\\
    & = V_{m, i}(t)\e^{\i\omega_{mi}t} \\
    \Rightarrow c_{m, i}^{(1)}(t)& = -\frac{\i}{\hbar}\int_{t_0}^{t}V_{m, i}(t)\e^{i\omega_{mi}t}\dd{t}\label{cmi1t-progress}
  \end{align}
  となり,$c_{m, i}^{(1)}(t)$が求まった.
  また,系の量子状態を$\lambda^1$の項までで近似すると,
  \begin{align}
    \ket{\psi(t)}_{\r{I}} &= \sum_{n}c_{n, i}(t)\ket{n} \\ 
    &\simeq \sum_{n} \qty(c_{n, i}^{(0)}(t) + c_{n, i}^{(1)}(t))\ket{n} \\ 
    &= \ket{i} + \sum_{n}c_{n, i}^{(1)}(t)\ket{n} \\ 
    &= \ket{i} + c_{i, i}^{(1)}(t)\ket{i} + \sum_{n\neq i}c_{n, i}^{(1)}(t)\ket{n}
  \end{align}
  と表される.なお今後のために,$c_{n, i}^{(1)}$で$n = i$と$n \neq i$に分けた.
  まとめると,$\ket{\psi(t_0)}_{\r{I}} = \ket{i}$の時間発展は以下のように書ける.
  \begin{itembox}[l]{$\ket{\psi(t_0)}_{\r{I}} = \ket{i}$の時間発展}
    \begin{align}
      \begin{dcases}
        \ket{\psi(t)}_{\r{I}}& = \qty(1 + c_{i, i}^{(1)}(t))\ket{i} + \sum_{n\neq i}c_{n, i}^{(1)}(t)\ket{n}\label{interaction-picture-time-development-of-qc} \\
        c_{n, i}^{(1)}(t) &= -\frac{\i}{\hbar}\int_{t_0}^{t}V_{n, i}\e^{i\omega_{ni}t}\dd{t}
      \end{dcases}
    \end{align}
  \end{itembox}
  また,始状態$\ket{i}$から終状態$\ket{f}\ (f\neq i)$への遷移確率は,
  \begin{align}
    \abs{\braket{f}{\psi(t)}_{\r{I}}}^2 &= \abs{\qty(1 + c_{i, i}^{(1)}(t))\braket{f}{i} + \sum_{n\neq i}c_{n, i}^{(1)}(t)\braket{f}{n}}^2 \\ 
    &= \abs{c_{f, i}^{(1)}(t)}^2
  \end{align}
  である.
  \par
  さて,このようにして得られた$\ket{\psi}_{\r{I}}$の時間発展について,次節では$\hat{V}$が一定のときを,次々節では$\hat{V}$が余弦関数で書けるときを議論する.
\end{document}
