\documentclass{report}
\input{../../../head.tex}
\begin{document}
  本節では,摂動の大きさが時間に依存しない場合を考える.
  摂動の大きさが時間に依存しないことは,定常摂動とは異なる.
  定常摂動では,量子系に摂動をかけ始めてから十分長い時間が経った状況のみ扱えるが,非定常摂動を考えると,量子系の時間変化を任意の時刻で追跡できる.
  \par
  さて,時刻$t = 0$から一定の摂動$\hat{V}$を加え始めたとき考える.$\hat{V}(t)$は,
  \begin{align}
    \hat{V}(t) =
    \begin{dcases}
      0 & t \leq 0 \\
      \hat{V} & t \geq 0
    \end{dcases}
  \end{align}
  と書ける.
  $V_{fi}(t)$は,
  \begin{align}
    V_{fi}(t) = \mel**{f}{\hat{V}(t)}{i} =
    \begin{dcases}
      0 & t \leq 0 \\
      \mel**{f}{\hat{V}}{i} & t \geq 0
    \end{dcases}
  \end{align}
  である.$V_{fi}$を
  \begin{align}
    V_{fi} \coloneqq \mel**{f}{\hat{V}}{i}
  \end{align}
  定義する.また,$t = 0$で量子状態は$\ket{i}$であったとする.
  時刻$t$で状態$\ket{f}$である確率を考える.
  \refe{interaction-picture-time-development-of-qc}の第2式より,
  \begin{align}
    c_{f, i}^{(1)}(t) &= -\frac{\i}{\hbar}\int_{0}^{t}V_{fi}\e^{\i\omega_{fi}t'}\dd{t'} \\
    &= -\frac{V_{fi}}{\hbar\omega_{fi}}\qty(\e^{\i\omega_{fi}t} - 1) \\ 
    &= -\frac{V_{fi}}{\hbar\omega_{fi}}\exp\qty(\frac{\i\omega_{fi}}{2}t)\qty[\exp\qty(\frac{\i\omega_{fi}}{2}t) - \exp\qty(-\frac{\i\omega_{fi}}{2}t)] \\ 
    &= 2\i\frac{V_{fi}}{\hbar\omega_{fi}}\exp\qty(\frac{\i\omega_{fi}}{2}t)\cdot\sin\qty(\frac{\omega_{fi}}{2}t)
  \end{align}
  と計算できるから,$\ket{i}$から$\ket{f}$への遷移確率$\abs{c_{f, i}^{(1)}(t)}^2$は,
  \begin{align}
    \abs{c_{f, i}^{(1)}(t)}^2 &= \frac{\abs{V_{fi}}^2}{\hbar^2}\qty[\frac{\sin\qty(\frac{\omega_{fi}}{2}t)}{\qty(\frac{\omega_{fi}}{2})}]^2 \\ 
    &= \frac{\abs{V_{fi}}^2}{\hbar^2}t^2\qty[\frac{\sin\qty(\frac{\omega_{fi}}{2}t)}{\qty(\frac{\omega_{fi}}{2}t)}]^2 \label{v-const-cfi-t}
  \end{align}
  である.\refe{v-const-cfi-t}において,$t$を固定して$\omega_{fi}$が満たすべき物理条件を考える.
  まず,$\abs{c_{f, i}^{(1)}(t)}^2$が0でない,つまり,$\ket{i}$から$\ket{f}$への遷移が起きるのは,$\sin\qty(\frac{\omega_{fi}}{2}t)$が0でないときである.
  また,$\qty[\frac{\sin\qty(\frac{\omega_{fi}}{2}t)}{\qty(\frac{\omega_{fi}}{2})}]^2$を$\omega_{fi}$の関数とみたとき,
  $\sin$が1周期の間のみで有効な値を持つ.
  言い換えれば,$\abs{\omega_{fi}} < \frac{2\pi}{t}$で有効な値をもつ.
  これは,$\sin$の2周期目の最大値が1周期目の最大値の$\qty(\frac{2}{3\pi})^2\simeq 4.4\ \%$程度であること依る.
  従って,$t$が小さいときはこの範囲が十分広く,$\omega_{fi} \neq 0$の状態への遷移が起こるが,
  $t$が大きいときは$\omega_{fi}\simeq 0$の状態($\sin$の1周期目)への遷移しか起きない.
  摂動を与えた瞬間に測定を行うことは困難であるという実験上の理由から,通常は$t \to \infty$としておけば良い.
  また,規格化されたデルタ関数$\tilde{\delta}$を用いると,$\r{sinc}$関数の性質から,
  \begin{align}
    \lim_{t\to\infty}\qty(\frac{\sin\qty(\frac{\omega_{fi} t}{2})}{\frac{\omega_{fi} t}{2}})^2 = \frac{2\pi}{t}\tilde{\delta}(\omega_{fi})
  \end{align}
  を使うと\footnote{$\delta(ax)=\frac{1}{\abs{a}}\delta(x)$},
  \begin{align}
    \lim_{t\to\infty}\abs{c_{f, i}^{(1)}(t)}^2 &= \frac{\abs{V_{fi}}^2}{\hbar^2}2\pi t\delta(\omega_{fi})\\
    &= \frac{2\pi}{\hbar}\abs{V_{fi}}^2\delta(E_f - E_i)t
  \end{align}
  が得られる.$t$をあまりにも大きくすると,$\abs{c_{f, i}^{(1)}(t)}^2 > 1$となり,$\abs{c_{f, i}^{(1)}(t)}^2$の時刻$t$での$\ket{i}$から$\ket{f}$への遷移確率という
  物理的な意味が失われることに注意しながら,$t\to \infty$の領域での単位時間当たりの$\ket{i}$から$\ket{f}$への遷移確率$\omega_{i\to f}$は,
  \begin{align}
    \omega_{i\to f} &= \frac{\abs{c_{f, i}^{(1)}(t)}^2}{t} \\
    &= \frac{2\pi}{\hbar}\abs{\mel**{f}{\hat{V}}{i}}^2\delta(E_f - E_i)
  \end{align}
  である.これを\textbf{Fermiの黄金律}という\footnote{Enrico Fermi(1901-1954)}.
  \begin{itembox}[l]{Fermiの黄金律}
    \begin{align}
      \omega_{i\to f} = \frac{2\pi}{\hbar}\abs{\mel**{f}{\hat{V}}{i}}^2\delta(E_f - E_i)\label{fermis-golden-rule}
    \end{align}
  \end{itembox}
  \begin{myex}{電子の弾性散乱}{}
    弾性散乱では散乱前後で粒子のエネルギーが保存される.電子のエネルギーは
    \begin{align}
      E_{\bm{k'}} = \frac{\hbar^2\abs{\bm{k'}}^2}{2m}
    \end{align}
    と表される.よって,終状態は多く存在する.始状態$i$から終状態集合 $\qty{f}$への遷移確率を求める.
    \begin{align}
      \omega_{i \to \qty{f}} &= \sum_{f}\omega_{i \to f}\\
      &= \sum_{f}\frac{2\pi}{\hbar}\abs{\mel**{f}{\hat{V}}{i}}^2\delta(E_f - E_i)\\
      &= \frac{2\pi}{\hbar}\overline{\abs{\mel**{f}{\hat{V}}{i}}^2}\sum_{f}\delta(E_f - E_i)\\
      &= \frac{2\pi}{\hbar}\overline{\abs{\mel**{f}{\hat{V}}{i}}^2}\rho(E_f)
    \end{align}
    $\overline{\abs{\mel**{f}{\hat{V}}{i}}^2}$は散乱体の性質を,状態密度$\rho(E_f)$は物質の性質を反映している.
    この関係もFermiの黄金律という.
  \end{myex}
\end{document}
