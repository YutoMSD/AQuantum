\documentclass{report}
\input{../../../head.tex}
\begin{document}
  時刻$t=0$から摂動$\hat{V}$を加え始めたとする.
  \begin{equation}
    \hat{V}(t)=
    \begin{cases}
      0\ (t\le0)\\
      \hat{V}\ (t>0)
    \end{cases}
  \end{equation}
  このとき
  \begin{equation}
    V_{fi}(t)=\bra{f}\hat{V}(t)\ket{i}=
    \begin{cases}
      0\ (t\le 0)\\
      \bra{f}\hat{V}\ket{i}\ (t>0)
    \end{cases}
  \end{equation}
  である.ここで$\bra{f}\hat{V}\ket{i}=V_{fi}$とする.$t=0$で量子状態は$\ket{i}$であったとする.
  \begin{align}
    c_{f,i}^1(t)&=-\frac{i}{\hbar}\int_{0}^{t}V_{fi}\mathrm{e}^{i\omega_{fi}t}dt\\
    &=2i\exp\qty(-i\frac{\omega_{fi}t}{2})\sin\qty(\frac{\omega_{fi}}{2}t)
  \end{align}
  よって$\ket{i}$から$\ket{f}$への遷移確率は
  \begin{equation}
    |c_{f,i}^1(t)|^2=\frac{|V_{fi}|^2}{\hbar^2}\qty(\frac{\sin\qty(\omega_{fi}t/2)}{\qty(\omega_{fi}/2)})^2
  \end{equation}
  であり,$\sin\qty(\omega_{fi}t/2)$が0でないときのみ$\ket{i}$から$\ket{f}$への遷移が起きることがわかる.また,
  $\qty(\frac{\sin\qty(\omega_{fi}t/2)}{\qty(\omega_{fi}/2)})^2$は$|\omega_{fi}|<2\pi/t$で有効な値をもつ.従って,$t$が小さいときはこの範囲が十分広く,$\omega_{fi}\ne0$の状態への遷移が起こりうる.しかし,
  $t$が大きいときは$\omega_{fi}\simeq0$の状態への遷移しか起きない.
  通常は$t\to\infty$と考えてよい.
  \begin{equation}
    \lim_{t\to\infty}\qty(\frac{\sin\qty(\omega t/2)}{\omega t/2})^2=2\pi t\delta(\omega)
  \end{equation}
  を使うと\footnote{$\delta(ax)=\frac{1}{|a|}\delta(x)$},
  \begin{align}
    \lim_{t\to\infty}|c_{f,i}^1(t)|^2&=\frac{|V_{fi}|^2}{\hbar^2}2\pi t\delta(\omega_{f,i})\\
    &=\frac{2\pi}{\hbar}|V_{fi}|^2\delta(E_f-E_i)t
  \end{align}
  が得られる.
  よって,単位時間当たりの$\ket{i}$から$\ket{f}$への遷移確率は
  \begin{align}
    \omega_{i\to f}& = |c_{f,i}^1(t)|^2/t\\
    &=\frac{2\pi}{\hbar}\qty|\bra{f}\hat{V}\ket{i}|^2\delta(E_f-E_i)
  \end{align}
  である.これを\textbf{Fermiの黄金律}という\footnote{Enrico Fermi(1901-1954)}.
  \begin{itembox}[l]{Fermiの黄金律}
    \begin{equation}
      \omega_{i\to f}=\frac{2\pi}{\hbar}\qty|\bra{f}\hat{V}\ket{i}|^2\delta(E_f-E_i)
    \end{equation}
  \end{itembox}
  \begin{myex}{}{}電子の弾性散乱
    弾性散乱では散乱前後で粒子のエネルギーが保存される.電子のエネルギーは
    \begin{equation}
      E_{\bm{k'}}=\frac{\hbar^2|\bm{k'}|^2}{2m}
    \end{equation}
    と表される.よって,終状態は多く存在する.始状態$i$から終状態グループ$\qty{f}$への遷移確率を求める.
    \begin{align}
      \omega_{i\to\qty{f}}&=\sum_{f}\omega_{i\to f}\\
      &=\sum_{f}\frac{2\pi}{\hbar}\qty|\bra{f}\hat{V}\ket{i}|^2\delta(E_f-E_i)\\
      &=\frac{2\pi}{\hbar}\overline{\qty|\bra{f}\hat{V}\ket{i}|^2}\sum_{f}\delta(E_f-E_i)\\
      &=\frac{2\pi}{\hbar}\overline{\qty|\bra{f}\hat{V}\ket{i}|^2}\rho(E_f)
    \end{align}
    $\overline{\qty|\bra{f}\hat{V}\ket{i}|^2}$は散乱体の性質を,状態密度$\rho(E_f)$は物質の性質を反映している.この関係もFermiの黄金律という.
  \end{myex}
\end{document}
