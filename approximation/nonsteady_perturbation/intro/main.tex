\documentclass{report}
\input{../../../head.tex}
\begin{document}
  摂動項が時間に依存する場合の摂動(time-dependent perturbation)\footnote{電磁波による摂動など.}を扱う.
  本節では,量子系の時間発展が状態ベクトルの時間変化で描像するSchr\"odinger描像で記述する.
  時間に依存するSchrödinger方程式,
  \begin{align}
    \i\hbar\dv{t}\ket{\psi(t)} = \hat{H}\ket{\psi(t)}\label{time-dependent-schroinger-eq}
  \end{align}
  を考える.なお,$\ket{\psi(t)}$は$\hat{H}^{(0)}$の固有ベクトルを用いて,
  \begin{align}
    \ket{\psi(t)} = \sum_{n}c_n(t)\ket{n}\label{psi-t-expantion}
  \end{align}
  と展開できたとする.$\hat{H}^{(0)}$の固有ベクトルが完全系を成すので,
  状態ベクトルが時間変化した空間は$\qty{\ket{n}\mid n = 0, 1, \cdots}$が張る空間の部分空間となることに注意する.
  $\hat{H}$の性質ごとに$\ket{\psi(t)}$の具体的な形を議論する.
  \begin{enumerate}
    \item $\hat{H} = \hat{H}^{(0)}$の場合\footnote{$\hat{H}^{(0)}$は厳密に解けるHamiltonian.}\par
      量子状態$\ket{\psi(t)}$の時間発展は時間発展演算子
      \footnote{
        $\Delta t \ll 1$として,
        \begin{align*}
          \ket{\psi(\Delta t)} = \exp\qty[-\i\frac{\hat{H}\Delta t}{\hbar}]\ket{\psi(0)} \approx\qty(\hat{I} - \i\frac{\hat{H}\Delta t}{\hbar})\ket{\psi(0)} 
          = \ket{\psi(0)} + \Delta t \left.\qty(\dv{t}\ket{\psi(t)})\right|_{t = 0}
        \end{align*}
        より微分の形で書けることから,確かに時間発展すると私は解釈する.}
      \footnote{
        演算子が交換するときは数字と同じ扱いをしても良いと考える.
        一般に演算子は,
        \begin{align*}
          \e^{\hat{A}}\e^{\hat{B}} = \exp{\hat{A} + \hat{B} + \frac{1}{2}\qty[\hat{A},\hat{B}] + \cdots}\neq\e^{\hat{A} + \hat{B}}
        \end{align*}
        なるBCH公式を満たす.
      }
      $\exp\qty(-\i\frac{\hat{H}t}{\hbar})$を用いて,
      \begin{align}
        \ket{\psi(t)} &= \exp\qty[-\i\frac{\hat{H}^{(0)}t}{\hbar}]\ket{\psi(0)} \\
        &= \sum_{n}c_n(0)\exp\qty[-\i\frac{\hat{H}^{(0)}t}{\hbar}]\ket{n} \\
        &= \sum_{n}c_n(0)\exp\qty[-\i\frac{E_n}{\hbar}t]\ket{n}
      \end{align}
      のように表すことができる.
      \par
      時間発展演算子を用いることなく計算することもできる.
      \refe{psi-t-expantion}を\refe{time-dependent-schroinger-eq}に代入すると,$\hat{H}$が時間に依存しないことに注意すれば,
      \begin{align}
        \i\hbar\dv{t}\qty(\sum_{n}c_n(t)\ket{n}) &= \hat{H}\qty(\sum_{n}c_n(t)\ket{n}) \label{naive-solution-to-decide-ct}\\ 
        \Leftrightarrow \sum_{n}\qty(\i\hbar\dv{c_n(t)}{t}\ket{n}) &= \sum_{n}\qty(c_n(t)E_n\ket{n}) \\ 
        \Leftrightarrow \forall n\ \i\hbar\dv{c_n(t)}{t}\ket{n} &= c_n(t)E_n\ket{n} \\ 
        \Leftrightarrow \forall n\ c_n(t) &= \exp\qty(-\i\frac{E_n}{\hbar}t)
      \end{align}
      を用いれば,
      \begin{align}
        \ket{\psi(t)} = \sum_{n}\exp\qty(-\i\frac{E_n}{\hbar}t)\ket{n}
      \end{align}
      を得る.
    \item $\hat{H} = \hat{H}^{(0)} + \hat{V}(t)$の場合\par
      このときは$\ket{\psi(t)}$を簡単な形で書き下すことが出来ないから,便宜的に,
      \begin{align}
        \psi(t) = \sum_{n}c_n(t)\exp\qty(-\i\frac{E_n}{\hbar}t)\ket{n}
      \end{align}
      と展開しておく.なお,$c_n(t)$が定数のときの$\ket{\psi(t)}$との整合性をとるために$\exp\qty(-\i\frac{E_n}{\hbar}t)$をかけてある.
      原理的には$c_n(t)$が求まれば量子系の時間発展の様子がわかる.% 暇になったら時間順序積のことを書く.
  \end{enumerate}
  \par
  さて,いずれの場合でも,量子系の性質を調べるには$c_n(t)$の具体的な形がわかればよいことを発見した.
  本節では,まずSchr\"odinger描像から相互作用表示に書き換え,$c_n(t)$を厳密に知ることが困難であることを知り,$c_n(t)$の近似解を導く.
\end{document}
