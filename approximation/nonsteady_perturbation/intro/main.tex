\documentclass{report}
\input{../../../head.tex}
\begin{document}
  摂動項が時間に依存する場合の摂動(time-dependent perturbation)\footnote{電磁波による摂動など.}を扱う.\\
  時間に依存するSchrödinger方程式
  \begin{align}
    \i\hbar\dv{t}\ket{\psi(t)}  =  \hat{H}\ket{\psi(t)}
  \end{align}
  \begin{enumerate}
    \item $\hat{H} = \hat{H}^0$の場合\footnote{$\hat{H}^0$は厳密に解けるHamiltonian.}\par
      量子状態の時間発展は時間発展演算子$\e^{-\i\frac{\hat{H}t}{\hbar}}$を用いて次のように表すことができる\footnote{
        $\Delta t \ll 1$として,$\ket{\psi(\Delta t)} = \e^{-\i\frac{\hat{H}\Delta t}{\hbar}}\ket{\psi(0)} \approx\qty(\hat{I} - \i\frac{\hat{H}\Delta t}{\hbar})\ket{\psi(0)} = \ket{\psi(0)} + \Delta t\left.\dv{t}\ket{\psi(t)}\right_0$
        より確かに時間発展する,と私は解釈する.
      }
      \footnote{演算子が交換するときは数字と同じ扱いをしても良いと考える.
        一般に,演算子は次のBCH公式を満たす.
        $\e^{\hat{A}}\e^{\hat{B}} = \exp{\hat{A} + \hat{B} + \frac{1}{2}\qty[\hat{A},\hat{B}] + \cdots}\neq\e^{\hat{A} + \hat{B}}$
      }.
      $\psi(0)$を固有ベクトルで展開すると,
      \begin{align}
        \ket{\psi(0)} = \sum_{n}c_n(0)\ket{n}
      \end{align}
      と書けたとする.$\ket{\psi(t)}$は,
      \begin{align}
        \ket{\psi(t)}& = \e^{-\i\hat{H}^0t/\hbar}\ket{\psi(0)}\\
        & = \sum_{n}c_n(0)\e^{-\i\hat{H}^0t/\hbar}\ket{n}\\
        & = \sum_{n}c_n(0)\e^{-\i E_nt/\hbar}\ket{n}
      \end{align}
      と書ける.
    \item $\hat{H} = \hat{H}^0+\hat{V}(t)$の場合\par
      $\hat{V}$の効果を$c_n(t)$に押し付け.
      \begin{align}
        \ket{\psi(t)} = \sum_{n}c_n(t)\e^{-iE_nt/\hbar}\ket{n}
      \end{align}
      とする.$c_n(t)$が求まれば量子系の時間発展がわかる.
  \end{enumerate}
\end{document}
