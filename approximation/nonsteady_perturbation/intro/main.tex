\documentclass{report}
\input{../../../head.tex}
\begin{document}
  摂動項が時間に依存する場合の摂動(time-dependent perturbation)\footnote{電磁波による摂動など}を扱う.\\
  時間に依存するSchrödinger方程式
  \begin{equation}
  i\hbar\frac{d}{dt}\ket{\psi(t)}=\hat{H}\ket{\psi(t)}
  \end{equation}
  (i)$\hat{H}=\hat{H}^0$の場合\footnote{$\hat{H}^0$は厳密に解けるHamiltonian.}\\
  量子状態の時間発展は時間発展演算子$\mathrm{e}^{-i\frac{\hat{H}t}{\hbar}}$を用いて次のように表すことができる\footnote{$\ket{\psi(t)}=\mathrm{e}^{-i\frac{\hat{H}t}{\hbar}}\ket{\psi(0)}\approx\qty(\hat{I}-i\frac{\hat{H}t}{\hbar})\ket{\psi(0)}=\qty(\hat{I}+t\frac{d}{dt})\ket{\psi(0)}$より確かに時間発展する,と私は解釈する.}
  \footnote{演算子が交換するときは数字と同じ扱いをしても良いと考える.一般に,演算子は次のBCH公式を満たす.$\mathrm{e}^{\hat{A}}\mathrm{B}^{\hat{B}}=\exp{\hat{A}+\hat{B}+\frac{1}{2}[\hat{A},\hat{B}]+\cdots}\ne\mathrm{e}^{\hat{A}+\hat{B}}$}.
  \begin{align}
    \ket{\psi(t)}&=\mathrm{e}^{-i\hat{H}^0t/\hbar}\ket{\psi(0)}\\
    &=\sum_{n}c_n(0)\mathrm{e}^{-i\hat{H}^0t/\hbar}\ket{n}\\
    &=\sum_{n}c_n(0)\mathrm{e}^{-iE_nt/\hbar}\ket{n}
  \end{align}
  (ii)$\hat{H}=\hat{H}^0+\hat{V}(t)$の場合\\
  $\hat{V}$の効果を$c_n(t)$に押し付け.
  \begin{equation}
    \ket{\psi(t)}=\sum_{n}c_n(t)\mathrm{e}^{-iE_nt/\hbar}\ket{n}
  \end{equation}
  とする.$c_n(t)$が求まれば量子系の時間発展がわかる.
\end{document}
