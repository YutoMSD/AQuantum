\documentclass{report}
\input{../../../head.tex}
\begin{document}
  電磁場中のHamiltonianは
  \begin{equation}
    H=\frac{1}{2m}\qty(\bm{p} + e\bm{A})^2 - e\phi
  \end{equation}
  である.今回は$\phi=0$とする.電磁場が十分弱いとし,
  \begin{align}
    \hat{H} &= \frac{1}{2m}\qty(\bm{p}+e\bm{A})^2\\
    &\simeq \frac{1}{2m}\qty(\bm{p}^2+e\bm{p}\cdot\bm{A} + e\bm{A} \cdot \bm{p})\\
  \end{align}
  と近似する.ここで,$\hat{H}^0 \equiv \frac{\bm{p}^2}{2m}$,摂動項$\hat{V}(t) \equiv \frac{e}{2m} \qty(\bm{p}\cdot\bm{A} + \bm{A} \cdot \bm{p})$
  とする.さらに,$\nabla\cdot\bm{A}=0$となるように$\bm{A}$を決める\footnote{Coulomb Gauge}.すると,
  \begin{align}
    \qty(\bm{p} \cdot \bm{A})\psi &= -\i\hbar \nabla \cdot \qty(\bm{A} \psi)\\
    &= \i\hbar\qty[\qty(\nabla \cdot \bm{A}) \psi + \bm{A} \cdot \qty(\nabla \psi)]\\
    &= \qty(\bm{A} \cdot \bm{p}) \psi
  \end{align}

  となる.よって,電子と電磁場の相互作用を摂動として加えたHamiltonian,
  \begin{equation}
    \hat{H} = \hat{H}^0 + \frac{e}{m}\qty(\bm{A} \cdot \bm{p})
  \end{equation}
  を得る.
  \begin{myex}{直線偏光}{}
    \begin{equation}
      \bm{A}(\bm{r},t) = 2A_0 \bm{e}_x \cos(\bm{k} \cdot \bm{r} - \omega t)
    \end{equation}
    を加える.ただし,$\bm{k} = \frac{\omega}{c}\bm{e}_z$とする.
    ベクトルポテンシャルと電磁場の関係より,
    \begin{equation}
      \begin{cases}
      \bm{E}(\bm{r},t) = \frac{\partial \bm{A}}{\partial t} = E_0 \bm{e}_x \sin(\bm{k} \cdot \bm{r} - \omega t)\\
      \bm{B}(\bm{r},t) = \nabla \times \bm{A} = \frac{E_0}{c}\bm{e}_y \sin(\bm{k} \cdot \bm{r} - \omega t)\\
      E_0 = -2 \omega A_0
      \end{cases}
    \end{equation}
    が成り立っている.
  \end{myex}
\end{document}