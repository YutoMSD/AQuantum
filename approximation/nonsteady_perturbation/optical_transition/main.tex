\documentclass{report}
\input{../../../head.tex}
\begin{document}
  電磁場中のHamiltonianは
  \begin{equation}
    H=\frac{1}{2m}\qty(\bm{p} + e\bm{A})^2 - e\phi
  \end{equation}
  である.今回は$\phi=0$とする.電磁場が十分弱いとし,
  \begin{align}
    \hat{H} &= \frac{1}{2m}\qty(\bm{p}+e\bm{A})^2\\
    &\simeq \frac{1}{2m}\qty(\bm{p}^2+e\bm{p}\cdot\bm{A} + e\bm{A} \cdot \bm{p})\\
  \end{align}
  と近似する.ここで,$\hat{H}^0 \equiv \frac{\bm{p}^2}{2m}$,摂動項$\hat{V}(t) \equiv \frac{e}{2m} \qty(\bm{p}\cdot\bm{A} + \bm{A} \cdot \bm{p})$
  とする.さらに,$\nabla\cdot\bm{A}=0$となるように$\bm{A}$を決める\footnote{Coulomb Gauge}.すると,
  \begin{align}
    \qty(\bm{p} \cdot \bm{A})\psi &= -\i\hbar \nabla \cdot \qty(\bm{A} \psi)\\
    &= \i\hbar\qty[\qty(\nabla \cdot \bm{A}) \psi + \bm{A} \cdot \qty(\nabla \psi)]\\
    &= \qty(\bm{A} \cdot \bm{p}) \psi
  \end{align}
  となる.よって,電子と電磁場の相互作用を摂動として加えたHamiltonian,
  \begin{equation}
    \hat{H} = \hat{H}^0 + \frac{e}{m}\qty(\bm{A} \cdot \bm{p})
  \end{equation}
  を得る.
  \begin{myex}{直線偏光}{}
    \begin{equation}
      \bm{A}(\bm{r},t) = 2A_0 \bm{e}_x \cos(\bm{k} \cdot \bm{r} - \omega t)
    \end{equation}
    を加える.ただし,$\bm{k} = \frac{\omega}{c}\bm{e}_z$とする.
    ベクトルポテンシャルと電磁場の関係より,
    \begin{equation}
      \begin{cases}
      \bm{E}(\bm{r},t) = \frac{\partial \bm{A}}{\partial t} = E_0 \bm{e}_x \sin(\bm{k} \cdot \bm{r} - \omega t)\\
      \bm{B}(\bm{r},t) = \nabla \times \bm{A} = \frac{E_0}{c}\bm{e}_y \sin(\bm{k} \cdot \bm{r} - \omega t)\\
      E_0 = -2 \omega A_0
      \end{cases}
    \end{equation}
    が成り立っている.摂動項は,
    \begin{align}
      \hat{V}(t) &= \frac{e}{m}(\bm{A}\cdot\bm{p}) \\
      &= \frac{2eA_0}{m}\cos(\bm{k} \cdot \bm{r} - \omega t)\bm{e}_x\cdot\bm{p}\\
      &= \frac{eA_0}{m}\qty[\e^{\i(\bm{k}\cdot\bm{r}-\omega t)} + \e^{-\i(\bm{k}\cdot\bm{r}-\omega t)}]p_x\\
      &= \frac{eA_0}{m}\qty(\e^{\i\bm{k}\cdot\bm{r}}p_x\e^{-\i\omega t} + \e^{-\i \bm{k} \cdot \bm{r}}p_x \e^{\i \omega t})
    \end{align}
    と表せる.光の吸収を考えるときは,,第1項$\frac{eA_0}{m}\e^{\i\bm{k}\cdot\bm{r}}p_x$が支配的なのでこの項を$\hat{V}$とする.
    単位時間当たりの遷移確率を計算する.
    \begin{align}
      \omega_{i\to f} &= \frac{2\pi}{\hbar} \qty|\bra{f}\hat{V}\ket{i}|^2\delta(E_f - E_i - \hbar \omega)\\
      &= \frac{2 \pi}{\hbar}\qty(\frac{e A_0}{m})^2 \qty|\bra{f} \e^{\i\bm{k} \cdot p_x \ket{i}}|^2 \delta(E_f - E_i - \hbar \omega)
    \end{align}
    と表せるので$\qty|\bra{f} \e^{\i\bm{k} \cdot p_x \ket{i}}|^2$を\textbf{電気双極子近似}を用いて計算する.
    原子の準位間隔は$E_f - E_i \~ 1$ eVである.これと相互作用する電磁場のエネルギーは$\hbar \omega \~ 1$ eVである.これを波長に換算すると,
    \begin{equation}
      \lambda = \frac{2\pi}{\hbar} = \frac{2\pi c}{\omega} \~ 1000\ \rm{nm}
    \end{equation}
    である.これは原子のスケール1 Åよりもはるかに大きいため,電子・原子を扱う上では電磁場は空間的に一様だとみなせる.よって,
    \begin{equation}
      \e^{\i\bm{k}\cdot\bm{r}}\simeq 1 + \i \bm{k} \cdot \bm{r} + \cdots \simeq 1
    \end{equation}
    と近似できる.これを電気双極子近似という.この近似を用いると,
    \begin{equation}
      \bra{f} \e^{\i \bm{k} \cdot \bm{r}} \ket{i} \simeq \bra{f} p_x \ket{i}
    \end{equation}
    を得る.さらに
    \begin{equation}
      [x,\hat{H}^0] = \frac{\i \hbar}{m} p_x
    \end{equation}
    であるため
    \begin{align}
      \bra{f} p_x \ket{i} &= \frac{m}{\i\hbar} \bra{f} [x,\hat{H}^0] \ket{i}\\
      &= \frac{m}{\i\hbar} \qty(\bra{f}xE_i\ket{i} - \bra{f} E_f x \ket{i})\\
      &= \frac{m}{\i\hbar} (E_i - E_f)\bra{f}x\ket{i}
    \end{align}
    を得る.よって,電磁場による単位時間当たりの遷移確率
    \begin{equation}
     \omega_{i \to f} = \frac{2 \pi}{\hbar^3} (eA_0)^2 (E_i - E_f)^2 |\bra{f} x \ket{i}|^2 \delta(E_f - E_i - \hbar\omega) 
    \end{equation}
    を得る.これは$\bra{f} x \ket{i} \ne 0$のときのみ$\omega_{i\to f} \ne 0$という選択則を表している.
  \end{myex}
\end{document}