\documentclass{report}
\input{../../../head.tex}
\begin{document}
<<<<<<< HEAD
  時間に依存する摂動を調和摂動という.
  以下の摂動を考える\footnote{光のイメージ}.
  \begin{align}
    \hat{V}(t) = 
    \begin{dcases}
      0 & t\leq 0 \\
      2V\cos\omega t & t > 0
    \end{dcases}
  \end{align}
  $2V\cos\omega t = V(\e^{i\omega t} + \e^{-i\omega t})$であるので,
  \begin{align}
    c^{(1)}_{f, i}(t) &= \frac{i}{\hbar}\int_{0}^{t}\mel**{f}{\hat{V}}{i}(\e^{\i\omega t} + \e^{-\i\omega t})\e^{\i\omega_{fi}t}\dd{t} \label{3.6} \\
    &= -\frac{V_{fi}}{\hbar}\qty(\frac{\e^{\i(\omega_{fi} + \omega)t} - 1}{\omega_{fi} + \omega} + \frac{\e^{\i(\omega_{fi} - \omega)t} - 1}{\omega_{fi} - \omega})
  \end{align}
  を得る.ここで$V_{fi} = \mel**{f}{\hat{V}}{i}$とした.
  \begin{enumerate}
    \item $\omega_{fi} - \omega \approx 0$のとき\par
      \refe{3.6}の第2項は第1項より十分大きい.よって,
      \begin{align}
        c_{f, i}^{(1)}(t)\approx - \frac{V_{fi}}{\hbar}\frac{\e^{\i(\omega_{fi} - \omega)t} - 1}{\omega_{fi} - \omega} = -\frac{V_{fi}}{\hbar}\e^{\i\frac{\omega_{fi} - \omega}{2}t}\frac{\sin\frac{\omega_{fi} - \omega}{2}t}{\frac{\omega_{fi} - \omega}{2}}
      \end{align}
      と近似できる.
      このとき$\ket{i}\to\ket{f}$の遷移確率は,
      \begin{align}
        \abs{c_{f, i}^{(1)}(t)}^2 = \frac{V_{fi}^2}{\hbar^2}\frac{\sin^2\frac{\omega_{fi} - \omega}{2}t}{\qty(\frac{\omega_{fi} - \omega}{2})^2}
      \end{align}
      $\frac{\sin^2\frac{\omega_{fi} - \omega}{2}t}{\qty(\frac{\omega_{fi} - \omega}{2})^2}$は$t\to\infty$で$2\pi t\delta(\omega_{fi} - \omega) = 2\pi t\hbar\delta(E_f - E_i - \hbar\omega)$
      と近似できるので
      単位時間当たりの遷移確率は,
      \begin{align}
        \omega_{f\to i} = \frac{\abs{c_{f, i}^{(1)}(t)}^2}{t} = \frac{2\pi}{\hbar}\abs{V_{fi}}^2\delta(E_f - E_i - \hbar\omega)
      \end{align}
      である.これもFermiの黄金律という.$E_f = E_i + \hbar\omega$へと遷移することがわかる.
    \item $\omega_{fi} + \omega\approx 0$のとき\par
      \refe{3.6}の第1項が支配的となる.上記の議論を$\omega\to\omega$と置き換えて繰り返すと単位時間当たりの遷移確率として
      \begin{align}
        \omega_{i\to f} = \frac{\abs{c_{f,i}^{(1)}(t)}^2}{t} = \frac{2\pi}{\hbar}\abs{V_{fi}}^2\delta(E_f - E_i + \hbar\omega)
      \end{align}
      を得る.$E_f = E_i - \hbar\omega$へと遷移することがわかる.
=======
  次に,時間に依存する摂動である調和摂動を議論する.
  定常状態の系に,$\hat{V}(t)$なる摂動が加わったこと考える.
  $\hat{V}(t)$を,
  \begin{align}
    \hat{V}(t) \coloneqq
    \begin{dcases}
      0 & t \leq 0 \\
      2V\cos\omega t & t \geq 0 
    \end{dcases}
  \end{align}
  と定義する\footnote{このような$\hat{V}(t)$は量子系に対して光を照射することを意味している.このような摂動はしばしば,レーザにおいて,反転分布を構成するために用いられる.}.
  $2V\cos\omega t = V(\e^{\i\omega t} + \e^{-\i\omega t})$であるので,\refe{interaction-picture-time-development-of-qc}より,
  \begin{align}
    c^{(1)}_{f, i}(t) &= \frac{\i}{\hbar}\int_{0}^{t}\mel**{f}{\hat{V}}{i}(\e^{\i\omega t} + \e^{-\i\omega t})\e^{\i\omega_{fi}t}\dd{t} \label{harmonic-perturbation-cn1t-progress} \\ 
    &= -\frac{V_{fi}}{\hbar}\qty(\frac{\e^{\i(\omega_{fi} + \omega)t} - 1}{\omega_{fi} + \omega} + \frac{\e^{\i(\omega_{fi}-\omega)t} - 1}{\omega_{fi} - \omega})
  \end{align}
  を得る.ただし,$V_{fi} \coloneqq \mel**{f}{\hat{V}}{i}$とした.
  \begin{enumerate}
    \item $\omega_{fi} - \omega \approx 0$のとき\par
      \refe{harmonic-perturbation-cn1t-progress}の第2項は第1項より十分大きいから,第1項を無視して,
      \begin{align}
        c_{f, i}^{(1)}(t) \approx - \frac{V_{fi}}{\hbar}\frac{\e^{\i(\omega_{fi} - \omega)t} - 1}{\omega_{fi} - \omega} = -\frac{V_{fi}}{\hbar}\exp\qty(\i\frac{\omega_{fi} - \omega}{2}t)\cfrac{\sin\qty(\cfrac{\omega_{fi} - \omega}{2}t)}{\cfrac{\omega_{fi} - \omega}{2}}
      \end{align}
      と近似できる.このとき$\ket{i}\to\ket{f}$の遷移確率は,
      \begin{align}
        \abs{c_{f, i}^{(1)}(t)}^2 = \frac{V_{fi}^2}{\hbar^2}\frac{\sin^2\frac{\omega_{fi} - \omega}{2}t}{\qty(\frac{\omega_{fi} - \omega}{2})^2}
      \end{align}
      $\frac{\sin^2\frac{\omega_{fi}-\omega}{2}t}{\qty(\frac{\omega_{fi} - \omega}{2})^2}$は$t\to\infty$で$2\pi t\tilde{\delta}(\omega_{fi} - \omega) = 2\pi t\hbar\tilde{\delta}(E_f - E_i - \hbar\omega)$
      と近似できるので単位時間当たりの遷移確率は,
      \begin{align}
        \omega_{f\to i} = \frac{\abs{c_{f, i}^{(1)}(t)}^2}{t} = \frac{2\pi}{\hbar}\abs{V_{fi}}^2\tilde{\delta}(E_f - E_i - \hbar\omega)\label{harmonic-perturbation-fermi-rule}
      \end{align}
      である.つまり,$E_i$から$E_f = E_i + \hbar\omega$なるエネルギー準位へ遷移することがわかる.また,\refe{harmonic-perturbation-fermi-rule}もFermiの黄金律という.
    \item $\omega_{fi} + \omega \approx 0$のとき\par
      \refe{harmonic-perturbation-cn1t-progress}の第1項が支配的となる.上記の議論を$\omega \to -\omega$と置き換えて繰り返すと単位時間当たりの遷移確率として
      \begin{align}
        \omega_{i\to f} = \frac{\abs{c_{f, i}^{(1)}(t)}^2}{t} = \frac{2\pi}{\hbar}\abs{V_{fi}}^2\delta(E_f - E_i + \hbar\omega)
      \end{align}
      を得る.つまり,$E_i$から$E_f = E_i - \hbar\omega$なるエネルギー準位へ遷移することがわかる.
>>>>>>> nonsteady_perturbation
  \end{enumerate}
\end{document}
