\documentclass{report}
\input{../../../head.tex}
\begin{document}
  時間に依存する摂動を調和摂動という.
  摂動$\hat{V}$が,
  \begin{align}
    \hat{V}(t) = 
    \begin{dcases}
      0 & t\leq 0 \\
      2\hat{V}\cos\omega t & t > 0
    \end{dcases}
  \end{align}
  で与えられるときを考える\footnote{光のイメージ}\footnote{2023年度期末試験第2問(3).}.
  $2\hat{V}\cos\omega t = \hat{V}(\e^{i\omega t} + \e^{-i\omega t})$であるので,
  \begin{align}
    c^{(1)}_{f, i}(t) &= -\frac{\i}{\hbar}\int_{0}^{t}V_{fi}(t')\e^{\i\omega_{fi}t'}\dd{t'} \\ 
    &= -\frac{i}{\hbar}\int_{0}^{t}\mel**{f}{\hat{V}}{i}(\e^{\i\omega t'} + \e^{-\i\omega t'})\e^{\i\omega_{fi}t'}\dd{t'} \\
    &= -\i\frac{V_{fi}}{\hbar}\qty(\frac{\e^{\i(\omega_{fi} + \omega)t} - 1}{\omega_{fi} + \omega} + \frac{\e^{\i(\omega_{fi} - \omega)t} - 1}{\omega_{fi} - \omega})\label{harmonic-perturbation-cnt}
  \end{align}
  を得る.ここで$V_{fi} = \mel**{f}{\hat{V}}{i}$とした.
  \begin{enumerate}
    \item $\omega_{fi} - \omega \approx 0$のとき\par
      \refe{harmonic-perturbation-cnt}の第2項は第1項より十分大きい.よって,
      \begin{align}
        c_{f, i}^{(1)}(t)\approx -\i\frac{V_{fi}}{\hbar}\frac{\e^{\i(\omega_{fi} - \omega)t} - 1}{\omega_{fi} - \omega} = -\i\frac{V_{fi}}{\hbar}\exp\qty(\i\frac{\omega_{fi} - \omega}{2}t)\frac{\sin\qty(\frac{\omega_{fi} - \omega}{2}t)}{\frac{\omega_{fi} - \omega}{2}}
      \end{align}
      と近似できる.
      このとき$\ket{i}\to\ket{f}$の遷移確率は,
      \begin{align}
        \abs{c_{f, i}^{(1)}(t)}^2 = \frac{V_{fi}^2}{\hbar^2}\frac{\sin^2\qty(\frac{\omega_{fi} - \omega}{2}t)}{\qty(\frac{\omega_{fi} - \omega}{2})^2}
      \end{align}
      $\frac{\sin^2\frac{\omega_{fi} - \omega}{2}t}{\qty(\frac{\omega_{fi} - \omega}{2})^2}$は$t\to\infty$で$2\pi t\delta(\omega_{fi} - \omega) = 2\pi t\hbar\delta(E_f - E_i - \hbar\omega)$
      と近似できるので
      単位時間当たりの遷移確率は,
      \begin{align}
        \omega_{f\to i} = \frac{\abs{c_{f, i}^{(1)}(t)}^2}{t} = \frac{2\pi}{\hbar}\abs{V_{fi}}^2\delta(E_f - E_i - \hbar\omega)
      \end{align}
      である.これもFermiの黄金律という.$E_f = E_i + \hbar\omega$へと遷移することがわかる.
    \item $\omega_{fi} + \omega\approx 0$のとき\par
      \refe{harmonic-perturbation-cnt}の第1項が支配的となる.
      $\omega_{fi} - \omega \approx 0$のときの議論を$\omega\to-\omega$と置き換えて繰り返すと単位時間当たりの遷移確率として
      \begin{align}
        \omega_{i\to f} = \frac{\abs{c_{f,i}^{(1)}(t)}^2}{t} = \frac{2\pi}{\hbar}\abs{V_{fi}}^2\delta(E_f - E_i + \hbar\omega)
      \end{align}
      を得る.$E_f = E_i - \hbar\omega$へと遷移することがわかる.
  \end{enumerate}
\end{document}
