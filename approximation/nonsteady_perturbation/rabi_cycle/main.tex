\documentclass{report}
\input{../../../head.tex}
\begin{document}
  前節の最後に,一般に時間発展する量子系を正確に追跡すること,すなわち,$c_n(t)$の厳密解を求めることが困難であると述べた.
  にもかかわらず,ある特殊な条件下では近似を行うことなく,厳密解を得ることができる.
  以下の\exref{rabi-cycle}では,そのような物理現象として\textbf{Rabi振動}(Rabi cycle)\footnote{I.I.Rabi(1898 - 1988)}
  \footnote{量子状態の振動をRabi振動という.}を議論する.
  \begin{myex}{Rabi振動}{rabi-cycle}
    厳密に解くことのできる2準位系,
    \begin{align}
      \ket{1} &\coloneqq \mqty(1\\ 0) \\
      \ket{2} &\coloneqq \mqty(0\\ 1) \\
      \hat{H}^{(0)} &\coloneqq \mqty(E_1 & 0\\ 0 & E_2)
    \end{align}
    を考える.明らかに,$E_1$と$E_2$は$\hat{H}^{(0)}$のエネルギー固有値で,それぞれに属する固有ベクトルは$\ket{1}$と$\ket{2}$である.
    この2準位系に,時刻$t = 0$から$\hat{V}(t)$なる摂動を加える.
    ただし,$\hat{V}(t)$は,
    \begin{align}
      \hat{V}(t) = \mqty(
        0 & \gamma\e^{\i\omega t} \\
        \gamma\e^{-\i\omega t} & 0
      )
    \end{align}
    によって与えられる\footnote{$\gamma$は摂動の強さを表す.}.
    系全体のハミルトニアンを$\hat{H} \coloneqq \hat{H}^{(0)} + \hat{V}(t)$とする.
    系の状態ベクトルを相互作用表示を用いて,
    \begin{align}
      \ket{\psi(t)}_{\r{I}} \coloneqq c_1(t)\ket{1} + c_2(t)\ket{2}
    \end{align}
    と書いたとき,以下の問いに答えよ.
    \begin{enumerate}
      \item $c_1(t)$と$c_2(t)$の時間発展を調べよ.ただし,初期条件は$c_1(0) = 1$,$c_2(0) = 0$とする.
      \item 1. の条件のもとで,$c_2(t)$の確率振幅が最大となる$\omega$を求めよ.
    \end{enumerate}
    \tcblower
    \begin{enumerate}
      \item $c_1(t)$と$c_2(t)$の時間発展\par
        今回の設定での$\omega_{mn}$や$V_{mn}$を計算すると,
        \begin{align}
          \omega_{11} &= \omega_{22} = 0 \\ 
          \omega_{21} &= -\omega_{12} = \frac{E_2 - E_1}{\hbar} \\ 
          V_{11} &= V_{22} = 0 \\ 
          V_{21} &= V_{12}^* = \gamma\e^{-\i\omega t}
        \end{align}
        である.非定常摂動の時間発展の式より,
        \begin{align}
          \begin{dcases}
            \i\hbar\dv{t}c_1(t) = c_1(t)V_{11}\e^{\i\omega_{11}t} + c_2(t)V_{12}\e^{\i\omega_{12}t} = c_2(t)\gamma\e^{\i\qty(\omega + \omega_{12}) t} \label{rabi-cnt-eq}\\ 
            \i\hbar\dv{t}c_2(t) = c_1(t)V_{21}\e^{\i\omega_{21}t} + c_2(t)V_{22}\e^{\i\omega_{22}t} = c_1(t)\gamma\e^{-\i\qty(\omega - \omega_{21}) t} 
          \end{dcases}
        \end{align}
        が成り立つ.
        次に,$\Delta\omega \coloneqq \omega - \omega_{21}$として\refe{rabi-cnt-eq}を解く.
        2式目を1式目に代入して$c_1(t)$を消去すると,
        \begin{align}
          \dv[2]{t}c_2(t) + \i\Delta\omega\dv{t}c_2(t) + \qty(\frac{\gamma}{\hbar})^2c_2 = 0\label{rabi-cnt-eq-c2}
        \end{align}
        2階の斉次微分方程式の解は$c_2(t) = \e^{\i\lambda t}$と書けるので,これを\refe{rabi-cnt-eq-c2}に代入すると,
        \begin{align}
          \lambda^2 + \Delta\omega\lambda - \qty(\frac{\gamma}{\hbar}) &= 0 \\
          \Rightarrow \lambda &= -\frac{\Delta\omega}{2}\pm\sqrt{\qty(\frac{\Delta\omega}{2})^2 + \qty(\frac{\gamma}{\hbar})^2}
        \end{align} 
        を得る.$\Omega$を,
        \begin{align}
          \Omega \coloneqq \sqrt{\qty(\frac{\Delta\omega}{2})^2 + \qty(\frac{\gamma}{\hbar})^2}\label{rabi-freq}
        \end{align}
        と定義して,これをRabi周波数と呼ぶ.
        $c_2$の一般解は,
        \begin{align}
          c_2(t) = \exp\qty(-\i\frac{\Delta\omega}{2}t)\qty(A\e^{\i\Omega t} + B\e^{-\i\Omega t})\label{c2t-general-solution}
        \end{align}
        と書ける.\refe{c2t-general-solution}を\refe{rabi-cnt-eq}の第2式に代入すると,
        \begin{align}
          c_1(t) = \frac{\hbar}{\gamma}\exp\qty(-\i\frac{\Delta\omega}{2}t)\qty[\frac{\Delta\omega}{2}\qty(A\e^{\i\Omega t} + B\e^{-\i\Omega t}) - \Omega\qty(A\e^{\i\Omega t} - B\e^{-\i\Omega t})]
        \end{align}
        となる.
        初期条件を考えると,
        \begin{align}
          \begin{dcases}
            A + B = 0 \\ 
            A - B = -\frac{\gamma}{\hbar\Omega}
          \end{dcases}
        \end{align}
        であるから,
        \begin{align}
          B = -A = \frac{\gamma}{2\hbar\Omega}
        \end{align}
        となる.よって,
        \begin{align}
          c_1(t) &= \exp\qty(-\i\frac{\Delta\omega}{2}t)\qty(\cos\Omega t - \i\frac{\Delta \omega}{2\Omega}\sin\Omega t) \\ 
          c_2(t) &=  - \i\frac{\gamma}{\hbar\Omega}\exp\qty(-\i\frac{\Delta\omega}{2}t)\sin\Omega t
        \end{align}
        である.時刻$t$で$\ket{1}$,$\ket{2}$に状態を見出す確率,$\abs{c_1(t)}^2$,$\abs{c_2(t)}^2$はそれぞれ,
        \begin{align}
          \abs{c_1(t)}^2 &= \cos^2\Omega t + \qty(\frac{\Delta \omega}{2})^2\frac{1}{\Omega^2}\sin^2\Omega t \\ 
          \abs{c_2(t)}^2 &= \frac{\gamma^2}{\hbar^2\Omega^2}\sin^2\Omega t
        \end{align}
        である.簡単な計算により,$\abs{c_1(t)}^2 + \abs{c_2(t)}^2 = 1$となることが容易に確かめられる.
      \item $c_2(t)$の振幅が最大となる$\omega$の値\par
        振幅の大きさは$\Omega$の定義\refe{rabi-freq}より,
        \begin{align}
          \frac{(\gamma/\hbar)^2}{(\gamma/\hbar)^2 + (\Delta\omega/2)^2}
        \end{align}
        と表されるので,$\Delta\omega = \omega - \omega_{21} = 0$のときに最大となる.つまり,摂動の周波数$\omega$と
        2準位のエネルギー差に由来する$\omega_{21}$が一致したときに遷移が起こりやすい.
    \end{enumerate}
  \end{myex}
  \exref{rabi-cycle}では,Rabi振動に関する2つの重要な物理量を得たので,下にまとめる.
  \begin{itembox}[l]{Rabi周波数}
    \begin{align}
    \Omega = \sqrt{\qty(\Delta\omega/2)^2 + \qty(\gamma/\hbar)^2}   
    \end{align}
  \end{itembox}
  \begin{itembox}[l]{共鳴条件}
    \begin{align}
      \omega = \omega_{21} = \frac{E_2 - E_1}{\hbar}
    \end{align}
  \end{itembox}
\end{document}
