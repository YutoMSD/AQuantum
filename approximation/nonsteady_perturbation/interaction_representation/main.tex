\documentclass{report}
\input{../../../head.tex}
\begin{document}
  Schr\"odinger表示の非定常摂動基本方程式は,\refe{time-dependent-schroinger-eq}と\refe{nonsteady-perturbation-cnt-eq}であり,
  \begin{align}
    \begin{dcases}
      \i\hbar\dv{t}\ket{\psi(t)} = \qty(\hat{H}^{(0)} + \hat{V}(t))\ket{\psi(t)} \label{TDSE} \\
      \ket{\psi(t)} = \sum_{n}c_n(t)\exp\qty(-\i\frac{E_n}{\hbar}t)\ket{n}
    \end{dcases}
  \end{align}
  と書けるのであった.
  \textbf{相互作用表示}(interaction picture)を\refe{interaction-picture}のような$\ket{\psi(t)} \to \ket{\psi(t)}_{\r{I}}$の変換を行って得られる状態ベクトルと定義する.
  \begin{itembox}[l]{相互作用表示}
    \begin{align}
      \ket{\psi(t)}_{\r{I}} \coloneqq \exp\qty(\i\frac{\hat{H}^{(0)}}{\hbar}t)\ket{\psi(t)}\label{interaction-picture}
    \end{align}
  \end{itembox}
  \refe{interaction-picture}を用いて,\refe{TDSE}等価な基本方程式,相互作用表示の非定常摂動基本方程式を導く.
  まず\refe{TDSE}の第2式を用いて,
  \begin{align}
    \ket{\psi(t)}_{\r{I}} &= \exp\qty(\i\frac{\hat{H}^{(0)}}{\hbar}t)\sum_{n}c_n(t)\exp\qty(-\i\frac{E_n}{\hbar}t)\ket{n}\\
    &= \sum_{n}c_n(t)\exp\qty(-\i\frac{E_n}{\hbar}t)\exp\qty(\i\frac{\hat{H}^{(0)}}{\hbar}t)\ket{n} \\ 
    &= \sum_{n}c_n(t)\exp\qty(-\i\frac{E_n}{\hbar}t)\exp\qty(\i\frac{E_n}{\hbar}t)\ket{n} \\ 
    &= \sum_{n}c_n(t)\ket{n}\label{TDSE-2nd-eq-interaction-picture}
  \end{align}
  と計算できる.
  \par
  次に,相互作用表示の時間微分を計算してみる.
  相互作用表示での摂動項$\hat{V}_{\r{I}}$を,
  \begin{align}
    \hat{V}_{\r{I}} \coloneqq \exp\qty(\i\frac{\hat{H}^{(0)}}{\hbar}t)\hat{V}(t)
  \end{align}
  とする.計算の途中で,\refe{TDSE}の第1式を用いると,
  \begin{align}
    \i\hbar\dv{t}\ket{\psi(t)}_{\r{I}} &= \dv{t}\qty[\exp\qty(\i\frac{\hat{H}^{(0)}}{\hbar}t)]\ket{\psi(t)} + \exp\qty(\i\frac{\hat{H}^{(0)}}{\hbar}t)\dv{t}\ket{\psi(t)} \\
    &= \i\frac{\hat{H}^{(0)}}{\hbar}\exp\qty(\i\frac{\hat{H}^{(0)}}{\hbar}t)\ket{\psi(t)} + \exp\qty(\i\frac{\hat{H}^{(0)}}{\hbar}t)\frac{1}{\i\hbar}\qty(\hat{H}^{(0)} + \hat{V}(t))\ket{\psi(t)} \\ 
    &= \frac{\i}{\hbar}\qty[\hat{H}^{(0)}, \exp\qty(\i\frac{\hat{H}^{(0)}}{\hbar}t)]\ket{\psi(t)} + \exp\qty(\i\frac{\hat{H}^{(0)}}{\hbar}t)\hat{V}(t)\ket{\psi(t)} \\ 
    &= \exp\qty(\i\frac{\hat{H}^{(0)}}{\hbar}t)\hat{V}(t)\ket{\psi(t)} \\
    &= \exp\qty(\i\frac{\hat{H}^{(0)}}{\hbar}t)\hat{V}(t)\exp\qty(-\i\frac{\hat{H}^{(0)}}{\hbar}t)\exp\qty(\i\frac{\hat{H}^{(0)}}{\hbar}t)\ket{\psi(t)} \\
    &= \exp\qty(\i\frac{\hat{H}^{(0)}}{\hbar}t)\hat{V}(t)\exp\qty(-\i\frac{\hat{H}^{(0)}}{\hbar}t)\ket{\psi(t)}_{\r{I}} \\
    &= \hat{V}_{\r{I}}(t)\ket{\psi(t)}_{\r{I}}\label{TDSE-1st-eq-interaction-picture}
  \end{align}
  を得る.\refe{TDSE-1st-eq-interaction-picture}と\refe{TDSE-2nd-eq-interaction-picture}はSch\"odinger表示の非定常摂動基本方程式と等価な方程式であるから,
  これを\textbf{朝永・Schwinger方程式}と呼ぶ.
  \footnote{Schrödinger描像は量子状態が時間発展するとみなす.Heisenberg描像は物理量が時間発展するとみなす.相互作用表示はその中間であるといえる.}  
  \footnote{朝永振一郎(1906-1979)}
  \footnote{Julian Schwinger(1918-1994)}
  \footnote{朝永とSchwingerは1964年にRichard Feynmannとともにノーベル賞を受賞.}.
  \begin{itembox}[l]{朝永・Schwinger方程式}
    \begin{align}
      \i\hbar\dv{t}\ket{\psi(t)}_{\r{I}}&=\hat{V}_{\r{I}}(t)\ket{\psi(t)}_{\r{I}}\label{Tomonaga-Schwinger}\\
      \hat{V}_{\r{I}}(t)&=\exp\qty(\i\frac{\hat{H}^{(0)}}{\hbar}t)\hat{V}(t)\exp\qty(-\i\frac{\hat{H}^{(0)}}{\hbar}t)
    \end{align}
  \end{itembox}
  \par
  \refe{Tomonaga}に左から$\bra{m}$を演算する($\hat{H}^{(0)}\ket{m} = E_m\ket{m}$).\\
  左辺は,
  \begin{align}
    &\bra{m}\i\hbar\dv{t}\sum_{n}c_n(t)\ket{n}\\
    & = \i\hbar\dv{t}c_m(t)
  \end{align}
  となる.
  右辺は.
  \begin{align}
    &\bra{m}\hat{V}_{\r{I}}(t)\sum_{n}c_n(t)\ket{n}\\
    & = \sum_{n}c_n(t)\e^{-\i\frac{(E_n - E_m)t}{\hbar}}\mel{m}{\hat{V}(t)}{n}
  \end{align}
  となる.よって,非定常摂動の時間発展は以下の式を満たす.
  \begin{itembox}[l]{$\hat{H}(t) = \hat{H}^{(0)} + \hat{V}(t)$}
    \begin{align}
      \ket{\psi(t)}_{\r{I}} &= \sum_{n}c_n(t)\ket{n}\\
      i\hbar\dv{t}c_m(t) &= \sum_{n}c_n(t)V_{mn}\e^{\i\omega_{mn}t}\\
      V_{mn} &= \bra{m}\hat{V}(t)\ket{n}\\
      \omega_{mn} &= \frac{E_m - E_n}{\hbar} = -\omega_{nm}
    \end{align}
  \end{itembox}
  これは$c_n$の連立方程式になっており解くことは困難である.よって近似を加える.
\end{document}
