\documentclass{report}
\input{../../../head.tex}
\begin{document}
  Schr\"odinger表示の非定常摂動基本方程式は,\refe{time-dependent-schrodinger-eq-re}と\refe{nonsteady-perturbation-cnt-eq}であり,
  \begin{align}
    \begin{dcases}
      \i\hbar\dv{t}\ket{\psi(t)} = \qty(\hat{H}^{(0)} + \hat{V}(t))\ket{\psi(t)} \label{time-dependent-schrodinger-eq-re} \\
      \ket{\psi(t)} = \sum_{n}c_n(t)\exp\qty(-\i\frac{E_n}{\hbar}t)\ket{n}
    \end{dcases}
  \end{align}
  と書けるのであった.
  \textbf{相互作用表示}(interaction picture)を\refe{interaction-picture}のような$\ket{\psi(t)} \to \ket{\psi(t)}_{\r{I}}$の変換を行って得られる状態ベクトルと定義する.
  \begin{itembox}[l]{相互作用表示}
    \begin{align}
      \ket{\psi(t)}_{\r{I}} \coloneqq \exp\qty(\i\frac{\hat{H}^{(0)}}{\hbar}t)\ket{\psi(t)}\label{interaction-picture}
    \end{align}
  \end{itembox}
  \refe{interaction-picture}を用いて,\refe{time-dependent-schrodinger-eq-re}と等価な基本方程式である,相互作用表示の非定常摂動基本方程式を導く.
  まず,\refe{time-dependent-schrodinger-eq-re}の第2式を用いて,
  \begin{align}
    \ket{\psi(t)}_{\r{I}} &= \exp\qty(\i\frac{\hat{H}^{(0)}}{\hbar}t)\sum_{n}c_n(t)\exp\qty(-\i\frac{E_n}{\hbar}t)\ket{n}\\
    &= \sum_{n}c_n(t)\exp\qty(-\i\frac{E_n}{\hbar}t)\exp\qty(\i\frac{\hat{H}^{(0)}}{\hbar}t)\ket{n} \\ 
    &= \sum_{n}c_n(t)\exp\qty(-\i\frac{E_n}{\hbar}t)\exp\qty(\i\frac{E_n}{\hbar}t)\ket{n} \\ 
    &= \sum_{n}c_n(t)\ket{n} \label{psi-t-cnt}
  \end{align}
  と計算できる.
  \par
  次に,相互作用表示の時間微分を計算してみる.
  相互作用表示での摂動項$\hat{V}_{\r{I}}$を,
  \begin{align}
    \hat{V}_{\r{I}} \coloneqq \exp\qty(\i\frac{\hat{H}^{(0)}}{\hbar}t)\hat{V}(t)\exp\qty(-\i\frac{\hat{H}^{(0)}}{\hbar}t) \label{time-dependent-schrodinger-eq-re-2nd-eq-interaction-picture}
  \end{align}
  とする.計算の途中で,\refe{time-dependent-schrodinger-eq-re}の第1式を用いると,
  \begin{align}
    \i\hbar\dv{t}\ket{\psi(t)}_{\r{I}} &= \dv{t}\qty[\exp\qty(\i\frac{\hat{H}^{(0)}}{\hbar}t)]\ket{\psi(t)} + \exp\qty(\i\frac{\hat{H}^{(0)}}{\hbar}t)\dv{t}\ket{\psi(t)} \\
    &= \i\frac{\hat{H}^{(0)}}{\hbar}\exp\qty(\i\frac{\hat{H}^{(0)}}{\hbar}t)\ket{\psi(t)} + \exp\qty(\i\frac{\hat{H}^{(0)}}{\hbar}t)\frac{1}{\i\hbar}\qty(\hat{H}^{(0)} + \hat{V}(t))\ket{\psi(t)} \\ 
    &= \frac{\i}{\hbar}\qty[\hat{H}^{(0)}, \exp\qty(\i\frac{\hat{H}^{(0)}}{\hbar}t)]\ket{\psi(t)} + \exp\qty(\i\frac{\hat{H}^{(0)}}{\hbar}t)\hat{V}(t)\ket{\psi(t)} \\ 
    &= \exp\qty(\i\frac{\hat{H}^{(0)}}{\hbar}t)\hat{V}(t)\ket{\psi(t)} \\
    &= \exp\qty(\i\frac{\hat{H}^{(0)}}{\hbar}t)\hat{V}(t)\exp\qty(-\i\frac{\hat{H}^{(0)}}{\hbar}t)\exp\qty(\i\frac{\hat{H}^{(0)}}{\hbar}t)\ket{\psi(t)} \\
    &= \exp\qty(\i\frac{\hat{H}^{(0)}}{\hbar}t)\hat{V}(t)\exp\qty(-\i\frac{\hat{H}^{(0)}}{\hbar}t)\ket{\psi(t)}_{\r{I}} \\
    &= \hat{V}_{\r{I}}(t)\ket{\psi(t)}_{\r{I}}\label{time-dependent-schrodinger-eq-re-1st-eq-interaction-picture}
  \end{align}
  を得る.\refe{time-dependent-schrodinger-eq-re-1st-eq-interaction-picture}と\refe{time-dependent-schrodinger-eq-re-2nd-eq-interaction-picture}はSch\"odinger表示の非定常摂動基本方程式と等価な方程式であるから,
  これを\textbf{朝永・Schwinger方程式}と呼ぶ.
  \footnote{Schrödinger描像は量子状態が時間発展するとみなす.Heisenberg描像は物理量が時間発展するとみなす.相互作用表示はその中間であるといえる.}  
  \footnote{朝永振一郎(1906-1979)}
  \footnote{Julian Schwinger(1918-1994)}
  \footnote{朝永とSchwingerは1964年にRichard Feynmannとともにノーベル賞を受賞.}.
  \begin{itembox}[l]{朝永・Schwinger方程式}
    \begin{align}
      \begin{dcases}
        \i\hbar\dv{t}\ket{\psi(t)}_{\r{I}} = \hat{V}_{\r{I}}(t)\ket{\psi(t)}_{\r{I}} \label{tomonaga-schwinger}\\
        \hat{V}_{\r{I}}(t) = \exp\qty(\i\frac{\hat{H}^{(0)}}{\hbar}t)\hat{V}(t)\exp\qty(-\i\frac{\hat{H}^{(0)}}{\hbar}t)
      \end{dcases}
    \end{align}
  \end{itembox}
  \par
  さて,定常状態のSchr\"odinger方程式が,
  \begin{align}
    \forall m\ \hat{H}^{(0)}\ket{m} = E_m\ket{m}
  \end{align}
  を満たすとする.
  \refe{tomonaga-schwinger}の第1式に左から$\bra{m}$を演算する.
  途中,\refe{psi-t-cnt}を用いて$\ket{\psi(t)}_{\r{I}}$を展開し,\refe{tomonaga-schwinger}の第2式を用いて$\hat{V}_{\r{I}}(t)$を$\hat{V}(t)$に戻す.
  また,$\hat{V}_{\r{I}}(t)$は$c_n(t)$に作用しないものとすると,
  \begin{align}
    \mel**{m}{\i\hbar\dv{t}}{\psi(t)}_{\r{I}} &= \mel**{m}{\hat{V}_{\r{I}}(t)}{\psi(t)}_{\r{I}} \\ 
    \Leftrightarrow \bra{m}\i\hbar\sum_{n}c_n(t)\ket{n} &= \sum_{n}c_n(t)\mel**{m}{\exp\qty(\i\frac{\hat{H}^{(0)}}{\hbar}t)\hat{V}(t)\exp\qty(-\i\frac{\hat{H}^{(0)}}{\hbar}t)}{n} \\ 
    \Leftrightarrow \i\hbar\sum_{n}\dv{t}c_n(t)\braket{m}{n} &= \sum_{n}c_n(t)\exp\qty(-\i\frac{E_n - E_m}{\hbar}t)\mel**{m}{\hat{V}(t)}{n} \\ 
    \Leftrightarrow \i\hbar \dv{t}c_m(t) &= \sum_{n}c_n(t)\exp\qty(\i\frac{E_m - E_n}{\hbar}t)\mel**{m}{\hat{V}(t)}{n}
  \end{align}
  となる.
  $\omega_{mn}$と$V_{mn}$を,
  \begin{align}
    \omega_{mn} &\coloneqq \frac{E_m - E_n}{\hbar} \\ 
    V_{mn}(t) &\coloneqq \mel**{m}{\hat{V}(t)}{n}
  \end{align}
  と定義する.
  非定常摂動量子系の時間発展は以下の式を満たす.
  \begin{itembox}[l]{非定常摂動量子系の時間発展}
    \begin{align}
      \i\hbar\dv{t}c_m(t) &= \sum_{n}c_n(t)V_{mn}(t)\e^{\i\omega_{mn}t}\label{cnt-eq}
    \end{align}
  \end{itembox}
  これは$c_n(t)$の連立方程式になっていて,一般に解くことは困難である.
  例えば$\hat{V}(t)$が有限次元であり,行列表示ができたとすると\refe{cnt-eq}は,
  \begin{align}
    \i\hbar\dv{t}\mqty(c_1 \\ c_2 \\ \vdots \\ c_n) 
    = \mqty(
      V_{11}(t) & V_{12}(t)\e^{\i\omega_{12}t} & \cdots & V_{1n(t)}\e^{\i\omega_{1n}t} \\ 
      (V_{12}(t)\e^{\i\omega_{12}t})^* & V_{22}(t) & \cdots & V_{2n}(t)\e^{\i\omega_{2n}t} \\ 
      \vdots & \vdots & \ddots & \vdots \\ 
      (V_{1n}(t)\e^{\i\omega_{1n}t})^* & (V_{2n}(t)\e^{\i\omega_{2n}t})^* & \cdots & V_{nn}(t)
      ) 
      \mqty(c_1 \\ c_2 \\ \vdots \\ c_n)\label{cnt-matrix}
  \end{align}
  となる.途中で,
  \begin{align}
    (V_{mn}(t)\e^{\i\omega_{mn}t})^* &= V_{mn}(t)^* \qty(\e^{\i\omega_{mn}t})^* \\ 
    &= \mel**{m}{\hat{V}(t)}{n}^*\exp\qty(-\i\frac{E_m - E_n}{\hbar}t) \\ 
    &= \mel**{n}{\hat{V}(t)}{m}\exp\qty(\i\frac{E_n - E_m}{\hbar}t) \\ 
    &= (V_{nm}(t)\e^{\i\omega_{nm}t})
  \end{align}
  を用いた.\refe{cnt-matrix}の行列の部分も時間に依存することを考えると$n$が大きなときに厳密に解くことは困難である.
\end{document}
