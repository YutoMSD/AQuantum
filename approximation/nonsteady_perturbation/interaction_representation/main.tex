\documentclass{report}
\input{../../../head.tex}
\begin{document}
  非定常摂動の運動はSchrödinger表示で
  \begin{align}
    \begin{dcases}
      \i\hbar\dv{t}\ket{\psi(t)} = \qty(\hat{H}^{(0)} + \hat{V}(t))\ket{\psi(t)}\label{TDSE}\\
      \ket{\psi(t)} = \sum_{n}c_n(t)\exp\qty(-\i\frac{E_n}{\hbar}t)\ket{n}
    \end{dcases}
  \end{align}
  と書ける.これを次の\textbf{相互作用表示}(interaction picture)を用いて書き直す.
  \begin{itembox}[l]{相互作用表示}
    \begin{align}
      \ket{\psi(t)}_{\r{I}} = \exp\qty(-\i\frac{E_n}{\hbar}t)\ket{\psi(t)}
    \end{align}  
  \end{itembox}
  実際,相互作用表示を用いると,
  \begin{align}
    \ket{\psi(t)}_{\r{I}} &= \exp\qty(-\i\frac{E_n}{\hbar}t)\sum_{n}c_n(t)\exp\qty(-\i\frac{E_n}{\hbar}t)\ket{n}\\
    &= \sum_{n}c_n(t)\ket{n}
  \end{align}
  であり.\refe{TDSE}の2式目と同じものが得られる.
  相互作用表示の時間微分を計算してみる.
  \begin{align}
    i\hbar\dv{t}\ket{\psi(t)}_{\r{I}} &= \i\hbar\qty(i\frac{\hat{H^{(0)}}}{\hbar})\exp\qty(-\i\frac{E_n}{\hbar}t)\ket{\psi(t)} + i\hbar\exp\qty(-\i\frac{E_n}{\hbar}t)\dv{t}\ket{\psi(t)}\\
    & = \exp\qty(-\i\frac{E_n}{\hbar}t)\hat{V}(t)\ket{\psi(t)}\\
    & = \exp\qty(-\i\frac{E_n}{\hbar}t)\hat{V}(t)\exp\qty(-\i\frac{\hat{H}^{(0)}}{\hbar}t)\exp\qty(-\i\frac{E_n}{\hbar}t)\ket{\psi(t)}\\
    & = \exp\qty(-\i\frac{E_n}{\hbar}t)\hat{V}(t)\exp\qty(-\i\frac{\hat{H}^{(0)}}{\hbar}t)\ket{\psi(t)}_{\r{I}}\\
    &\equiv\hat{V}_{\r{I}}(t)\ket{\psi(t)}_{\r{I}}
  \end{align}
  Schrödinger方程式に似た式が得られた.これを\textbf{朝永・Schwinger方程式}という
  \footnote{Schrödinger描像は量子状態が時間発展するとみなす.Heisenberg描像は物理量が時間発展するとみなす.相互作用表示はその中間であるといえる.}  
  \footnote{朝永振一郎(1906-1979)}
  \footnote{Julian Schwinger(1918-1994)}
  \footnote{朝永とSchwingerは1964年にRichard Feynmannとともにノーベル賞を受賞.}.
  \begin{itembox}[l]{朝永・Schwinger方程式}
    \begin{align}
      \i\hbar\dv{t}\ket{\psi(t)}_{\r{I}} &= \hat{V}_{\r{I}}(t)\ket{\psi(t)}_{\r{I}}\label{Tomonaga}\\
      \hat{V}_{\r{I}}(t)& = \exp\qty(-\i\frac{E_n}{\hbar}t)\hat{V}(t)\exp\qty(-\i\frac{\hat{H}^{(0)}}{\hbar}t)
    \end{align}
  \end{itembox}
  \refe{Tomonaga}に左から$\bra{m}$を演算する($\hat{H}^{(0)}\ket{m} = E_m\ket{m}$).\\
  左辺は,
  \begin{align}
    &\bra{m}\i\hbar\dv{t}\sum_{n}c_n(t)\ket{n}\\
    & = \i\hbar\dv{t}c_m(t)
  \end{align}
  となる.
  右辺は.
  \begin{align}
    &\bra{m}\hat{V}_{\r{I}}(t)\sum_{n}c_n(t)\ket{n}\\
    & = \sum_{n}c_n(t)\e^{-\i\frac{(E_n - E_m)t}{\hbar}}\mel{m}{\hat{V}(t)}{n}
  \end{align}
  となる.よって,非定常摂動の時間発展は以下の式を満たす.
  \begin{itembox}[l]{$\hat{H}(t) = \hat{H}^{(0)} + \hat{V}(t)$}
    \begin{align}
      \ket{\psi(t)}_{\r{I}} &= \sum_{n}c_n(t)\ket{n}\\
      i\hbar\dv{t}c_m(t) &= \sum_{n}c_n(t)V_{mn}\e^{\i\omega_{mn}t}\\
      V_{mn} &= \bra{m}\hat{V}(t)\ket{n}\\
      \omega_{mn} &= \frac{E_m - E_n}{\hbar} = -\omega_{nm}
    \end{align}
  \end{itembox}
  これは$c_n$の連立方程式になっており解くことは困難である.よって近似を加える.
\end{document}
