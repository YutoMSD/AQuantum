\documentclass{report}
\input{../../../head.tex}
\begin{document}
  非定常摂動の運動はSchrödinger表示で
  \begin{equation}
    \label{TDSE}
    \begin{cases}
    i\hbar\frac{d}{dt}\ket{\psi(t)}=\qty(\hat{H}^0+\hat{V}(t))\ket{\psi(t)}\\
    \ket{\psi(t)}=\sum_{n}c_n(t)\mathrm{e}^{-iE_nt/\hbar}\ket{n}
    \end{cases}
  \end{equation}
  と書ける.これを次の\textbf{相互作用表示}(interaction picture)を用いて書き直す.
  \begin{itembox}[l]{相互作用表示}
    \begin{equation}
      \ket{\psi(t)}_I=\mathrm{e}^{i\hat{H}^0t/\hbar}\ket{\psi(t)}
    \end{equation}  
  \end{itembox}
  実際,相互作用表示を用いると,
  \begin{align}
    \ket{\psi(t)}_I&=\mathrm{e}^{i\hat{H}^0t/\hbar}\sum_{n}c_n(t)\mathrm{e}^{-iE_nt/\hbar}\ket{n}\\
    &=\sum_{n}c_n(t)\ket{n}
  \end{align}
  であり.式(\ref{TDSE})の2式目と同じものが得られる.\\
  相互作用表示の時間微分を計算してみる.
  \begin{align}
    i\hbar\frac{d}{dt}\ket{\psi(t)}_I&=i\hbar\qty(i\frac{\hat{H^0}}{\hbar})\mathrm{e}^{i\hat{H}^0t/\hbar}\ket{\psi(t)}+i\hbar\mathrm{e}^{i\hat{H}^0t/\hbar}\frac{d}{dt}\ket{\psi(t)}\\
    &=\mathrm{e}^{i\hat{H}^0t/\hbar}\hat{V}(t)\ket{\psi(t)}\\
    &=\mathrm{e}^{i\hat{H}^0t/\hbar}\hat{V}(t)\mathrm{e}^{-i\hat{H}^0t/\hbar}\mathrm{e}^{i\hat{H}^0t/\hbar}\ket{\psi(t)}\\
    &=\mathrm{e}^{i\hat{H}^0t/\hbar}\hat{V}(t)\mathrm{e}^{-i\hat{H}^0t/\hbar}\ket{\psi(t)}_I\\
    &\equiv\hat{V}_I(t)\ket{\psi(t)}_I
  \end{align}
  Schrödinger方程式に似た式が得られた.これを\textbf{朝永・Schwinger方程式}という
  \footnote{Schrödinger描像は量子状態が時間発展するとみなす.Heisenberg描像は物理量が時間発展するとみなす.相互作用表示はその中間であるといえる.}  
  \footnote{朝永振一郎(1906-1979)}
  \footnote{Julian Schwinger(1918-1994)}
  \footnote{朝永とSchwingerは1964年にRichard Feynmannとともにノーベル賞を受賞.}.
  \begin{itembox}[l]{朝永・Schwinger方程式}
    \begin{align}
      \label{Tomonaga}
      i\hbar\frac{d}{dt}\ket{\psi(t)}_I&=\hat{V}_I(t)\ket{\psi(t)}_I\\
      \hat{V}_I(t)&=\mathrm{e}^{i\hat{H}^0t/\hbar}\hat{V}(t)\mathrm{e}^{-i\hat{H}^0t/\hbar}
    \end{align}
  \end{itembox}
\end{document}
