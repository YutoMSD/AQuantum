\documentclass{standalone}
\input{../../../head.tex}
\begin{document}
  厳密に解くことのできる2準位系
  \begin{equation}
    \ket{1}=\begin{pmatrix}
      1\\
      0
    \end{pmatrix},\ 
    \ket{2}=\begin{pmatrix}
      0\\
      1
    \end{pmatrix},\ 
    \hat{H}^0=\begin{pmatrix}
      E_1 & 0\\
      0 & E_2
    \end{pmatrix}
  \end{equation}
  に摂動を加える($E_1$,$E_2$はそれぞれ$\ket{1}$,$\ket{2}$のエネルギー固有値).摂動は次のようにする\footnote{$\gamma$は摂動の強さを表す.}.
  \begin{equation}
    \hat{V}(t)=\begin{pmatrix}
      0& \gamma\mathrm{e}^{i\omega t}\\
      \gamma \mathrm{e}^{-i\omega t}&0
    \end{pmatrix}
  \end{equation}
  よって,Hamiltonianは,
  \begin{equation}
    \hat{H}=\hat{H}^0+\hat{V}(t)=\begin{pmatrix}
      E_1&\gamma\mathrm{e}^{i\omega t}\\
      \gamma\mathrm{e}^{-i\omega t}&E_2
    \end{pmatrix}
  \end{equation}
  である.$t$に依存する非対角項が存在するため$\ket{1}$と$\ket{2}$が$t$に依存して混ざってしまう.\\
  この系の量子状態は相互作用表示を使って,
  \begin{equation}
    \ket{\psi(t)}_I=c_1(t)\ket{1}+c_2(t)\ket{2}
  \end{equation}
  と表される.$V_{11}=V_{22}=0,\ V_{12}=V_{21}=\gamma\mathrm{e}^{-i\omega t}$であるから,
  \begin{equation}
    \begin{cases}
      i\hbar\frac{d}{dt}c_1=c_2\gamma\mathrm{e}^{i(\omega-\omega_{21})t}\\
      i\hbar\frac{d}{dt}c_2=c_1\gamma\mathrm{e}^{-i(\omega-\omega_{21})t}
    \end{cases}
  \end{equation}
  が成り立つ.次に,$\Delta\omega\equiv\omega-\omega_{21}$とし,上の連立微分方程式を解く.\\
  2式目を1式目に代入して$c_1$を消去すると
  \begin{equation}
    \ddot{c_2}+i\Delta\omega\dot{c_2}+\qty(\frac{\gamma}{\hbar})^2c_2=0
  \end{equation}
  $c_2(t)=\mathrm{e}^{i\lambda t}$を代入すると
  \begin{align}
    \lambda^2+\Delta\omega\lambda-\qty(\frac{\gamma}{\hbar})=0\\
    \lambda=-\frac{\Delta\omega}{2}\pm\sqrt{\qty(\Delta\omega/2)^2+\qty(\gamma/\hbar)^2}
  \end{align}
  を得る.ここで
  \begin{equation}
    \Omega=\sqrt{\qty(\Delta\omega/2)^2+\qty(\gamma/\hbar)^2}
  \end{equation}
  を\textbf{Rabi周波数}(Rabi frequency)という\footnote{I.I.Rabi(1898-1988)}\footnote{量子状態の振動を\textbf{Rabi振動}(Rabi oscillation)という.}.
  \begin{itembox}[l]{Rabi周波数}
    \begin{equation}
    \Omega=\sqrt{\qty(\Delta\omega/2)^2+\qty(\gamma/\hbar)^2}   
    \end{equation}
  \end{itembox}
  よって,$c_2$の一般解は
  \begin{equation}
    c_2(t)=\mathrm{e}^{-i\Delta\omega t/2}\qty(a\mathrm{e}^{i\Omega t}+b\mathrm{e}^{-i\Omega t})
  \end{equation}
  となる.初期条件を$c_1(0)=1,\ c_2(0)=0$とすると,
  \begin{equation}
    c_2(t)=-i\frac{\gamma}{\hbar\Omega}\mathrm{e}^{-i\Delta\omega t/2}\sin\Omega t
  \end{equation}
  である.この系の量子状態は
  \begin{equation}
    \ket{\psi(t)}_I=c_1(t)\ket{1}+c_2(t)\ket{2}
  \end{equation}
  だから,時刻$t$で$\ket{2}$に状態を見出す確率は
  \begin{equation}
    |c_2(t)|^2=\frac{(\gamma/\hbar)^2}{\Omega^2}\sin^2\Omega t
  \end{equation}
  である.確率が周期的に変化することがわかる.\\
  振幅の大きさは
  \begin{equation}
    \frac{(\gamma/\hbar)^2}{(\gamma/\hbar)^2+(\Delta\omega/2)^2}
  \end{equation}
  と表されるので,$\Delta\omega=\omega-\omega_{12}=0$のときに最大となる.つまり,摂動の周波数$\omega$と
  2準位のエネルギー差に由来する$\omega_{12}$が一致したときに遷移が起こりやすい.
  \begin{itembox}[l]{共鳴条件}
    \begin{equation}
      \omega=\omega_{12}=\frac{E_2-E_1}{\hbar}
    \end{equation}  
  \end{itembox}
\end{document}
