\documentclass{report}
\input{../../../head.tex}
\begin{document}
  厳密に解くことのできる2準位系
  \begin{align}
    \ket{1} = \mqty(
      1\\
      0
    ),\ 
    \ket{2} = \mqty(
      0\\
      1
    ),\ 
    \hat{H}^0 = \mqty(
      E_1  &  0\\
      0  &  E_2
    )
  \end{align}
  に摂動を加える($E_1$,$E_2$はそれぞれ$\ket{1}$,$\ket{2}$のエネルギー固有値).摂動は次のようにする\footnote{$\gamma$は摂動の強さを表す.}.
  \begin{align}
    \hat{V}(t) = \mqty(
      0 &  \gamma\e^{\i\omega t}\\
      \gamma \e^{ - \i\omega t} & 0
    )
  \end{align}
  よって,Hamiltonianは,
  \begin{align}
    \hat{H} = \hat{H}^0 + \hat{V}(t) = \mqty(
      E_1 & \gamma\e^{\i\omega t}\\
      \gamma\e^{ - \i\omega t} & E_2
    )
  \end{align}
  である.$t$に依存する非対角項が存在するため$\ket{1}$と$\ket{2}$が$t$に依存して混ざってしまう.\\
  この系の量子状態は相互作用表示を使って,
  \begin{align}
    \ket{\psi(t)}_I = c_1(t)\ket{1} + c_2(t)\ket{2}
  \end{align}
  と表される.$V_{11} = V_{22} = 0,\ V_{12} = V_{21} = \gamma\e^{ - \i\omega t}$であるから,
  \begin{align}
    \begin{cases}
      \i\hbar\frac{d}{dt}c_1 = c_2\gamma\e^{\i(\omega - \omega_{21})t}\\
      \i\hbar\frac{d}{dt}c_2 = c_1\gamma\e^{ - \i(\omega - \omega_{21})t}
    \end{cases}
  \end{align}
  が成り立つ.次に,$\Delta\omega\equiv\omega - \omega_{21}$とし,上の連立微分方程式を解く.\\
  2式目を1式目に代入して$c_1$を消去すると
  \begin{align}
    \ddot{c_2} + \i\Delta\omega\dot{c_2} + \qty(\frac{\gamma}{\hbar})^2c_2 = 0
  \end{align}
  $c_2(t) = \e^{\i\lambda t}$を代入すると
  \begin{align}
    \lambda^2 + \Delta\omega\lambda - \qty(\frac{\gamma}{\hbar}) = 0\\
    \lambda =  - \frac{\Delta\omega}{2}\pm\sqrt{\qty(\Delta\omega/2)^2 + \qty(\gamma/\hbar)^2}
  \end{align}
  を得る.ここで
  \begin{align}
    \Omega = \sqrt{\qty(\Delta\omega/2)^2 + \qty(\gamma/\hbar)^2}
  \end{align}
  を\textbf{Rabi周波数}(Rabi frequency)という\footnote{I.I.Rabi(1898 - 1988)}\footnote{量子状態の振動を\textbf{Rabi振動}(Rabi oscillation)という.}.
  \begin{itembox}[l]{Rabi周波数}
    \begin{align}
    \Omega = \sqrt{\qty(\Delta\omega/2)^2 + \qty(\gamma/\hbar)^2}   
    \end{align}
  \end{itembox}
  よって,$c_2$の一般解は
  \begin{align}
    c_2(t) = \e^{ - \i\Delta\omega t/2}\qty(a\e^{\i\Omega t} + b\e^{ - \i\Omega t})
  \end{align}
  となる.初期条件を$c_1(0) = 1,\ c_2(0) = 0$とすると,
  \begin{align}
    c_2(t) =  - \i\frac{\gamma}{\hbar\Omega}\e^{ - \i\Delta\omega t/2}\sin\Omega t
  \end{align}
  である.この系の量子状態は
  \begin{align}
    \ket{\psi(t)}_I = c_1(t)\ket{1} + c_2(t)\ket{2}
  \end{align}
  だから,時刻$t$で$\ket{2}$に状態を見出す確率は
  \begin{align}
    |c_2(t)|^2 = \frac{(\gamma/\hbar)^2}{\Omega^2}\sin^2\Omega t
  \end{align}
  である.確率が周期的に変化することがわかる.\\
  振幅の大きさは
  \begin{align}
    \frac{(\gamma/\hbar)^2}{(\gamma/\hbar)^2 + (\Delta\omega/2)^2}
  \end{align}
  と表されるので,$\Delta\omega = \omega - \omega_{12} = 0$のときに最大となる.つまり,摂動の周波数$\omega$と
  2準位のエネルギー差に由来する$\omega_{12}$が一致したときに遷移が起こりやすい.
  \begin{itembox}[l]{共鳴条件}
    \begin{align}
      \omega = \omega_{12} = \frac{E_2 - E_1}{\hbar}
    \end{align}  
  \end{itembox}
\end{document}
