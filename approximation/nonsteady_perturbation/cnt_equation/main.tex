\documentclass{standalone}
\input{../../../head.tex}
\begin{document}
  式(\ref{Tomonaga})に左から$\bra{m}$を演算する($\hat{H}^0\ket{m}=E_m\ket{m}$).\\
  左辺は
  \begin{align}
    &\bra{m}i\hbar\frac{d}{dt}\sum_nc_n(t)\ket{n}\\
    &=i\hbar\frac{d}{dt}c_m(t)
  \end{align}
  右辺は
  \begin{align}
    &\bra{m}\hat{V}_I(t)\sum_{n}c_n(t)\ket{n}\\
    &=\sum_{n}c_n(t)\mathrm{e}^{-i\frac{(E_n-E_m)t}{\hbar}}\bra{m}\hat{V}(t)\ket{n}
  \end{align}
  となる.よって,非定常摂動の時間発展は以下の式を満たす.
  \begin{itembox}[l]{$\hat{H}(t)=\hat{H}^0+\hat{V}(t)$}
    \begin{align}
      \ket{\psi(t)}_I&=\sum_{n}c_n(t)\ket{n}\\
      i\hbar\frac{d}{dt}c_m(t)&=\sum_{n}c_n(t)V_{mn}\mathrm{e}^{i\omega_{mn}t}\\
      V_{mn}&=\bra{m}\hat{V}(t)\ket{n}\\
      \omega_{mn}&=\frac{E_m-E_n}{\hbar}=-\omega_{nm}
    \end{align}
  \end{itembox}
  これは$c_n$の連立方程式になっており解くことは困難である.よって近似を加える.
\end{document}
