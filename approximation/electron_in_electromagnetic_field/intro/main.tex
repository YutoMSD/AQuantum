\documentclass{report}
\input{../../../head.tex}
\begin{document}
  以下ではまず,古典の電磁気学を考えよう.
  電磁場中の電子の運動方程式は,
  \begin{align}
    m\dv{\bm{v}}{t} = -e\qty(\bm{E} + \bm{v}\times\bm{B})
  \end{align}
  と書けるのであった.
  電磁場中の電子のハミルトニアンは,
  \begin{itembox}[l]{電磁場中の電子のハミルトニアン}
    \begin{align}
      H = \frac{1}{2m}\qty(\bm{p} + e\bm{A})^2 - e\phi
    \end{align}
  \end{itembox}
  である.ただし$\bm{A}$はベクトルポテンシャルで,
  \begin{align}
    \bm{E} &= -\pdv{\bm{A}}{t} - \grad\phi \\ 
    \bm{B} &= \curl\bm{A}
  \end{align}
  を満たす.
  本節では$U(1)$\footnote{Unitary}Gauge対称性を扱い説明し電磁場の起源を探る.
\end{document}
