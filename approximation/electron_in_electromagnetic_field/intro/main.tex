\documentclass{report}
\input{../../../head.tex}
\begin{document}
  電磁場中の電子の運動方程式は
  \begin{equation}
    m\frac{d\bm{v}}{dt}=-e\qty(\bm{E}+\bm{v}\times\bm{B})
  \end{equation}
  である.電磁場中の電子のHamiltonianは
  \begin{itembox}[l]{電磁場中の電子のHamiltonian}
    \begin{equation}
      H=\frac{1}{2m}\qty(\bm{p}+e\bm{A})^2-e\phi
    \end{equation}
  \end{itembox}
  である.ここで$\bm{A}$はベクトルポテンシャルで
  \begin{equation}
    \bm{E}=-\frac{\partial \bm{A}}{\partial t}-\nabla\phi,\ \bm{B}=\nabla\times\bm{A}
  \end{equation}
  を満たす.\\
  本節では$U(1)$\footnote{Unitary}Gauge対称性を扱い説明し電磁場の起源を探る.
\end{document}
