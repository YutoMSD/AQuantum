\documentclass{report}
\input{../../../head.tex}
\begin{document}
  波動関数$\psi(\bm{r})$を次のように変換する.
  \begin{align}
    \psi(\bm{r}) \to \psi'(\bm{r}) \coloneqq \e^{\i\alpha(\bm{r})}\psi(\bm{r})
  \end{align}
  この場合,運動量がゲージ不変でなくなってしまう.
  例えば波動関数の微分を計算すると
  \begin{align}
    \grad\psi'(\bm{r}) &= \i\qty(\grad\alpha(\bm{r})\e^{i\alpha(\bm{r})})\psi(\bm{r}) + \e^{\i\alpha(\bm{r})}\grad\psi(\bm{r}) \\ 
    &= \e^{i\alpha(\bm{r  })}(\grad + \i\grad\alpha(\bm{r}))\psi(\bm{r}) 
  \end{align}
  余分な項が加わってしまう.よって,
  \begin{align}
    \begin{dcases}
      \i\hbar\pdv{t}\psi' \neq -\frac{\hbar^2}{2m}\laplacian\psi'\\
      \mel**{\psi}{\hat{A}}{\psi} \neq \mel**{\psi'}{\hat{A}}{\psi'}
    \end{dcases}
  \end{align}
  である.したがって,局所ゲージ変換に対して物理は不変ではない.そこで,\textbf{局所ゲージ不変性を基本原理とする物理を再構築する.}
  \par
  まずは,微分を次の共変微分として再定義する.
  \begin{align}
    \bm{D} \coloneqq \grad + \i\frac{e}{\hbar}\bm{A}
  \end{align}
  ただし,$\psi$と$\bm{A}$はゲージ変換により
  \begin{align}
    \psi &\to \psi' = \e^{\i(\alpha{\bm{r}})}\psi\\
    \bm{A} &\to \bm{A}' = \bm{A} - \frac{\hbar}{e}\grad\alpha(\bm{r})
  \end{align}
  このように微分を定義すると,
  \begin{align}
    \bm{D}'\psi'(\bm{r}) = \e^{\i\alpha(\bm{r})}\bm{D}\psi(\bm{r})
  \end{align}
  つまり,局所ゲージ変換は波動関数の微分を
  \begin{align}
    \bm{D}\psi(\bm{r}) \to \e^{\i\alpha(\bm{r})}\bm{D}\psi(\bm{r})
  \end{align}
  と変換することがわかる.
  これは大域的ゲージ変換による$\grad\psi(\bm{r}) \to \e^{\i\alpha}\grad\psi(\bm{r})$と同じ形をしている.
  \par
  次に,$\alpha(\bm{r})$に時間依存性を持たせ,$\alpha(\bm{r},t)$とする.つまり波動関数を
  \begin{align}
    \psi \to \psi' = \e^{\i\alpha(\bm{r},t)}\psi
  \end{align}
  と変換する.
  このとき,時間についての偏微分を
  \begin{align}
    D_t = \pdv{t} - \i\frac{e}{\hbar}\phi
  \end{align}
  と定義する.ただし,
  \begin{align}
    \phi \to \phi' + \frac{\hbar}{e}\pdv{t}\alpha(\bm{r}, t)
  \end{align}
  である.以上で定義した共変微分と時間微分を用いるとSchrödinger方程式は
  \begin{align}
    \i\hbar D'_t\psi' &= -\frac{\hbar^2}{2m}\bm{D}'^2\psi' \\
    \i\hbar\pdv{t}\psi &= -\frac{\hbar^2}{2m}\qty(\grad + \i\frac{e}{\hbar}\bm{A})^2\psi - e\phi\psi\\
    &= \qty[-\frac{\hbar^2}{2m}\qty(\frac{\i}{\hbar}\bm{p} + \i\frac{e}{\hbar}\bm{A})^2 - e\phi]\psi
  \end{align}
  と変換される.
  よって,
  \begin{itembox}[l]{局所ゲージ変換に対して不変なSchrödinger方程式}
    \begin{align}
      \i\hbar\pdv{t}\psi=\qty[\frac{1}{2m}\qty(\bm{p} + e\bm{A})^2 - e\phi]\psi
    \end{align}
  \end{itembox}
  を得る.
  \par
  以上の流れをまとめると,局所ゲージ不変性を要請した.それによりゲージ場$\bm{A}$が導入された.よって,電磁場の起源は局所ゲージ不変性であるといえる.
\end{document}