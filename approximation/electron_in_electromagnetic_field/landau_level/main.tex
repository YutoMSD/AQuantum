\documentclass{report}
\input{../../../head.tex}
\begin{document}
  一様な磁場中の荷電粒子の運動を考える.磁場は$z$軸と平行で大きさは$B$とする.これを満たすベクトルポテンシャルとして,
  \begin{align}
    \begin{dcases}
      A_x = 0\\
      A_y = Bx\\
      A_z = 0
    \end{dcases}
  \end{align}
  を採用する.$z$に対しては対称なので,以降は$xy$平面内での運動を考える.ハミルトニアンは,
  \begin{align}
    \hat{H} &= \frac{1}{2m}\qty[\hat{p}_x^2 + \qty(\hat{p}_y + eB\hat{x})^2]\\
    &= \frac{\hat{p}_x^2}{2m} + \frac{(eB)^2}{2m}\qty(\hat{x} + \frac{\hat{p}_y}{eB})^2
  \end{align}
  である.このハミルトニアンは$\hat{y}$を含まないので,
  \begin{align}
    \qty[\hat{H}, \hat{p}_y] = 0
  \end{align}
  であることがわかる.$y$方向に関しては自由粒子の運動となる.また,$p_y$の固有値が定まるので,これを$\hbar k_y$とする.ハミルトニアンは,
  \begin{align}
    \hat{H} = \frac{\hat{p}_x^2}{2m} + \frac{(eB)^2}{2m}\qty(\hat{x} + \frac{\hbar k_y}{eB})^2
  \end{align}
  と書き換えられ,これは調和振動子型のハミルトニアン,
  \begin{align}
    \hat{H} = \frac{\hat{p}_x^2}{2m} + \frac{m\omega^2\hat{x}^2}{2}
  \end{align}
  と同じ形をしている.よって,エネルギー固有値は,
  \begin{align}
    E &= \hbar\omega_{\r{c}} \qty(n + \frac{1}{2})\ (n = 0,1,2,\dots)\\
    \omega_{\r{c}} &= \frac{\abs{e}B}{m}
  \end{align}
  である.平面内の自由粒子のエネルギーが垂直磁場を加えることで離散化された.これを\textbf{Landau準位}(Landau level)という\footnote{この現象は量子Hall効果へとつながる.}.また,$\omega_{\r{c}}$を\textbf{サイクロトロン周波数}という.
  \par
  次に,$x$方向に長さ$L_x$,$y$方向に長さ$L_y$の周期的境界条件を課す.すると,
  \begin{align}
    k_y = \frac{2\pi}{L_y}l\ (l = 0,\pm 1, \pm2,\dots)
  \end{align}
  となる.また$x$方向の調和振動子は$\frac{\hbar k_y}{eB}$だけずれている.これが0と$L_x$の間にあるためには,
  \begin{align}
    0 \leq \frac{\hbar k_y}{eB} = \frac{\hbar}{eB} \frac{2\pi}{L_y}l \leq L_x
  \end{align}
  でなければならない.これは,
  \begin{align}
    0 \leq l \leq \frac{eB}{2\pi\hbar}L_xL_y
  \end{align}
  と直される.よって,$l$は$\frac{|e|B}{2\pi\hbar}L_xL_y$通りの値を取ることがわかる.したがって,単位面積当たりの縮退度は,
  \begin{align}
    \frac{|e|B}{2\pi\hbar}
  \end{align}
  である.
\end{document}