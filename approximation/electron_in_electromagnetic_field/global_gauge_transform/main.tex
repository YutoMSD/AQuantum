\documentclass{report}
\input{../../../head.tex}
\begin{document}
  まず,大域的Gauge変換を議論する.
  波動関数$\psi(\bm{r})$を,
  \begin{align}
    \psi(\bm{r}) \to \psi'(\bm{r}) \coloneqq \e^{\i\alpha}\psi(\bm{r})
  \end{align}
  のように変換することを大域的ゲージ変換という.
  これをSchrödinger方程式に代入すると,
  \begin{align}
    \i\hbar\pdv{t}\psi &\to \i\hbar\pdv{t}\psi' = \e^{\i\alpha}\i\hbar\pdv{t}\psi \\ 
    -\frac{\hbar^2}{2m}\laplacian\psi &\to -\frac{\hbar^2}{2m}\laplacian\psi' = -\e^{i\alpha}\frac{\hbar^2}{2m}\laplacian\psi
  \end{align}
  となる.
  よって,
  \begin{align}
    \i\hbar\pdv{t}\psi' = -\frac{\hbar^2}{2m}\laplacian\psi'
  \end{align}
  が成り立つので,大域的ゲージ変換を行う前と後でSchr\"odinger方程式の形が変わらないことが分かる.
  当然期待値も,
  \begin{align}
    \mel**{\psi'}{\hat{A}}{\psi'} &= \mel{\psi}{\e^{-\i\alpha}\hat{A}\e^{\i\alpha}}{\psi} \\ 
    &= \mel**{\psi}{\hat{A}}{\psi}
  \end{align}
  となるから,物理は大域的ゲージ変換に対して不変であると言える.
\end{document}
