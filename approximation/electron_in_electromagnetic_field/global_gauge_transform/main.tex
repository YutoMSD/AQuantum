\documentclass{report}
\input{../../../head.tex}
\begin{document}
  まず,大域的Gauge変換を議論する.
  波動関数$\psi(\bm{r},t)$を,
  \begin{align}
    \psi(\bm{r},t) \to \e^{\i\alpha}\psi(\bm{r},t) \eqqcolon \psi'(\bm{r},t) 
  \end{align}
  のように変換することを大域的ゲージ変換という.
  $\psi(\bm{r},t) = \e^{-\i\alpha}\psi'(\bm{r}, t)$をSchrödinger方程式に代入すると,
  \begin{align}
    \i\hbar\pdv{t}\psi(\bm{r},t) &= -\frac{\hbar^2}{2m}\laplacian\psi(\bm{r},t) \\ 
    \Leftrightarrow \i\hbar\pdv{t}\e^{-\i\alpha}\psi'(\bm{r},t) &= -\frac{\hbar^2}{2m}\laplacian\e^{-\i\alpha}\psi'(\bm{r},t) \\ 
    \Leftrightarrow \e^{-\i\alpha}\i\hbar\pdv{t}\psi'(\bm{r},t) &= -\e^{-\i\alpha}\frac{\hbar^2}{2m}\laplacian\psi'(\bm{r},t) \\ 
    \Leftrightarrow \i\hbar\pdv{t}\psi'(\bm{r},t) &= -\frac{\hbar^2}{2m}\laplacian\psi'(\bm{r},t)
  \end{align}
  となる.
  つまり,大域的ゲージ変換によって,Schr\"odinger方程式は影響を受けない.
  当然,期待値も,
  \begin{align}
    \mel**{\psi'}{\hat{A}}{\psi'} &= \mel**{\psi}{\e^{-\i\alpha}\hat{A}\e^{\i\alpha}}{\psi} \\ 
    &= \mel**{\psi}{\hat{A}}{\psi}
  \end{align}
  となるから,物理は大域的ゲージ変換に対して不変であると言える.
\end{document}
