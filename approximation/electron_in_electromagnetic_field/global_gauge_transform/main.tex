\documentclass{report}
\input{../../../head.tex}
\begin{document}
  波動関数$\psi(\bm{r})$を次のように変換する.
  \begin{equation}
    \psi(\bm{r})\to\psi'(\bm{r})=\mathrm{e}^{i\alpha}\psi(\bm{r})
  \end{equation}
  これをSchrödinger方程式に代入する.
  \begin{align}
    i\hbar\frac{\partial}{\partial t}\psi&\to i\hbar\frac{\partial}{\partial t}\psi'=\mathrm{e}^{i\alpha}i\hbar\frac{\partial}{\partial t}\psi\\
    -\frac{\hbar^2}{2m}\nabla^2\psi&\to-\frac{\hbar^2}{2m}\nabla^2\psi'=-\mathrm{e}^{i\alpha}\frac{\hbar^2}{2m}\nabla^2\psi
  \end{align}
  よって,
  \begin{equation}
    i\hbar\frac{\partial}{\partial t}\psi'=-\frac{\hbar^2}{2m}\nabla^2\psi'
  \end{equation}
  が成り立つことがわかる.また期待値も
  \begin{equation}
    \bra{\psi}\hat{A}\ket{\psi}=\bra{\psi'}\hat{A}\ket{\psi'}
  \end{equation}
  である.つまり,物理は大域的Gauge変換に対して不変である.
\end{document}
