\documentclass{report}
\input{../../../head.tex}
\begin{document}
  \refe{hamiltonian-in-electro-magnetic-field-cm}より,電磁場中のハミルトニアンは,
  \begin{align}
    H = \frac{1}{2m}\qty(\bm{p} + e\bm{A})^2 - e\phi\label{hamiltonian-in-electro-magnetic-field-cm-re}
  \end{align}
  と書けるのであった.
  今回は$\phi = 0$とする.
  電磁場が十分弱いという条件のもと,\refe{hamiltonian-in-electro-magnetic-field-cm-re}を量子化すると,
  \begin{align}
    \hat{H} &= \frac{1}{2m}\qty(\hat{\bm{p}} + e\hat{\bm{A}})^2 \\
    &\simeq \frac{1}{2m}\qty(\hat{\bm{p}}^2 + e\hat{\bm{p}}\cdot\hat{\bm{A}} + e\hat{\bm{A}} \cdot \hat{\bm{p}})\\
  \end{align}
  と近似する.ここで,$\hat{H}^{(0)} \coloneqq \frac{\bm{p}^2}{2m}$,摂動項$\hat{V}(t) \coloneqq \frac{e}{2m} \qty(\hat{\bm{p}}\cdot\hat{\bm{A}} + \hat{\bm{A}} \cdot \hat{\bm{p}})$
  とする.さらに,$\div\bm{A} = 0$となるように$\bm{A}$を決める\footnote{Coulomb ゲージ}.すると,
  \begin{align}
    \qty(\hat{\bm{p}} \cdot \bm{A})\psi &= -\i\hbar \div\qty(\hat{\bm{A}} \psi)\\
    &= \i\hbar\qty[\qty(\div\hat{\bm{A}}) \psi + \hat{\bm{A}} \cdot \qty(\grad \psi)]\\
    &= \qty(\hat{\bm{A}} \cdot \hat{\bm{p}}) \psi
  \end{align}
  となる.よって,電子と電磁場の相互作用を摂動として加えたハミルトニアン,
  \begin{align}
    \hat{H} = \hat{H}^{(0)} + \frac{e}{m}\qty(\hat{\bm{A}} \cdot \hat{\bm{p}})
  \end{align}
  を得る.
  \begin{myex}{直線偏光}{}
    ベクトルポテンシャル$\hat{A}$が,
    \begin{align}
      \hat{\bm{A}}(\bm{r}, t) = 2A_0 \bm{e}_x \cos(\bm{k} \cdot \bm{r} - \omega t)
    \end{align}
    であるときを考える.
    ただし,$\bm{k} = \frac{\omega}{c}\bm{e}_z$とする.
    ベクトルポテンシャルと電磁場の関係より,
    \begin{align}
      \begin{cases}
      \bm{E}(\bm{r}, t) = \pdv{\hat{\bm{A}}}{t} = E_0 \bm{e}_x \sin(\bm{k} \cdot \bm{r} - \omega t)\\
      \bm{B}(\bm{r}, t) = \curl\bm{A} = \frac{E_0}{c}\bm{e}_y \sin(\bm{k} \cdot \bm{r} - \omega t)\\
      E_0 = -2 \omega A_0
      \end{cases}
    \end{align}
    が成り立っている.摂動項は,
    \begin{align}
      \hat{V}(t) &= \frac{e}{m}(\hat{\bm{A}}\cdot\hat{\bm{p}}) \\
      &= \frac{2eA_0}{m}\cos(\bm{k} \cdot \bm{r} - \omega t)\bm{e}_x\cdot\hat{\bm{p}} \\
      &= \frac{eA_0}{m}\qty[\e^{\i(\bm{k}\cdot\bm{r} - \omega t)} + \e^{-\i(\bm{k}\cdot\bm{r} - \omega t)}]\hat{p}_x\\
      &= \frac{eA_0}{m}\qty(\e^{\i\bm{k}\cdot\bm{r}}\hat{p}_x\e^{-\i\omega t} + \e^{-\i \bm{k}\cdot\bm{r}}\hat{p}_x \e^{\i \omega t})
    \end{align}
    と表せる.光の吸収を考えるときは,第1項$\frac{eA_0}{m}\e^{\i\bm{k}\cdot\bm{r}}\hat{p}_x$が支配的なのでこの項を$\hat{V}$とする.
    単位時間当たりの遷移確率を計算する.
    \refe{fermis-golden-rule}より,$\omega_{i\to f}$は,
    \begin{align}
      \omega_{i\to f} &= \frac{2\pi}{\hbar} \abs{\mel**{f}{\hat{V}}{i}}^2\delta(E_f - E_i - \hbar \omega) \\
      &= \frac{2 \pi}{\hbar}\qty(\frac{e A_0}{m})^2 \abs{\mel**{f}{\e^{\i\bm{k}\cdot \bm{r}}\hat{p}_x}{i}}^2 \delta(E_f - E_i - \hbar \omega)
    \end{align}
    と表せる.
    さて,$\abs{\mel**{f}{\e^{\i\bm{k}\cdot\bm{r}}\hat{p}_x}{i}}^2$を\textbf{電気双極子近似}を用いて計算する.
    電気双極子近似とは,電磁場の変化の項である$\e^{\i\bm{k}\cdot\bm{r}}$を$1$とみなす近似である.
    これは次のような議論から正当化される.
    原子の準位間隔は$E_f - E_i \sim 1\ \r{eV}$である.これと相互作用する電磁場のエネルギーは$\hbar \omega\sim 1\ \r{eV}$ である.これを波長に換算すると,
    \begin{align}
      \lambda = \frac{2\pi}{\hbar} = \frac{2\pi c}{\omega} \sim 1000\ \r{nm}
    \end{align}
    である.これは原子のスケール$1\ \si{\angstrom}$よりもはるかに大きいため,電子・原子を扱う上では電磁場は空間的に一様だとみなせる.よって,
    \begin{align}
      \e^{\i\bm{k}\cdot\bm{r}}\simeq 1 + \i \bm{k} \cdot \bm{r} + \cdots \simeq 1
    \end{align}
    と近似できる.
    \par
    電気双極子近似を用いると,
    \begin{align}
      \mel**{f}{\e^{\i \bm{k} \cdot \bm{r}}\hat{p}_x}{{i}} \simeq \mel**{f}{\hat{p}_x}{i}
    \end{align}
    を得る.さらに
    \begin{align}
      \qty[\hat{x}, \hat{H}^{(0)}] = \frac{\i \hbar}{m}\hat{p}_x
    \end{align}
    であるため
    \begin{align}
      \mel**{f}{\hat{p}_x}{i} &= \frac{m}{\i\hbar} \mel**{f}{\qty[\hat{x}, \hat{H}^{(0)}]}{i} \\ 
      &= \frac{m}{\i\hbar} \qty(\mel**{f}{\hat{x}\hat{H}^{(0)}}{i} - \mel**{f}{\hat{H}^{(0)}\hat{x}}{i}) \\ 
      &= \frac{m}{\i\hbar} \qty(\mel**{f}{\hat{x}E_i}{i} - \mel**{f}{E_f\hat{x}}{i}) \\ 
      &= \frac{m}{\i\hbar} (E_i - E_f)\mel**{f}{\hat{x}}{i}
    \end{align}
    を得る.よって,電磁場による単位時間当たりの遷移確率
    \begin{align}
      \omega_{i \to f} = \frac{2\pi}{\hbar^3} (eA_0)^2 (E_i - E_f)^2 \abs{\mel**{f}{\hat{x}}{i}}^2 \delta(E_f - E_i - \hbar\omega) 
    \end{align}
    を得る.これは$\mel**{f}{\hat{x}}{i} \neq 0$のときのみ$\omega_{i\to f} \neq 0$であるという選択則を表している.
  \end{myex}
\end{document}