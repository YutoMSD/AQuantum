\documentclass{report}
\input{../head.tex}
\begin{document}
  \maketitle
  \begin{abstract}
    物理情報工学科2024年度秋学期前半「応用量子物性」(担当:安藤和也先生)の学生による自作講義ノートである.
    2年生秋学期後半「量子力学入門」,3年生春学期前半「量子力学」を履修済みであることが望ましい.

    Chapter1近似法では変分法と摂動法を学習する.厳密に解くことのできないハミルトニアンに近似を加え,
    固有状態及び固有エネルギーを求める手法を説明する.

    Chapter2は散乱理論である.散乱は物体の微視的構造を探る非常に有効な方法である.

    Chaper3ではDirac方程式を基礎方程式とする相対論的量子論を学ぶ.特殊相対論を簡単に解説したあと,量子論が
    Lorentz共変性をもつように修正する.
  \end{abstract}
  \tableofcontents
  \thispagestyle{empty}
\end{document}