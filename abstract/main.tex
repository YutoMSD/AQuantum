\documentclass{report}
\input{../head.tex}
\begin{document}
  \maketitle
  \begin{abstract}
    物理情報工学科2024年度秋学期前半「応用量子物性」(担当:安藤和也先生)の学生による自作講義ノートである.
    2年生秋学期後半「量子力学入門」,3年生春学期前半「量子力学」を履修済みであることが望ましい.
    また,2年生秋学期「分布系の数理」を履修していると,散乱理論で用いたGreen関数の扱いやHelmholtz方程式,Legendre陪多項式の扱いを知っていると途中の数学的手続きを理解することができる.
    \par 
    Chapter1は近似法として変分法と摂動法を学習する.厳密に解くことのできないハミルトニアンに近似を行い,
    固有状態及び固有エネルギーを求める手法を説明する.
    \par
    Chapter2は散乱理論である.散乱は物体の微視的構造を探る非常に有効な方法である.
    \par
    Chaper3ではDirac方程式を基礎方程式とする相対論的量子論を学ぶ.特殊相対論を簡単に解説したあと,量子論が
    Lorentz共変性をもつように修正する.
  \end{abstract}
  \tableofcontents
  \thispagestyle{empty}
\end{document}