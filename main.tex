\documentclass{report}
\input{head.tex}
\begin{document}
  \documentclass{report}
\input{../head.tex}
\begin{document}
  \maketitle
  \begin{abstract}
    「応用量子物性」の講義ノート(勝手に)である.
    Griffithと書いてある例題はDavid J. Griffith, \textit{Introduction to Quantum Mechanics 3rd Edition}から拾ってきた.
  \end{abstract}
  \tableofcontents
  \thispagestyle{empty}
\end{document}
  \maketitle
  \tableofcontents
  \chapter{近似法}
    \section{変分法}
      \documentclass{report}
\input{../../../head.tex}
\begin{document}
  \begin{align}
    \hat{H}\ket{k} = E_k\ket{k}\label{SE}
  \end{align}
  変分法(variational principle)とはHamiltonianの基底エネルギー$E_0$の近似法である
  \footnote{
    近似法には摂動法と変分法がある.
    摂動法はHamiltonianが厳密に解ける項$\hat{H}^0$と摂動項$\hat{\delta}$を用いて,$\hat{H}=\hat{H}^0+\hat{\delta}$と表され,摂動項が小さいときのみ有効である.
    これに対し,変分法はどんなときでも有効である.
  }.
\end{document}

      \subsection{基本原理}
        \documentclass{standalone}
\input{../../../head.tex}
\begin{document}
  \begin{myprop}{変分法の基本原理}{}
    任意の状態ベクトル$\ket{\psi}$に対して以下の不等式が成り立つ.
    \begin{align}
      E(\psi) = \cfrac{\bra{\psi}\hat{H}\ket{\psi}}{\bra{\psi}\ket{\psi}} \ge E_0 \label{ne}
    \end{align}
    \tcblower
    \begin{proof}
      任意の状態ベクトル$\ket{\psi}$を
      \begin{align}
        \ket{\psi} = \sum_{k} c_k\ket{k} \label{sv}
      \end{align}
      と展開する.左から$\bra{k'}$を作用させると
      \begin{align}
        \bra{k'}\ket{\psi} = \sum_{k}c_k\bra{k'}\ket{k} = \sum_{k}c_k\delta_{k',k} = c_{k'}
      \end{align}
      を得る.これは任意の$k$に対して成り立つので\refe{sv}は以下のように変形できる.
      \begin{align}
        \ket{\psi}&=\sum_{k}\bra{k}\ket{\psi}\ket{k}\\
        &=\sum_{k}\ket{k}\bra{k}\ket{\psi}
      \end{align}
      これを用いて\refe{ne}の分母を以下のように変形する.
      \begin{align}
        \bra{\psi}\hat{H}\ket{\psi}&=\bra{\psi}\hat{H}\sum_{k}\ket{k}\bra{k}\ket{\psi}\\
        &=\sum_{k}\bra{\psi}\hat{H}\ket{k}\bra{k}\ket{\psi}\\
        &=\sum_{k}E_k\bra{\psi}\ket{k}\bra{k}\ket{\psi}\\
        &=\sum_{k}E_k \abs{\bra{k}\ket{\psi}}^2
      \end{align}
      また,
      \begin{align}
        \braket{\psi}{\psi}=\sum_{k}\abs{\braket{k}{\psi}}^2
      \end{align}
      であるので,
      \begin{align}
        E(\psi) &= \cfrac{\bra{\psi}\hat{H}\ket{\psi}}{\braket{\psi}{\psi} } \\ 
        &= \cfrac{\sum_{k}E_k\abs{\braket{k}{\psi}}^2}{\sum_{k}\abs{\braket{k}{\psi}}^2} \\ 
        &\geq \cfrac{\sum_{k}E_0\abs{\braket{k}{\psi}}^2}{\sum_{k}\abs{\braket{k}{\psi}}^2} = E_0
      \end{align}
      が示される.
    \end{proof}
  \end{myprop}
  \refe{ne}よりあらゆる状態ベクトル$\ket{\psi}$のエネルギーは基底エネルギー$E_0$以上である.
  変分法は,
  \begin{screen}
    \begin{enumerate}
      \item \textbf{試行関数}$\ket{\psi}$をたくさん用意し,
      \item それぞれのエネルギー$E(\psi)$を計算し,
      \item その中で最小の$E(\psi)$を$E_0$の近似解とする
    \end{enumerate}
  \end{screen}
  近似法である.
  \begin{myex}{}{}
    ポテンシャル$V(x)=\lambda x^4$中に粒子がある系を考える.
    この系のHamiltonianは
    \begin{align}
      \hat{H} = -\cfrac{\hbar^2}{2m}\cfrac{d^2}{dx^2} + \lambda x^4
    \end{align}
    である.予想される基底状態が満たすべき条件は
    \begin{itemize}
      \item $x = 0$で存在確率が最大
      \item $\abs{x} \to \infty$で存在確率が0
      \item 節がない\footnote{節があると微係数が大きい点が存在し,これは運動エネルギーを大きくしてしまう.}
    \end{itemize}
    である.この条件と変分法を用いて,エネルギーの近似値を求めよ.
    \tcblower
    試行関数として$\psi(x,\alpha) = \e^{-\tfrac{\alpha x^2}{2}}$,$\alpha > 0$を考える.
    \refe{ne}の右辺を計算すると,
    \begin{align}
      E(\alpha) &= \cfrac{\int \psi^{*}\hat{H}\psi \dd{x}}{\int \psi^{*}\psi \dd{x}} \label{E1} \\ 
      &= \cfrac{\hbar^2 \alpha}{4m} + \cfrac{3\lambda}{4\alpha^2}
    \end{align}
    を得る.
    第1項は運動エネルギーを,第2項はポテンシャルエネルギーを,それぞれ表している\footnote{
      ポテンシャルエネルギーの項は$\alpha$が大きくなるほど小さくなる.
      これは,波動関数が狭まり$x=0$での存在確率が大きくなるためである.
      一方,運動エネルギーの項は$\alpha$が大きくなるほど大きくなる.
      これは,不確定性関係$\Delta x\Delta p \ge \cfrac{\hbar}{2}$より,運動量のばらつきが大きくなるためである.
    }.
    \refe{E1}の最小値が基底エネルギー$E_0$の近似解である.よって,$\dv{\alpha}E(\alpha_0) = 0$となる$\alpha_0$を\refe{E1}に代入することで近似解,
    \begin{align}
      E(\alpha_0) = \cfrac{3}{8}\qty(\cfrac{6\hbar^4\lambda}{m^2})^{1/3}
    \end{align}
    を得る.
  \end{myex}
\end{document}
      \subsection{ヘリウム原子}
        \documentclass{report}
\input{../../../head.tex}
\begin{document}
  本節では,変分法の威力を確認するために,ヘリウム原子の基底エネルギーを考える.
  ヘリウム原子において,$m/M \to 0$であり,原子核が動かない(原子核の運動エネルギーが無視できる)とする.
  これをBorn-Oppenheimer近似という.
  ヘリウム原子は電荷$2e$の原子核と電荷$-e$の電子を2つもつので,ハミルトニアンは,
  \begin{align}
    \hat{H} = -\cfrac{\hbar^2}{2m}\grad_{1}^{2} -\cfrac{\hbar^2}{2m}\grad_{2}^{2} - \cfrac{2e^2}{4\pi \epsilon_0 r_1} - \cfrac{2e^2}{4\pi \epsilon_0 r_2} + \cfrac{e^2}{4\pi \epsilon_0 r_{12}}
  \end{align}
  である.
  第1項から第4項は水素陽原子のハミルトニアン $\hat{H}^0$であり厳密に解くことが出来ることを利用して,第5項を無視して考えたときと,試行関数を定めて変分法を用いたときを比較する.
  なお,実験によりヘリウム原子の基底エネルギーは$-78.6\ \r{eV}$と求まっている.
  \begin{figure}[htbp]
    \centering
    \includegraphics[width=0.5\columnwidth]{fig/helium.pdf}
    \caption{ヘリウム原子の構造}\label{helium-atom}
  \end{figure}
  \begin{myex}{ヘリウム原子の基底エネルギー(荒い近似)}{rough-helium}
    計算を行うと,ヘリウムの原子番号を$Z$として$\hat{H}^0$の基底波動関数,
    \begin{align}
      \psi = \cfrac{Z^3}{\pi a_0^3}\exp\qty(-Z\frac{r_1 + r_2}{a_0})\label{helium}
    \end{align}
    と$\hat{H}^0$の基底エネルギー,
    \begin{align}
      E = -8\ \r{Ry}\approx -108.8\ \r{eV}\label{helium-rough-min}
    \end{align}
    が求まる\footnote{
      $a_0 = \cfrac{4\pi\epsilon_0\hbar^2}{me^2} \approx 5.29\times 10^{-11}\ \r{m}$: Bohr半径
    }\footnote{
      $Z = 2$
    }\footnote{
      $\r{Ry} = \cfrac{\hbar^2}{2m\omega^2} \approx 13.6 \ \r{eV}$: Rydberg定数
    }.
  \end{myex}
  \begin{myex}{ヘリウム原子の基底エネルギー(変分法)}{}
    \exref{rough-helium}の結果とヘリウム原子の基底エネルギーの測定結果は$-78.6\ \r{eV}$と大きく異なっているため,相互作用の項を取り入れた近似を考える.
    \refe{helium}を試行関数$\psi(Z)$とする.
    $\psi(Z)$を用いてエネルギーを計算する.
    \begin{align}
      E(Z) &= \cfrac{\displaystyle\int\psi^{*}\hat{H}\psi \dd{\bm{r}_1}\dd{\bm{r}_2}}{\displaystyle\int\psi^{*}\psi \dd{\bm{r}_1}\dd{\bm{r}_2}} \\
      &= -2\qty(4Z - Z^2 - \cfrac{5}{8}Z)\ \r{Ry}\label{helium-energy-function} 
    \end{align}
    となる.\refe{helium-energy-function}が最小となるような$Z$を$Z_0$とすると$Z_0 = 27/16$であったので,
    \begin{align}
      E(Z) \geq E\qty(Z_0) = -77.5\ \r{eV}\label{helium-energy-varidation-min}
    \end{align}
    となった.\refe{helium-energy-varidation-min}と\refe{helium-rough-min}を比べると,変分法による近似の方が真の基底エネルギ$-78.6\ \r{eV}$に近い値が得られた\footnote{
      $Z_0 < 2$は遮蔽効果により有効電荷が$2e$より小さくなったことを意味する.
    }\footnote{
      積分の計算はDavid J. Griffith, \textit{Introduction to Quantum Mechanics}, pp. 333-334にある.
    }.
  \end{myex}
\end{document}
      \subsection{変分法の誤差}
        \documentclass{report}
\input{../../../head.tex}
\begin{document}
  真の基底状態$\ket{E_0}$に第1励起状態$\ket{E_1}$を10\%含んだ試行関数$\ket{\psi} = \ket{E_0} + \cfrac{1}{10}\ket{E_1}$を使ってエネルギーを計算する.
  \begin{align}
    E(\psi) &= \frac{\mel**{\psi}{\hat{H}}{\psi}}{\braket{\psi}} \\
    &= \cfrac{\mel**{E_0}{\hat{H}}{E_0} + \cfrac{1}{100}\me**l{E_1}{\hat{H}}{E_1}}{1 + \cfrac{1}{100}} \\
    &= \cfrac{E_0 + 0.01E_1}{1.01}\\
    &\approx 0.99E_0 + 0.01E_1
  \end{align}
  試行関数で10\%含まれていた誤差がエネルギーでは1\%に収まっている.
  \begin{myex}{}{}
    無限井戸型ポテンシャル$[-a, a]$を考える.
    この問題を厳密に解けば$n$番目のエネルギー準位は,
    \begin{align}
      E_n = \cfrac{\hbar^2}{2m}\qty(\cfrac{n\pi}{2a})^2
    \end{align}
    と計算できるが,ここでは変分法を用いて近似解を求める.
    予想される試行関数の条件は
    \begin{itemize}
      \item $\psi(a) = \psi(-a) = 0$
      \item 節がない
    \end{itemize}
    である.よって今回は
    \begin{align}
      \psi(x) = a^2 - x^2
    \end{align}
    を採用する.この試行関数を用いたときの基底エネルギーを見積もれ.
    \tcblower
    \begin{align}
      E(\psi) &= \cfrac{\displaystyle \int_{-a}^{a}\qty(a^2 - x^2)\qty(-\frac{\hbar^2}{2m}\dv[2]{x})\qty(a^2 - x^2)\dd{x}}{\displaystyle \int_{-a}^{a}\qty(a^2 - x^2)^2\dd{x}} \\
      &= \cfrac{10}{\pi^2}E_1 \\
      &\approx 1.01E_1
    \end{align}
    真の基底エネルギー$E_1$に近い値が得られた\footnote{このくらいの計算が期末試験に出たことがある.}.
  \end{myex}
\end{document}
      \subsection{練習問題}
        \documentclass{report}
\input{../../../head.tex}
\begin{document}
  \begin{myexc}{Griffith Example 8.1}{}
    1次元調和振動子$\hat{H} = -\cfrac{\hbar^2}{2m}\dv[2]{x} + \cfrac{1}{2}m\omega^2 x^2$の基底エネルギーを見積もれ.
    ただし,試行関数を$\psi(x) = \qty(\cfrac{2b}{\pi})^{1/4}\e^{-bx^2}$とせよ.試行関数は規格化されている.
    \tcblower
    \begin{align}
      E(b) = \bra{\psi}\hat{H}\ket{\psi} &= \qty(\cfrac{2b}{\pi})^{1/2}\int_{-\infty}^{\infty}\e^{-bx^2}\qty(-\cfrac{\hbar^2}{2m}\dv[2]{x} + \cfrac{1}{2}m\omega^2x^2)\e^{-bx^2}\dd{x}\\
      &=\cfrac{\hbar^2b}{2m}+\cfrac{m\omega^2}{8b}
    \end{align}
    次に$E(b)$の最小値を求める.
    \begin{align}
      \dv{b}E(b_0) = \cfrac{\hbar^2}{2m} - \cfrac{m\omega^2}{8b_{0}^2} = 0\Rightarrow b_0 = \cfrac{m\omega}{2\hbar}
    \end{align}
    \begin{align}
      E(b_0) = \cfrac{1}{2}\hbar\omega
    \end{align}
    偶然にも試行関数は基底エネルギーの固有関数となっていたため,$E(b_0)$は基底エネルギーと一致した.
  \end{myexc}
  \begin{myexc}{Griffith Example 8.2}{}
    デルタ関数型ポテンシャル$\hat{H} = -\frac{\hbar^2}{2m}\dv[2]{x}-\alpha \delta(x)$の基底エネルギーを見積もれ.
    ただし,試行関数を$\psi(x) = \qty(\cfrac{2b}{\pi})^{1/4}\e^{-bx^2}$とせよ.試行関数は規格化されている.
    \tcblower
    \begin{align}
      \ev{V} &= -\alpha \qty(\cfrac{2b}{\pi})^{1/2}\int_{-\infty}^{\infty}\e^{-2bx^2}\delta(x)\dd{x} = -\alpha\qty(\cfrac{2b}{\pi})^{1/2}\\
      \ev{T} &= \cfrac{\hbar^2 b}{2m}\\
      E(b) &= \cfrac{\hbar^2 b}{2m} - \alpha\qty(\cfrac{2b}{\pi})^{1/2}
    \end{align}
    $E(b)$の最小値を求める.
    \begin{align}
      \dv{b}E(b_0) = \cfrac{\hbar^2}{2m} - \cfrac{\alpha}{\sqrt{2\pi b_0}} = 0\Rightarrow b_0=\cfrac{2m^2\alpha^2}{\pi\hbar^4}
    \end{align}
    よって,基底エネルギーの近似解として
    \begin{align}
      E(b_0) = -\cfrac{m\alpha^2}{\pi\hbar^2}
    \end{align}
    を得る\footnote{
      厳密解を求めることができ,$\psi(x) = \cfrac{\sqrt{m\alpha}}{\hbar}\e^{-m\alpha\abs{x}/\hbar^2},\ E_0 = -\cfrac{m\alpha^2}{2\hbar^2}$である.
    }.
  \end{myexc}
  \begin{myexc}{Griffith Example 8.3}{}
    $[0, a]$の無限井戸型ポテンシャルの基底エネルギーを見積もれ.ただし,試行関数を
    \begin{align}
      \psi(x)=
      \begin{dcases*}
        Ax & if $0 \leq x \leq a/2$ \\
        A(a - x) & if $a/2\leq x \leq a$ \\ 
        0 & otherwise 
      \end{dcases*}
    \end{align}
    とせよ.
    \tcblower
    規格化条件より,$A = \cfrac{2}{a}\sqrt{\cfrac{3}{a}}$を得る.波動関数の導関数は
    \begin{align}
      \dv[2]{\psi}{x} =
        \begin{dcases*}
          Ax & if $0 \leq x \leq a/2$ \\
          A(a - x) & if $a/2\leq x \leq a$ \\ 
          0 & otherwise 
        \end{dcases*}
    \end{align}
    である.よって,2次の微係数として
    \begin{align}
      \dv[2]{\psi}{x} = A\delta(x) - 2A\delta\qty(x - \cfrac{a}{2}) + A\delta(x - a)
    \end{align}
    を得る.したがって近似解は
    \begin{align}
      E &= \int_{0}^{a}\psi(x)\qty(-\cfrac{\hbar^2}{2m}\dv[2]{x})\psi(x)\dd{x}\\
      &= -\cfrac{\hbar^2}{2m}\int_{0}^{a} A\qty[\delta(x)-\delta\qty(x-\cfrac{a}{2}) + \delta(x - a)]\psi(x)\dd{x}\\
      &= \cfrac{12\hbar^2}{2ma^2}
    \end{align}
    である\footnote{
      厳密解は$E_0 = \cfrac{\pi^2\hbar^2}{2ma^2}$
    }.
  \end{myexc}
  \begin{myexc}{Griffith Problem8.4 (a)}{}
    試行関数$\ket{\psi}$が基底状態と直交するとき,つまり$\braket{\psi}{0}$のとき,
    \begin{align}
      E(\psi)\geq E_1
    \end{align}
    であることを示せ\footnote{
      例えば偶関数のポテンシャルに対し奇関数の試行関数で計算すれば第1励起状態のエネルギーの近似解が得られる.
    }.
    ただし$E_1$は第1励起状態のエネルギーである.$\ket{\psi}$は規格化されている.
    \tcblower
    \begin{proof}
      \begin{align}
        E(\psi) &= \sum_{k = 0}E_k\abs{\braket{\psi}{k}}^2 \\
        &= E_0\abs{\braket{\psi}{0}}^2 + \sum_{k = 1}\abs{\braket{\psi}{k}}^2 \\
        &= 0 + \sum_{k = 1}\abs{\braket{\psi}{k}}^2 \\
        &\geq E_1\sum_{k = 1}\abs{\braket{\psi}{k}}^2 = E_1
      \end{align}
    \end{proof}
  \end{myexc}
\end{document}

    \section{摂動I(定常摂動)}
      \documentclass{report}
\input{../../../head.tex}
\begin{document}
  ハミルトニアンが時間に依存しない定常摂動(time-independent perturbation)を扱う.
\end{document}

      \subsection{準備}
        \documentclass{standalone}
\input{../../../head.tex}
\begin{document}
  次の式は厳密に解くことができるとする.
  \begin{equation}
    \hat{H}^{(0)}\ket{n^{(0)}}=E_0^{(0)}\ket{n^{(0)}}
  \end{equation}
  ここに摂動$\hat{V}$を加える\footnote{摂動の例: 光,電場}.
  \begin{equation}
    \qty(\hat{H}^{(0)}+\hat{V})\ket{n}=E_n\ket{n}
  \end{equation}
  $\hat{H}=\hat{H}^{(0)}+\lambda\hat{V}$とする.$\lambda\to0$ならば,
  \begin{equation}
    \begin{cases}
      \ket{n}\to\ket{n^{(0)}}\\
      E_n\to E_n^{(0)}
    \end{cases}
  \end{equation}
  である.
  ここで,$\hat{H}=\hat{H}^{(0)}+\lambda\hat{V}$の解を次のようにおく.\footnote{$\hat{H},n,E$の肩の()を今後は省略する.}
  \begin{equation}
    \begin{cases}
      \ket{n}=\ket{n^0}+\lambda\ket{n^1}+\lambda^2\ket{n^2}+\cdots\\
      E_n=E_n^0+\lambda+E_n^1+\lambda^2+E_n^2+\cdots
    \end{cases}
  \end{equation}
  $\ket{n^1},\ket{n^2},E_n^1,E_n^2$を求める.
  ここで,規格化条件として
  \begin{equation}
    \braket{n^0}{n}=1
  \end{equation}
  を定める.
  以上の$\hat{H},\ket{n},E_n$を用いて,Schrödinger方程式を立て,整理すると,
  $\lambda^0$,$\lambda^1$,$\lambda^2$の係数としてそれぞれ
  \begin{align}
    &(E_n^0-\hat{H}^0)\ket{n^0}=0\\
    \label{1st}
    &(E_n^0-\hat{H}^0)\ket{n^1}+E_n^1\ket{n^0}=\hat{V}\ket{n^0}\\
    \label{2nd}
    &(E_n^0-\hat{H}^0)\ket{n^2}+E_n^1\ket{n^1}+E_n^2\ket{n^0}=\hat{V}\ket{n^1}\\
  \end{align}
  を得る.
\end{document}

      \subsection{1次摂動}
        \documentclass{report}
\input{../../../head.tex}
\begin{document}
  まずエネルギー補正$E_n^{(1)}$について考える.
  \refe{1st-perturbation}の両辺に$\bra{n^{(0)}}$を作用すると,
  \begin{align}
    \mel**{n^{(0)}}{\qty(E_n^{(0)} - \hat{H}^{(0)})}{n^{(1)}} + \mel**{n^{(1)}}{E_n^{(1)}}{n^{(0)}} &= \mel{n^{(0)}}{\hat{V}}{n^{(0)}} \\ 
    \Leftrightarrow E_n^{(0)}\braket{n^{(0)}}{n^{(1)}} - E_n^{(0)}\braket{n^{(0)}}{n^{(1)}} + E_n^{(1)}\braket{n^{(1)}}{n^{(0)}} &= \mel{n^{(0)}}{\hat{V}}{n^{(0)}} \\ 
    \Leftrightarrow 0 + E_n^{(1)}\braket{n^{(0)}} &= \ev**{\hat{V}}{n^{(0)}} \\ 
    \Leftrightarrow E_n^{(1)} &= \ev**{\hat{V}}{n^{(0)}}
  \end{align}
  を得る.よって,1次摂動によるエネルギー補正は,
  \begin{itembox}[l]{1次摂動によるエネルギー補正}
    \begin{align}
      E_n^{(1)} = \ev**{\hat{V}}{n^{(0)}}
    \end{align}
  \end{itembox}
  である.
  \par
  次に固有ベクトル$\ket{n^{(1)}}$の補正を求める.
  \refe{1st-perturbation}の両辺に$\bra{m^{(0)}}$を左から作用すると,
  \begin{align}
    \mel**{m^{(0)}}{\qty(E_n^{(0)} - \hat{H}^{(0)})}{n^{(1)}} + \mel**{m^{(0)}}{E_n^{(1)}}{n^{(0)}} &= \mel**{m^{(0)}}{\hat{V}}{n^{(0)}} \\ 
    E_n^{(0)}\braket{m^{(0)}}{n^{(1)}} - E_m^{(0)}\braket{m^{(0)}}{n^{(1)}} + 0 &= \mel**{m^{(0)}}{\hat{V}}{n^{(0)}}\\
    \qty(E_n^{(0)} - E_m^{(0)})\braket{m^{(0)}}{n^{(1)}} &= \mel**{m^{(0)}}{\hat{V}}{n^{(0)}} \label{1st-perturbation-prep}
  \end{align}
  となる.\refe{1st-perturbation-prep}に$\bra{m^{(0)}}$をかけて,$m$に関して和を取れば,
  \begin{align}
    \sum_{m}\qty(E_n^{(0)} - E_m^{(0)})\ket{m^{(0)}}\braket{m^{(0)}}{n^{(0)}} &= \sum_{m}\mel**{m^{(0)}}{\hat{V}}{n^{(0)}}\ket{m^{(0)}} \\ 
    \Leftrightarrow \sum_{m}\qty(E_n^{(0)} - E_m^{(0)})\ket{n^{(0)}} &= \sum_{m}\mel**{m^{(0)}}{\hat{V}}{n^{(0)}}\ket{m^{(0)}} \\ 
    \Leftrightarrow \ket{n^{(0)}}\sum_{m \neq n}\qty(E_n^{(0)} - E_m^{(0)}) &= \sum_{m \neq n}\mel**{m^{(0)}}{\hat{V}}{n^{(0)}}\ket{m^{(0)}} \\
    \Leftrightarrow \ket{n^{(0)}} &= \sum_{m\neq n}\cfrac{\mel**{m^{(0)}}{\hat{V}}{n^{(0)}}}{E_n^{(0)} - E_m^{(0)}}\ket{m^{(0)}}
  \end{align}
  となる.
  途中の式変形でエネルギー縮退がないので,
  \begin{align}
    E_n^{(0)} - E_m^{(0)} 
    \begin{dcases}
      = 0 & n = m \label{no-degeneracy} \\ 
      \neq 0 & n \neq m
    \end{dcases}
  \end{align}
  とした.
  また,Hermite演算子である$\hat{H}^{(0)}$の固有ベクトルに関する完全性より,
  \begin{align}
    I = \sum_{m}\ketbra{m^{(0)}}{m^{(0)}}\label{completeness}
  \end{align}
  を用いた.
  \begin{itembox}[l]{1次摂動による固有ベクトル補正}
    \begin{align}
      \ket{n^{(1)}} = \sum_{m\neq n}\cfrac{\mel**{m^{(0)}}{\hat{V}}{n^{(0)}}}{E_n^{(0)} - E_m^{(0)}}\ket{m^{(0)}}\label{1st-order-eigenvector}
    \end{align}
  \end{itembox}
  を得る.
  \refe{1st-order-eigenvector}において,$\ket{n^{(1)}}$と$\ket{n^{(0)}}$は直交することに注意する.
\end{document}

      \subsection{ヘリウム原子(再考)}
        \documentclass{report}
\input{../../../head.tex}
\begin{document}
  \begin{myex}{ヘリウム原子の基底エネルギー}{}
    \begin{align}
      \hat{H} = \hat{H}^0 + \frac{e^2}{4\pi\epsilon_0r_{12}}\equiv\hat{H}^0 + \hat{V}
    \end{align}
    $\hat{H}^0$の基底エネルギーは
    \begin{align}
      \psi^0 = \frac{Z^3}{\pi a_0^3}\e^{-Z(r_1+r_2)/a_0}
    \end{align}
    である.よって,$\hat{V}$による1次のエネルギー補正は以下のように計算できる.
    \begin{align}
      E^1 &= \bra{\psi^0}\hat{V}\ket{\psi^0}\\
      &= \int{\psi^0}^{*}\frac{e^2}{4\pi\epsilon_0r_{12}}\psi^0\dd{\bm{r}_1}\dd{\bm{r}_2} \\
      &= \frac{5}{4}Z\ \mathrm{Ry}
    \end{align}
    よって,基底エネルギー
    \begin{align}
      E_0&=E^0+E^1\\
      &= -8\ \mathrm{Ry}+\frac{5}{4}\times{2}\ \mathrm{Ry}\\
      &= -74.8\ \mathrm{eV}
    \end{align}
    を得る\footnote{測定値は$-78.6\ \mathrm{eV}$}.
  \end{myex}
\end{document}

      \subsection{2次摂動}
        \documentclass{report}
\input{../../../head.tex}
\begin{document}
  式(\ref{2nd})の両辺に$\bra{n^0}$を作用することで,2次摂動によるエネルギー補正を得る.
  \begin{align}
    0+0&+\bra{n^0}E_n^2\ket{n^0}=\bra{n^0}\hat{V}\ket{n^1}\\
    E_n^2&=\bra{n^0}\hat{V}\ket{n^1}
  \end{align}
  \begin{itembox}[l]{2次摂動によるエネルギー補正}
  \begin{equation}
    E_n^2=\sum_{m\ne n}\frac{\lvert \bra{m^0}\hat{V}\ket{n^0}\rvert^2}{E_n^0-E_m^0}
  \end{equation}
  \end{itembox}
  また,基底状態においては$E_{n=0}^0 < E_m^0$である.常に$\frac{\lvert \bra{m^0}\hat{V}\ket{n^0}\rvert^2}{E_n^0-E_m^0}$の
  分母は負であるため,\textbf{基底状態のエネルギーは2次摂動により必ず下がる}.
  \begin{myex}{Mott insulator\footnote{
      Nevill Francis Mott (1905-1996)
    }\footnote{
      バンドギャップが大きくギャップ内にフェルミ準位があるバンド絶縁体と異なり,運動エネルギーが小さくCoulomb力が大きいため電子が移動できない絶縁体である.
    }\footnote{
      $R\mathrm{NiO_3,TaS_2,Sr_2IrO_4}$など
    }}{}
    Coulomb力が強い4つのサイトに電子を4つ入れる.
    $\uparrow \ \uparrow\ \uparrow\ \uparrow$と$\uparrow\ \downarrow\ \uparrow\ \downarrow$のどちらが基底状態としてふさわしいだろうか.
    サイト間の電子の飛び移りを摂動として扱う.
    ここで重要なのは,基底状態のエネルギーは2次摂動により必ず下がるということである.
    $\uparrow \ \uparrow\ \uparrow\ \uparrow$に摂動を加えたとしてもPauliの排他律により電子の飛び移りは起こらない.
    摂動によってエネルギーは変化しない.しかし,$\uparrow\ \downarrow\ \uparrow\ \downarrow$は電子が反平行であるため電子のサイト間での飛び移りが許される.
    これは,2次摂動によるエネルギーの低下を引き起こす.
    よって,$\uparrow\ \downarrow\ \uparrow\ \downarrow$の方が基底状態としてふさわしい\footnote{
      Mott insulatorは反強磁絶縁体である.
    }.
  \end{myex}
\end{document}

      \subsection{量子閉じ込めStark効果}
        \documentclass{report}
\input{../../../head.tex}
\begin{document}
  ここでは,2次摂動を用いた例題としてStark効果\footnote{Johanes Stark(1874-1957)}\footnote{電場によるエネルギー準位の変化をStark効果という.}を考えよう.
  \begin{myex}{量子閉じ込めStark効果}{stark}
    定常状態ののハミルトニアン$\hat{H}^{(0)}$に,電場による摂動$\hat{V}$を加えたハミルトニアン$\hat{H}$を考える.
    ただし,定常状態のポテンシャルは,長さ$L$の無限井戸型ポテンシャル$\hat{U}$である.
    $\hat{U}$,$\hat{V}$,$\hat{H}^{(0)}$,$\hat{H}$は,
    \begin{align}
      U(x) &\coloneqq
      \begin{dcases}
        0 & \ \abs{x}\leq L/2\\
        \infty & \text{otherwise}
      \end{dcases} \\ 
      V(x) &\coloneqq -e\phi(x) = eEx\ (e>0) \\ 
      \hat{H}^{(0)} &\coloneqq -\frac{\hbar^2}{2m}\dv[2]{x} + \hat{U}(x) \\ 
      \hat{H} &\coloneqq \hat{H}^{(0)} + \hat{V}(x) \\ 
    \end{align}
    と定義される.
    また,$\hat{H}^{(0)}$の固有エネルギーとそれに属する固有関数は,
    \begin{align}
      E_n^{(0)} &= \frac{\hbar^2}{2m}\qty(\frac{\pi}{L})^2n^2,\ (n = 1, 2, 3,\cdots)\\
      \phi_n(x)&=
      \begin{dcases}
        \sqrt{\frac{2}{L}}\cos\qty(\frac{n\pi}{L}x) & n: \r{odd}\\
        \sqrt{\frac{2}{L}}\sin\qty(\frac{n\pi}{L}x) & n: \r{even}
      \end{dcases}
    \end{align}
    のようになっている.
    このとき,2次の摂動まで用いて$\hat{V}$の影響による$n = 1$のエネルギー補正を計算せよ.
    \tcblower
    1次摂動によるエネルギー補正は奇関数の積分になるため0である.\footnote{もし0でないならば,電場をかける向きによりエネルギーが変わることを意味するが,これは対称性より不合理である.}.\\
    2次摂動によるエネルギー補正は,
    \begin{align}
      E_{1}^{(2)} &= \sum_{m\neq 1}\frac{\abs{V_{m1}}^2}{E_1^{(0)} - E_m^{(0)}}\\
      V_{m1} &= eE\int\phi_m^*x\phi_{1}\dd{x}\\
      &
      \begin{dcases}
      = 0 & n: \r{odd}\\
      \neq 0 & n: \r{even}
      \end{dcases}\\
      E_{1}^{(2)}&=\frac{\abs{V_{21}}^2}{E_1^{(0)} - E_2^{(0)}} + \frac{\abs{V_{41}}^2}{E_1^{(0)} - E_4^{(0)}} + \cdots\\
      &\approx \frac{\abs{V_{21}}^2}{E_1^{(0)} - E_2^{(0)}}\\
      &=-\frac{256}{234\pi^4}\frac{(eEL)^2}{E_1^{(0)}}
    \end{align}
    と計算できて,2次の摂動を考えるとエネルギーは低下することがわかる.
  \end{myex}
\end{document}

      \subsection{縮退がある場合の摂動論}
        \documentclass{standalone}
\input{../../../head.tex}
\begin{document}
  1次摂動の式
  \begin{align}
    \label{1ji}
    (E_n^0-\hat{H}^0)\ket{n^1}+E_n^1\ket{n^0}=\hat{V}\ket{n^0}\\
    \ket{n^1}=\sum_{m\ne n}\ket{m^0}\frac{\bra{m^0}\hat{V}\ket{n^0}}{E_n^0-E_m^0}
  \end{align}
  これは$E_n^0=E_m^0$となる$m\ne n$が存在すると発散してしまう.そのため,発散する項は別で扱う必要がある.
  以下のような2重縮退がある場合を考える.
  \begin{align}
    \hat{H}^0\ket{n_a^0}&=E_n^0\ket{n_a^0}\\
    \hat{H}^0\ket{n_b^0}&=E_n^0\ket{n_b^0}
  \end{align}
  ただし,$\braket{n_i^0}{n_j^0}=\delta_{ij}$とする.$\ket{n_a^0}$と$\ket{n_b^0}$は同じ固有値をもつため,これらの線形結合
  $\ket{n_0}=\alpha\ket{n_a^0}+\beta\ket{n_b^0}$も解となる.\\
  まず,式(\ref{1ji})の両辺に左から$\bra{n_a^0}$を作用する.
  \begin{equation}
    \bra{n_a^0}(E_n^0-\hat{H}^0)\ket{n^1}+\bra{n_a^0}E_n^1\ket{n^0}=\bra{n_a}\hat{V}\ket{n^0}
  \end{equation}
  第1項は$E_n^0-E_n^0$より0.ここで,$\bra{n_i^0}\hat{V}\ket{n_j^0}=\omega_{ij}$とおけば
  \begin{equation}
    \alpha E_n^1=\alpha\omega_{aa}+\beta\omega_{ab}
  \end{equation}
  を得る.式(\ref{1ji})の両辺に左から$\bra{n_b^0}$を作用することも考えることにより,合わせて
  \begin{equation}
    \begin{cases}
      \alpha\omega_{aa}+\beta\omega_{ab}=\alpha E_n^1\\
      \alpha\omega_{ba}+\beta\omega_{bb}=\beta E_n^1
    \end{cases}
  \end{equation}
  を得る.これは行列を用いて以下のように書き直される.
  \begin{equation}
    \label{gyouretu}
    \begin{pmatrix}
      \omega_{aa}-E_n^1&\omega_{ab}\\
      \omega_{ba}&\omega_{bb}-E_n^1
    \end{pmatrix}
    \begin{pmatrix}
      \alpha\\
      \beta
    \end{pmatrix}
    =
    \begin{pmatrix}
      0\\0
    \end{pmatrix}
  \end{equation}
  $(\alpha,\beta)=(0,0)$以外の解を持つには行列式が0となればいいので,
  \begin{equation}
    \begin{vmatrix}
      \omega_{aa}-E_n^1&\omega_{ab}\\
      \omega_{ba}&\omega_{bb}-E_n^1
    \end{vmatrix}
    =0
  \end{equation}
  である.よって,1次の摂動エネルギーとして
  \begin{screen}
  \begin{equation}
    \label{syukutai}
    E_n^1=\frac{1}{2}\qty[(\omega_{aa}+\omega_{bb}\pm\sqrt{(\omega_{aa}-\omega_{bb})^2+4|\omega_{ab}|^2})]
  \end{equation}
  \end{screen}
  を得る.縮退が解けてエネルギーが2つに分かれている.
  \begin{myex}{}{}$\omega_{aa}=\omega_{bb}=0,\omega_{ab}=\omega_{ba}=\omega$の場合
  式(\ref{gyouretu})は
  \begin{equation}
    \begin{pmatrix}
      -E_n^1&\omega\\
      \omega&-E_n^1
    \end{pmatrix}
    \begin{pmatrix}
      \alpha\\
      \beta
    \end{pmatrix}
    =\begin{pmatrix}
      0\\0
    \end{pmatrix}
  \end{equation}
  となる.よって,
  \begin{equation}
    E_n^1=
    \begin{cases}
      +\omega\ (\alpha=1,\beta=1)\\
      -\omega\ (\alpha=1,\beta=-1)
    \end{cases}
  \end{equation}
  である.これは摂動を加える前に縮退していた2つの状態$\ket{n^0}=\ket{n_a^0}\pm\ket{n_b^0}$の縮退が解け,エネルギー$E_n^0+\omega$をもつ状態$\ket{n_a^0}+\ket{n_b^0}$
  とエネルギー$E_n^0-\omega$をもつ状態$\ket{n_a^0}-\ket{n_b^0}$に分かれたことを意味している.
  \end{myex}
\end{document}

      \subsection{物質中の電子}
        \documentclass{standalone}
\input{../../../head.tex}
\begin{document}
  結晶中の周期ポテンシャルによりバンドギャップができることを確認する.\\
  長さ$L$の周期ポテンシャル中の1次元自由電子の運動を考える.\\
  波動関数は
  \begin{equation}
    \phi_k(x)=\braket{x}{k}=\frac{1}{\sqrt{L}}\mathrm{e}^{ikx}
  \end{equation}
  エネルギー固有値は
  \begin{equation}
    \epsilon_0(k)=\frac{\hbar^2k^2}{2m}\ (k=\frac{2\pi}{L}N,\ N=0,\pm1,\pm2,...)
  \end{equation}
  である.\\
  ここで$V(x+a)=V(x)$を満たす結晶の周期ポテンシャルを摂動として加える.
  \begin{align}
    V(x)&=2V\cos\frac{2\pi}{a}x\\
    &=V(\mathrm{e}^{i\frac{2\pi}{a}x}+\mathrm{e}^{-i\frac{2\pi}{a}x})\\
    &=V(\mathrm{e}^{igx}+\mathrm{e}^{-igx})\\
    g&\equiv\frac{2\pi}{a}
  \end{align}
  2次摂動まで含めるとエネルギーは次のようになる.
  \begin{equation}
    \label{2ji}
    E_n=E_n^0+\bra{n^0}\hat{V}\ket{n^0}+\sum_{m\ne n}\frac{\qty|\bra{m^0}\hat{V}\ket{n^0}|^2}{E_n^0-E_m^0}
  \end{equation}
  ここで,状態及びエネルギーのラベリングを$n$から$k$に変更する.$V_{k'k}=\bra{k'}\hat{V}\ket{k}$として式(\ref{2ji})を書き換える.
  \begin{equation}
    E(k)=\epsilon^0(k)+V_{kk}+\sum_{k'\ne k}\frac{|V_{k'k}|^2}{\epsilon^0(k)-\epsilon^0(k')}
  \end{equation}
  摂動によるエネルギーは
  \begin{align}
    V_{k'k}&=\frac{V}{L}\int_{-L/2}^{L/2}\phi_{k'}^{*}(x)\hat{V}(x)\phi_k(x)dx\\
    &=V\qty[\frac{\sin\frac{qL}{2}}{\frac{qL}{2}}+\frac{\sin\frac{q'L}{2}}{\frac{q'L}{2}}]\\
  \end{align}
  と計算される.($q\equiv -k'+g+k,\ q'\equiv -k'-g+k$)摂動によるエネルギーはsinc関数の形になっているので$L\to\infty$ではデルタ関数に近似できる.よって,
  \begin{equation}
    V_{k'k}=V(\delta_{q,0}+\delta_{q',0})=
    \begin{cases}
      V\ \mathrm{if}\ k'=k+g\ \mathrm{or}\ k'=k-g\\
      0\ \mathrm{otherwise}
    \end{cases}
  \end{equation}
  である.したがってエネルギーは
  \begin{equation}
    \label{Ebr}
    E(k)=\epsilon^0(k)+\frac{V^2}{\epsilon(k)-\epsilon(k+g)}+\frac{V^2}{\epsilon(k)-\epsilon(k-g)}
  \end{equation}
  となる.この振る舞いをを1st Brillouin Zoneの内外で確認する(対称性から右側のみ).\\
  \\
  1st Brillouin Zone内側($k=k_1<\frac{\pi}{a}$)\\
  $\epsilon(k)$は放物線なので
  \begin{equation}
    \begin{cases}
      \epsilon(k_1)\ll\epsilon^0(k_1+g)\\
      \epsilon(k_1)<\epsilon^0(k_1-g)
    \end{cases}
  \end{equation}
  が成りたつ.よって,$E(k)<\epsilon^0(k)$.摂動が加わった後のエネルギーは加わる前のエネルギーより小さくなる.\\
  1st Brillouin Zone外側($k=k_2>\frac{\pi}{a}$)\\
  同様に考え,
  \begin{equation}
    \begin{cases}
      \epsilon(k_2)\ll\epsilon^0(k_2+g)\\
      \epsilon(k_2)>\epsilon^0(k_2-g)
    \end{cases}
  \end{equation}
  である.よって,$E(k)>\epsilon^0(k)$が成り立ち,摂動が加わった後のエネルギーは加わる前のエネルギーより大きくなる.\\
  以上の議論により,結晶中の周期ポテンシャルによりバンドギャップが形成されることがわかった.しかし,式(\ref{Ebr})に$k=\pm\frac{\pi}{a}$を代入すると発散してしまう.以下では
  2重縮退があるときの摂動を考えバンドギャップ$\Delta E$を求める.\\
  式(\ref{syukutai})で$a=\frac{\pi}{a}$,$b=-\frac{\pi}{a}$とする.$V_{kk}=0,V_{ab}=V$なので,2次摂動によるエネルギーは
  \begin{align}
    E&=\pm\frac{1}{2}\sqrt{4|V|^2}\\
    &=\pm V
  \end{align}
  である.よって,
  \begin{equation}
    \Delta E=2V
  \end{equation}
  を得る.また,$E=\pm V$に対応する波動関数はそれぞれ
  \begin{align}
    \psi_{+}&=\phi_{k=\pi/a}+\phi_{k=-\pi/a}\sim\cos\frac{\pi}{a}x\\
    \psi_{-}&=\phi_{k=\pi/a}-\phi_{k=-\pi/a}\sim\sin\frac{\pi}{a}x
  \end{align}
  であり,定在波が生じる.\\
  バンドギャップの起源はBragg反射である.Bragg反射は以下の式を満たす.
  \begin{equation}
    2a\sin\theta=\lambda
  \end{equation}
  今回の場合は1次元なので$\theta=\pi/2$であり,波数は$k=2\pi/\lambda$である.よって,Bragg条件は
  \begin{equation}
    k=\frac{\pi}{a}
  \end{equation}
  と書き換えられる.上の条件を満たす波数のみが反射し,定在波をつくる.sinで表される波動関数はポテンシャルが最小となる波数で確率振幅が最大となる.
  cosで表される波動関数はポテンシャルが最大となる波数で確率振幅が最大となる.よって,sinの方はエネルギーが低くなり,cosの方は高くなる.これによりバンドギャップが生じる.
\end{document}

      \subsection{練習問題}
        \documentclass{report}
\input{../../../head.tex}
\begin{document}
  \begin{myexc}{Griffith Example 7.3}{}
    2次元調和振動子$\hat{H^0}=\frac{\hat{p}^2}{2m}+\frac{1}{2}m\omega^2(\hat{x}^2+\hat{y}^2)$の第1励起状態は縮退している.
    \begin{align}
      \psi^0_a&=\psi_0(x)\psi_1(y)=\sqrt{\frac{2}{\pi}}\frac{m\omega}{\hbar}y\exp\qty(\frac{m\omega}{2\hbar}(x^2+y^2))\\
      \psi^0_a&=\psi_1(x)\psi_0(y)=\sqrt{\frac{2}{\pi}}\frac{m\omega}{\hbar}x\exp\qty(\frac{m\omega}{2\hbar}(x^2+y^2))
    \end{align}
    ここに摂動$\hat{H'}=\epsilon m\omega^2xy$を加える.
    \begin{equation}
      \begin{pmatrix}
        \omega_{aa}-E_n^1&\omega_{ab}\\
        \omega_{ba}&\omega_{bb}-E_n^1
      \end{pmatrix}
      \begin{pmatrix}
        \alpha\\
        \beta
      \end{pmatrix}
      =
      \begin{pmatrix}
        0\\0
      \end{pmatrix}
    \end{equation}
    を用いて摂動を加えた後の固有関数及び摂動による補正エネルギーを求めよ.
    \tcblower
    \begin{align}
      W_{aa}&=\int\int\psi_a^0\hat{H'}\psi_a^0dxdy\\
      &=\epsilon m\omega^2\int|\psi_0(x)|^2xdx\int|\psi_1(x)|^2ydy\\
      &=0=W_{bb}
    \end{align}
    \begin{align}
      W_{ab}&=\qty[\int\psi_0(x)\epsilon m\omega^2x\psi_1(x)dx]^2\\
      &=\epsilon\frac{\hbar\omega}{2}\qty[\int\psi_0(x)(\hat{a}_{+}+\hat{a}_{-})\psi_1(x)dx]^2\\
      &=\epsilon\frac{\hbar\omega}{2}\qty[\int\psi_0(x)\psi_0(x)dx]^2\\
      &=\epsilon\frac{\hbar\omega}{2}
    \end{align}
    よって,摂動による補正エネルギーは$E_1=\pm\epsilon\frac{\hbar\omega}{2}$,固有関数は
    \begin{equation}
      \psi_{\pm}^0=\frac{1}{\sqrt{2}}\qty(\psi_b^0\pm\psi_a^0)
    \end{equation}
    である.
  \end{myexc}
\end{document}

    \section{摂動II(非定常摂動)}
      \documentclass{report}
\input{../../../head.tex}
\begin{document}
  摂動項が時間に依存する場合の摂動(time-dependent perturbation)\footnote{電磁波による摂動など.}を扱う.
  本節では,量子系の時間発展が状態ベクトルの時間変化で描像するSchr\"odinger描像で記述する.
  時間に依存するSchrödinger方程式,
  \begin{align}
    \i\hbar\dv{t}\ket{\psi(t)} = \hat{H}\ket{\psi(t)}\label{time-dependent-schroinger-eq}
  \end{align}
  を考える.なお,$\ket{\psi(t)}$は$\hat{H}^{(0)}$の固有ベクトルを用いて,
  \begin{align}
    \ket{\psi(t)} = \sum_{n}c_n(t)\ket{n}\label{psi-t-expantion}
  \end{align}
  と展開できたとする.$\hat{H}^{(0)}$の固有ベクトルが完全系を成すので,
  状態ベクトルが時間変化した空間は$\qty{\ket{n}\mid n = 0, 1, \cdots}$が張る空間の部分空間となることに注意する.
  $\hat{H}$の性質ごとに$\ket{\psi(t)}$の具体的な形を議論する.
  \begin{enumerate}
    \item $\hat{H} = \hat{H}^{(0)}$の場合\footnote{$\hat{H}^{(0)}$は厳密に解けるHamiltonian.}\par
      量子状態$\ket{\psi(t)}$の時間発展は時間発展演算子
      \footnote{
        $\Delta t \ll 1$として,
        \begin{align*}
          \ket{\psi(\Delta t)} = \exp\qty[-\i\frac{\hat{H}\Delta t}{\hbar}]\ket{\psi(0)} \approx\qty(\hat{I} - \i\frac{\hat{H}\Delta t}{\hbar})\ket{\psi(0)} 
          = \ket{\psi(0)} + \Delta t \left.\qty(\dv{t}\ket{\psi(t)})\right|_{t = 0}
        \end{align*}
        より微分の形で書けることから,確かに時間発展すると私は解釈する.}
      \footnote{
        演算子が交換するときは数字と同じ扱いをしても良いと考える.
        一般に演算子は,
        \begin{align*}
          \e^{\hat{A}}\e^{\hat{B}} = \exp{\hat{A} + \hat{B} + \frac{1}{2}\qty[\hat{A},\hat{B}] + \cdots}\neq\e^{\hat{A} + \hat{B}}
        \end{align*}
        なるBCH公式を満たす.
      }
      $\exp\qty(-\i\frac{\hat{H}t}{\hbar})$を用いて,
      \begin{align}
        \ket{\psi(t)} &= \exp\qty[-\i\frac{\hat{H}^{(0)}t}{\hbar}]\ket{\psi(0)} \\
        &= \sum_{n}c_n(0)\exp\qty[-\i\frac{\hat{H}^{(0)}t}{\hbar}]\ket{n} \\
        &= \sum_{n}c_n(0)\exp\qty[-\i\frac{E_n}{\hbar}t]\ket{n}
      \end{align}
      のように表すことができる.
      \par
      時間発展演算子を用いることなく計算することもできる.
      \refe{psi-t-expantion}を\refe{time-dependent-schroinger-eq}に代入すると,$\hat{H}$が時間に依存しないことに注意すれば,
      \begin{align}
        \i\hbar\dv{t}\qty(\sum_{n}c_n(t)\ket{n}) &= \hat{H}\qty(\sum_{n}c_n(t)\ket{n}) \label{naive-solution-to-decide-ct}\\ 
        \Leftrightarrow \sum_{n}\qty(\i\hbar\dv{c_n(t)}{t}\ket{n}) &= \sum_{n}\qty(c_n(t)E_n\ket{n}) \\ 
        \Leftrightarrow \forall n\ \i\hbar\dv{c_n(t)}{t}\ket{n} &= c_n(t)E_n\ket{n} \\ 
        \Leftrightarrow \forall n\ c_n(t) &= \exp\qty(-\i\frac{E_n}{\hbar}t)
      \end{align}
      を用いれば,
      \begin{align}
        \ket{\psi(t)} = \sum_{n}\exp\qty(-\i\frac{E_n}{\hbar}t)\ket{n}
      \end{align}
      を得る.
    \item $\hat{H} = \hat{H}^{(0)} + \hat{V}(t)$の場合\par
      このときは$\ket{\psi(t)}$を簡単な形で書き下すことが出来ないから,便宜的に,
      \begin{align}
        \psi(t) = \sum_{n}c_n(t)\exp\qty(-\i\frac{E_n}{\hbar}t)\ket{n}
      \end{align}
      と展開しておく.なお,$c_n(t)$が定数のときの$\ket{\psi(t)}$との整合性をとるために$\exp\qty(-\i\frac{E_n}{\hbar}t)$をかけてある.
      原理的には$c_n(t)$が求まれば量子系の時間発展の様子がわかる.% 暇になったら時間順序積のことを書く.
  \end{enumerate}
  \par
  さて,いずれの場合でも,量子系の性質を調べるには$c_n(t)$の具体的な形がわかればよいことを発見した.
  本節では,まずSchr\"odinger描像から相互作用表示に書き換え,$c_n(t)$を厳密に知ることが困難であることを知り,$c_n(t)$の近似解を導く.
\end{document}

      \subsection{相互作用表示}
        \documentclass{report}
\input{../../../head.tex}
\begin{document}
  Schr\"odinger表示の非定常摂動基本方程式は,\refe{time-dependent-schroinger-eq}と\refe{nonsteady-perturbation-cnt-eq}であり,
  \begin{align}
    \begin{dcases}
      \i\hbar\dv{t}\ket{\psi(t)} = \qty(\hat{H}^{(0)} + \hat{V}(t))\ket{\psi(t)} \label{TDSE} \\
      \ket{\psi(t)} = \sum_{n}c_n(t)\exp\qty(-\i\frac{E_n}{\hbar}t)\ket{n}
    \end{dcases}
  \end{align}
  と書けるのであった.
  \textbf{相互作用表示}(interaction picture)を\refe{interaction-picture}のような$\ket{\psi(t)} \to \ket{\psi(t)}_{\r{I}}$の変換を行って得られる状態ベクトルと定義する.
  \begin{itembox}[l]{相互作用表示}
    \begin{align}
      \ket{\psi(t)}_{\r{I}} \coloneqq \exp\qty(\i\frac{\hat{H}^{(0)}}{\hbar}t)\ket{\psi(t)}\label{interaction-picture}
    \end{align}
  \end{itembox}
  \refe{interaction-picture}を用いて,\refe{TDSE}等価な基本方程式,相互作用表示の非定常摂動基本方程式を導く.
  まず\refe{TDSE}の第2式を用いて,
  \begin{align}
    \ket{\psi(t)}_{\r{I}} &= \exp\qty(\i\frac{\hat{H}^{(0)}}{\hbar}t)\sum_{n}c_n(t)\exp\qty(-\i\frac{E_n}{\hbar}t)\ket{n}\\
    &= \sum_{n}c_n(t)\exp\qty(-\i\frac{E_n}{\hbar}t)\exp\qty(\i\frac{\hat{H}^{(0)}}{\hbar}t)\ket{n} \\ 
    &= \sum_{n}c_n(t)\exp\qty(-\i\frac{E_n}{\hbar}t)\exp\qty(\i\frac{E_n}{\hbar}t)\ket{n} \\ 
    &= \sum_{n}c_n(t)\ket{n}\label{TDSE-2nd-eq-interaction-picture}
  \end{align}
  と計算できる.
  \par
  次に,相互作用表示の時間微分を計算してみる.
  相互作用表示での摂動項$\hat{V}_{\r{I}}$を,
  \begin{align}
    \hat{V}_{\r{I}} \coloneqq \exp\qty(\i\frac{\hat{H}^{(0)}}{\hbar}t)\hat{V}(t)
  \end{align}
  とする.計算の途中で,\refe{TDSE}の第1式を用いると,
  \begin{align}
    \i\hbar\dv{t}\ket{\psi(t)}_{\r{I}} &= \dv{t}\qty[\exp\qty(\i\frac{\hat{H}^{(0)}}{\hbar}t)]\ket{\psi(t)} + \exp\qty(\i\frac{\hat{H}^{(0)}}{\hbar}t)\dv{t}\ket{\psi(t)} \\
    &= \i\frac{\hat{H}^{(0)}}{\hbar}\exp\qty(\i\frac{\hat{H}^{(0)}}{\hbar}t)\ket{\psi(t)} + \exp\qty(\i\frac{\hat{H}^{(0)}}{\hbar}t)\frac{1}{\i\hbar}\qty(\hat{H}^{(0)} + \hat{V}(t))\ket{\psi(t)} \\ 
    &= \frac{\i}{\hbar}\qty[\hat{H}^{(0)}, \exp\qty(\i\frac{\hat{H}^{(0)}}{\hbar}t)]\ket{\psi(t)} + \exp\qty(\i\frac{\hat{H}^{(0)}}{\hbar}t)\hat{V}(t)\ket{\psi(t)} \\ 
    &= \exp\qty(\i\frac{\hat{H}^{(0)}}{\hbar}t)\hat{V}(t)\ket{\psi(t)} \\
    &= \exp\qty(\i\frac{\hat{H}^{(0)}}{\hbar}t)\hat{V}(t)\exp\qty(-\i\frac{\hat{H}^{(0)}}{\hbar}t)\exp\qty(\i\frac{\hat{H}^{(0)}}{\hbar}t)\ket{\psi(t)} \\
    &= \exp\qty(\i\frac{\hat{H}^{(0)}}{\hbar}t)\hat{V}(t)\exp\qty(-\i\frac{\hat{H}^{(0)}}{\hbar}t)\ket{\psi(t)}_{\r{I}} \\
    &= \hat{V}_{\r{I}}(t)\ket{\psi(t)}_{\r{I}}\label{TDSE-1st-eq-interaction-picture}
  \end{align}
  を得る.\refe{TDSE-1st-eq-interaction-picture}と\refe{TDSE-2nd-eq-interaction-picture}はSch\"odinger表示の非定常摂動基本方程式と等価な方程式であるから,
  これを\textbf{朝永・Schwinger方程式}と呼ぶ.
  \footnote{Schrödinger描像は量子状態が時間発展するとみなす.Heisenberg描像は物理量が時間発展するとみなす.相互作用表示はその中間であるといえる.}  
  \footnote{朝永振一郎(1906-1979)}
  \footnote{Julian Schwinger(1918-1994)}
  \footnote{朝永とSchwingerは1964年にRichard Feynmannとともにノーベル賞を受賞.}.
  \begin{itembox}[l]{朝永・Schwinger方程式}
    \begin{align}
      \i\hbar\dv{t}\ket{\psi(t)}_{\r{I}}&=\hat{V}_{\r{I}}(t)\ket{\psi(t)}_{\r{I}}\label{Tomonaga-Schwinger}\\
      \hat{V}_{\r{I}}(t)&=\exp\qty(\i\frac{\hat{H}^{(0)}}{\hbar}t)\hat{V}(t)\exp\qty(-\i\frac{\hat{H}^{(0)}}{\hbar}t)
    \end{align}
  \end{itembox}
  \par
  \refe{Tomonaga}に左から$\bra{m}$を演算する($\hat{H}^{(0)}\ket{m} = E_m\ket{m}$).\\
  左辺は,
  \begin{align}
    &\bra{m}\i\hbar\dv{t}\sum_{n}c_n(t)\ket{n}\\
    & = \i\hbar\dv{t}c_m(t)
  \end{align}
  となる.
  右辺は.
  \begin{align}
    &\bra{m}\hat{V}_{\r{I}}(t)\sum_{n}c_n(t)\ket{n}\\
    & = \sum_{n}c_n(t)\e^{-\i\frac{(E_n - E_m)t}{\hbar}}\mel{m}{\hat{V}(t)}{n}
  \end{align}
  となる.よって,非定常摂動の時間発展は以下の式を満たす.
  \begin{itembox}[l]{$\hat{H}(t) = \hat{H}^{(0)} + \hat{V}(t)$}
    \begin{align}
      \ket{\psi(t)}_{\r{I}} &= \sum_{n}c_n(t)\ket{n}\\
      i\hbar\dv{t}c_m(t) &= \sum_{n}c_n(t)V_{mn}\e^{\i\omega_{mn}t}\\
      V_{mn} &= \bra{m}\hat{V}(t)\ket{n}\\
      \omega_{mn} &= \frac{E_m - E_n}{\hbar} = -\omega_{nm}
    \end{align}
  \end{itembox}
  これは$c_n$の連立方程式になっており解くことは困難である.よって近似を加える.
\end{document}

      \subsection{$c_n(t)$に関する連立方程式}
        \documentclass{standalone}
\input{../../../head.tex}
\begin{document}
  式(\ref{Tomonaga})に左から$\bra{m}$を演算する($\hat{H}^0\ket{m}=E_m\ket{m}$).\\
  左辺は
  \begin{align}
    &\bra{m}i\hbar\frac{d}{dt}\sum_nc_n(t)\ket{n}\\
    &=i\hbar\frac{d}{dt}c_m(t)
  \end{align}
  右辺は
  \begin{align}
    &\bra{m}\hat{V}_I(t)\sum_{n}c_n(t)\ket{n}\\
    &=\sum_{n}c_n(t)\mathrm{e}^{-i\frac{(E_n-E_m)t}{\hbar}}\bra{m}\hat{V}(t)\ket{n}
  \end{align}
  となる.よって,非定常摂動の時間発展は以下の式を満たす.
  \begin{itembox}[l]{$\hat{H}(t)=\hat{H}^0+\hat{V}(t)$}
    \begin{align}
      \ket{\psi(t)}_I&=\sum_{n}c_n(t)\ket{n}\\
      i\hbar\frac{d}{dt}c_m(t)&=\sum_{n}c_n(t)V_{mn}\mathrm{e}^{i\omega_{mn}t}\\
      V_{mn}&=\bra{m}\hat{V}(t)\ket{n}\\
      \omega_{mn}&=\frac{E_m-E_n}{\hbar}=-\omega_{nm}
    \end{align}
  \end{itembox}
  これは$c_n$の連立方程式になっており解くことは困難である.よって近似を加える.
\end{document}

      \subsection{2準位系}
        \documentclass{report}
\input{../../../head.tex}
\begin{document}
  厳密に解くことのできる2準位系
  \begin{align}
    \ket{1} = \mqty(
      1\\
      0
    ),\ 
    \ket{2} = \mqty(
      0\\
      1
    ),\ 
    \hat{H}^0 = \mqty(
      E_1  &  0\\
      0  &  E_2
    )
  \end{align}
  に摂動を加える($E_1$,$E_2$はそれぞれ$\ket{1}$,$\ket{2}$のエネルギー固有値).摂動は次のようにする\footnote{$\gamma$は摂動の強さを表す.}.
  \begin{align}
    \hat{V}(t) = \mqty(
      0 &  \gamma\e^{\i\omega t}\\
      \gamma \e^{ - \i\omega t} & 0
    )
  \end{align}
  よって,Hamiltonianは,
  \begin{align}
    \hat{H} = \hat{H}^0 + \hat{V}(t) = \mqty(
      E_1 & \gamma\e^{\i\omega t}\\
      \gamma\e^{ - \i\omega t} & E_2
    )
  \end{align}
  である.$t$に依存する非対角項が存在するため$\ket{1}$と$\ket{2}$が$t$に依存して混ざってしまう.\\
  この系の量子状態は相互作用表示を使って,
  \begin{align}
    \ket{\psi(t)}_I = c_1(t)\ket{1} + c_2(t)\ket{2}
  \end{align}
  と表される.$V_{11} = V_{22} = 0,\ V_{12} = V_{21} = \gamma\e^{ - \i\omega t}$であるから,
  \begin{align}
    \begin{cases}
      \i\hbar\frac{d}{dt}c_1 = c_2\gamma\e^{\i(\omega - \omega_{21})t}\\
      \i\hbar\frac{d}{dt}c_2 = c_1\gamma\e^{ - \i(\omega - \omega_{21})t}
    \end{cases}
  \end{align}
  が成り立つ.次に,$\Delta\omega\equiv\omega - \omega_{21}$とし,上の連立微分方程式を解く.\\
  2式目を1式目に代入して$c_1$を消去すると
  \begin{align}
    \ddot{c_2} + \i\Delta\omega\dot{c_2} + \qty(\frac{\gamma}{\hbar})^2c_2 = 0
  \end{align}
  $c_2(t) = \e^{\i\lambda t}$を代入すると
  \begin{align}
    \lambda^2 + \Delta\omega\lambda - \qty(\frac{\gamma}{\hbar}) = 0\\
    \lambda =  - \frac{\Delta\omega}{2}\pm\sqrt{\qty(\Delta\omega/2)^2 + \qty(\gamma/\hbar)^2}
  \end{align}
  を得る.ここで
  \begin{align}
    \Omega = \sqrt{\qty(\Delta\omega/2)^2 + \qty(\gamma/\hbar)^2}
  \end{align}
  を\textbf{Rabi周波数}(Rabi frequency)という\footnote{I.I.Rabi(1898 - 1988)}\footnote{量子状態の振動を\textbf{Rabi振動}(Rabi oscillation)という.}.
  \begin{itembox}[l]{Rabi周波数}
    \begin{align}
    \Omega = \sqrt{\qty(\Delta\omega/2)^2 + \qty(\gamma/\hbar)^2}   
    \end{align}
  \end{itembox}
  よって,$c_2$の一般解は
  \begin{align}
    c_2(t) = \e^{ - \i\Delta\omega t/2}\qty(a\e^{\i\Omega t} + b\e^{ - \i\Omega t})
  \end{align}
  となる.初期条件を$c_1(0) = 1,\ c_2(0) = 0$とすると,
  \begin{align}
    c_2(t) =  - \i\frac{\gamma}{\hbar\Omega}\e^{ - \i\Delta\omega t/2}\sin\Omega t
  \end{align}
  である.この系の量子状態は
  \begin{align}
    \ket{\psi(t)}_I = c_1(t)\ket{1} + c_2(t)\ket{2}
  \end{align}
  だから,時刻$t$で$\ket{2}$に状態を見出す確率は
  \begin{align}
    |c_2(t)|^2 = \frac{(\gamma/\hbar)^2}{\Omega^2}\sin^2\Omega t
  \end{align}
  である.確率が周期的に変化することがわかる.\\
  振幅の大きさは
  \begin{align}
    \frac{(\gamma/\hbar)^2}{(\gamma/\hbar)^2 + (\Delta\omega/2)^2}
  \end{align}
  と表されるので,$\Delta\omega = \omega - \omega_{12} = 0$のときに最大となる.つまり,摂動の周波数$\omega$と
  2準位のエネルギー差に由来する$\omega_{12}$が一致したときに遷移が起こりやすい.
  \begin{itembox}[l]{共鳴条件}
    \begin{align}
      \omega = \omega_{12} = \frac{E_2 - E_1}{\hbar}
    \end{align}  
  \end{itembox}
\end{document}

      \subsection{近似解}
        \documentclass{report}
\input{../../../head.tex}
\begin{document}
  時間に依存する摂動がある量子系の時間発展は次の式で表されることを確認した.
  \begin{equation}
    \begin{cases}
      \label{3.4}
    \ket{\psi(t)}_I=\sum_{n}c_n(t)\ket{n}\\
    i\hbar\frac{d}{dt}c_m(t)=\sum_{n}V_{mn}(t)\mathrm{e}^{i\omega_{mn}t}c_n(t)
    \end{cases}
  \end{equation}
  一般に式(\ref{3.4})を解くことはできない.そこで近似を加える.\\
  $\hat{V}(t)\to\lambda\hat{V}(t)$として$c_n(t)$をべき級数展開する.
  \begin{equation}
    c_n(t)=c_n^0(t)+\lambda c_n^1(t)+\lambda^2c_n^2(t)+\cdots
  \end{equation}
  これを式(\ref{3.4})の第2式に代入して整理すると
  $\lambda^0$の係数比較から
  \begin{equation}
    \label{3.40ji}
    i\hbar\frac{d}{dt}c_m^0(t)=0
  \end{equation}
  $\lambda^1$の係数比較から
  \begin{equation}
    \label{3.41ji}
    i\hbar\frac{d}{dt}c_m^1(t)=\sum_{n}V_{mn}(t)\mathrm{e}^{i\omega_{mn}t}c_n^0(t)
  \end{equation}
  を得る.式(\ref{3.40ji})より,
  \begin{equation}
    \label{3.4const}
    c_m^0(t)=\rm{const.}
  \end{equation}
  である.以下では,$t=t_0$から摂動$\hat{V}(t)$を加え始めたとする.また,$t=t_0$で系の量子状態が$\ket{i}$であったとする.
  このとき
  \begin{equation}
    \begin{cases}
      c_i^0(t_0)=1\\
      c_m^0(t_0)=0\ (m\ne i)
    \end{cases}
  \end{equation}
  である.式(\ref{3.4const})から0次の係数は定数なので
  \begin{equation}
    \begin{cases}
      c_i^0(t)=1\\
      c_m^0(t)=0\ (m\ne i)
    \end{cases}
  \end{equation}
  が得られる.この系の状態は初期状態$\ket{i}$に依ることがわかったのでこれからは$c_m(t)$を$c_{m,i}(t)$と表記する.
  式(\ref{3.41ji})から
  \begin{align}
    i\hbar\frac{d}{dt}c_{m,i}^1(t)&=\sum_{n}V_{mn}(t)\mathrm{e}^{i\omega_{mn}t}c_{n,i}^0(t)\\
    &=V_{m,i}(t)\mathrm{e}^{o\omega_{mi}t}\\
    c_{m,i}^1(t)&=-\frac{i}{\hbar}\int_{t_0}^{t}V_{m,i}(t)\mathrm{e}^{i\omega_{mi}t}dt
  \end{align}
  である.また,系の量子状態は
  \begin{align}
    \ket{\psi(t)}_I&=\sum_{n}c_{n,i}(t)\ket{n}\\
    &\simeq\sum_{n}[c_{n,i}^0(t)+c_{n,i}^1(t)]\ket{n}\\
    &=\ket{i}+c_{i,i}^1(t)\ket{i}+\sum_{n\ne i}c_{n,i}^1(t)\ket{n}
  \end{align}
  と表される.よって
  \begin{itembox}[l]{$\ket{\psi(t_0)}_I=\ket{i}$の時間発展}
    \begin{align}
      \ket{\psi(t)}_I&=(1+c_{i,i}^1(t))\ket{i}+\sum_{n\ne i}c_{n,i}^1(t)\ket{n}\\
      c_{n,i}^1(t)&=-\frac{i}{\hbar}\int_{t_0}^{t}V_{n,i}\mathrm{e}^{i\omega_{ni}t}dt
    \end{align}
  \end{itembox}
  が得られる.このとき,始状態$\ket{i}$から終状態$\ket{f}\ (f\ne i)$への遷移確率は
  \begin{equation}
    |\braket{f}{\psi(t)}_I|^2=|(1+c_{i,i}^1(t))\braket{f}{i}+\sum_{n\ne i}c_{n,i}^1(t)\braket{f}{n}|^2=|c_{f,i}^1(t)|^2
  \end{equation}
  である.
\end{document}

      \subsection{一定の摂動}
        \documentclass{report}
\input{../../../head.tex}
\begin{document}
  時刻$t=0$から摂動$\hat{V}$を加え始めたとする.
  \begin{equation}
    \hat{V}(t)=
    \begin{cases}
      0\ (t\le0)\\
      \hat{V}\ (t>0)
    \end{cases}
  \end{equation}
  このとき
  \begin{equation}
    V_{fi}(t)=\bra{f}\hat{V}(t)\ket{i}=
    \begin{cases}
      0\ (t\le 0)\\
      \bra{f}\hat{V}\ket{i}\ (t>0)
    \end{cases}
  \end{equation}
  である.ここで$\bra{f}\hat{V}\ket{i}=V_{fi}$とする.$t=0$で量子状態は$\ket{i}$であったとする.
  \begin{align}
    c_{f,i}^1(t)&=-\frac{i}{\hbar}\int_{0}^{t}V_{fi}\mathrm{e}^{i\omega_{fi}t}dt\\
    &=2i\exp\qty(-i\frac{\omega_{fi}t}{2})\sin\qty(\frac{\omega_{fi}}{2}t)
  \end{align}
  よって$\ket{i}$から$\ket{f}$への遷移確率は
  \begin{equation}
    |c_{f,i}^1(t)|^2=\frac{|V_{fi}|^2}{\hbar^2}\qty(\frac{\sin\qty(\omega_{fi}t/2)}{\qty(\omega_{fi}/2)})^2
  \end{equation}
  であり,$\sin\qty(\omega_{fi}t/2)$が0でないときのみ$\ket{i}$から$\ket{f}$への遷移が起きることがわかる.また,
  $\qty(\frac{\sin\qty(\omega_{fi}t/2)}{\qty(\omega_{fi}/2)})^2$は$|\omega_{fi}|<2\pi/t$で有効な値をもつ.従って,$t$が小さいときはこの範囲が十分広く,$\omega_{fi}\ne0$の状態への遷移が起こりうる.しかし,
  $t$が大きいときは$\omega_{fi}\simeq0$の状態への遷移しか起きない.
  通常は$t\to\infty$と考えてよい.
  \begin{equation}
    \lim_{t\to\infty}\qty(\frac{\sin\qty(\omega t/2)}{\omega t/2})^2=2\pi t\delta(\omega)
  \end{equation}
  を使うと\footnote{$\delta(ax)=\frac{1}{|a|}\delta(x)$},
  \begin{align}
    \lim_{t\to\infty}|c_{f,i}^1(t)|^2&=\frac{|V_{fi}|^2}{\hbar^2}2\pi t\delta(\omega_{f,i})\\
    &=\frac{2\pi}{\hbar}|V_{fi}|^2\delta(E_f-E_i)t
  \end{align}
  が得られる.
  よって,単位時間当たりの$\ket{i}$から$\ket{f}$への遷移確率は
  \begin{align}
    \omega_{i\to f}& = |c_{f,i}^1(t)|^2/t\\
    &=\frac{2\pi}{\hbar}\qty|\bra{f}\hat{V}\ket{i}|^2\delta(E_f-E_i)
  \end{align}
  である.これを\textbf{Fermiの黄金律}という\footnote{Enrico Fermi(1901-1954)}.
  \begin{itembox}[l]{Fermiの黄金律}
    \begin{equation}
      \omega_{i\to f}=\frac{2\pi}{\hbar}\qty|\bra{f}\hat{V}\ket{i}|^2\delta(E_f-E_i)
    \end{equation}
  \end{itembox}
  \begin{myex}{}{}電子の弾性散乱
    弾性散乱では散乱前後で粒子のエネルギーが保存される.電子のエネルギーは
    \begin{equation}
      E_{\bm{k'}}=\frac{\hbar^2|\bm{k'}|^2}{2m}
    \end{equation}
    と表される.よって,終状態は多く存在する.始状態$i$から終状態グループ$\qty{f}$への遷移確率を求める.
    \begin{align}
      \omega_{i\to\qty{f}}&=\sum_{f}\omega_{i\to f}\\
      &=\sum_{f}\frac{2\pi}{\hbar}\qty|\bra{f}\hat{V}\ket{i}|^2\delta(E_f-E_i)\\
      &=\frac{2\pi}{\hbar}\overline{\qty|\bra{f}\hat{V}\ket{i}|^2}\sum_{f}\delta(E_f-E_i)\\
      &=\frac{2\pi}{\hbar}\overline{\qty|\bra{f}\hat{V}\ket{i}|^2}\rho(E_f)
    \end{align}
    $\overline{\qty|\bra{f}\hat{V}\ket{i}|^2}$は散乱体の性質を,状態密度$\rho(E_f)$は物質の性質を反映している.この関係もFermiの黄金律という.
  \end{myex}
\end{document}

      \subsection{調和摂動}
        \documentclass{standalone}
\input{../../../head.tex}
\begin{document}
  時間に依存する摂動を調和摂動という.
  以下の摂動を考える\footnote{光のイメージ}.
  \begin{equation}
    \hat{V}(t)=
    \begin{cases}
    0\ (t\le0)\\
    2V\cos\omega t\ (t>0)
    \end{cases}
  \end{equation}
  $2V\cos\omega t=V(\mathrm{e}^{i\omega t}+\mathrm{e}^{-i\omega t})$であるので,
  \begin{align}
    c^1_{f,i}(t)&=\frac{i}{\hbar}\int_{0}^{t}\bra{f}\hat{V}\ket{i}(\mathrm{e}^{i\omega t}+\mathrm{e}^{-i\omega t})\mathrm{e}^{i\omega_{fi}t}dt\\
    \label{3.6}
    &=-\frac{V_{fi}}{\hbar}\qty(\frac{\mathrm{e}^{i(\omega_{fi}+\omega)t}-1}{\omega_{fi}+\omega}+\frac{\mathrm{e}^{i(\omega_{fi}-\omega)t}-1}{\omega_{fi}-\omega})
  \end{align}
  を得る.ここで$V_{fi}=\bra{f}\hat{V}\ket{i}$とした.\\
  (i)$\omega_{fi}-\omega\approx0$のとき\\
  式(\ref{3.6})の第2項は第1項より十分大きい.よって,
  \begin{equation}
    c_{f,i}^1(t)\approx-\frac{V_{fi}}{\hbar}\frac{\mathrm{e}^{i(\omega_{fi}-\omega)t}-1}{\omega_{fi}-\omega}=-\frac{V_{fi}}{\hbar}\mathrm{e}^{i\frac{\omega_{fi}-\omega}{2}t}\frac{\sin\frac{\omega_{fi}-\omega}{2}t}{\frac{\omega_{fi}-\omega}{2}}
  \end{equation}
  と近似できる.このとき$\ket{i}\to\ket{f}$の遷移確率は
  \begin{equation}
    |c_{f,i}^1(t)|^2=\frac{V_{fi}^2}{\hbar^2}\frac{\sin^2\frac{\omega_{fi}-\omega}{2}t}{\qty(\frac{\omega_{fi}-\omega}{2})^2}
  \end{equation}
  $\frac{\sin^2\frac{\omega_{fi}-\omega}{2}t}{\qty(\frac{\omega_{fi}-\omega}{2})^2}$は$t\to\infty$で$2\pi t\delta(\omega_{fi}-\omega)=2\pi t\hbar\delta(E_f-E_i-\hbar\omega)$
  と近似できるので
  単位時間当たりの遷移確率は
  \begin{equation}
    \omega_{f\to i}=\frac{|c_{f,i}^1(t)|^2}{t}=\frac{2\pi}{\hbar}|V_{fi}|^2\delta(E_f-E_i-\hbar\omega)
  \end{equation}
  である.これもFermiの黄金律という.$E_f=E_i+\hbar\omega$へと遷移することがわかる.\\
  (ii)$\omega_{fi}+\omega\approx0$のとき
  式(\ref{3.6})の第1項が支配的となる.上記の議論を$\omega\to\omega$と置き換えて繰り返すと単位時間当たりの遷移確率として
  \begin{equation}
    \omega_{i\to f}=\frac{|c_{f,i}^1(t)|^2}{t}=\frac{2\pi}{\hbar}|V_{fi}|^2\delta(E_f-E_i+\hbar\omega)
  \end{equation}
  を得る.$E_f=E_i-\hbar\omega$へと遷移することがわかる.
\end{document}

      \subsection{電磁場中の電子}
        \documentclass{report}
\input{../../head.tex}
\begin{document}
  全角運動量以外の保存量として\textbf{ヘリシティ}がある.
  \begin{align}
    \bm{\bm{S}} = 
    \mqty(
      \bm{\sigma} & 0\\
      0 & \bm{\sigma}
    )
  \end{align}
  として,ヘリシティは,
  \begin{align}
    h \coloneqq \frac{\bm{\bm{S}} \cdot \bm{p}}{\abs{\bm{p}}}
  \end{align}
  と定義される.
  これが保存量であることは,
  \begin{align}
    \qty(\bm{\alpha} \cdot \hat{\bm{p}}) \hat{h} 
    &= \frac{1}{p}\mqty(
      0 & \bm{\sigma} \cdot \hat{\bm{p}}\\
      \bm{\sigma} \cdot \hat{\bm{p}} & 0
    )\mqty(
      \bm{\sigma} \cdot \hat{\bm{p}} & 0\\
      0 & \bm{\sigma} \cdot \hat{\bm{p}}
    ) \\
    &= \frac{1}{p}\mqty(
      0 & (\bm{\sigma} \cdot \hat{\bm{p}})(\bm{\sigma} \cdot \hat{\bm{p}})\\
      (\bm{\sigma} \cdot \hat{\bm{p}})(\bm{\sigma} \cdot \hat{\bm{p}}) & 0
    ) \\
    &= \frac{1}{p}\mqty(
      0 & \hat{\bm{p}}^2\\
      \hat{\bm{p}}^2 & 0
    ) \\
    &= \hat{h}(\bm{\alpha}\cdot \hat{\bm{p}})
  \end{align}
  であるから,
  \begin{align}
    \qty[\bm{\alpha} \cdot \hat{\bm{p}}, \hat{h}] = 0\\
    \qty[\beta, \hat{h}] = 0
  \end{align}
  となることより,
  \begin{align}
    \qty[\hat{H}, \hat{h}] = 0
  \end{align}
  したがって,
  Dirac方程式においてヘリシティは保存量である.
\end{document}
        \subsubsection{大域的Gauge変換}
          \documentclass{report}
\input{../../head.tex}
\begin{document}
  全角運動量以外の保存量として\textbf{ヘリシティ}がある.
  \begin{align}
    \bm{\bm{S}} = 
    \mqty(
      \bm{\sigma} & 0\\
      0 & \bm{\sigma}
    )
  \end{align}
  として,ヘリシティは,
  \begin{align}
    h \coloneqq \frac{\bm{\bm{S}} \cdot \bm{p}}{\abs{\bm{p}}}
  \end{align}
  と定義される.
  これが保存量であることは,
  \begin{align}
    \qty(\bm{\alpha} \cdot \hat{\bm{p}}) \hat{h} 
    &= \frac{1}{p}\mqty(
      0 & \bm{\sigma} \cdot \hat{\bm{p}}\\
      \bm{\sigma} \cdot \hat{\bm{p}} & 0
    )\mqty(
      \bm{\sigma} \cdot \hat{\bm{p}} & 0\\
      0 & \bm{\sigma} \cdot \hat{\bm{p}}
    ) \\
    &= \frac{1}{p}\mqty(
      0 & (\bm{\sigma} \cdot \hat{\bm{p}})(\bm{\sigma} \cdot \hat{\bm{p}})\\
      (\bm{\sigma} \cdot \hat{\bm{p}})(\bm{\sigma} \cdot \hat{\bm{p}}) & 0
    ) \\
    &= \frac{1}{p}\mqty(
      0 & \hat{\bm{p}}^2\\
      \hat{\bm{p}}^2 & 0
    ) \\
    &= \hat{h}(\bm{\alpha}\cdot \hat{\bm{p}})
  \end{align}
  であるから,
  \begin{align}
    \qty[\bm{\alpha} \cdot \hat{\bm{p}}, \hat{h}] = 0\\
    \qty[\beta, \hat{h}] = 0
  \end{align}
  となることより,
  \begin{align}
    \qty[\hat{H}, \hat{h}] = 0
  \end{align}
  したがって,
  Dirac方程式においてヘリシティは保存量である.
\end{document}
        \subsubsection{局所的Gauge変換}
          \documentclass{report}
\input{../../../head.tex}
\begin{document}
  波動関数$\psi(\bm{r})$を次のように変換する.
  \begin{equation}
    \psi(\bm{r})\to\psi'(\bm{r})=\mathrm{e}^{i\alpha(\bm{r})}\psi(\bm{r})
  \end{equation}
  この場合,運動量がGauge不変でなくなってしまう.
  例えば波動関数の微分を計算すると
  \begin{equation}
    \nabla\psi'(\bm{r})=i(\nabla\alpha(\bm{r})\mathrm{e}^{i\alpha(\bm{r})})\psi(\bm{r})+\mathrm{e}^{i\alpha(\bm{r})}\nabla\psi(\bm{r})=\mathrm{e}^{i\alpha(\bm(r))}(\nabla+i\nabla\alpha(\bm{r}))\psi(\bm{r}) 
  \end{equation}
  余分な項が加わってしまう.よって,
  \begin{equation}
    \begin{cases}
      i\hbar\frac{\partial}{\partial t}\psi'\ne -\frac{\hbar^2}{2m}\nabla^2\psi'\\
      \bra{\psi}\hat{A}\ket{\psi}\ne\bra{\psi'}\hat{A}\ket{\psi'}
    \end{cases}
  \end{equation}
  である.したがって,局所Gauge変換に対して物理は不変ではない.\\
  \textbf{局所Gauge不変性を基本原理とする物理を再構築する.}
\end{document}

  \chapter{散乱理論}
    \section{立体角}
      \documentclass{report}
\input{../../head.tex}
\begin{document}
  全角運動量以外の保存量として\textbf{ヘリシティ}がある.
  \begin{align}
    \bm{\bm{S}} = 
    \mqty(
      \bm{\sigma} & 0\\
      0 & \bm{\sigma}
    )
  \end{align}
  として,ヘリシティは,
  \begin{align}
    h \coloneqq \frac{\bm{\bm{S}} \cdot \bm{p}}{\abs{\bm{p}}}
  \end{align}
  と定義される.
  これが保存量であることは,
  \begin{align}
    \qty(\bm{\alpha} \cdot \hat{\bm{p}}) \hat{h} 
    &= \frac{1}{p}\mqty(
      0 & \bm{\sigma} \cdot \hat{\bm{p}}\\
      \bm{\sigma} \cdot \hat{\bm{p}} & 0
    )\mqty(
      \bm{\sigma} \cdot \hat{\bm{p}} & 0\\
      0 & \bm{\sigma} \cdot \hat{\bm{p}}
    ) \\
    &= \frac{1}{p}\mqty(
      0 & (\bm{\sigma} \cdot \hat{\bm{p}})(\bm{\sigma} \cdot \hat{\bm{p}})\\
      (\bm{\sigma} \cdot \hat{\bm{p}})(\bm{\sigma} \cdot \hat{\bm{p}}) & 0
    ) \\
    &= \frac{1}{p}\mqty(
      0 & \hat{\bm{p}}^2\\
      \hat{\bm{p}}^2 & 0
    ) \\
    &= \hat{h}(\bm{\alpha}\cdot \hat{\bm{p}})
  \end{align}
  であるから,
  \begin{align}
    \qty[\bm{\alpha} \cdot \hat{\bm{p}}, \hat{h}] = 0\\
    \qty[\beta, \hat{h}] = 0
  \end{align}
  となることより,
  \begin{align}
    \qty[\hat{H}, \hat{h}] = 0
  \end{align}
  したがって,
  Dirac方程式においてヘリシティは保存量である.
\end{document}
  \chapter{相対論的量子論}
\end{document}
