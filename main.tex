\documentclass{report}
\input{head.tex}
\begin{document}
  \documentclass{report}
\input{../head.tex}
\begin{document}
  \maketitle
  \begin{abstract}
    物理情報工学科2024年度秋前半「応用量子物性」(担当:安藤和也先生)の講義ノート(勝手に)である.2秋「量子力学入門」,3春「量子力学」を履修済みであることが望ましい?
  \end{abstract}
  \tableofcontents
  \thispagestyle{empty}
\end{document}
  \maketitle
  \tableofcontents
  \chapter{近似法}
    \section{変分法}
      \documentclass{report}
\input{../../../head.tex}
\begin{document}
  \begin{align}
    \hat{H}\ket{k} = E_k\ket{k}\label{SE}
  \end{align}
  変分法(variational principle)とはHamiltonianの基底エネルギー$E_0$の近似法である
  \footnote{
    近似法には摂動法と変分法がある.
    摂動法はHamiltonianが厳密に解ける項$\hat{H}^0$と摂動項$\hat{\delta}$を用いて,$\hat{H}=\hat{H}^0+\hat{\delta}$と表され,摂動項が小さいときのみ有効である.
    これに対し,変分法はどんなときでも有効である.
  }.
\end{document}

      \subsection{基本原理}
        \documentclass{standalone}
\input{../../../head.tex}
\begin{document}
  \begin{myprop}{変分法の基本原理}{}
    任意の状態ベクトル$\ket{\psi}$に対して以下の不等式が成り立つ.
    \begin{align}
      E(\psi) = \cfrac{\bra{\psi}\hat{H}\ket{\psi}}{\bra{\psi}\ket{\psi}} \ge E_0 \label{ne}
    \end{align}
    \tcblower
    \begin{proof}
      任意の状態ベクトル$\ket{\psi}$を
      \begin{align}
        \ket{\psi} = \sum_{k} c_k\ket{k} \label{sv}
      \end{align}
      と展開する.左から$\bra{k'}$を作用させると
      \begin{align}
        \bra{k'}\ket{\psi} = \sum_{k}c_k\bra{k'}\ket{k} = \sum_{k}c_k\delta_{k',k} = c_{k'}
      \end{align}
      を得る.これは任意の$k$に対して成り立つので\refe{sv}は以下のように変形できる.
      \begin{align}
        \ket{\psi}&=\sum_{k}\bra{k}\ket{\psi}\ket{k}\\
        &=\sum_{k}\ket{k}\bra{k}\ket{\psi}
      \end{align}
      これを用いて\refe{ne}の分母を以下のように変形する.
      \begin{align}
        \bra{\psi}\hat{H}\ket{\psi}&=\bra{\psi}\hat{H}\sum_{k}\ket{k}\bra{k}\ket{\psi}\\
        &=\sum_{k}\bra{\psi}\hat{H}\ket{k}\bra{k}\ket{\psi}\\
        &=\sum_{k}E_k\bra{\psi}\ket{k}\bra{k}\ket{\psi}\\
        &=\sum_{k}E_k \abs{\bra{k}\ket{\psi}}^2
      \end{align}
      また,
      \begin{align}
        \braket{\psi}{\psi}=\sum_{k}\abs{\braket{k}{\psi}}^2
      \end{align}
      であるので,
      \begin{align}
        E(\psi) &= \cfrac{\bra{\psi}\hat{H}\ket{\psi}}{\braket{\psi}{\psi} } \\ 
        &= \cfrac{\sum_{k}E_k\abs{\braket{k}{\psi}}^2}{\sum_{k}\abs{\braket{k}{\psi}}^2} \\ 
        &\geq \cfrac{\sum_{k}E_0\abs{\braket{k}{\psi}}^2}{\sum_{k}\abs{\braket{k}{\psi}}^2} = E_0
      \end{align}
      が示される.
    \end{proof}
  \end{myprop}
  \refe{ne}よりあらゆる状態ベクトル$\ket{\psi}$のエネルギーは基底エネルギー$E_0$以上である.
  変分法は,
  \begin{screen}
    \begin{enumerate}
      \item \textbf{試行関数}$\ket{\psi}$をたくさん用意し,
      \item それぞれのエネルギー$E(\psi)$を計算し,
      \item その中で最小の$E(\psi)$を$E_0$の近似解とする
    \end{enumerate}
  \end{screen}
  近似法である.
  \begin{myex}{}{}
    ポテンシャル$V(x)=\lambda x^4$中に粒子がある系を考える.
    この系のHamiltonianは
    \begin{align}
      \hat{H} = -\cfrac{\hbar^2}{2m}\cfrac{d^2}{dx^2} + \lambda x^4
    \end{align}
    である.予想される基底状態が満たすべき条件は
    \begin{itemize}
      \item $x = 0$で存在確率が最大
      \item $\abs{x} \to \infty$で存在確率が0
      \item 節がない\footnote{節があると微係数が大きい点が存在し,これは運動エネルギーを大きくしてしまう.}
    \end{itemize}
    である.この条件と変分法を用いて,エネルギーの近似値を求めよ.
    \tcblower
    試行関数として$\psi(x,\alpha) = \e^{-\tfrac{\alpha x^2}{2}}$,$\alpha > 0$を考える.
    \refe{ne}の右辺を計算すると,
    \begin{align}
      E(\alpha) &= \cfrac{\int \psi^{*}\hat{H}\psi \dd{x}}{\int \psi^{*}\psi \dd{x}} \label{E1} \\ 
      &= \cfrac{\hbar^2 \alpha}{4m} + \cfrac{3\lambda}{4\alpha^2}
    \end{align}
    を得る.
    第1項は運動エネルギーを,第2項はポテンシャルエネルギーを,それぞれ表している\footnote{
      ポテンシャルエネルギーの項は$\alpha$が大きくなるほど小さくなる.
      これは,波動関数が狭まり$x=0$での存在確率が大きくなるためである.
      一方,運動エネルギーの項は$\alpha$が大きくなるほど大きくなる.
      これは,不確定性関係$\Delta x\Delta p \ge \cfrac{\hbar}{2}$より,運動量のばらつきが大きくなるためである.
    }.
    \refe{E1}の最小値が基底エネルギー$E_0$の近似解である.よって,$\dv{\alpha}E(\alpha_0) = 0$となる$\alpha_0$を\refe{E1}に代入することで近似解,
    \begin{align}
      E(\alpha_0) = \cfrac{3}{8}\qty(\cfrac{6\hbar^4\lambda}{m^2})^{1/3}
    \end{align}
    を得る.
  \end{myex}
\end{document}
      \subsection{(変分法の例題)ヘリウム原子}
        \documentclass{report}
\input{../../../head.tex}
\begin{document}
  本節では,変分法の威力を確認するために,ヘリウム原子の基底エネルギーを考える.
  ヘリウム原子において,$m/M \to 0$であり,原子核が動かない(原子核の運動エネルギーが無視できる)とする.
  これをBorn-Oppenheimer近似という.
  ヘリウム原子は電荷$2e$の原子核と電荷$-e$の電子を2つもつので,ハミルトニアンは,
  \begin{align}
    \hat{H} = -\cfrac{\hbar^2}{2m}\grad_{1}^{2} -\cfrac{\hbar^2}{2m}\grad_{2}^{2} - \cfrac{2e^2}{4\pi \epsilon_0 r_1} - \cfrac{2e^2}{4\pi \epsilon_0 r_2} + \cfrac{e^2}{4\pi \epsilon_0 r_{12}}
  \end{align}
  である.
  第1項から第4項は水素陽原子のハミルトニアン $\hat{H}^0$であり厳密に解くことが出来ることを利用して,第5項を無視して考えたときと,試行関数を定めて変分法を用いたときを比較する.
  なお,実験によりヘリウム原子の基底エネルギーは$-78.6\ \r{eV}$と求まっている.
  \begin{figure}[htbp]
    \centering
    \includegraphics[width=0.5\columnwidth]{fig/helium.pdf}
    \caption{ヘリウム原子の構造}\label{helium-atom}
  \end{figure}
  \begin{myex}{ヘリウム原子の基底エネルギー(荒い近似)}{rough-helium}
    計算を行うと,ヘリウムの原子番号を$Z$として$\hat{H}^0$の基底波動関数,
    \begin{align}
      \psi = \cfrac{Z^3}{\pi a_0^3}\exp\qty(-Z\frac{r_1 + r_2}{a_0})\label{helium}
    \end{align}
    と$\hat{H}^0$の基底エネルギー,
    \begin{align}
      E = -8\ \r{Ry}\approx -108.8\ \r{eV}\label{helium-rough-min}
    \end{align}
    が求まる\footnote{
      $a_0 = \cfrac{4\pi\epsilon_0\hbar^2}{me^2} \approx 5.29\times 10^{-11}\ \r{m}$: Bohr半径
    }\footnote{
      $Z = 2$
    }\footnote{
      $\r{Ry} = \cfrac{\hbar^2}{2m\omega^2} \approx 13.6 \ \r{eV}$: Rydberg定数
    }.
  \end{myex}
  \begin{myex}{ヘリウム原子の基底エネルギー(変分法)}{}
    \exref{rough-helium}の結果とヘリウム原子の基底エネルギーの測定結果は$-78.6\ \r{eV}$と大きく異なっているため,相互作用の項を取り入れた近似を考える.
    \refe{helium}を試行関数$\psi(Z)$とする.
    $\psi(Z)$を用いてエネルギーを計算する.
    \begin{align}
      E(Z) &= \cfrac{\displaystyle\int\psi^{*}\hat{H}\psi \dd{\bm{r}_1}\dd{\bm{r}_2}}{\displaystyle\int\psi^{*}\psi \dd{\bm{r}_1}\dd{\bm{r}_2}} \\
      &= -2\qty(4Z - Z^2 - \cfrac{5}{8}Z)\ \r{Ry}\label{helium-energy-function} 
    \end{align}
    となる.\refe{helium-energy-function}が最小となるような$Z$を$Z_0$とすると$Z_0 = 27/16$であったので,
    \begin{align}
      E(Z) \geq E\qty(Z_0) = -77.5\ \r{eV}\label{helium-energy-varidation-min}
    \end{align}
    となった.\refe{helium-energy-varidation-min}と\refe{helium-rough-min}を比べると,変分法による近似の方が真の基底エネルギ$-78.6\ \r{eV}$に近い値が得られた\footnote{
      $Z_0 < 2$は遮蔽効果により有効電荷が$2e$より小さくなったことを意味する.
    }\footnote{
      積分の計算はDavid J. Griffith, \textit{Introduction to Quantum Mechanics}, pp. 333-334にある.
    }.
  \end{myex}
\end{document}
      \subsection{変分法の誤差の評価}
        \documentclass{report}
\input{../../../head.tex}
\begin{document}
  真の基底状態$\ket{E_0}$に第1励起状態$\ket{E_1}$を10\%含んだ試行関数$\ket{\psi} = \ket{E_0} + \cfrac{1}{10}\ket{E_1}$を使ってエネルギーを計算する.
  \begin{align}
    E(\psi) &= \frac{\mel**{\psi}{\hat{H}}{\psi}}{\braket{\psi}} \\
    &= \cfrac{\mel**{E_0}{\hat{H}}{E_0} + \cfrac{1}{100}\me**l{E_1}{\hat{H}}{E_1}}{1 + \cfrac{1}{100}} \\
    &= \cfrac{E_0 + 0.01E_1}{1.01}\\
    &\approx 0.99E_0 + 0.01E_1
  \end{align}
  試行関数で10\%含まれていた誤差がエネルギーでは1\%に収まっている.
  \begin{myex}{}{}
    無限井戸型ポテンシャル$[-a, a]$を考える.
    この問題を厳密に解けば$n$番目のエネルギー準位は,
    \begin{align}
      E_n = \cfrac{\hbar^2}{2m}\qty(\cfrac{n\pi}{2a})^2
    \end{align}
    と計算できるが,ここでは変分法を用いて近似解を求める.
    予想される試行関数の条件は
    \begin{itemize}
      \item $\psi(a) = \psi(-a) = 0$
      \item 節がない
    \end{itemize}
    である.よって今回は
    \begin{align}
      \psi(x) = a^2 - x^2
    \end{align}
    を採用する.この試行関数を用いたときの基底エネルギーを見積もれ.
    \tcblower
    \begin{align}
      E(\psi) &= \cfrac{\displaystyle \int_{-a}^{a}\qty(a^2 - x^2)\qty(-\frac{\hbar^2}{2m}\dv[2]{x})\qty(a^2 - x^2)\dd{x}}{\displaystyle \int_{-a}^{a}\qty(a^2 - x^2)^2\dd{x}} \\
      &= \cfrac{10}{\pi^2}E_1 \\
      &\approx 1.01E_1
    \end{align}
    真の基底エネルギー$E_1$に近い値が得られた\footnote{このくらいの計算が期末試験に出たことがある.}.
  \end{myex}
\end{document}
      \subsection{練習問題}
        \documentclass{report}
\input{../../../head.tex}
\begin{document}
  \begin{myexc}{Griffith Example 8.1}{}
    1次元調和振動子$\hat{H} = -\cfrac{\hbar^2}{2m}\dv[2]{x} + \cfrac{1}{2}m\omega^2 x^2$の基底エネルギーを見積もれ.
    ただし,試行関数を$\psi(x) = \qty(\cfrac{2b}{\pi})^{1/4}\e^{-bx^2}$とせよ.試行関数は規格化されている.
    \tcblower
    \begin{align}
      E(b) = \bra{\psi}\hat{H}\ket{\psi} &= \qty(\cfrac{2b}{\pi})^{1/2}\int_{-\infty}^{\infty}\e^{-bx^2}\qty(-\cfrac{\hbar^2}{2m}\dv[2]{x} + \cfrac{1}{2}m\omega^2x^2)\e^{-bx^2}\dd{x}\\
      &=\cfrac{\hbar^2b}{2m}+\cfrac{m\omega^2}{8b}
    \end{align}
    次に$E(b)$の最小値を求める.
    \begin{align}
      \dv{b}E(b_0) = \cfrac{\hbar^2}{2m} - \cfrac{m\omega^2}{8b_{0}^2} = 0\Rightarrow b_0 = \cfrac{m\omega}{2\hbar}
    \end{align}
    \begin{align}
      E(b_0) = \cfrac{1}{2}\hbar\omega
    \end{align}
    偶然にも試行関数は基底エネルギーの固有関数となっていたため,$E(b_0)$は基底エネルギーと一致した.
  \end{myexc}
  \begin{myexc}{Griffith Example 8.2}{}
    デルタ関数型ポテンシャル$\hat{H} = -\frac{\hbar^2}{2m}\dv[2]{x}-\alpha \delta(x)$の基底エネルギーを見積もれ.
    ただし,試行関数を$\psi(x) = \qty(\cfrac{2b}{\pi})^{1/4}\e^{-bx^2}$とせよ.試行関数は規格化されている.
    \tcblower
    \begin{align}
      \ev{V} &= -\alpha \qty(\cfrac{2b}{\pi})^{1/2}\int_{-\infty}^{\infty}\e^{-2bx^2}\delta(x)\dd{x} = -\alpha\qty(\cfrac{2b}{\pi})^{1/2}\\
      \ev{T} &= \cfrac{\hbar^2 b}{2m}\\
      E(b) &= \cfrac{\hbar^2 b}{2m} - \alpha\qty(\cfrac{2b}{\pi})^{1/2}
    \end{align}
    $E(b)$の最小値を求める.
    \begin{align}
      \dv{b}E(b_0) = \cfrac{\hbar^2}{2m} - \cfrac{\alpha}{\sqrt{2\pi b_0}} = 0\Rightarrow b_0=\cfrac{2m^2\alpha^2}{\pi\hbar^4}
    \end{align}
    よって,基底エネルギーの近似解として
    \begin{align}
      E(b_0) = -\cfrac{m\alpha^2}{\pi\hbar^2}
    \end{align}
    を得る\footnote{
      厳密解を求めることができ,$\psi(x) = \cfrac{\sqrt{m\alpha}}{\hbar}\e^{-m\alpha\abs{x}/\hbar^2},\ E_0 = -\cfrac{m\alpha^2}{2\hbar^2}$である.
    }.
  \end{myexc}
  \begin{myexc}{Griffith Example 8.3}{}
    $[0, a]$の無限井戸型ポテンシャルの基底エネルギーを見積もれ.ただし,試行関数を
    \begin{align}
      \psi(x)=
      \begin{dcases*}
        Ax & if $0 \leq x \leq a/2$ \\
        A(a - x) & if $a/2\leq x \leq a$ \\ 
        0 & otherwise 
      \end{dcases*}
    \end{align}
    とせよ.
    \tcblower
    規格化条件より,$A = \cfrac{2}{a}\sqrt{\cfrac{3}{a}}$を得る.波動関数の導関数は
    \begin{align}
      \dv[2]{\psi}{x} =
        \begin{dcases*}
          Ax & if $0 \leq x \leq a/2$ \\
          A(a - x) & if $a/2\leq x \leq a$ \\ 
          0 & otherwise 
        \end{dcases*}
    \end{align}
    である.よって,2次の微係数として
    \begin{align}
      \dv[2]{\psi}{x} = A\delta(x) - 2A\delta\qty(x - \cfrac{a}{2}) + A\delta(x - a)
    \end{align}
    を得る.したがって近似解は
    \begin{align}
      E &= \int_{0}^{a}\psi(x)\qty(-\cfrac{\hbar^2}{2m}\dv[2]{x})\psi(x)\dd{x}\\
      &= -\cfrac{\hbar^2}{2m}\int_{0}^{a} A\qty[\delta(x)-\delta\qty(x-\cfrac{a}{2}) + \delta(x - a)]\psi(x)\dd{x}\\
      &= \cfrac{12\hbar^2}{2ma^2}
    \end{align}
    である\footnote{
      厳密解は$E_0 = \cfrac{\pi^2\hbar^2}{2ma^2}$
    }.
  \end{myexc}
  \begin{myexc}{Griffith Problem8.4 (a)}{}
    試行関数$\ket{\psi}$が基底状態と直交するとき,つまり$\braket{\psi}{0}$のとき,
    \begin{align}
      E(\psi)\geq E_1
    \end{align}
    であることを示せ\footnote{
      例えば偶関数のポテンシャルに対し奇関数の試行関数で計算すれば第1励起状態のエネルギーの近似解が得られる.
    }.
    ただし$E_1$は第1励起状態のエネルギーである.$\ket{\psi}$は規格化されている.
    \tcblower
    \begin{proof}
      \begin{align}
        E(\psi) &= \sum_{k = 0}E_k\abs{\braket{\psi}{k}}^2 \\
        &= E_0\abs{\braket{\psi}{0}}^2 + \sum_{k = 1}\abs{\braket{\psi}{k}}^2 \\
        &= 0 + \sum_{k = 1}\abs{\braket{\psi}{k}}^2 \\
        &\geq E_1\sum_{k = 1}\abs{\braket{\psi}{k}}^2 = E_1
      \end{align}
    \end{proof}
  \end{myexc}
\end{document}

    \section{摂動I(定常摂動)}
      \documentclass{report}
\input{../../../head.tex}
\begin{document}
  ハミルトニアンが時間に依存しない定常摂動(time-independent perturbation)を扱う.
\end{document}

      \subsection{準備}
        \documentclass{standalone}
\input{../../../head.tex}
\begin{document}
  次の式は厳密に解くことができるとする.
  \begin{equation}
    \hat{H}^{(0)}\ket{n^{(0)}}=E_0^{(0)}\ket{n^{(0)}}
  \end{equation}
  ここに摂動$\hat{V}$を加える\footnote{摂動の例: 光,電場}.
  \begin{equation}
    \qty(\hat{H}^{(0)}+\hat{V})\ket{n}=E_n\ket{n}
  \end{equation}
  $\hat{H}=\hat{H}^{(0)}+\lambda\hat{V}$とする.$\lambda\to0$ならば,
  \begin{equation}
    \begin{cases}
      \ket{n}\to\ket{n^{(0)}}\\
      E_n\to E_n^{(0)}
    \end{cases}
  \end{equation}
  である.
  ここで,$\hat{H}=\hat{H}^{(0)}+\lambda\hat{V}$の解を次のようにおく.\footnote{$\hat{H},n,E$の肩の()を今後は省略する.}
  \begin{equation}
    \begin{cases}
      \ket{n}=\ket{n^0}+\lambda\ket{n^1}+\lambda^2\ket{n^2}+\cdots\\
      E_n=E_n^0+\lambda+E_n^1+\lambda^2+E_n^2+\cdots
    \end{cases}
  \end{equation}
  $\ket{n^1},\ket{n^2},E_n^1,E_n^2$を求める.
  ここで,規格化条件として
  \begin{equation}
    \braket{n^0}{n}=1
  \end{equation}
  を定める.
  以上の$\hat{H},\ket{n},E_n$を用いて,Schrödinger方程式を立て,整理すると,
  $\lambda^0$,$\lambda^1$,$\lambda^2$の係数としてそれぞれ
  \begin{align}
    &(E_n^0-\hat{H}^0)\ket{n^0}=0\\
    \label{1st}
    &(E_n^0-\hat{H}^0)\ket{n^1}+E_n^1\ket{n^0}=\hat{V}\ket{n^0}\\
    \label{2nd}
    &(E_n^0-\hat{H}^0)\ket{n^2}+E_n^1\ket{n^1}+E_n^2\ket{n^0}=\hat{V}\ket{n^1}\\
  \end{align}
  を得る.
\end{document}

      \subsection{1次摂動}
        \documentclass{report}
\input{../../../head.tex}
\begin{document}
  まずエネルギー補正$E_n^{(1)}$について考える.
  \refe{1st-perturbation}の両辺に$\bra{n^{(0)}}$を作用すると,
  \begin{align}
    \mel**{n^{(0)}}{\qty(E_n^{(0)} - \hat{H}^{(0)})}{n^{(1)}} + \mel**{n^{(1)}}{E_n^{(1)}}{n^{(0)}} &= \mel{n^{(0)}}{\hat{V}}{n^{(0)}} \\ 
    \Leftrightarrow E_n^{(0)}\braket{n^{(0)}}{n^{(1)}} - E_n^{(0)}\braket{n^{(0)}}{n^{(1)}} + E_n^{(1)}\braket{n^{(1)}}{n^{(0)}} &= \mel{n^{(0)}}{\hat{V}}{n^{(0)}} \\ 
    \Leftrightarrow 0 + E_n^{(1)}\braket{n^{(0)}} &= \ev**{\hat{V}}{n^{(0)}} \\ 
    \Leftrightarrow E_n^{(1)} &= \ev**{\hat{V}}{n^{(0)}}
  \end{align}
  を得る.よって,1次摂動によるエネルギー補正は,
  \begin{itembox}[l]{1次摂動によるエネルギー補正}
    \begin{align}
      E_n^{(1)} = \ev**{\hat{V}}{n^{(0)}}
    \end{align}
  \end{itembox}
  である.
  \par
  次に固有ベクトル$\ket{n^{(1)}}$の補正を求める.
  \refe{1st-perturbation}の両辺に$\bra{m^{(0)}}$を左から作用すると,
  \begin{align}
    \mel**{m^{(0)}}{\qty(E_n^{(0)} - \hat{H}^{(0)})}{n^{(1)}} + \mel**{m^{(0)}}{E_n^{(1)}}{n^{(0)}} &= \mel**{m^{(0)}}{\hat{V}}{n^{(0)}} \\ 
    E_n^{(0)}\braket{m^{(0)}}{n^{(1)}} - E_m^{(0)}\braket{m^{(0)}}{n^{(1)}} + 0 &= \mel**{m^{(0)}}{\hat{V}}{n^{(0)}}\\
    \qty(E_n^{(0)} - E_m^{(0)})\braket{m^{(0)}}{n^{(1)}} &= \mel**{m^{(0)}}{\hat{V}}{n^{(0)}} \label{1st-perturbation-prep}
  \end{align}
  となる.\refe{1st-perturbation-prep}に$\bra{m^{(0)}}$をかけて,$m$に関して和を取れば,
  \begin{align}
    \sum_{m}\qty(E_n^{(0)} - E_m^{(0)})\ket{m^{(0)}}\braket{m^{(0)}}{n^{(0)}} &= \sum_{m}\mel**{m^{(0)}}{\hat{V}}{n^{(0)}}\ket{m^{(0)}} \\ 
    \Leftrightarrow \sum_{m}\qty(E_n^{(0)} - E_m^{(0)})\ket{n^{(0)}} &= \sum_{m}\mel**{m^{(0)}}{\hat{V}}{n^{(0)}}\ket{m^{(0)}} \\ 
    \Leftrightarrow \ket{n^{(0)}}\sum_{m \neq n}\qty(E_n^{(0)} - E_m^{(0)}) &= \sum_{m \neq n}\mel**{m^{(0)}}{\hat{V}}{n^{(0)}}\ket{m^{(0)}} \\
    \Leftrightarrow \ket{n^{(0)}} &= \sum_{m\neq n}\cfrac{\mel**{m^{(0)}}{\hat{V}}{n^{(0)}}}{E_n^{(0)} - E_m^{(0)}}\ket{m^{(0)}}
  \end{align}
  となる.
  途中の式変形でエネルギー縮退がないので,
  \begin{align}
    E_n^{(0)} - E_m^{(0)} 
    \begin{dcases}
      = 0 & n = m \label{no-degeneracy} \\ 
      \neq 0 & n \neq m
    \end{dcases}
  \end{align}
  とした.
  また,Hermite演算子である$\hat{H}^{(0)}$の固有ベクトルに関する完全性より,
  \begin{align}
    I = \sum_{m}\ketbra{m^{(0)}}{m^{(0)}}\label{completeness}
  \end{align}
  を用いた.
  \begin{itembox}[l]{1次摂動による固有ベクトル補正}
    \begin{align}
      \ket{n^{(1)}} = \sum_{m\neq n}\cfrac{\mel**{m^{(0)}}{\hat{V}}{n^{(0)}}}{E_n^{(0)} - E_m^{(0)}}\ket{m^{(0)}}\label{1st-order-eigenvector}
    \end{align}
  \end{itembox}
  を得る.
  \refe{1st-order-eigenvector}において,$\ket{n^{(1)}}$と$\ket{n^{(0)}}$は直交することに注意する.
\end{document}

      \subsection{(1次摂動の例題)ヘリウム原子}
        \documentclass{report}
\input{../../../head.tex}
\begin{document}
  \begin{myex}{ヘリウム原子の基底エネルギー}{}
    \begin{align}
      \hat{H} = \hat{H}^0 + \frac{e^2}{4\pi\epsilon_0r_{12}}\equiv\hat{H}^0 + \hat{V}
    \end{align}
    $\hat{H}^0$の基底エネルギーは
    \begin{align}
      \psi^0 = \frac{Z^3}{\pi a_0^3}\e^{-Z(r_1+r_2)/a_0}
    \end{align}
    である.よって,$\hat{V}$による1次のエネルギー補正は以下のように計算できる.
    \begin{align}
      E^1 &= \bra{\psi^0}\hat{V}\ket{\psi^0}\\
      &= \int{\psi^0}^{*}\frac{e^2}{4\pi\epsilon_0r_{12}}\psi^0\dd{\bm{r}_1}\dd{\bm{r}_2} \\
      &= \frac{5}{4}Z\ \mathrm{Ry}
    \end{align}
    よって,基底エネルギー
    \begin{align}
      E_0&=E^0+E^1\\
      &= -8\ \mathrm{Ry}+\frac{5}{4}\times{2}\ \mathrm{Ry}\\
      &= -74.8\ \mathrm{eV}
    \end{align}
    を得る\footnote{測定値は$-78.6\ \mathrm{eV}$}.
  \end{myex}
\end{document}

      \subsection{2次摂動}
        \documentclass{report}
\input{../../../head.tex}
\begin{document}
  式(\ref{2nd})の両辺に$\bra{n^0}$を作用することで,2次摂動によるエネルギー補正を得る.
  \begin{align}
    0+0&+\bra{n^0}E_n^2\ket{n^0}=\bra{n^0}\hat{V}\ket{n^1}\\
    E_n^2&=\bra{n^0}\hat{V}\ket{n^1}
  \end{align}
  \begin{itembox}[l]{2次摂動によるエネルギー補正}
  \begin{equation}
    E_n^2=\sum_{m\ne n}\frac{\lvert \bra{m^0}\hat{V}\ket{n^0}\rvert^2}{E_n^0-E_m^0}
  \end{equation}
  \end{itembox}
  また,基底状態においては$E_{n=0}^0 < E_m^0$である.常に$\frac{\lvert \bra{m^0}\hat{V}\ket{n^0}\rvert^2}{E_n^0-E_m^0}$の
  分母は負であるため,\textbf{基底状態のエネルギーは2次摂動により必ず下がる}.
  \begin{myex}{Mott insulator\footnote{
      Nevill Francis Mott (1905-1996)
    }\footnote{
      バンドギャップが大きくギャップ内にフェルミ準位があるバンド絶縁体と異なり,運動エネルギーが小さくCoulomb力が大きいため電子が移動できない絶縁体である.
    }\footnote{
      $R\mathrm{NiO_3,TaS_2,Sr_2IrO_4}$など
    }}{}
    Coulomb力が強い4つのサイトに電子を4つ入れる.
    $\uparrow \ \uparrow\ \uparrow\ \uparrow$と$\uparrow\ \downarrow\ \uparrow\ \downarrow$のどちらが基底状態としてふさわしいだろうか.
    サイト間の電子の飛び移りを摂動として扱う.
    ここで重要なのは,基底状態のエネルギーは2次摂動により必ず下がるということである.
    $\uparrow \ \uparrow\ \uparrow\ \uparrow$に摂動を加えたとしてもPauliの排他律により電子の飛び移りは起こらない.
    摂動によってエネルギーは変化しない.しかし,$\uparrow\ \downarrow\ \uparrow\ \downarrow$は電子が反平行であるため電子のサイト間での飛び移りが許される.
    これは,2次摂動によるエネルギーの低下を引き起こす.
    よって,$\uparrow\ \downarrow\ \uparrow\ \downarrow$の方が基底状態としてふさわしい\footnote{
      Mott insulatorは反強磁絶縁体である.
    }.
  \end{myex}
\end{document}

      \subsection{(2次摂動の例題)量子閉じ込めStark効果}
        \documentclass{report}
\input{../../../head.tex}
\begin{document}
  ここでは,2次摂動を用いた例題としてStark効果\footnote{Johanes Stark(1874-1957)}\footnote{電場によるエネルギー準位の変化をStark効果という.}を考えよう.
  \begin{myex}{量子閉じ込めStark効果}{stark}
    定常状態ののハミルトニアン$\hat{H}^{(0)}$に,電場による摂動$\hat{V}$を加えたハミルトニアン$\hat{H}$を考える.
    ただし,定常状態のポテンシャルは,長さ$L$の無限井戸型ポテンシャル$\hat{U}$である.
    $\hat{U}$,$\hat{V}$,$\hat{H}^{(0)}$,$\hat{H}$は,
    \begin{align}
      U(x) &\coloneqq
      \begin{dcases}
        0 & \ \abs{x}\leq L/2\\
        \infty & \text{otherwise}
      \end{dcases} \\ 
      V(x) &\coloneqq -e\phi(x) = eEx\ (e>0) \\ 
      \hat{H}^{(0)} &\coloneqq -\frac{\hbar^2}{2m}\dv[2]{x} + \hat{U}(x) \\ 
      \hat{H} &\coloneqq \hat{H}^{(0)} + \hat{V}(x) \\ 
    \end{align}
    と定義される.
    また,$\hat{H}^{(0)}$の固有エネルギーとそれに属する固有関数は,
    \begin{align}
      E_n^{(0)} &= \frac{\hbar^2}{2m}\qty(\frac{\pi}{L})^2n^2,\ (n = 1, 2, 3,\cdots)\\
      \phi_n(x)&=
      \begin{dcases}
        \sqrt{\frac{2}{L}}\cos\qty(\frac{n\pi}{L}x) & n: \r{odd}\\
        \sqrt{\frac{2}{L}}\sin\qty(\frac{n\pi}{L}x) & n: \r{even}
      \end{dcases}
    \end{align}
    のようになっている.
    このとき,2次の摂動まで用いて$\hat{V}$の影響による$n = 1$のエネルギー補正を計算せよ.
    \tcblower
    1次摂動によるエネルギー補正は奇関数の積分になるため0である.\footnote{もし0でないならば,電場をかける向きによりエネルギーが変わることを意味するが,これは対称性より不合理である.}.\\
    2次摂動によるエネルギー補正は,
    \begin{align}
      E_{1}^{(2)} &= \sum_{m\neq 1}\frac{\abs{V_{m1}}^2}{E_1^{(0)} - E_m^{(0)}}\\
      V_{m1} &= eE\int\phi_m^*x\phi_{1}\dd{x}\\
      &
      \begin{dcases}
      = 0 & n: \r{odd}\\
      \neq 0 & n: \r{even}
      \end{dcases}\\
      E_{1}^{(2)}&=\frac{\abs{V_{21}}^2}{E_1^{(0)} - E_2^{(0)}} + \frac{\abs{V_{41}}^2}{E_1^{(0)} - E_4^{(0)}} + \cdots\\
      &\approx \frac{\abs{V_{21}}^2}{E_1^{(0)} - E_2^{(0)}}\\
      &=-\frac{256}{234\pi^4}\frac{(eEL)^2}{E_1^{(0)}}
    \end{align}
    と計算できて,2次の摂動を考えるとエネルギーは低下することがわかる.
  \end{myex}
\end{document}

      \subsection{(2次摂動の例題)Mott絶縁体}
        \documentclass{report}
\input{../../../head.tex}
\begin{document}
  \begin{myex}{Mott insulator\footnote{
      Nevill Francis Mott (1905-1996)
    }\footnote{
      バンドギャップが大きくギャップ内にフェルミ準位があるバンド絶縁体と異なり,運動エネルギーが小さくCoulomb力が大きいため電子が移動できない絶縁体である.
    }\footnote{
      $R\mathrm{NiO_3,TaS_2,Sr_2IrO_4}$など
    }}{}
    Coulomb力が強い4つのサイトに電子を4つ入れる.
    $\uparrow \ \uparrow\ \uparrow\ \uparrow$と$\uparrow\ \downarrow\ \uparrow\ \downarrow$のどちらが基底状態としてふさわしいだろうか.
    サイト間の電子の飛び移りを摂動として扱う.
    ここで重要なのは,基底状態のエネルギーは2次摂動により必ず下がるということである.
    $\uparrow \ \uparrow\ \uparrow\ \uparrow$に摂動を加えたとしてもPauliの排他律により電子の飛び移りは起こらない.
    摂動によってエネルギーは変化しない.しかし,$\uparrow\ \downarrow\ \uparrow\ \downarrow$は電子が反平行であるため電子のサイト間での飛び移りが許される.
    これは,2次摂動によるエネルギーの低下を引き起こす.
    よって,$\uparrow\ \downarrow\ \uparrow\ \downarrow$の方が基底状態としてふさわしい\footnote{
      Mott insulatorは反強磁絶縁体である.
    }.
  \end{myex}
\end{document}

      \subsection{縮退がある場合の摂動論}
        \documentclass{standalone}
\input{../../../head.tex}
\begin{document}
  1次摂動の式
  \begin{align}
    \label{1ji}
    (E_n^0-\hat{H}^0)\ket{n^1}+E_n^1\ket{n^0}=\hat{V}\ket{n^0}\\
    \ket{n^1}=\sum_{m\ne n}\ket{m^0}\frac{\bra{m^0}\hat{V}\ket{n^0}}{E_n^0-E_m^0}
  \end{align}
  これは$E_n^0=E_m^0$となる$m\ne n$が存在すると発散してしまう.そのため,発散する項は別で扱う必要がある.
  以下のような2重縮退がある場合を考える.
  \begin{align}
    \hat{H}^0\ket{n_a^0}&=E_n^0\ket{n_a^0}\\
    \hat{H}^0\ket{n_b^0}&=E_n^0\ket{n_b^0}
  \end{align}
  ただし,$\braket{n_i^0}{n_j^0}=\delta_{ij}$とする.$\ket{n_a^0}$と$\ket{n_b^0}$は同じ固有値をもつため,これらの線形結合
  $\ket{n_0}=\alpha\ket{n_a^0}+\beta\ket{n_b^0}$も解となる.\\
  まず,式(\ref{1ji})の両辺に左から$\bra{n_a^0}$を作用する.
  \begin{equation}
    \bra{n_a^0}(E_n^0-\hat{H}^0)\ket{n^1}+\bra{n_a^0}E_n^1\ket{n^0}=\bra{n_a}\hat{V}\ket{n^0}
  \end{equation}
  第1項は$E_n^0-E_n^0$より0.ここで,$\bra{n_i^0}\hat{V}\ket{n_j^0}=\omega_{ij}$とおけば
  \begin{equation}
    \alpha E_n^1=\alpha\omega_{aa}+\beta\omega_{ab}
  \end{equation}
  を得る.式(\ref{1ji})の両辺に左から$\bra{n_b^0}$を作用することも考えることにより,合わせて
  \begin{equation}
    \begin{cases}
      \alpha\omega_{aa}+\beta\omega_{ab}=\alpha E_n^1\\
      \alpha\omega_{ba}+\beta\omega_{bb}=\beta E_n^1
    \end{cases}
  \end{equation}
  を得る.これは行列を用いて以下のように書き直される.
  \begin{equation}
    \label{gyouretu}
    \begin{pmatrix}
      \omega_{aa}-E_n^1&\omega_{ab}\\
      \omega_{ba}&\omega_{bb}-E_n^1
    \end{pmatrix}
    \begin{pmatrix}
      \alpha\\
      \beta
    \end{pmatrix}
    =
    \begin{pmatrix}
      0\\0
    \end{pmatrix}
  \end{equation}
  $(\alpha,\beta)=(0,0)$以外の解を持つには行列式が0となればいいので,
  \begin{equation}
    \begin{vmatrix}
      \omega_{aa}-E_n^1&\omega_{ab}\\
      \omega_{ba}&\omega_{bb}-E_n^1
    \end{vmatrix}
    =0
  \end{equation}
  である.よって,1次の摂動エネルギーとして
  \begin{screen}
  \begin{equation}
    \label{syukutai}
    E_n^1=\frac{1}{2}\qty[(\omega_{aa}+\omega_{bb}\pm\sqrt{(\omega_{aa}-\omega_{bb})^2+4|\omega_{ab}|^2})]
  \end{equation}
  \end{screen}
  を得る.縮退が解けてエネルギーが2つに分かれている.
  \begin{myex}{}{}$\omega_{aa}=\omega_{bb}=0,\omega_{ab}=\omega_{ba}=\omega$の場合
  式(\ref{gyouretu})は
  \begin{equation}
    \begin{pmatrix}
      -E_n^1&\omega\\
      \omega&-E_n^1
    \end{pmatrix}
    \begin{pmatrix}
      \alpha\\
      \beta
    \end{pmatrix}
    =\begin{pmatrix}
      0\\0
    \end{pmatrix}
  \end{equation}
  となる.よって,
  \begin{equation}
    E_n^1=
    \begin{cases}
      +\omega\ (\alpha=1,\beta=1)\\
      -\omega\ (\alpha=1,\beta=-1)
    \end{cases}
  \end{equation}
  である.これは摂動を加える前に縮退していた2つの状態$\ket{n^0}=\ket{n_a^0}\pm\ket{n_b^0}$の縮退が解け,エネルギー$E_n^0+\omega$をもつ状態$\ket{n_a^0}+\ket{n_b^0}$
  とエネルギー$E_n^0-\omega$をもつ状態$\ket{n_a^0}-\ket{n_b^0}$に分かれたことを意味している.
  \end{myex}
\end{document}

      \subsection{(定常摂動の例題)物質中の電子}
        \documentclass{standalone}
\input{../../../head.tex}
\begin{document}
  結晶中の周期ポテンシャルによりバンドギャップができることを確認する.\\
  長さ$L$の周期ポテンシャル中の1次元自由電子の運動を考える.\\
  波動関数は
  \begin{equation}
    \phi_k(x)=\braket{x}{k}=\frac{1}{\sqrt{L}}\mathrm{e}^{ikx}
  \end{equation}
  エネルギー固有値は
  \begin{equation}
    \epsilon_0(k)=\frac{\hbar^2k^2}{2m}\ (k=\frac{2\pi}{L}N,\ N=0,\pm1,\pm2,...)
  \end{equation}
  である.\\
  ここで$V(x+a)=V(x)$を満たす結晶の周期ポテンシャルを摂動として加える.
  \begin{align}
    V(x)&=2V\cos\frac{2\pi}{a}x\\
    &=V(\mathrm{e}^{i\frac{2\pi}{a}x}+\mathrm{e}^{-i\frac{2\pi}{a}x})\\
    &=V(\mathrm{e}^{igx}+\mathrm{e}^{-igx})\\
    g&\equiv\frac{2\pi}{a}
  \end{align}
  2次摂動まで含めるとエネルギーは次のようになる.
  \begin{equation}
    \label{2ji}
    E_n=E_n^0+\bra{n^0}\hat{V}\ket{n^0}+\sum_{m\ne n}\frac{\qty|\bra{m^0}\hat{V}\ket{n^0}|^2}{E_n^0-E_m^0}
  \end{equation}
  ここで,状態及びエネルギーのラベリングを$n$から$k$に変更する.$V_{k'k}=\bra{k'}\hat{V}\ket{k}$として式(\ref{2ji})を書き換える.
  \begin{equation}
    E(k)=\epsilon^0(k)+V_{kk}+\sum_{k'\ne k}\frac{|V_{k'k}|^2}{\epsilon^0(k)-\epsilon^0(k')}
  \end{equation}
  摂動によるエネルギーは
  \begin{align}
    V_{k'k}&=\frac{V}{L}\int_{-L/2}^{L/2}\phi_{k'}^{*}(x)\hat{V}(x)\phi_k(x)dx\\
    &=V\qty[\frac{\sin\frac{qL}{2}}{\frac{qL}{2}}+\frac{\sin\frac{q'L}{2}}{\frac{q'L}{2}}]\\
  \end{align}
  と計算される.($q\equiv -k'+g+k,\ q'\equiv -k'-g+k$)摂動によるエネルギーはsinc関数の形になっているので$L\to\infty$ではデルタ関数に近似できる.よって,
  \begin{equation}
    V_{k'k}=V(\delta_{q,0}+\delta_{q',0})=
    \begin{cases}
      V\ \mathrm{if}\ k'=k+g\ \mathrm{or}\ k'=k-g\\
      0\ \mathrm{otherwise}
    \end{cases}
  \end{equation}
  である.したがってエネルギーは
  \begin{equation}
    \label{Ebr}
    E(k)=\epsilon^0(k)+\frac{V^2}{\epsilon(k)-\epsilon(k+g)}+\frac{V^2}{\epsilon(k)-\epsilon(k-g)}
  \end{equation}
  となる.この振る舞いをを1st Brillouin Zoneの内外で確認する(対称性から右側のみ).\\
  \\
  1st Brillouin Zone内側($k=k_1<\frac{\pi}{a}$)\\
  $\epsilon(k)$は放物線なので
  \begin{equation}
    \begin{cases}
      \epsilon(k_1)\ll\epsilon^0(k_1+g)\\
      \epsilon(k_1)<\epsilon^0(k_1-g)
    \end{cases}
  \end{equation}
  が成りたつ.よって,$E(k)<\epsilon^0(k)$.摂動が加わった後のエネルギーは加わる前のエネルギーより小さくなる.\\
  1st Brillouin Zone外側($k=k_2>\frac{\pi}{a}$)\\
  同様に考え,
  \begin{equation}
    \begin{cases}
      \epsilon(k_2)\ll\epsilon^0(k_2+g)\\
      \epsilon(k_2)>\epsilon^0(k_2-g)
    \end{cases}
  \end{equation}
  である.よって,$E(k)>\epsilon^0(k)$が成り立ち,摂動が加わった後のエネルギーは加わる前のエネルギーより大きくなる.\\
  以上の議論により,結晶中の周期ポテンシャルによりバンドギャップが形成されることがわかった.しかし,式(\ref{Ebr})に$k=\pm\frac{\pi}{a}$を代入すると発散してしまう.以下では
  2重縮退があるときの摂動を考えバンドギャップ$\Delta E$を求める.\\
  式(\ref{syukutai})で$a=\frac{\pi}{a}$,$b=-\frac{\pi}{a}$とする.$V_{kk}=0,V_{ab}=V$なので,2次摂動によるエネルギーは
  \begin{align}
    E&=\pm\frac{1}{2}\sqrt{4|V|^2}\\
    &=\pm V
  \end{align}
  である.よって,
  \begin{equation}
    \Delta E=2V
  \end{equation}
  を得る.また,$E=\pm V$に対応する波動関数はそれぞれ
  \begin{align}
    \psi_{+}&=\phi_{k=\pi/a}+\phi_{k=-\pi/a}\sim\cos\frac{\pi}{a}x\\
    \psi_{-}&=\phi_{k=\pi/a}-\phi_{k=-\pi/a}\sim\sin\frac{\pi}{a}x
  \end{align}
  であり,定在波が生じる.\\
  バンドギャップの起源はBragg反射である.Bragg反射は以下の式を満たす.
  \begin{equation}
    2a\sin\theta=\lambda
  \end{equation}
  今回の場合は1次元なので$\theta=\pi/2$であり,波数は$k=2\pi/\lambda$である.よって,Bragg条件は
  \begin{equation}
    k=\frac{\pi}{a}
  \end{equation}
  と書き換えられる.上の条件を満たす波数のみが反射し,定在波をつくる.sinで表される波動関数はポテンシャルが最小となる波数で確率振幅が最大となる.
  cosで表される波動関数はポテンシャルが最大となる波数で確率振幅が最大となる.よって,sinの方はエネルギーが低くなり,cosの方は高くなる.これによりバンドギャップが生じる.
\end{document}

      \subsection{練習問題}
        \documentclass{report}
\input{../../../head.tex}
\begin{document}
  \begin{myexc}{Griffith Example 7.3}{}
    2次元調和振動子$\hat{H^0}=\frac{\hat{p}^2}{2m}+\frac{1}{2}m\omega^2(\hat{x}^2+\hat{y}^2)$の第1励起状態は縮退している.
    \begin{align}
      \psi^0_a&=\psi_0(x)\psi_1(y)=\sqrt{\frac{2}{\pi}}\frac{m\omega}{\hbar}y\exp\qty(\frac{m\omega}{2\hbar}(x^2+y^2))\\
      \psi^0_a&=\psi_1(x)\psi_0(y)=\sqrt{\frac{2}{\pi}}\frac{m\omega}{\hbar}x\exp\qty(\frac{m\omega}{2\hbar}(x^2+y^2))
    \end{align}
    ここに摂動$\hat{H'}=\epsilon m\omega^2xy$を加える.
    \begin{equation}
      \begin{pmatrix}
        \omega_{aa}-E_n^1&\omega_{ab}\\
        \omega_{ba}&\omega_{bb}-E_n^1
      \end{pmatrix}
      \begin{pmatrix}
        \alpha\\
        \beta
      \end{pmatrix}
      =
      \begin{pmatrix}
        0\\0
      \end{pmatrix}
    \end{equation}
    を用いて摂動を加えた後の固有関数及び摂動による補正エネルギーを求めよ.
    \tcblower
    \begin{align}
      W_{aa}&=\int\int\psi_a^0\hat{H'}\psi_a^0dxdy\\
      &=\epsilon m\omega^2\int|\psi_0(x)|^2xdx\int|\psi_1(x)|^2ydy\\
      &=0=W_{bb}
    \end{align}
    \begin{align}
      W_{ab}&=\qty[\int\psi_0(x)\epsilon m\omega^2x\psi_1(x)dx]^2\\
      &=\epsilon\frac{\hbar\omega}{2}\qty[\int\psi_0(x)(\hat{a}_{+}+\hat{a}_{-})\psi_1(x)dx]^2\\
      &=\epsilon\frac{\hbar\omega}{2}\qty[\int\psi_0(x)\psi_0(x)dx]^2\\
      &=\epsilon\frac{\hbar\omega}{2}
    \end{align}
    よって,摂動による補正エネルギーは$E_1=\pm\epsilon\frac{\hbar\omega}{2}$,固有関数は
    \begin{equation}
      \psi_{\pm}^0=\frac{1}{\sqrt{2}}\qty(\psi_b^0\pm\psi_a^0)
    \end{equation}
    である.
  \end{myexc}
\end{document}

    \section{摂動II(非定常摂動)}
      \documentclass{report}
\input{../../../head.tex}
\begin{document}
  摂動項が時間に依存する場合の摂動(time-dependent perturbation)\footnote{電磁波による摂動など.}を扱う.
  本節では,量子系の時間発展が状態ベクトルの時間変化で描像するSchr\"odinger描像で記述する.
  時間に依存するSchrödinger方程式,
  \begin{align}
    \i\hbar\dv{t}\ket{\psi(t)} = \hat{H}\ket{\psi(t)}\label{time-dependent-schroinger-eq}
  \end{align}
  を考える.なお,$\ket{\psi(t)}$は$\hat{H}^{(0)}$の固有ベクトルを用いて,
  \begin{align}
    \ket{\psi(t)} = \sum_{n}c_n(t)\ket{n}\label{psi-t-expantion}
  \end{align}
  と展開できたとする.$\hat{H}^{(0)}$の固有ベクトルが完全系を成すので,
  状態ベクトルが時間変化した空間は$\qty{\ket{n}\mid n = 0, 1, \cdots}$が張る空間の部分空間となることに注意する.
  $\hat{H}$の性質ごとに$\ket{\psi(t)}$の具体的な形を議論する.
  \begin{enumerate}
    \item $\hat{H} = \hat{H}^{(0)}$の場合\footnote{$\hat{H}^{(0)}$は厳密に解けるHamiltonian.}\par
      量子状態$\ket{\psi(t)}$の時間発展は時間発展演算子
      \footnote{
        $\Delta t \ll 1$として,
        \begin{align*}
          \ket{\psi(\Delta t)} = \exp\qty[-\i\frac{\hat{H}\Delta t}{\hbar}]\ket{\psi(0)} \approx\qty(\hat{I} - \i\frac{\hat{H}\Delta t}{\hbar})\ket{\psi(0)} 
          = \ket{\psi(0)} + \Delta t \left.\qty(\dv{t}\ket{\psi(t)})\right|_{t = 0}
        \end{align*}
        より微分の形で書けることから,確かに時間発展すると私は解釈する.}
      \footnote{
        演算子が交換するときは数字と同じ扱いをしても良いと考える.
        一般に演算子は,
        \begin{align*}
          \e^{\hat{A}}\e^{\hat{B}} = \exp{\hat{A} + \hat{B} + \frac{1}{2}\qty[\hat{A},\hat{B}] + \cdots}\neq\e^{\hat{A} + \hat{B}}
        \end{align*}
        なるBCH公式を満たす.
      }
      $\exp\qty(-\i\frac{\hat{H}t}{\hbar})$を用いて,
      \begin{align}
        \ket{\psi(t)} &= \exp\qty[-\i\frac{\hat{H}^{(0)}t}{\hbar}]\ket{\psi(0)} \\
        &= \sum_{n}c_n(0)\exp\qty[-\i\frac{\hat{H}^{(0)}t}{\hbar}]\ket{n} \\
        &= \sum_{n}c_n(0)\exp\qty[-\i\frac{E_n}{\hbar}t]\ket{n}
      \end{align}
      のように表すことができる.
      \par
      時間発展演算子を用いることなく計算することもできる.
      \refe{psi-t-expantion}を\refe{time-dependent-schroinger-eq}に代入すると,$\hat{H}$が時間に依存しないことに注意すれば,
      \begin{align}
        \i\hbar\dv{t}\qty(\sum_{n}c_n(t)\ket{n}) &= \hat{H}\qty(\sum_{n}c_n(t)\ket{n}) \label{naive-solution-to-decide-ct}\\ 
        \Leftrightarrow \sum_{n}\qty(\i\hbar\dv{c_n(t)}{t}\ket{n}) &= \sum_{n}\qty(c_n(t)E_n\ket{n}) \\ 
        \Leftrightarrow \forall n\ \i\hbar\dv{c_n(t)}{t}\ket{n} &= c_n(t)E_n\ket{n} \\ 
        \Leftrightarrow \forall n\ c_n(t) &= \exp\qty(-\i\frac{E_n}{\hbar}t)
      \end{align}
      を用いれば,
      \begin{align}
        \ket{\psi(t)} = \sum_{n}\exp\qty(-\i\frac{E_n}{\hbar}t)\ket{n}
      \end{align}
      を得る.
    \item $\hat{H} = \hat{H}^{(0)} + \hat{V}(t)$の場合\par
      このときは$\ket{\psi(t)}$を簡単な形で書き下すことが出来ないから,便宜的に,
      \begin{align}
        \psi(t) = \sum_{n}c_n(t)\exp\qty(-\i\frac{E_n}{\hbar}t)\ket{n}
      \end{align}
      と展開しておく.なお,$c_n(t)$が定数のときの$\ket{\psi(t)}$との整合性をとるために$\exp\qty(-\i\frac{E_n}{\hbar}t)$をかけてある.
      原理的には$c_n(t)$が求まれば量子系の時間発展の様子がわかる.% 暇になったら時間順序積のことを書く.
  \end{enumerate}
  \par
  さて,いずれの場合でも,量子系の性質を調べるには$c_n(t)$の具体的な形がわかればよいことを発見した.
  本節では,まずSchr\"odinger描像から相互作用表示に書き換え,$c_n(t)$を厳密に知ることが困難であることを知り,$c_n(t)$の近似解を導く.
\end{document}

      \subsection{相互作用表示}
        \documentclass{report}
\input{../../../head.tex}
\begin{document}
  Schr\"odinger表示の非定常摂動基本方程式は,\refe{time-dependent-schroinger-eq}と\refe{nonsteady-perturbation-cnt-eq}であり,
  \begin{align}
    \begin{dcases}
      \i\hbar\dv{t}\ket{\psi(t)} = \qty(\hat{H}^{(0)} + \hat{V}(t))\ket{\psi(t)} \label{TDSE} \\
      \ket{\psi(t)} = \sum_{n}c_n(t)\exp\qty(-\i\frac{E_n}{\hbar}t)\ket{n}
    \end{dcases}
  \end{align}
  と書けるのであった.
  \textbf{相互作用表示}(interaction picture)を\refe{interaction-picture}のような$\ket{\psi(t)} \to \ket{\psi(t)}_{\r{I}}$の変換を行って得られる状態ベクトルと定義する.
  \begin{itembox}[l]{相互作用表示}
    \begin{align}
      \ket{\psi(t)}_{\r{I}} \coloneqq \exp\qty(\i\frac{\hat{H}^{(0)}}{\hbar}t)\ket{\psi(t)}\label{interaction-picture}
    \end{align}
  \end{itembox}
  \refe{interaction-picture}を用いて,\refe{TDSE}等価な基本方程式,相互作用表示の非定常摂動基本方程式を導く.
  まず\refe{TDSE}の第2式を用いて,
  \begin{align}
    \ket{\psi(t)}_{\r{I}} &= \exp\qty(\i\frac{\hat{H}^{(0)}}{\hbar}t)\sum_{n}c_n(t)\exp\qty(-\i\frac{E_n}{\hbar}t)\ket{n}\\
    &= \sum_{n}c_n(t)\exp\qty(-\i\frac{E_n}{\hbar}t)\exp\qty(\i\frac{\hat{H}^{(0)}}{\hbar}t)\ket{n} \\ 
    &= \sum_{n}c_n(t)\exp\qty(-\i\frac{E_n}{\hbar}t)\exp\qty(\i\frac{E_n}{\hbar}t)\ket{n} \\ 
    &= \sum_{n}c_n(t)\ket{n}\label{TDSE-2nd-eq-interaction-picture}
  \end{align}
  と計算できる.
  \par
  次に,相互作用表示の時間微分を計算してみる.
  相互作用表示での摂動項$\hat{V}_{\r{I}}$を,
  \begin{align}
    \hat{V}_{\r{I}} \coloneqq \exp\qty(\i\frac{\hat{H}^{(0)}}{\hbar}t)\hat{V}(t)
  \end{align}
  とする.計算の途中で,\refe{TDSE}の第1式を用いると,
  \begin{align}
    \i\hbar\dv{t}\ket{\psi(t)}_{\r{I}} &= \dv{t}\qty[\exp\qty(\i\frac{\hat{H}^{(0)}}{\hbar}t)]\ket{\psi(t)} + \exp\qty(\i\frac{\hat{H}^{(0)}}{\hbar}t)\dv{t}\ket{\psi(t)} \\
    &= \i\frac{\hat{H}^{(0)}}{\hbar}\exp\qty(\i\frac{\hat{H}^{(0)}}{\hbar}t)\ket{\psi(t)} + \exp\qty(\i\frac{\hat{H}^{(0)}}{\hbar}t)\frac{1}{\i\hbar}\qty(\hat{H}^{(0)} + \hat{V}(t))\ket{\psi(t)} \\ 
    &= \frac{\i}{\hbar}\qty[\hat{H}^{(0)}, \exp\qty(\i\frac{\hat{H}^{(0)}}{\hbar}t)]\ket{\psi(t)} + \exp\qty(\i\frac{\hat{H}^{(0)}}{\hbar}t)\hat{V}(t)\ket{\psi(t)} \\ 
    &= \exp\qty(\i\frac{\hat{H}^{(0)}}{\hbar}t)\hat{V}(t)\ket{\psi(t)} \\
    &= \exp\qty(\i\frac{\hat{H}^{(0)}}{\hbar}t)\hat{V}(t)\exp\qty(-\i\frac{\hat{H}^{(0)}}{\hbar}t)\exp\qty(\i\frac{\hat{H}^{(0)}}{\hbar}t)\ket{\psi(t)} \\
    &= \exp\qty(\i\frac{\hat{H}^{(0)}}{\hbar}t)\hat{V}(t)\exp\qty(-\i\frac{\hat{H}^{(0)}}{\hbar}t)\ket{\psi(t)}_{\r{I}} \\
    &= \hat{V}_{\r{I}}(t)\ket{\psi(t)}_{\r{I}}\label{TDSE-1st-eq-interaction-picture}
  \end{align}
  を得る.\refe{TDSE-1st-eq-interaction-picture}と\refe{TDSE-2nd-eq-interaction-picture}はSch\"odinger表示の非定常摂動基本方程式と等価な方程式であるから,
  これを\textbf{朝永・Schwinger方程式}と呼ぶ.
  \footnote{Schrödinger描像は量子状態が時間発展するとみなす.Heisenberg描像は物理量が時間発展するとみなす.相互作用表示はその中間であるといえる.}  
  \footnote{朝永振一郎(1906-1979)}
  \footnote{Julian Schwinger(1918-1994)}
  \footnote{朝永とSchwingerは1964年にRichard Feynmannとともにノーベル賞を受賞.}.
  \begin{itembox}[l]{朝永・Schwinger方程式}
    \begin{align}
      \i\hbar\dv{t}\ket{\psi(t)}_{\r{I}}&=\hat{V}_{\r{I}}(t)\ket{\psi(t)}_{\r{I}}\label{Tomonaga-Schwinger}\\
      \hat{V}_{\r{I}}(t)&=\exp\qty(\i\frac{\hat{H}^{(0)}}{\hbar}t)\hat{V}(t)\exp\qty(-\i\frac{\hat{H}^{(0)}}{\hbar}t)
    \end{align}
  \end{itembox}
  \par
  \refe{Tomonaga}に左から$\bra{m}$を演算する($\hat{H}^{(0)}\ket{m} = E_m\ket{m}$).\\
  左辺は,
  \begin{align}
    &\bra{m}\i\hbar\dv{t}\sum_{n}c_n(t)\ket{n}\\
    & = \i\hbar\dv{t}c_m(t)
  \end{align}
  となる.
  右辺は.
  \begin{align}
    &\bra{m}\hat{V}_{\r{I}}(t)\sum_{n}c_n(t)\ket{n}\\
    & = \sum_{n}c_n(t)\e^{-\i\frac{(E_n - E_m)t}{\hbar}}\mel{m}{\hat{V}(t)}{n}
  \end{align}
  となる.よって,非定常摂動の時間発展は以下の式を満たす.
  \begin{itembox}[l]{$\hat{H}(t) = \hat{H}^{(0)} + \hat{V}(t)$}
    \begin{align}
      \ket{\psi(t)}_{\r{I}} &= \sum_{n}c_n(t)\ket{n}\\
      i\hbar\dv{t}c_m(t) &= \sum_{n}c_n(t)V_{mn}\e^{\i\omega_{mn}t}\\
      V_{mn} &= \bra{m}\hat{V}(t)\ket{n}\\
      \omega_{mn} &= \frac{E_m - E_n}{\hbar} = -\omega_{nm}
    \end{align}
  \end{itembox}
  これは$c_n$の連立方程式になっており解くことは困難である.よって近似を加える.
\end{document}

      \subsection{(相互作用表示の例題)Rabi振動}
        \documentclass{report}
\input{../../../head.tex}
\begin{document}
  前節の最後に,一般に時間発展する量子系を正確に追跡すること,すなわち,$c_n(t)$の厳密解を求めることが困難であると述べた.
  にもかかわらず,ある特殊な条件下では近似を行うことなく,厳密解を得ることができる.
  以下の\exref{rabi-cycle}では,そのような物理現象として\textbf{Rabi振動}(Rabi cycle)\footnote{I.I.Rabi(1898 - 1988)}
  \footnote{量子状態の振動をRabi振動という.}を議論する.
  \begin{myex}{Rabi振動}{rabi-cycle}
    厳密に解くことのできる2準位系,
    \begin{align}
      \ket{1} &\coloneqq \mqty(1\\ 0) \\
      \ket{2} &\coloneqq \mqty(0\\ 1) \\
      \hat{H}^{(0)} &\coloneqq \mqty(E_1 & 0\\ 0 & E_2)
    \end{align}
    を考える.明らかに,$E_1$と$E_2$は$\hat{H}^{(0)}$のエネルギー固有値で,それぞれに属する固有ベクトルは$\ket{1}$と$\ket{2}$である.
    この2準位系に,時刻$t = 0$から$\hat{V}(t)$なる摂動を加える.
    ただし,$\hat{V}(t)$は,
    \begin{align}
      \hat{V}(t) = \mqty(
        0 & \gamma\e^{\i\omega t} \\
        \gamma\e^{-\i\omega t} & 0
      )
    \end{align}
    によって与えられる\footnote{$\gamma$は摂動の強さを表す.}.
    系全体のハミルトニアンを$\hat{H} \coloneqq \hat{H}^{(0)} + \hat{V}(t)$とする.
    系の状態ベクトルを相互作用表示を用いて,
    \begin{align}
      \ket{\psi(t)}_{\r{I}} \coloneqq c_1(t)\ket{1} + c_2(t)\ket{2}
    \end{align}
    と書いたとき,以下の問いに答えよ.
    \begin{enumerate}
      \item $c_1(t)$と$c_2(t)$の時間発展を調べよ.ただし,初期条件は$c_1(0) = 1$,$c_2(0) = 0$とする.
      \item 1. の条件のもとで,$c_2(t)$の確率振幅が最大となる$\omega$を求めよ.
    \end{enumerate}
    \tcblower
    \begin{enumerate}
      \item $c_1(t)$と$c_2(t)$の時間発展\par
        今回の設定での$\omega_{mn}$や$V_{mn}$を計算すると,
        \begin{align}
          \omega_{11} &= \omega_{22} = 0 \\ 
          \omega_{21} &= -\omega_{12} = \frac{E_2 - E_1}{\hbar} \\ 
          V_{11} &= V_{22} = 0 \\ 
          V_{21} &= V_{12}^* = \gamma\e^{-\i\omega t}
        \end{align}
        である.非定常摂動の時間発展の式より,
        \begin{align}
          \begin{dcases}
            \i\hbar\dv{t}c_1(t) = c_1(t)V_{11}\e^{\i\omega_{11}t} + c_2(t)V_{12}\e^{\i\omega_{12}t} = c_2(t)\gamma\e^{\i\qty(\omega + \omega_{12}) t} \label{rabi-cnt-eq}\\ 
            \i\hbar\dv{t}c_2(t) = c_1(t)V_{21}\e^{\i\omega_{21}t} + c_2(t)V_{22}\e^{\i\omega_{22}t} = c_1(t)\gamma\e^{-\i\qty(\omega - \omega_{21}) t} 
          \end{dcases}
        \end{align}
        が成り立つ.
        次に,$\Delta\omega \coloneqq \omega - \omega_{21}$として\refe{rabi-cnt-eq}を解く.
        2式目を1式目に代入して$c_1(t)$を消去すると,
        \begin{align}
          \dv[2]{t}c_2(t) + \i\Delta\omega\dv{t}c_2(t) + \qty(\frac{\gamma}{\hbar})^2c_2 = 0\label{rabi-cnt-eq-c2}
        \end{align}
        2階の斉次微分方程式の解は$c_2(t) = \e^{\i\lambda t}$と書けるので,これを\refe{rabi-cnt-eq-c2}に代入すると,
        \begin{align}
          \lambda^2 + \Delta\omega\lambda - \qty(\frac{\gamma}{\hbar}) &= 0 \\
          \Rightarrow \lambda &= -\frac{\Delta\omega}{2}\pm\sqrt{\qty(\frac{\Delta\omega}{2})^2 + \qty(\frac{\gamma}{\hbar})^2}
        \end{align} 
        を得る.$\Omega$を,
        \begin{align}
          \Omega \coloneqq \sqrt{\qty(\frac{\Delta\omega}{2})^2 + \qty(\frac{\gamma}{\hbar})^2}\label{rabi-freq}
        \end{align}
        と定義して,これをRabi周波数と呼ぶ.
        $c_2$の一般解は,
        \begin{align}
          c_2(t) = \exp\qty(-\i\frac{\Delta\omega}{2}t)\qty(A\e^{\i\Omega t} + B\e^{-\i\Omega t})\label{c2t-general-solution}
        \end{align}
        と書ける.\refe{c2t-general-solution}を\refe{rabi-cnt-eq}の第2式に代入すると,
        \begin{align}
          c_1(t) = \frac{\hbar}{\gamma}\exp\qty(-\i\frac{\Delta\omega}{2}t)\qty[\frac{\Delta\omega}{2}\qty(A\e^{\i\Omega t} + B\e^{-\i\Omega t}) - \Omega\qty(A\e^{\i\Omega t} - B\e^{-\i\Omega t})]
        \end{align}
        となる.
        初期条件を考えると,
        \begin{align}
          \begin{dcases}
            A + B = 0 \\ 
            A - B = -\frac{\gamma}{\hbar\Omega}
          \end{dcases}
        \end{align}
        であるから,
        \begin{align}
          B = -A = \frac{\gamma}{2\hbar\Omega}
        \end{align}
        となる.よって,
        \begin{align}
          c_1(t) &= \exp\qty(-\i\frac{\Delta\omega}{2}t)\qty(\cos\Omega t - \i\frac{\Delta \omega}{2\Omega}\sin\Omega t) \\ 
          c_2(t) &=  - \i\frac{\gamma}{\hbar\Omega}\exp\qty(-\i\frac{\Delta\omega}{2}t)\sin\Omega t
        \end{align}
        である.時刻$t$で$\ket{1}$,$\ket{2}$に状態を見出す確率,$\abs{c_1(t)}^2$,$\abs{c_2(t)}^2$はそれぞれ,
        \begin{align}
          \abs{c_1(t)}^2 &= \cos^2\Omega t + \qty(\frac{\Delta \omega}{2})^2\frac{1}{\Omega^2}\sin^2\Omega t \\ 
          \abs{c_2(t)}^2 &= \frac{\gamma^2}{\hbar^2\Omega^2}\sin^2\Omega t
        \end{align}
        である.簡単な計算により,$\abs{c_1(t)}^2 + \abs{c_2(t)}^2 = 1$となることが容易に確かめられる.
      \item $c_2(t)$の振幅が最大となる$\omega$の値\par
        振幅の大きさは$\Omega$の定義\refe{rabi-freq}より,
        \begin{align}
          \frac{(\gamma/\hbar)^2}{(\gamma/\hbar)^2 + (\Delta\omega/2)^2}
        \end{align}
        と表されるので,$\Delta\omega = \omega - \omega_{21} = 0$のときに最大となる.つまり,摂動の周波数$\omega$と
        2準位のエネルギー差に由来する$\omega_{21}$が一致したときに遷移が起こりやすい.
    \end{enumerate}
  \end{myex}
  \exref{rabi-cycle}では,Rabi振動に関する2つの重要な物理量を得たので,下にまとめる.
  \begin{itembox}[l]{Rabi周波数}
    \begin{align}
    \Omega = \sqrt{\qty(\Delta\omega/2)^2 + \qty(\gamma/\hbar)^2}   
    \end{align}
  \end{itembox}
  \begin{itembox}[l]{共鳴条件}
    \begin{align}
      \omega = \omega_{21} = \frac{E_2 - E_1}{\hbar}
    \end{align}
  \end{itembox}
\end{document}

      \subsection{非定常摂動の近似解}
        \documentclass{report}
\input{../../../head.tex}
\begin{document}
  時間に依存する摂動がある量子系の時間発展は次の式で表されることを確認した.
  \begin{equation}
    \begin{cases}
      \label{3.4}
    \ket{\psi(t)}_I=\sum_{n}c_n(t)\ket{n}\\
    i\hbar\frac{d}{dt}c_m(t)=\sum_{n}V_{mn}(t)\mathrm{e}^{i\omega_{mn}t}c_n(t)
    \end{cases}
  \end{equation}
  一般に式(\ref{3.4})を解くことはできない.そこで近似を加える.\\
  $\hat{V}(t)\to\lambda\hat{V}(t)$として$c_n(t)$をべき級数展開する.
  \begin{equation}
    c_n(t)=c_n^0(t)+\lambda c_n^1(t)+\lambda^2c_n^2(t)+\cdots
  \end{equation}
  これを式(\ref{3.4})の第2式に代入して整理すると
  $\lambda^0$の係数比較から
  \begin{equation}
    \label{3.40ji}
    i\hbar\frac{d}{dt}c_m^0(t)=0
  \end{equation}
  $\lambda^1$の係数比較から
  \begin{equation}
    \label{3.41ji}
    i\hbar\frac{d}{dt}c_m^1(t)=\sum_{n}V_{mn}(t)\mathrm{e}^{i\omega_{mn}t}c_n^0(t)
  \end{equation}
  を得る.式(\ref{3.40ji})より,
  \begin{equation}
    \label{3.4const}
    c_m^0(t)=\rm{const.}
  \end{equation}
  である.以下では,$t=t_0$から摂動$\hat{V}(t)$を加え始めたとする.また,$t=t_0$で系の量子状態が$\ket{i}$であったとする.
  このとき
  \begin{equation}
    \begin{cases}
      c_i^0(t_0)=1\\
      c_m^0(t_0)=0\ (m\ne i)
    \end{cases}
  \end{equation}
  である.式(\ref{3.4const})から0次の係数は定数なので
  \begin{equation}
    \begin{cases}
      c_i^0(t)=1\\
      c_m^0(t)=0\ (m\ne i)
    \end{cases}
  \end{equation}
  が得られる.この系の状態は初期状態$\ket{i}$に依ることがわかったのでこれからは$c_m(t)$を$c_{m,i}(t)$と表記する.
  式(\ref{3.41ji})から
  \begin{align}
    i\hbar\frac{d}{dt}c_{m,i}^1(t)&=\sum_{n}V_{mn}(t)\mathrm{e}^{i\omega_{mn}t}c_{n,i}^0(t)\\
    &=V_{m,i}(t)\mathrm{e}^{o\omega_{mi}t}\\
    c_{m,i}^1(t)&=-\frac{i}{\hbar}\int_{t_0}^{t}V_{m,i}(t)\mathrm{e}^{i\omega_{mi}t}dt
  \end{align}
  である.また,系の量子状態は
  \begin{align}
    \ket{\psi(t)}_I&=\sum_{n}c_{n,i}(t)\ket{n}\\
    &\simeq\sum_{n}[c_{n,i}^0(t)+c_{n,i}^1(t)]\ket{n}\\
    &=\ket{i}+c_{i,i}^1(t)\ket{i}+\sum_{n\ne i}c_{n,i}^1(t)\ket{n}
  \end{align}
  と表される.よって
  \begin{itembox}[l]{$\ket{\psi(t_0)}_I=\ket{i}$の時間発展}
    \begin{align}
      \ket{\psi(t)}_I&=(1+c_{i,i}^1(t))\ket{i}+\sum_{n\ne i}c_{n,i}^1(t)\ket{n}\\
      c_{n,i}^1(t)&=-\frac{i}{\hbar}\int_{t_0}^{t}V_{n,i}\mathrm{e}^{i\omega_{ni}t}dt
    \end{align}
  \end{itembox}
  が得られる.このとき,始状態$\ket{i}$から終状態$\ket{f}\ (f\ne i)$への遷移確率は
  \begin{equation}
    |\braket{f}{\psi(t)}_I|^2=|(1+c_{i,i}^1(t))\braket{f}{i}+\sum_{n\ne i}c_{n,i}^1(t)\braket{f}{n}|^2=|c_{f,i}^1(t)|^2
  \end{equation}
  である.
\end{document}

      \subsection{(非定常摂動の近似解の例題)一定の摂動}
        \documentclass{report}
\input{../../../head.tex}
\begin{document}
  時刻$t=0$から摂動$\hat{V}$を加え始めたとする.
  \begin{equation}
    \hat{V}(t)=
    \begin{cases}
      0\ (t\le0)\\
      \hat{V}\ (t>0)
    \end{cases}
  \end{equation}
  このとき
  \begin{equation}
    V_{fi}(t)=\bra{f}\hat{V}(t)\ket{i}=
    \begin{cases}
      0\ (t\le 0)\\
      \bra{f}\hat{V}\ket{i}\ (t>0)
    \end{cases}
  \end{equation}
  である.ここで$\bra{f}\hat{V}\ket{i}=V_{fi}$とする.$t=0$で量子状態は$\ket{i}$であったとする.
  \begin{align}
    c_{f,i}^1(t)&=-\frac{i}{\hbar}\int_{0}^{t}V_{fi}\mathrm{e}^{i\omega_{fi}t}dt\\
    &=2i\exp\qty(-i\frac{\omega_{fi}t}{2})\sin\qty(\frac{\omega_{fi}}{2}t)
  \end{align}
  よって$\ket{i}$から$\ket{f}$への遷移確率は
  \begin{equation}
    |c_{f,i}^1(t)|^2=\frac{|V_{fi}|^2}{\hbar^2}\qty(\frac{\sin\qty(\omega_{fi}t/2)}{\qty(\omega_{fi}/2)})^2
  \end{equation}
  であり,$\sin\qty(\omega_{fi}t/2)$が0でないときのみ$\ket{i}$から$\ket{f}$への遷移が起きることがわかる.また,
  $\qty(\frac{\sin\qty(\omega_{fi}t/2)}{\qty(\omega_{fi}/2)})^2$は$|\omega_{fi}|<2\pi/t$で有効な値をもつ.従って,$t$が小さいときはこの範囲が十分広く,$\omega_{fi}\ne0$の状態への遷移が起こりうる.しかし,
  $t$が大きいときは$\omega_{fi}\simeq0$の状態への遷移しか起きない.
  通常は$t\to\infty$と考えてよい.
  \begin{equation}
    \lim_{t\to\infty}\qty(\frac{\sin\qty(\omega t/2)}{\omega t/2})^2=2\pi t\delta(\omega)
  \end{equation}
  を使うと\footnote{$\delta(ax)=\frac{1}{|a|}\delta(x)$},
  \begin{align}
    \lim_{t\to\infty}|c_{f,i}^1(t)|^2&=\frac{|V_{fi}|^2}{\hbar^2}2\pi t\delta(\omega_{f,i})\\
    &=\frac{2\pi}{\hbar}|V_{fi}|^2\delta(E_f-E_i)t
  \end{align}
  が得られる.
  よって,単位時間当たりの$\ket{i}$から$\ket{f}$への遷移確率は
  \begin{align}
    \omega_{i\to f}& = |c_{f,i}^1(t)|^2/t\\
    &=\frac{2\pi}{\hbar}\qty|\bra{f}\hat{V}\ket{i}|^2\delta(E_f-E_i)
  \end{align}
  である.これを\textbf{Fermiの黄金律}という\footnote{Enrico Fermi(1901-1954)}.
  \begin{itembox}[l]{Fermiの黄金律}
    \begin{equation}
      \omega_{i\to f}=\frac{2\pi}{\hbar}\qty|\bra{f}\hat{V}\ket{i}|^2\delta(E_f-E_i)
    \end{equation}
  \end{itembox}
  \begin{myex}{}{}電子の弾性散乱
    弾性散乱では散乱前後で粒子のエネルギーが保存される.電子のエネルギーは
    \begin{equation}
      E_{\bm{k'}}=\frac{\hbar^2|\bm{k'}|^2}{2m}
    \end{equation}
    と表される.よって,終状態は多く存在する.始状態$i$から終状態グループ$\qty{f}$への遷移確率を求める.
    \begin{align}
      \omega_{i\to\qty{f}}&=\sum_{f}\omega_{i\to f}\\
      &=\sum_{f}\frac{2\pi}{\hbar}\qty|\bra{f}\hat{V}\ket{i}|^2\delta(E_f-E_i)\\
      &=\frac{2\pi}{\hbar}\overline{\qty|\bra{f}\hat{V}\ket{i}|^2}\sum_{f}\delta(E_f-E_i)\\
      &=\frac{2\pi}{\hbar}\overline{\qty|\bra{f}\hat{V}\ket{i}|^2}\rho(E_f)
    \end{align}
    $\overline{\qty|\bra{f}\hat{V}\ket{i}|^2}$は散乱体の性質を,状態密度$\rho(E_f)$は物質の性質を反映している.この関係もFermiの黄金律という.
  \end{myex}
\end{document}

      \subsection{(非定常摂動の近似解の例題)調和摂動}
        \documentclass{standalone}
\input{../../../head.tex}
\begin{document}
  時間に依存する摂動を調和摂動という.
  以下の摂動を考える\footnote{光のイメージ}.
  \begin{equation}
    \hat{V}(t)=
    \begin{cases}
    0\ (t\le0)\\
    2V\cos\omega t\ (t>0)
    \end{cases}
  \end{equation}
  $2V\cos\omega t=V(\mathrm{e}^{i\omega t}+\mathrm{e}^{-i\omega t})$であるので,
  \begin{align}
    c^1_{f,i}(t)&=\frac{i}{\hbar}\int_{0}^{t}\bra{f}\hat{V}\ket{i}(\mathrm{e}^{i\omega t}+\mathrm{e}^{-i\omega t})\mathrm{e}^{i\omega_{fi}t}dt\\
    \label{3.6}
    &=-\frac{V_{fi}}{\hbar}\qty(\frac{\mathrm{e}^{i(\omega_{fi}+\omega)t}-1}{\omega_{fi}+\omega}+\frac{\mathrm{e}^{i(\omega_{fi}-\omega)t}-1}{\omega_{fi}-\omega})
  \end{align}
  を得る.ここで$V_{fi}=\bra{f}\hat{V}\ket{i}$とした.\\
  (i)$\omega_{fi}-\omega\approx0$のとき\\
  式(\ref{3.6})の第2項は第1項より十分大きい.よって,
  \begin{equation}
    c_{f,i}^1(t)\approx-\frac{V_{fi}}{\hbar}\frac{\mathrm{e}^{i(\omega_{fi}-\omega)t}-1}{\omega_{fi}-\omega}=-\frac{V_{fi}}{\hbar}\mathrm{e}^{i\frac{\omega_{fi}-\omega}{2}t}\frac{\sin\frac{\omega_{fi}-\omega}{2}t}{\frac{\omega_{fi}-\omega}{2}}
  \end{equation}
  と近似できる.このとき$\ket{i}\to\ket{f}$の遷移確率は
  \begin{equation}
    |c_{f,i}^1(t)|^2=\frac{V_{fi}^2}{\hbar^2}\frac{\sin^2\frac{\omega_{fi}-\omega}{2}t}{\qty(\frac{\omega_{fi}-\omega}{2})^2}
  \end{equation}
  $\frac{\sin^2\frac{\omega_{fi}-\omega}{2}t}{\qty(\frac{\omega_{fi}-\omega}{2})^2}$は$t\to\infty$で$2\pi t\delta(\omega_{fi}-\omega)=2\pi t\hbar\delta(E_f-E_i-\hbar\omega)$
  と近似できるので
  単位時間当たりの遷移確率は
  \begin{equation}
    \omega_{f\to i}=\frac{|c_{f,i}^1(t)|^2}{t}=\frac{2\pi}{\hbar}|V_{fi}|^2\delta(E_f-E_i-\hbar\omega)
  \end{equation}
  である.これもFermiの黄金律という.$E_f=E_i+\hbar\omega$へと遷移することがわかる.\\
  (ii)$\omega_{fi}+\omega\approx0$のとき
  式(\ref{3.6})の第1項が支配的となる.上記の議論を$\omega\to\omega$と置き換えて繰り返すと単位時間当たりの遷移確率として
  \begin{equation}
    \omega_{i\to f}=\frac{|c_{f,i}^1(t)|^2}{t}=\frac{2\pi}{\hbar}|V_{fi}|^2\delta(E_f-E_i+\hbar\omega)
  \end{equation}
  を得る.$E_f=E_i-\hbar\omega$へと遷移することがわかる.
\end{document}

    \section{電磁場中の電子}Z
      \documentclass{report}
\input{../../../head.tex}
\begin{document}
  以下ではまず,古典の電磁気学を考えよう.
  電磁場中の電子の運動方程式は,
  \begin{align}
    m\dv{\bm{v}}{t} = -e\qty(\bm{E} + \bm{v}\times\bm{B})
  \end{align}
  と書けるのであった.
  電磁場中の電子のハミルトニアンは,
  \begin{itembox}[l]{電磁場中の電子のハミルトニアン}
    \begin{align}
      H = \frac{1}{2m}\qty(\bm{p} + e\bm{A})^2 - e\phi
    \end{align}
  \end{itembox}
  である.ただし$\bm{A}$はベクトルポテンシャルで,
  \begin{align}
    \bm{E} &= -\pdv{\bm{A}}{t} - \grad\phi \\ 
    \bm{B} &= \curl\bm{A}
  \end{align}
  を満たす.
  本節では$U(1)$\footnote{Unitary}Gauge対称性を扱い説明し電磁場の起源を探る.
\end{document}

      \subsection{大域的Gauge変換}
        \documentclass{report}
\input{../../../head.tex}
\begin{document}
  まず,大域的Gauge変換を議論する.
  波動関数$\psi(\bm{r},t)$を,
  \begin{align}
    \psi(\bm{r},t) \to \e^{\i\alpha}\psi(\bm{r},t) \eqqcolon \psi'(\bm{r},t) 
  \end{align}
  のように変換することを大域的ゲージ変換という.
  $\psi(\bm{r},t) = \e^{-\i\alpha}\psi'(\bm{r}, t)$をSchrödinger方程式に代入すると,
  \begin{align}
    \i\hbar\pdv{t}\psi(\bm{r},t) &= -\frac{\hbar^2}{2m}\laplacian\psi(\bm{r},t) \\ 
    \Leftrightarrow \i\hbar\pdv{t}\e^{-\i\alpha}\psi'(\bm{r},t) &= -\frac{\hbar^2}{2m}\laplacian\e^{-\i\alpha}\psi'(\bm{r},t) \\ 
    \Leftrightarrow \e^{-\i\alpha}\i\hbar\pdv{t}\psi'(\bm{r},t) &= -\e^{-\i\alpha}\frac{\hbar^2}{2m}\laplacian\psi'(\bm{r},t) \\ 
    \Leftrightarrow \i\hbar\pdv{t}\psi'(\bm{r},t) &= -\frac{\hbar^2}{2m}\laplacian\psi'(\bm{r},t)
  \end{align}
  となる.
  つまり,大域的ゲージ変換によって,Schr\"odinger方程式は影響を受けない.
  当然,期待値も,
  \begin{align}
    \mel**{\psi'}{\hat{A}}{\psi'} &= \mel**{\psi}{\e^{-\i\alpha}\hat{A}\e^{\i\alpha}}{\psi} \\ 
    &= \mel**{\psi}{\hat{A}}{\psi}
  \end{align}
  となるから,物理は大域的ゲージ変換に対して不変であると言える.
\end{document}

      \subsection{局所的Gauge変換}
        \documentclass{report}
\input{../../../head.tex}
\begin{document}
  次に局所ゲージ変換を議論する.
  波動関数$\psi(\bm{r}, t)$を,
  \begin{align}
    \psi(\bm{r}, t) \to \psi'(\bm{r}, t) \coloneqq \e^{\i\alpha(\bm{r}), t}\psi(\bm{r}, t)
  \end{align}
  のように変換することを局所ゲージ変換という.
  局所ゲージ変換では,$\alpha$が$\bm{r}$に依存することに注意する.
  簡単な考察により,局所ゲージ変換にSchr\"odinger方程式は影響を受けることが分かる.
  \par
  例えば波動関数の勾配を計算すると,
  \begin{align}
    \grad\psi'(\bm{r}, t) &= \psi(\bm{r})\grad\e^{\i\alpha(\bm{r}, t)} + \e^{\i\alpha(\bm{r}, t)}\grad\qty(\psi(\bm{r}, t)) \\ 
    &= \i\qty(\grad\alpha(\bm{r}, t))  \e^{i\alpha(\bm{r}, t)}\psi(\bm{r}) + \e^{\i\alpha(\bm{r}, t)}\grad\psi(\bm{r}, t) \\ 
    &= \e^{\i\alpha(\bm{r}, t)}(\grad + \i\grad\alpha(\bm{r}, t))\psi(\bm{r}, t) 
  \end{align}
  となり,$\grad\psi$と$\grad\psi'$が一致しない.
  時間発展するSchr\"odinger方程式と物理量$\hat{A}$の期待値についても同様に計算を行えば,
  \begin{align}
    \begin{dcases}
      \i\hbar\pdv{t}\psi' \neq - \frac{\hbar^2}{2m}\laplacian\psi' \\
      \mel**{\psi}{\hat{A}}{\psi} \neq \mel**{\psi'}{\hat{A}}{\psi'}
    \end{dcases}
  \end{align}
  である.したがって,局所ゲージ変換に対して物理は不変ではない.そこで,\textbf{局所ゲージ不変性を基本原理とする物理を再構築する.}% モチベ??相対論?
  \par
  今まで用いていた物理の模型の欠点は,積の微分をおこなったときに余分な項が出てきてしまうことであった.
  そこで,局所ゲージ不変性を満たすように物理の模型を変更するために,微分の定義を変更することを考える.
  以下では,今まで考えていた模型において空間に対する微分であった勾配を,共変微分$\hat{\bm{D}}$で,時間微分を$\hat{D}_t$で書き換えることよって,局所ゲージ不変な物理模型を作る.
  $\hat{\bm{D}}$と$\hat{D}_t$を,
  \begin{align}
    \hat{\bm{D}} &\coloneqq \grad + \i\frac{e}{\hbar}\hat{\bm{A}} \label{hatd-def}\\ 
    \hat{D}_t &\coloneqq \pdv{t} - \i\frac{e}{\hbar}\phi \label{hatdt-def}
  \end{align}
  と定義する.
  また,$\hat{\bm{A}}$と$\phi$の局所ゲージ変換則を,
  \begin{align}
    \hat{\bm{A}} &\to \hat{\bm{A}} - \frac{\hbar}{e}\grad\alpha(\bm{r}, t) \eqqcolon \hat{\bm{A}}' \\ 
    \phi &\to \phi + \frac{\hbar}{e}\pdv{t}\alpha(\bm{r}, t) \eqqcolon \phi'  
  \end{align}
  と定義する.$\hat{\bm{D}}$と$\hat{D}_t$を局所ゲージ変換すると,
  \begin{align}
    \hat{\bm{D}} &\to \hat{\bm{D}}' = \grad + \i\frac{e}{\hbar}\hat{\bm{A}'} = \grad + \i\frac{e}{\hbar}\bm{\hat{A}} - \i\grad\alpha(\bm{r}, t) \\ 
    \hat{D}_t &\to \hat{D}_t' = \pdv{t} - \i\frac{e}{\hbar}\phi' = \pdv{t} - \i\frac{e}{\hbar}\phi - \i\pdv{t}\alpha(\bm{r}, t)
  \end{align}
  となる.
  このように微分を定義すると,
  \begin{align}
    \hat{\bm{D}}'\psi'(\bm{r}, t) &= \e^{\i\alpha(\bm{r}, t)}\hat{\bm{D}}\psi(\bm{r}, t) \label{d-prime}\\ 
    \hat{\bm{D}}'^2\psi'(\bm{r}, t) &= \e^{\i\alpha(\bm{r}, t)}\hat{\bm{D}}^2\psi(\bm{r}, t) \label{d-prime-power}\\ 
    \hat{D}_t'\psi'(\bm{r}, t) &= \e^{\i\alpha(\bm{r}, t)}\hat{D}_t\psi(\bm{r}, t) \label{dt-prime}
  \end{align}
  となることが確かめられる.
  \par
  \refe{d-prime-power}や\refe{dt-prime}を使いながら,元のSchr\"odinger方程式において,勾配を,$\hat{\bm{D}}$で,時間微分を$\hat{D}_t$で書き換えると,
  \begin{align}
    \i\hbar D'_t\psi' &= -\frac{\hbar^2}{2m}\bm{D}'^2\psi' \\
    \Leftrightarrow \e^{\i\alpha(\bm{r}, t)}\i\hbar D_t\psi &= -\e^{\i\alpha(\bm{r}, t)}\frac{\hbar^2}{2m}\bm{D}^2\psi\label{local-gauge-psi}
  \end{align}
  となり,局所ゲージ変換に対して不変なSchrödinger方程式となった.
  また,\refe{local-gauge-psi}において\refe{hatd-def}と\refe{hatdt-def}を\refe{local-gauge-psi}に代入すると,
  \begin{itembox}[l]{局所ゲージ変換に対して不変なSchrödinger方程式}
    \begin{align}
      \i\hbar\pdv{t}\psi = \qty[\frac{1}{2m}\qty(\hat{\bm{p}} + e\hat{\bm{A}})^2 - e\phi]\psi
    \end{align}
  \end{itembox}
  を得る.
  \par
  以上の流れをまとめると,局所ゲージ不変性を要請することにより,ゲージ場$\bm{A}$が導入された.逆に,電磁場の起源は局所ゲージ不変性であるといえる.

  \begin{myexc}{電磁場中のSchrödinger方程式}{}
  \begin{enumerate}
    \item 共変微分
      \begin{align}
        \hat{\bm{D}} = \nabla + \i\frac{e}{\hbar}\bm{A}
      \end{align}
      において,
      \begin{align}
        \hat{\bm{D}} \times \hat{\bm{D}} = \i\frac{e}{\hbar}\bm{B}
      \end{align}
      が成り立つことを示せ.
    \item 電磁場中のSchrödinger方程式
    \begin{align}
      \i\hbar\pdv{t}\psi = \qty[\frac{1}{2m}\qty(\hat{\bm{p}} + e\hat{\bm{A}})^2 - e\phi]\psi
    \end{align}
    においては
    \begin{align}
      \frac{\partial}{\partial t} \rho + \nabla\cdot \bm{j} = 0 \label{prob-current-exercise}
    \end{align}
    が成り立たない.ここで,
    \begin{align}
      \bm{j} = \frac{\i\hbar}{2m}\qty[(\nabla \psi^{*})\psi - \psi^{*} (\nabla\psi)],\ \rho = \psi^{*}\psi
    \end{align}
    である.\refe{prob-current-exercise}が成り立つように確率の流れを再定義せよ.
  \end{enumerate}
  \tcblower
  \begin{enumerate}
    \item \begin{align}
      (\hat{\bm{D}} \times \hat{\bm{D}})\psi &= \qty(\nabla + \i\frac{e}{\hbar}\bm{A}) \times \qty(\nabla + \i\frac{e}{\hbar}\bm{A})\psi\\
      &= \qty(\nabla\times\nabla + \i\frac{e}{\hbar}\bm{A}\times\nabla + \i\frac{e}{\hbar}\nabla\times\bm{A} - \frac{e^2}{\hbar^2})\psi\\
      &= \i\frac{e}{\hbar}\bm{A}\times(\nabla\psi) + \i\frac{e}{\hbar}(\nabla\psi)\times\bm{A} + \i\frac{e}{\hbar}(\nabla\times\bm{A})\psi\\
      &= \bm{B}\psi
    \end{align}
    \item まず,確率密度の時間微分は
      \begin{align}
        \frac{\partial}{\partial t}\rho = \qty(\frac{\partial}{\partial t}\psi^{*}) + \psi^{*}\qty(\frac{\partial}{\partial t}\psi)
      \end{align}
      である.次に,電磁場中のSchrödinger方程式より,
      \begin{align}
        \i\hbar \frac{\partial}{\partial t} \psi &= -\frac{\hbar^2}{2m}\qty(\nabla + \i\frac{e}{\hbar}\bm{A})^2\psi - e\phi\psi\\
        \i\hbar \frac{\partial}{\partial t} \psi^{*} &= \frac{\hbar^2}{2m}\qty(\nabla - \i\frac{e}{\hbar}\bm{A})^2\psi^{*} - e\phi\psi^{*}
      \end{align}
      を得る.ここで,
      \begin{align}
        \qty(\nabla - \i\frac{e}{\hbar}\bm{A})^2\psi^{*} &= \nabla^2\psi^{*} - \i\frac{e}{\hbar}\bm{A}\cdot(\nabla\psi^{*}) - \i\frac{e}{\hbar}(\nabla\psi^{*})\cdot\bm{A} - \i\frac{e}{\hbar}(\nabla\cdot\bm{A})\psi^{*} - \frac{e^2}{\hbar^2}\psi^{*}\\
        &= \nabla^2\psi^{*} - 2\i\frac{e}{\hbar}\bm{A}\cdot(\nabla\psi^{*}) - \i\frac{e}{\hbar}(\nabla\cdot\bm{A})\psi^{*} - \frac{e^2}{\hbar^2}\psi^{*}\\
        \qty(\nabla + \i\frac{e}{\hbar}\bm{A})^2\psi &=\nabla^2\psi + 2\i\frac{e}{\hbar}\bm{A}\cdot(\nabla\psi) + \i\frac{e}{\hbar}(\nabla\cdot\bm{A})\psi - \frac{e^2}{\hbar^2}\psi
      \end{align}
      である.よって,
      \begin{align}
        2m\i\frac{\partial}{\partial t}\rho &= -\nabla\cdot\qty[\psi^{*}\nabla\psi - (\nabla\psi^{*})\psi - 2\i \frac{e}{\hbar}\bm{A}\psi^{*}\psi]\\
        &= -\nabla\cdot\qty[\psi^{*}(\nabla - \i\frac{e}{\hbar} \bm{A})\psi - ((\nabla - \i \frac{e}{\hbar}\bm{A})\psi)^{*}\psi]
      \end{align}
      となるため,
      \begin{align}
        \bm{j} = \frac{\i}{2m}\qty[\psi^{*}(\bm{D}\psi) - (\bm{D}^{*}\psi^{*})\psi]
      \end{align}
      と定義すればよいことがわかる.
  \end{enumerate}
  \end{myexc}
\end{document}
      \subsection{電磁場中の電子の摂動論}
        \documentclass{report}
\input{../../../head.tex}
\begin{document}
  \refe{hamiltonian-in-electro-magnetic-field-cm}より,電磁場中のハミルトニアンは,
  \begin{align}
    H = \frac{1}{2m}\qty(\bm{p} + e\bm{A})^2 - e\phi\label{hamiltonian-in-electro-magnetic-field-cm-re}
  \end{align}
  と書けるのであった.
  今回は$\phi = 0$とする.
  電磁場が十分弱いという条件のもと,\refe{hamiltonian-in-electro-magnetic-field-cm-re}を量子化すると,
  \begin{align}
    \hat{H} &= \frac{1}{2m}\qty(\hat{\bm{p}} + e\hat{\bm{A}})^2 \\
    &\simeq \frac{1}{2m}\qty(\hat{\bm{p}}^2 + e\hat{\bm{p}}\cdot\hat{\bm{A}} + e\hat{\bm{A}} \cdot \hat{\bm{p}})\\
  \end{align}
  と近似する.ここで,$\hat{H}^{(0)} \coloneqq \frac{\bm{p}^2}{2m}$,摂動項$\hat{V}(t) \coloneqq \frac{e}{2m} \qty(\hat{\bm{p}}\cdot\hat{\bm{A}} + \hat{\bm{A}} \cdot \hat{\bm{p}})$
  とする.さらに,$\div\bm{A} = 0$となるように$\bm{A}$を決める\footnote{Coulomb ゲージ}.すると,
  \begin{align}
    \qty(\hat{\bm{p}} \cdot \bm{A})\psi &= -\i\hbar \div\qty(\hat{\bm{A}} \psi)\\
    &= \i\hbar\qty[\qty(\div\hat{\bm{A}}) \psi + \hat{\bm{A}} \cdot \qty(\grad \psi)]\\
    &= \qty(\hat{\bm{A}} \cdot \hat{\bm{p}}) \psi
  \end{align}
  となる.よって,電子と電磁場の相互作用を摂動として加えたハミルトニアン,
  \begin{align}
    \hat{H} = \hat{H}^{(0)} + \frac{e}{m}\qty(\hat{\bm{A}} \cdot \hat{\bm{p}})
  \end{align}
  を得る.
  \begin{myex}{直線偏光}{}
    ベクトルポテンシャル$\hat{A}$が,
    \begin{align}
      \hat{\bm{A}}(\bm{r}, t) = 2A_0 \bm{e}_x \cos(\bm{k} \cdot \bm{r} - \omega t)
    \end{align}
    であるときを考える.
    ただし,$\bm{k} = \frac{\omega}{c}\bm{e}_z$とする.
    ベクトルポテンシャルと電磁場の関係より,
    \begin{align}
      \begin{cases}
      \bm{E}(\bm{r}, t) = \pdv{\hat{\bm{A}}}{t} = E_0 \bm{e}_x \sin(\bm{k} \cdot \bm{r} - \omega t)\\
      \bm{B}(\bm{r}, t) = \curl\bm{A} = \frac{E_0}{c}\bm{e}_y \sin(\bm{k} \cdot \bm{r} - \omega t)\\
      E_0 = -2 \omega A_0
      \end{cases}
    \end{align}
    が成り立っている.摂動項は,
    \begin{align}
      \hat{V}(t) &= \frac{e}{m}(\hat{\bm{A}}\cdot\hat{\bm{p}}) \\
      &= \frac{2eA_0}{m}\cos(\bm{k} \cdot \bm{r} - \omega t)\bm{e}_x\cdot\hat{\bm{p}} \\
      &= \frac{eA_0}{m}\qty[\e^{\i(\bm{k}\cdot\bm{r} - \omega t)} + \e^{-\i(\bm{k}\cdot\bm{r} - \omega t)}]\hat{p}_x\\
      &= \frac{eA_0}{m}\qty(\e^{\i\bm{k}\cdot\bm{r}}\hat{p}_x\e^{-\i\omega t} + \e^{-\i \bm{k}\cdot\bm{r}}\hat{p}_x \e^{\i \omega t})
    \end{align}
    と表せる.光の吸収を考えるときは,第1項$\frac{eA_0}{m}\e^{\i\bm{k}\cdot\bm{r}}\hat{p}_x$が支配的なのでこの項を$\hat{V}$とする.
    単位時間当たりの遷移確率を計算する.
    \refe{fermis-golden-rule}より,$\omega_{i\to f}$は,
    \begin{align}
      \omega_{i\to f} &= \frac{2\pi}{\hbar} \abs{\mel**{f}{\hat{V}}{i}}^2\delta(E_f - E_i - \hbar \omega) \\
      &= \frac{2 \pi}{\hbar}\qty(\frac{e A_0}{m})^2 \abs{\mel**{f}{\e^{\i\bm{k}\cdot \bm{r}}\hat{p}_x}{i}}^2 \delta(E_f - E_i - \hbar \omega)
    \end{align}
    と表せる.
    さて,$\abs{\mel**{f}{\e^{\i\bm{k}\cdot\bm{r}}\hat{p}_x}{i}}^2$を\textbf{電気双極子近似}を用いて計算する.
    電気双極子近似とは,電磁場の変化の項である$\e^{\i\bm{k}\cdot\bm{r}}$を$1$とみなす近似である.
    これは次のような議論から正当化される.
    原子の準位間隔は$E_f - E_i \sim 1\ \r{eV}$である.これと相互作用する電磁場のエネルギーは$\hbar \omega\sim 1\ \r{eV}$ である.これを波長に換算すると,
    \begin{align}
      \lambda = \frac{2\pi}{\hbar} = \frac{2\pi c}{\omega} \sim 1000\ \r{nm}
    \end{align}
    である.これは原子のスケール$1\ \si{\angstrom}$よりもはるかに大きいため,電子・原子を扱う上では電磁場は空間的に一様だとみなせる.よって,
    \begin{align}
      \e^{\i\bm{k}\cdot\bm{r}}\simeq 1 + \i \bm{k} \cdot \bm{r} + \cdots \simeq 1
    \end{align}
    と近似できる.
    \par
    電気双極子近似を用いると,
    \begin{align}
      \mel**{f}{\e^{\i \bm{k} \cdot \bm{r}}\hat{p}_x}{{i}} \simeq \mel**{f}{\hat{p}_x}{i}
    \end{align}
    を得る.さらに
    \begin{align}
      \qty[\hat{x}, \hat{H}^{(0)}] = \frac{\i \hbar}{m}\hat{p}_x
    \end{align}
    であるため
    \begin{align}
      \mel**{f}{\hat{p}_x}{i} &= \frac{m}{\i\hbar} \mel**{f}{\qty[\hat{x}, \hat{H}^{(0)}]}{i} \\ 
      &= \frac{m}{\i\hbar} \qty(\mel**{f}{\hat{x}\hat{H}^{(0)}}{i} - \mel**{f}{\hat{H}^{(0)}\hat{x}}{i}) \\ 
      &= \frac{m}{\i\hbar} \qty(\mel**{f}{\hat{x}E_i}{i} - \mel**{f}{E_f\hat{x}}{i}) \\ 
      &= \frac{m}{\i\hbar} (E_i - E_f)\mel**{f}{\hat{x}}{i}
    \end{align}
    を得る.よって,電磁場による単位時間当たりの遷移確率
    \begin{align}
      \omega_{i \to f} = \frac{2\pi}{\hbar^3} (eA_0)^2 (E_i - E_f)^2 \abs{\mel**{f}{\hat{x}}{i}}^2 \delta(E_f - E_i - \hbar\omega) 
    \end{align}
    を得る.これは$\mel**{f}{\hat{x}}{i} \neq 0$のときのみ$\omega_{i\to f} \neq 0$であるという選択則を表している.
  \end{myex}
\end{document}
  \chapter{散乱理論}
    \section{立体角}
      \documentclass{report}
\input{../../head.tex}
\begin{document}
  今後の議論のために\textbf{立体角}を導入する.
  立体角とは1点を中心としたときの広がり具合を表す指標である.
  2次元の場合,微小円弧と半径の比は角度にのみ依り,$r$に依らない.
  つまり,
  \begin{align}
    \frac{dl_1}{r_1}=\frac{dl_2}{r_2}=\frac{d\theta}{1}
  \end{align}
  が成り立つ.この関係から平面角$d\theta=\frac{dl}{r}$を定義する.これを3次元に拡張する.図\ref{solid_angle}に模式図を示す.単位球上の面積を考えると,
  \begin{align}
    \frac{dS_1}{r_1^2}=\frac{dS_2}{r_2^2}=\frac{dS}{1}
  \end{align}
  である.ここから立体角を$d\Omega=\frac{dS}r^2$と定義する.また,これを球座標表示に変換すると,
  \begin{align}
    d\Omega = \frac{dS}{r^2} = \frac{r^2\sin\theta d\theta d\phi}{r^2} = \sin \theta d\theta d\phi
  \end{align}
  である.

  \begin{figure}[htbp]
    \centering
    \includegraphics[width=0.6\columnwidth]{fig/solid_angle.pdf}
    \caption{立体角}
    \label{solid_angle}
  \end{figure}

\end{document}
    \section{散乱断面積}
      \documentclass{report}
\input{../../head.tex}
\begin{document}
  単位時間単位面積当たり$N$個の粒子を$z$軸方向に入射する.
  \refe{scattering_detection}に散乱の様子を示す.
  単位時間内に散乱体からの位置$(r,\theta,\phi)$にある面積$\dd{S}$の検出器に到達する粒子数は,
  \begin{align}
    \dd{N} \propto N \frac{\dd{S}}{r^2} = N\dd{\Omega}
  \end{align}
  を満たす.比例係数を$\sigma (\theta, \phi)$とすると,
  \begin{align}
    \dd{N} = \sigma(\theta, \phi) N \dd{\Omega}
  \end{align}
  と書ける.
  $\sigma(\theta, \phi)$を\textbf{微分断面積}という.
  散乱が$z$軸まわりに軸対称なとき,$\phi$依存性を取り除き$\sigma(\theta, \phi) = \sigma (\theta)$とできる.
  また,全断面積を,
  \begin{align}
    \sigma^{\r{tot}} \coloneqq \int \sigma(\theta) \dd{\Omega} = 2\pi \int_{0}^{\pi} \sigma(\theta) \sin \theta \dd{\theta}\label{total-area}
  \end{align}
  のように定義する.
  \begin{figure}[H]
    \centering
    \includegraphics[width = 0.6\columnwidth]{fig/scattering_diagram.pdf}
    \caption{散乱体と検出器}\label{scattering_detection}
  \end{figure}
\end{document}
    \section{古典力学における散乱}
      \documentclass{report}
\input{../../head.tex}
\begin{document}
  $z$軸に沿って粒子を単位時間単位面積当たり$n$個入射する.
  \begin{figure}[H]
    \centering
    \includegraphics[width=0.7\columnwidth]{fig/scattering_cm.pdf}
    \caption{古典力学における散乱}
    \label{sc-in-cm-image}
  \end{figure}
  $z$軸から距離$b$ (衝突パラメータ),角度$\dd{\phi}$,面積$\dd{S'}$のスリットを単位時間当たりに通過する粒子数は,
  \begin{align}
    n\dd{S'} = n \dd{\phi} (b \dd{b})
  \end{align}
  を満たす.
  また,単位時間に検出器に到達する粒子数は微分断面積の定義から,
  \begin{align}
    \dd{N} = \sigma(\theta) n \dd{\Omega}
  \end{align}
  である.
  古典力学ではスリットを通過した粒子は,必ず検出器で検出されるので,
  \begin{align}
    n\dd{S'} &= \dd{N} \\ 
    \Leftrightarrow \sigma (\theta) n \dd{\Omega} &= n \dd{\phi} (b \dd{\phi})
  \end{align}
  を得る.よって,微分断面積は,
  \begin{align}
    \sigma(\theta) = \frac{1}{\sin \theta} b \abs{\frac{\dd{b}}{\dd{\theta}}}
  \end{align}
  と表される.
  \begin{myex}{剛体球}{}
    散乱体を半径$a$の剛体球のポテンシャル$V(r)$を,
    \begin{align}
      V(r) = 
      \begin{dcases}
        \infty & r < a \\
        0 & r > a
      \end{dcases}
    \end{align}
    とする.
    衝突パラメータを$b$として,粒子が散乱体の角度$\phi$の位置で散乱し,その散乱角を$\theta$とする.
    これらのパラメータは,
    \begin{align}
      \begin{dcases}
        2\phi + \theta = \pi\\
        b = a\sin \phi
      \end{dcases}
    \end{align}
    を満たすため,
    \begin{align}
      b = a\cos\qty(\frac{\theta}{2})
    \end{align}
    を得る.よって,微分断面積は,
    \begin{align}
      \sigma(\theta) &= \frac{1}{\sin \theta} b \abs{\dv{b}{\theta}} \\
      &= \frac{a\cos\qty(\frac{\theta}{2})}{\sin\theta}\abs{-\frac{a}{2}\sin\qty(\frac{\theta}{2})} \\ 
      &= \frac{a^2}{4}
    \end{align}
    となる.$\theta$に依存しない等方散乱であることがわかる.また,全断面積は,
    \begin{align}
      \sigma^{\r{tot}} = \int \sigma(\theta) \dd{\Omega} = \pi a^2
    \end{align}
    である.剛体球の断面積と一致する.
  \end{myex}
\end{document}
    \section{量子力学における散乱}
      \documentclass{report}
\input{../../head.tex}
\begin{document}
  本節では量子力学的に散乱を議論する.量子力学では波動関数を用いて散乱体の様子を調べる.
  まず,散乱帯のポテンシャルを紹介して,散乱振幅$f(\theta)$を求めればいいことを知る.
  次に,散乱振幅と微分断面積の関係を調べる.
  最後に,具体的に散乱振幅の表式を導く.
  \subsection{散乱振幅}
    散乱体が球対称ポテンシャル$V(r)$を持つとする.Schrödinger方程式は
    \begin{align}
      \qty(-\frac{\hbar^2}{2m}\laplacian + V(r))\psi(\bm{r}) = E \psi(\bm{r})\label{se-scattering}
    \end{align}
    である.
    ここで,$V(r)$は$r \to \infty$で$r^{-1}$より早く$V \to 0$となるとする.
    なお,$r \to \infty$で$r^{-1}$より早く$0$に収束することを,十分早く収束するということにする.
    $z$軸に沿う入射波は平面波なので,
    \begin{align}
      \psi_{\r{in}} = \e^{\i k z}\ (z \to \infty)
    \end{align}
    と表せる.
    また,散乱波は外向きの球面波となるので,
    \begin{align}
      \psi_{\r{sc}} \simeq \frac{\e^{\i kr}}{r}\ (r \to \infty)
    \end{align}
    である.散乱問題とは,$r \to \infty$で
    \begin{align}
      \psi(\bm{r}) &= \psi_{\r{in}} + \psi_{\r{sc}} \\ 
      &= \e^{\i kz} + f(\theta) \frac{\e^{\i kr}}{r}
    \end{align}
    を満たす\refe{se-scattering}の定常解を求めることである.$f(\theta)$を散乱振幅という.上式は重要なので強調しておく.
    \begin{itembox}[l]{散乱問題の境界条件}
      \begin{equation}
        \psi(\bm{r}) = \e^{\i kz} + f(\theta) \frac{\e^{\i kr}}{r}
      \end{equation}
    \end{itembox}
  \subsection{散乱振幅と微分断面積の関係}
    この問題を解くための準備として確率密度$\rho \coloneqq \psi^{*} \psi$の時間変化を考える.
    時間変化するSchr\"odinger方程式より,
    \begin{align}
      \i\hbar\pdv{\psi}{t} &= \hat{H}\psi \\ 
      \Leftrightarrow \pdv{\psi}{t} &= \frac{1}{\i\hbar}\hat{H}\psi \\ 
      \Leftrightarrow \pdv{\psi^*}{t} &= -\frac{1}{\i\hbar}\hat{H}\psi^*
    \end{align}
    であることを用いると,
    \begin{align}
      \pdv{t}\rho &= \pdv{t}(\psi^{*}\psi) \\
      &= \pdv{\psi^*}{t}\psi + \psi^{*}\pdv{\psi}{t} \\
      &= -\frac{1}{\i\hbar}\qty[\qty(-\frac{\hbar^2}{2m}\laplacian + V)\psi^{*}]\psi + \psi^{*}\frac{1}{\i\hbar}\qty[\qty(-\frac{\hbar^2}{2m}\laplacian + V)\psi] \\
      &= \frac{\hbar}{2\i m}\qty[\qty(\laplacian \psi^{*})\psi - \psi^{*}\qty(\laplacian\psi)] \\
      &= -\frac{\hbar}{2\i m}\div \qty(\psi^{*}\grad\psi - \psi \grad\psi^{*})\label{prob-density}
    \end{align}
    確率流密度を,
    \begin{align}
      \bm{j} &\coloneqq \frac{\hbar}{2\i m} \qty(\psi^{*}\grad\psi - \psi \grad\psi^{*})\label{prob-flow} \\
      &= \frac{\hbar}{m}\Im \qty(\psi^{*}\grad\psi)
    \end{align}
    と定義する.
    \refe{prob-flow}を\refe{prob-density}に代入すると,確率密度に対する連続の式である,
    \begin{align}
      \pdv{t} \rho = -\div\bm{j}
    \end{align}
    を得る\footnote{両辺に電荷をかければ電荷保存の法則となる. }.
    \par
    次に微分断面積と散乱振幅の関係を考える.
    $z$軸に平行に入射しているので入射波は,
    \begin{align}
      \psi_{\r{in}} = \e^{\i kz}
    \end{align}
    と書けるのであった.
    入射波は$z$軸方向にしか存在しないので,その確率流密度の$z$成分$j_z$は\refe{psi-in-grad}より,
    \begin{align}
      j_z &= \frac{\hbar}{m}\Im \qty(\psi^{*}\pdv{z}\psi) \\ 
      &= \frac{\hbar}{m} \Im \qty(\e^{-\i kz} \i k e^{\i kz}) \\
      &= \frac{\hbar k}{m}
    \end{align}% メモ: 図にr軸を追加する.
    である.
    \par
    散乱波は,
    \begin{align}
      \psi_{\r{sc}} = \frac{f(\theta)}{r}\e^{\i kr}
    \end{align}
    と書けるのであった.
    散乱波は$r$軸方向にしか存在しないので,その確率流密度の$r$成分$j_r(\theta)$は,
    \begin{align}
      j_r(\theta) &= \frac{\hbar}{m}\Im \qty(\psi^{*}_{\r{sc}}\pdv{r}\psi_{\r{sc}}) \\
      &= \frac{\hbar}{m} \Im \qty[\frac{f(\theta)^*}{r} \e^{-\i kr}\qty(\frac{\i k}{r} - \frac{1}{r^2})f(\theta)\e^{\i kr}] \\
      &= \frac{\hbar}{m} \Im \qty[-\frac{\abs{f(\theta)}^2}{r^3} + \i k\frac{\abs{f(\theta)}^2}{r^2}] \\
      &= \frac{\hbar}{m} k\frac{\abs{f(\theta)}^2}{r^2} \\
      &= \frac{\abs{f(\theta)}^2}{r^2}j_z
    \end{align}
    である.
    \par
    さて,微分断面積と散乱振幅の関係について考えよう.
    微分断面積と粒子数の関係の両辺を$n$で割ると,
    \begin{align}
      \dd{N} &= \sigma (\theta) n \dd{\Omega} \\ 
      \Leftrightarrow \frac{\dd{N}}{n} &= \sigma (\theta) \dd{\Omega}
    \end{align}
    となる.$\frac{\dd{N}}{n}$は,
    \begin{align}
      \frac{\dd{N}}{n} &= \frac{\qty(\text{単位時間に位置$(r,\theta)$にある$検出器\dd{S}$に入射する粒子数})}{\qty(\text{単位時間単位面積当たりの入射粒子数})} \\ 
      &= \frac{\qty(\text{単位時間に位置$(r,\theta)$にある検出器$\dd{S}$に粒子が入射する確率})}{\qty(\text{単位時間単位面積当たりに粒子が入射する確率})} \\ 
      &= \frac{j_r \dd{S}}{j_z}
    \end{align}
    と書けるので,
    \begin{align}
      \sigma(\theta)\dd{\Omega} = \abs{f(\theta)}^2\frac{\dd{S}}{r^2}
    \end{align}
    が得られる.
    \refe{solid-angle-def}より,$\dd{\Omega} = \frac{\dd{S}}{r^2}$であるから,
    \begin{align}
      \sigma(\theta) = \abs{f(\theta)}^2
    \end{align}
    という関係が成り立つ.これは,散乱振幅から微分断面積が求められることを意味する.
    \begin{itembox}[l]{散乱振幅と微分断面積の関係}
      \begin{align}
        \sigma(\theta) = \abs{f(\theta)}^2
      \end{align}
    \end{itembox}
  \subsection{散乱振幅の表式}
    最後に散乱振幅$f(\theta)$の表式を求める.
    散乱のSchrödinger方程式は,
    \begin{align}
      \qty(-\frac{\hbar^2}{2m}\laplacian + V(r))\psi(\bm{r}) = E\psi(\bm{r})
    \end{align}
    であった.
    $\kappa$,$U(\bm{r})$を,
    \begin{align}
        \kappa &\coloneqq \frac{\sqrt{2mE}}{\hbar} \\
        U(\bm{r}) &\coloneqq \frac{2m}{\hbar^2} V(\bm{r})
    \end{align}
    と定義すると,
    \begin{align}
      \qty(\laplacian + \kappa^2)\psi(\bm{r}) = U(\bm{r}) \psi(\bm{r}) \label{helmholtz-eq}
    \end{align}
    と表せる.\refe{helmholtz-eq}の解は,Helmholtz方程式型の斉次式,
    \begin{align}
      \qty(\laplacian + \kappa^2)\phi(\bm{r}) = 0
    \end{align}
    の$z$方向に入射した場合の一般解$\phi(\bm{r}) = \e^{\i kz}$と,
    非斉次の方程式,
    \begin{align}
      \qty(\laplacian + \kappa^2)\chi(\bm{r}) = U(\bm{r}) \chi(\bm{r})\label{particular-solution}
    \end{align}
    の解となる$\chi(\bm{r})$の和である.
    \par
    では,\refe{particular-solution}の特解を求めよう.レシピはこうである.
    \begin{enumerate}
      \item $\qty(\laplacian + \kappa^2)G_0(\bm{r}) = \delta(\bm{r})$を満たすGreen関数$G_0$を求める.
      \item $\chi(\bm{r}) = \int\dd{\bm{r'}} G_0(\bm{r} - \bm{r'}) U(\bm{r'}) \psi(\bm{r'})$から特解を求める.
    \end{enumerate}
    2. から特解が求められるのは,
    \begin{align}
      \qty(\laplacian + \kappa^2)\chi(\bm{r}) &= \int \dd{\bm{r'}}\qty(\laplacian + \kappa^2)G_0(\bm{r} - \bm{r'}) U(\bm{r'}) \psi(\bm{r'}) \\
      &= \int\dd{\bm{r'}} \delta(\bm{r} - \bm{r'}) U(\bm{r'}) \psi(\bm{r'}) \\
      &= U(\bm{r}) \psi(\bm{r})
    \end{align}
    が成り立つからだ.
    \par
    まず,1. のステップで定義したGreen関数の関係式,
    \begin{align}
      \qty(\laplacian + \kappa^2)G_0(\bm{r}) = \delta(\bm{r})
    \end{align}
    の両辺をFourier変換すると,
    \begin{align}
      \int\dd{\bm{r}} \qty(\laplacian + \kappa^2) G_0(\bm{r}) \e^{-\i\bm{k} \cdot \bm{r}} &= \frac{1}{(2\pi)^3}\int\dd{\bm{r}} \delta(\bm{r})\e^{-\i\bm{k}\cdot\bm{r}} \\
      \Leftrightarrow \qty[(-\i\bm{k})^2 + \kappa^2] \int\dd{\bm{r}} G_0(\bm{r}) \e^{-\i \bm{k}\cdot \bm{r}} &= \frac{1}{(2\pi)^3}
    \end{align}
    となる.Green関数$G_0(\bm{r})$のFourier変換を$\tilde{G}_0(\bm{k})$と書くと,
    \begin{align}
      \tilde{G}_0(\bm{k}) = \frac{1}{(2\pi)^3}\frac{1}{\kappa^2 - k^2}
    \end{align}
    を得る.よってGreen関数は$\tilde{G}_0(\bm{k})$を逆Fourier変換して,
    \begin{align}
      G_0(\bm{r}) &= \int\dd{\bm{k}} \frac{1}{(2\pi)^3}\frac{\e^{\i\bm{k} \cdot \bm{r}}}{\kappa^2 - k^2}\label{g-tilde-to-gr}
    \end{align}
    と表される.極座標に変換し\refe{g-tilde-to-gr}の積分を実行すると,
    \begin{align}
      G_0(\bm{r}) &= \int\dd{\bm{k}} \frac{1}{(2\pi)^3}\frac{\e^{\i\bm{k} \cdot \bm{r}}}{\kappa^2 - k^2}\\ 
      &= \frac{1}{(2\pi)^3} \int\dd{\bm{k}} \frac{\exp\qty(\i kr\cos\theta)}{\kappa^2 - k^2} \\ 
      &= \frac{1}{(2\pi)^3} \int_{\phi = 0}^{2\pi}\dd{\phi}\int_{k = 0}^{\infty}\dd{r}\int_{\theta = 0}^{\pi}\dd{\theta} \frac{\exp(\i kr \cos \theta)}{\kappa^2 - k^2} k^2\sin\theta \\
      &= \frac{1}{(2\pi)^3} 2\pi \int_{k = 0}^{\infty} k^2 \dd{k} \int_{\theta = 0}^{\pi}\dd{\theta} \sin\theta\frac{\exp(\i k r\cos\theta)}{\kappa^2 - k^2} \\
      &= \frac{1}{(2\pi)^2} \int_{0}^{\infty}k^2\frac{\e^{\i kr} - \e^{-\i kr}}{\i kr\qty(\kappa^2 - k^2)}\dd{k} \\
      &= \frac{1}{4\pi^2\i r} \int_{0}^{\infty}k \frac{\e^{\i kr} - \e^{-\i kr}}{(\kappa - k)(\kappa + k)}\dd{k} \\
      &= \frac{1}{8\pi^2\i r} \int_{-\infty}^{\infty}k \frac{\e^{\i kr} - \e^{-\i kr}}{(\kappa - k)(\kappa + k)}\dd{k} \\
      &= \frac{1}{8\pi^2\i r} \int_{-\infty}^{\infty}k \qty[\frac{\e^{\i kr}}{(\kappa - k)(\kappa + k)} - \frac{\e^{-\i kr}}{(\kappa - k)(\kappa + k)}]\dd{k} \\
      &= \frac{1}{4\pi^2 \i r}I
    \end{align}
    となる.ただし,$I$は,
    \begin{align}
      I \coloneqq \int_{-\infty}^{\infty} \frac{k \e^{\i kr}}{(\kappa - k)(\kappa + k)} \dd{k}
    \end{align}
    である.
    式変形の途中で,
    \begin{align}
      -\int_{-\infty}^{\infty}k \frac{\e^{-\i kr}}{(\kappa - k)(\kappa + k)}\dd{k} &= -\int_{\infty}^{-\infty}\qty(-k) \frac{\e^{-\i \qty(-k)r}}{\qty(\kappa - \qty(-k))\qty(\kappa + \qty(-k))}\qty(-\dd{k}) \\ 
      &= \int_{-\infty}^{\infty}k \frac{\e^{\i kr}}{(\kappa - k)(\kappa + k)}\dd{k} = I
    \end{align}
    なる関係を用いた.
    $I$を計算するには留数定理を用いる.
    \begin{itembox}[l]{留数定理}
      複素関数$f(z)$が閉経路$C$内に$m$個の特異点$b_1, \cdots b_m$を持つとすると,
      \begin{align}
        \oint_C f(z) \dd{z} = 2\pi\i\sum_{k = 1}^{m}\r{Res}\qty(b_k; f)
      \end{align}
      が成立する.
    \end{itembox}
    今回の場合は特異点は$\kappa$と$-\kappa$である.
    しかし,極の避け方にはいくつかのパターンがあり,それに応じてGreen関数の計算結果は複数存在する.
    これは何ら不自然なことではない.
    今計算しているのは,$\qty(\laplacian + \kappa)$なる演算子の固有関数$G_0$を求めているので,複数存在してもよい.
    まず,\reff{Integral}のように$-\kappa$を上に避け,$\kappa$を下に避けると,
    積分経路$C$で囲まれている領域に存在する特異点は,$k = \kappa$のみであるので,
    \begin{align}
      I &= \int_{-\infty}^{\infty} \frac{k \e^{\i kr}}{(\kappa - k)(\kappa + k)} \dd{k} \\ 
      &= 2\pi\i\frac{\e^{\i\kappa r}}{2} \\ 
      &= \pi\i\e^{\i\kappa r}
    \end{align}
    となるから,
    \begin{align}
      G_0(\bm{r}) &= \frac{1}{4\pi^2 \i r}I \\ 
      &= \frac{\e^{\i\kappa r}}{4\pi r}
    \end{align}
    となる\footnote{
      積分経路$C$に$-\kappa$と$\kappa$がそれぞれ含まれるか否かを分類して求めたGreen関数は,
      \begin{align}
        G_0(\bm{r}) = 
        \begin{dcases}
          \text{不適} & $-\kappa$: 含まない,$\kappa$: 含まない \\ 
          \frac{\e^{\i\kappa r}}{4\pi r} & $-\kappa$: 含まない,$\kappa$: 含む \\ 
          \frac{\e^{-\i\kappa r}}{4\pi r} & $-\kappa$: 含む,$\kappa$: 含まない \\ 
          \i\frac{\sin\qty(\kappa r)}{4\pi r} & $-\kappa$: 含む,$\kappa$: 含む
        \end{dcases}
      \end{align}
      $-\kappa$と$\kappa$をともに含まない経路で計算したとき,$I = 0$となるが,固有関数は0であってはいけないから,不適である.
    }.
    計算しているのは,外向き球面波解であるため,\reff{Integral}で示した経路で積分した結果が最も合理的であるから,このGreen関数を採用して以下の計算を行う.
    Green関数を用いると\refe{se-scattering}の形式解は,
    \begin{align}
      \psi(\bm{r}) &= \phi(\bm{r}) + \chi(\bm{r}) \\ 
      &= \e^{\i kz} + \int\dd{\bm{r'}} G_0(\bm{r} - \bm{r'}) U(\bm{r'}) \phi(\bm{r'}) \\ 
      &= \e^{\i kz} + \int\dd{\bm{r'}} \frac{\e^{\i\kappa \abs{\bm{r} - \bm{r'}}}}{4\pi \abs{\bm{r} - \bm{r'}}} \cdot \frac{2m}{\hbar^2} V(\bm{r}') \cdot \psi(\bm{r'}) \\ 
      &= \e^{\i kz} - \frac{1}{4\pi} \int \frac{\e^{\i \kappa \abs{\bm{r} - \bm{r'}}}}{\abs{\bm{r} - \bm{r'}}} \frac{2m}{\hbar^2}V(r')\psi(\bm{r'}) \dd{r'}\label{formal-soultion}
    \end{align}
    である.
    \begin{figure}[H]
      \centering
      \includegraphics[width = 0.7\columnwidth]{fig/integral-green-tex.pdf}
      \caption{積分経路}\label{Integral}
    \end{figure}
    \par
    次に\refe{formal-soultion}から散乱振幅$f(\theta)$を求める.仮定として,$V(r')$は$r'<a$でのみ$V\neq 0$とする.
    \reff{formal-soultion}において
    $r \gg r'$では
    \begin{align}
      \abs{\bm{r} - \bm{r'}} &= \sqrt{r^2 - 2\bm{r} \cdot \bm{r'} + r'^2}\\
      &= r \sqrt{1 - \frac{2\bm{r'} \cdot \bm{n}}{r + \qty(\cfrac{r'}{r})^2}}\\
      &\simeq r - \bm{n}\cdot\bm{r'}
    \end{align}
    が成り立つ.よって,
    \begin{align}
      \e^{\i k \abs{\bm{r} - \bm{r'}}} &\simeq \e^{\i k \qty(r - \bm{n} \cdot \bm{r'})}\\
      &= \e^{\i k r} - \e^{-\i \bm{k'} \cdot \bm{r'}}
    \end{align}
    を得る.ここで,$\bm{k'} = k \bm{n}$を$z$軸と角度$\theta$をなす散乱方向の波数ベクトルとする.また,
    \begin{align}
      \frac{1}{\abs{\bm{r} - \bm{r'}}} &\simeq \frac{1}{r\qty(1 - \frac{\bm{n}\cdot\bm{r'}}{r})}\\
      &= \simeq \frac{1}{r}
    \end{align}
    である.以上より,
    \begin{align}
      \psi(\bm{r}) = \e^{\i kz} - \qty(\frac{1}{4\pi} \int \dd{\bm{r'}} \e^{- \i \bm{k'} \cdot \bm{r'}} \frac{2m}{\hbar^2}V(r')\psi(r'))\frac{\e^{\i kr}}{r}
    \end{align}
    である.したがって,散乱振幅は
    \begin{align}
      f(\theta) = -\frac{1}{4\pi} \int \dd{\bm{r'}} \e^{- \i \bm{k'} \cdot \bm{r'}} \frac{2m}{\hbar^2}V(r')\psi(r')
    \end{align}
    となる.
\end{document}

    \section{Born近似}
      \documentclass{report}
\input{../../head.tex}
\begin{document}
  一般に,散乱の波動関数は,
  \begin{align}
    \psi(\bm{r}) = \e^{\i kz} + \int G_0 (\bm{r} - \bm{r'}) \frac{2m}{\hbar^2} V(\bm{r'}) \psi(\bm{r'}) \dd{\bm{r'}}\label{wave-funtion-of-scattering}
  \end{align}
  と表されるのであった.
  この波動関数を厳密に求めることは困難であるため近似を考える.
  まず,\refe{wave-funtion-of-scattering}を簡略化して,
  \begin{align}
    \psi(\bm{r}) = \psi_0 + \int g V \psi(\bm{r}') \dd{\bm{r}'}\label{simple-wave-funtion-of-scattering}
  \end{align}
  と表現する.
  ただし,
  \begin{align}
    g(\bm{r}) \coloneqq G_0(\bm{r})\frac{2m}{\hbar^2}
  \end{align}
  である.
  \refe{simple-wave-funtion-of-scattering}を再帰的に代入すると,
  \begin{align}
    \psi(\bm{r}) &= \psi_0 + \int g V \psi(\bm{r'}) \dd{\bm{r'}} \\
    &= \psi_0 + \int gV \qty(\psi_0 + \int g V \psi(\bm{r''}) \dd{\bm{r''}}) \dd{\bm{r'}} \\
    &= \psi_0 + \int g V \psi_0(\bm{r'}) \dd{\bm{r'}} + \iint gVgV \psi(\bm{r''}) \dd{\bm{r'}} \dd{\bm{r''}} \\
    &= \psi_0 + \int g V \psi_0(\bm{r'}) \dd{\bm{r'}} + \iint gVgV \psi_0(\bm{r''}) \dd{\bm{r'}} \dd{\bm{r''}} + \iiint gVgVgV \psi_0(\bm{r'''}) \dd{\bm{r'}} \dd{\bm{r''}} \dd{\bm{r'''}} + \cdots\label{dozens-of-gv}
  \end{align}
  を得る.
  \refe{dozens-of-gv}を第1項までで近似して,残りの項を捨てる.
  これは散乱の波動関数を平面波で近似することに相当する.
  つまり,
  \begin{align}
    \psi \simeq \psi_0
  \end{align}
  とする.これを\textbf{第1 Born近似}という\footnote{Max Born(1882-1970)}\footnote{砂川,散乱の量子論,
  「第1 Born近似がとくによく利用される理由は,何といってもその簡単さにある.したがって,ある散乱問題を手がけたとき,だれもが最初に試してみるのが,この近似である.
  そして思わしい結果がえられないとき,他の近似法を考えるのである.」}\footnote{2023年度期末試験第3問(4).}.
  \par
  Born近似を用いて散乱振幅を求める.$\bm{k} \coloneqq k\bm{e}_z$,$\bm{r} \coloneqq z\bm{e}_z$とすると,
  $\psi(\bm{r}') = \e^{\i \bm{k}\cdot\bm{r}'}$となるから,
  \begin{align}
    f^{(1)}(\theta) &= -\frac{1}{4\pi} \int \e^{\i \bm{k'}\cdot\bm{r'}} \frac{2m}{\hbar^2} V(\bm{r'}) \psi(\bm{r'}) \dd{\bm{r'}} \\
    &= -\frac{1}{4\pi}\frac{2m}{\hbar^2}\int\e^{-\i (\bm{k'} - \bm{k})\cdot\bm{r}'} V(\bm{r'}) \dd{\bm{r}'} \\
    &= -\frac{1}{4\pi}\frac{2m}{\hbar^2}\int\e^{-\i \bm{q} \cdot \bm{r}'} V(\bm{r'}) \dd{\bm{r}'}\label{potential-fourier-transform}
  \end{align}
  となる.ただし,散乱による運動量変化に対応する物理量を$\bm{q} \coloneqq \bm{k'} - \bm{k}$と定義した.
  \refe{potential-fourier-transform}を見ると,散乱振幅はポテンシャル$V(r)$のFourier変換から得られることがわかる\footnote{
    $f^{(n)}$は第$n$ Born近似による散乱振幅を意味する.
  }.
  また,球対称ポテンシャルのとき\refe{potential-fourier-transform}は簡略化できて,$V(\bm{r'}) \to V(r')$としてよいから,
  \begin{align}
    f^{(1)}(\theta) &= -\frac{1}{4\pi}\frac{2m}{\hbar^2}\int\e^{-\i \bm{q}\cdot \bm{r}'} V(r') \dd{\bm{r}'}\\
    &= -\frac{1}{4\pi}\frac{2m}{\hbar^2} \int_{\phi' = 0}^{2\pi}\dd{\phi'} \int_{\theta' = 0}^{\pi}\dd{\theta'}\int_{r' = 0}^{\infty}r'^2 \sin\theta' \dd{r'} \e^{-\i qr' \cos\theta'} V(r') \\
    &= -\frac{1}{4\pi}\frac{2m}{\hbar^2} 2\pi \int_{r' = 0}^{\infty}V(r')r'^2\int_{\theta' = 0}^{\pi} \e^{-\i qr' \cos\theta'} \sin\theta' \dd{r'}\dd{\theta'} \\ 
    &= -\frac{m}{\hbar^2} \int_{0}^{\infty} r'^2V(r') \frac{\e^{\i qr'} - \e^{-\i qr'}}{\i qr'} \dd{r'} \\ 
    &= -\frac{m}{\hbar^2} \int_{0}^{\infty} rV(r) \frac{\e^{\i qr} - \e^{-\i qr}}{\i q} \dd{r} \\ 
    &= -\frac{2m}{\hbar^2 q} \int_{0}^{\infty} rV(r)\sin\qty(qr) \dd{r}
  \end{align}
  となる.
  \begin{itembox}[l]{球対称ポテンシャルの散乱振幅}
    \begin{align}
      f^{(1)}(\theta) &= -\frac{m}{\hbar^2} \int_{0}^{\infty} rV(r) \frac{\e^{\i qr} - \e^{-\i qr}}{\i q} \dd{r}\label{SCamp1st-exp} \\ 
      &= -\frac{2m}{\hbar^2 q} \int_{0}^{\infty} rV(r)\sin\qty(qr) \dd{r}\label{SCamp1st-sin}
    \end{align}
  \end{itembox}
  \begin{myex}{湯川ポテンシャル}{}
    球対称ポテンシャル$V(r)$が,湯川ポテンシャルの場合を考える.
    湯川ポテンシャル,
    \begin{align}
      V(r) = V_0 \frac{\e^{-\mu r}}{\mu r}
    \end{align}
    による散乱を考える\footnote{湯川秀樹(1907-1981)}.$V(r)$は到達距離が$\mu^{-1}$ほどであり,核子同士に働く力を表す.
    物質中では,伝導電子に遮蔽された不純物のCoulombポテンシャルを表す.
    $\mu = 1,2$及びCoulombポテンシャルのグラフを\reff{yukawa-potential-graph}に示す.
    このポテンシャルの下で散乱振幅$f^{(1)}(\theta)$と散乱断面積$\sigma^{(1)}(\theta)$を求めよ.
    \tcblower
    散乱振幅は\refe{SCamp1st-exp}より,
    \begin{align}
      f^{(1)}(\theta) &= -\frac{m}{\hbar^2} \int_{0}^{\infty} rV(r) \frac{\e^{\i qr} - \e^{-\i qr}}{\i q} \dd{r} \\ 
      &= -\frac{m}{\i \hbar^2 q \mu} \int_{0}^{\infty} rV_0\frac{e^{-\mu r}}{\mu r} \frac{\e^{\i qr} - \e^{-\i qr}}{\i q}\dd{r} \\
      &= -\frac{mV_0}{\i \hbar^2 q \mu} \int_{0}^{\infty} \qty[\exp\qty{(-\mu + \i q)r} - \exp\qty{(-\mu - \i q)r}] \dd{r} \\
      &= - \frac{2m V_0}{\hbar^2 \mu} \frac{1}{\mu^2 + q^2}
    \end{align}
    \par
    散乱振幅は,
    \begin{align}
      \sigma^{(1)}(\theta) &= \abs{f^{(1)}(\theta)}^2\\
      &= \qty(\frac{2m V_0}{\hbar^2 \mu})^2 \frac{1}{(\mu^2 + q^2)^2}\\
      &= \qty(\frac{2m V_0}{\hbar^2 \mu})^2 \frac{1}{\qty[\mu^2 + 4k^2 \sin^2\qty(\theta/2)]^2}
    \end{align}
    である.なお,$\bm{k'}$と$\bm{k}$のなす角が$\theta$であるので余弦定理より,$q = 2k\sin\theta/2$であることを用いた.
    \begin{figure}[H]
      \centering
      \includegraphics[width = 0.5\columnwidth]{fig/yukawa_potential.jpg}
      \caption{湯川ポテンシャルとCoulombポテンシャルの比較}\label{yukawa-potential-graph}
    \end{figure}
  \end{myex}
  \begin{myex}{Rutherford散乱(Griffith Example 10.6)}{}
    湯川ポテンシャルで$V_0 = \frac{q_1q_2}{4\pi \epsilon_0}$,$\mu = 0$とするとCoulombポテンシャル
    \begin{align}
      V(r) = \frac{q_1q_2}{4\pi\epsilon_0}
    \end{align}
    と一致する.\refe{SCamp1st-sin}に代入して散乱振幅を求める.
    \begin{align}
      f(\theta) &= - \frac{2m}{\hbar^2q} \int_{0}^{\infty} r \frac{q_1q_2}{4\pi\epsilon_0 r} \sin\qty(qr) \dd{r}\\
      &= -\frac{2m}{\hbar^2q}\frac{q_1q_2}{4\pi\epsilon_0}\int_{0}^{\infty} \sin\qty(qr) \dd{r} \\
      &=  -\frac{2m}{\hbar^2q}\frac{q_1q_2}{4\pi\epsilon_0}\qty(-\frac{1}{q}\qty[\cos\qty(qr)]_0^{\infty})\\
      &\simeq - \frac{mq_1q_1}{2\pi\epsilon_0\hbar^2q^2}\\
      &= - \frac{mq_1q_1}{8\pi\epsilon_0\hbar^2 \sin^2 \theta/2}
    \end{align}
  \end{myex}
\end{document}
    \section{部分波展開}
      \documentclass{report}
\input{../../head.tex}
\begin{document}
  近似法の一つである部分波展開を扱う.
  まず,半古典論を用いて散乱が起こる条件から,方位量子数$l$ごとに波動関数を展開して,$l$が小さい波動関数のみ考えればよいことが分かる.
  次に,波動関数の基底展開について議論を行う.
  前節の議論より,球Bessel関数と球Neumann関数を用いて任意の関数が展開できるのであった.
  また,散乱には$\phi$依存性が無いという条件を用いれば,球Neumann関数の成分は0であると分かる.
  さらに,球Bessel関数にかかる係数のラベルは方位量子数のラベルと一致することが分かる.
  \subsection{散乱の条件}
    古典力学において角運動量$\bm{L}$は,
    \begin{align}
      \bm{L} = \bm{r} \times \bm{p} = m\bm{r} \times \bm{v}
    \end{align}
    と書けるのであった.
    球対称ポテンシャルの下では,
    \begin{align}
      \dv{\bm{L}}{t} &= m\dv{\bm{r}}{t} \times \bm{v} + m\bm{r} \times \dv{\bm{v}}{t} \\
      &= \bm{r} \times \bm{F} \\
      &= \bm{r} \times (-\grad V(r)) \\
      &= \bm{r} \times \abs{\grad V}\frac{\bm{r}}{r} \\
      &= 0
    \end{align}
    より,角運動量は保存される.
    衝突パラメータを$b$,運動量を$\bm{p}$とした粒子の角運動量は,
    \begin{align}
      \abs{\bm{L}} &= \abs{\bm{r} \times \bm{p}} \\
      &= pb
    \end{align}
    である.散乱体を半径$a$の球とすると,衝突の条件は
    \begin{align}
      b &< a \\
      L/p &< a \\
      L &< pa \label{ConditionofSC}
    \end{align}
    である.つまり,角運動量が小さい粒子のみ散乱することがわかる.
    \par
    \refe{ConditionofSC}を半古典的な散乱条件へ書き直すことを考える.
    量子力学では,角運動量の大きさ$L$は,$l = 0, 1, 2, \cdots$の値を取る方位量子数$l$を用いて$L = \hbar \sqrt{l(l + 1)}$と書ける.
    また,\refe{assos-legendre-diff-eq}と,\refe{a-condition-use-lm}の第2項以降を見ると,球面分布函数$Y_{lm}(\theta, \phi)$がラプラシアンの角度部分の固有関数に
    なっていて,ラプラシアンの角度部分は,角運動量演算子を$\hbar$で割ったものであるから角運動量の固有値が$\hbar\sqrt{l(l + 1)}$であることが分かる.
    衝突の条件は,
    \begin{align}
      L = \hbar \sqrt{l(l + 1)} < pa = \hbar ka\label{scattering-condition-of-l}
    \end{align}
    と書ける.\refe{scattering-condition-of-l}を見れば,散乱の影響を受けるのは$l$が小さいときのみであることがわかる.
    よって波動関数を,
    \begin{align}
      \psi = \psi^{(l = 0)} + \psi^{(l = 1)} + \cdots
    \end{align}
    のように異なる$l$に属する固有関数で展開し,$l$が小さい状態についてだけ散乱の影響を考える.これを\textbf{部分波展開}という.
  \subsection{部分波展開の計算}
    波動関数は\refe{u-expand-form2}より,
    \begin{align}
      \psi(r, \theta, \phi) = \sum_{m = -\infty}^{\infty}\sum_{l = \abs{m}}^{\infty}\qty[a_{lm}j_l(\kappa r) + b_{lm}y_l(\kappa r)](-1)^m\sqrt{\frac{2l + 1}{4\pi}\frac{(l - m)!}{(l + m)!}}P_l^m(x)\e^{\i m\phi}
    \end{align}
    と書けるのであった.
    ただし,角運動量演算子の固有関数とラベルを合わせるために$n \to l$とした.
    散乱において,$\phi$依存性が無いので,$m = 0$のみ和を考えればよく,
    \begin{align}
      \psi(r, \theta) = \sum_{l = 0}^{\infty}\qty[a_lj_l(\kappa r) + b_ly_l(\kappa r)]\sqrt{\frac{2l + 1}{4\pi}}P_l(x)
    \end{align} 
    となる.
    $a_l \coloneqq a_{l0}$,$b_l \coloneqq b_{l0}$である.また,$P_l^0 = P_l$であることに注意する.
    球Bessel関数と球Neumann関数は$x \to 0$で,
    \begin{align}
      j_l(r) &\sim \frac{r^l}{(2l + 1)!}\qty(1 - \frac{r^2}{2(2l + 3)}\cdots) \\
      n_l(r) &\sim -\frac{(2l - 1)!!}{r^{l + 1}}
    \end{align}
    となることが知られている.
    $r = 0$で$\psi(r, \theta)$が発散しないために,$\forall l\ b_l = 0$とする.
    まず,平面波を部分波展開すると,Rayleighの公式より,
    \begin{align}
      \e^{\i kr\cos\theta} = \sum_{l = 0}^{\infty} (2l + 1)\i^l j_l(kr)P_l(\cos\theta)
    \end{align}
    と書ける.
    \par
    球面波は未定係数$a_l$をつけて,
    \begin{align}
      \sum_{l = 0}^{\infty} (2l + 1)a_l P_l(\cos\theta) \frac{\e^{\i kr}}{r}
    \end{align}
    と展開される.
    したがって,部分波展開した散乱の波動関数は,
    \begin{align}
      \psi(\bm{r}) = \sum_{l = 0}^{\infty} (2l + 1)\i^l j_l(kr)P_l(\cos\theta) + \sum_{l = 0}^{\infty} (2l + 1)a_l P_l(\cos\theta) \frac{\e^{\i kr}}{r}
    \end{align}
    である.
    \begin{itembox}[l]{部分波展開した散乱の波動関数}
      \begin{align}
        \psi(\bm{r}) = \sum_{l = 0}^{\infty} (2l + 1)\i^l j_l(kr)P_l(\cos\theta) + \sum_{l = 0}^{\infty} (2l + 1)a_l P_l(\cos\theta) \frac{\e^{\i kr}}{r}\label{PartialWave}
      \end{align}
    \end{itembox}
    \begin{itembox}[l]{部分波展開した散乱の散乱振幅}
      \begin{align}
        f(\theta) = \sum_{l = 0}^{\infty} (2l + 1)a_l P_l(\cos\theta) \label{PartialWave-amp}
      \end{align}
    \end{itembox}
  \subsection{散乱断面積と全断面積}
    部分波展開した波動関数より計算した散乱断面積は,
    \begin{align}
      \sigma(\theta) &= \abs{f(\theta)}^2 \\
      &= \sum_{l} \sum_{l'} (2l+1)(2l'+1) a_l^{*}a_{l'} P_l(\cos\theta) P_{l'}(\cos\theta)
    \end{align}
    である.全断面積は,
    \begin{align}
      \sigma^{\r{tot}} &= 2\pi \int \sigma(\theta) \sin\theta \dd{\theta} \\
      &= 2\pi \sum_{l = 0}^{\infty} (2l + 1)^2\abs{a_l}^2\qty(\frac{2}{2l + 1}) \\
      &= 4\pi \sum_{l = 0}^{\infty} (2l + 1) \abs{a_l}^2
    \end{align}
    である.ここで,Legendre多項式の直交性,
    \begin{align}
      \int_{0}^{\pi} P_l(\cos\theta)P_{l'}(\cos\theta) \sin\theta \dd{\theta} = \frac{2}{2l+1}\delta_{l, l'}
    \end{align}
    を用いた.
    以上の議論から,部分波展開を用いた散乱問題は未定係数$a_l$を求めることに帰着する.
\end{document}
  \chapter{相対論的量子論}
    \section{特殊相対論}
      \documentclass{report}
\input{../../head.tex}
\begin{document}
  特殊相対性理論(Special Relativity)は次の2つの事柄を原理とする.
  \begin{itembox}[l]{特殊相対性原理}
    あらゆる慣性系で同じ物理法則が成り立つ.
  \end{itembox}
  \begin{itembox}[l]{光速度不変の原理}
    あらゆる慣性形で真空中の光の速さは同一である.
  \end{itembox}
  \subsection{Lorentz変換}
    特殊相対性原理と光速度不変の原理の下で成り立つ座標変換の法則(Lorentz変換)を導く.
    まず,慣性系$X$系の原点$O$と$X'$系の原点$O'$が$t = t' = 0$で一致している.$t = t' = 0$で光が原点($O = O'$)を通過したとする.
    $X$系の空間座標を$(x, y, z)$,$X'$系の空間座標を$(x', y', z')$とすると,光速度不変の原理より,
    \begin{align}
      \frac{\sqrt{x^2 + y^2 + z^2}}{t} = \frac{\sqrt{x'^2 + y'^2 + z'^2}}{t'} = c
    \end{align}
    が成り立つ.上式から\textbf{世界長さ}(spacetime interval)
    \begin{align}
      s^2 \coloneqq x^2 + y^2 + z^2 - (ct)^2
    \end{align}
    が不変量であることが導かれる.
    \par
    次に,世界長さ不変性から,慣性系間の座標変換の法則である\textbf{Lorentz変換}(Lorentz Transformation)を導く
    慣性系$X'$が$x$軸正の方向に速さ$v$で移動しているとする.このとき$y=y',z=z'$である.わかりやすいように$T=\i t$とおく.
    世界長さは座標系に依らないから,
    \begin{align}
      x^2 - (ct)^2 &= x'^2 - (ct')^2\\
      x^2 + (cT)^2 &= x'^2 + (cT)^2
    \end{align}
    が成り立つ.これが回転座標変換と類似していることから,
    \begin{align}
      \mqty(cT' \\ x') 
      = \mqty(
        \cos\theta & -\sin\theta\\
        \sin\theta & \cos\theta
      )
      \mqty(cT \\ x)
    \end{align}
    と書ける.表示を$t$に戻すと,
    \begin{align}
      \mqty(ct' \\ x')
      =
      \mqty(
        \cos\theta & \i\sin\theta\\
        \i\sin\theta & \cos\theta
      )
      \mqty(ct\\ x)
    \end{align}
    である.さらに,$\theta = \i\phi$とすると
    \begin{align}
      \mqty(ct' \\ x')
      =
      \mqty(
        \cosh\phi & -\sinh\phi\\
        -\sinh\phi & \cosh\phi
      )
      \mqty(ct\\ x)
    \end{align}
    となる.よって,
    \begin{align}
      x' = (-\sinh\phi)ct + (\cosh\phi)x
    \end{align}
    を得る.$X$系において時刻$t$が経過したとする.$X$系から見る$X'$系の原点の位置は$x=vt$である.一方,$X'$系から見ると$x'=0$である.
    よって上式から,
    \begin{align}
      0 = (-\sinh\phi)ct + (\cosh\phi)vt
    \end{align}
    が成り立つ.よって,
    \begin{align}
      \frac{v}{c} = \frac{\sinh\phi}{\cosh\phi} = \tanh\phi
    \end{align}
    である.以上より,
    \begin{align}
      \begin{dcases}
        \sinh\phi = \frac{v/c}{\sqrt{1- (v/c)^2}}\\
        \cosh\phi = \frac{1}{\sqrt{1- (v/c)^2}}
      \end{dcases}
    \end{align}
    であることがわかる.したがって,
    \begin{itembox}[l]{Lorentz変換}
      \begin{align}
        \mqty(ct' \\ x')
        =
        \frac{1}{\sqrt{1 - (v/c)^2}}
        \mqty(
          1 & -v/c\\
          -v/c & 1
        )
        \mqty(ct\\ x)\label{LorentzTransformation}
      \end{align}
    \end{itembox}
    を得る.Lorentz変換
    \begin{align}
      x' = \frac{x-vt}{\sqrt{1-(v/c)^2}}
    \end{align}
    において$v \ll c$とすると
    \begin{align}
      x' = x - vt
    \end{align}
    となる.これはGalilei変換と一致している.

    \begin{myexc}{Lorentz変換の問題}{}
      \begin{enumerate}
        \item Lorentz変換\refe{LorentzTransformation}において,$x^2 - (ct)^2$が不変であることを確認せよ.
        \item $X'$系の原点が$X$系から見ると速さ$v$で$x$軸正方向へ運動していることを示せ.
      \end{enumerate}
      \tcblower
      \begin{enumerate}
        \item \refe{LorentzTransformation}より,
          \begin{align}
            x'^2 - (ct')^2 &= \frac{x^2 -2xvt + v^2t^2}{1 - (v/c)^2} - \frac{c^2t^2 - 2xvt + (v/c)^2x^2}{1 - (v/c)^2}\\
            &= \frac{[1-(v/c)^2]x^2 - [1 - (v/c)^2](ct)^2}{1 - (v/c)^2}\\
            &= x^2 - (ct)^2
          \end{align}
        \item $X'$系の原点は
          \begin{align}
            x' = \frac{-vt + x}{\sqrt{1-(v/c)^2}} = 0
          \end{align}
          より,$X$系から見ると
          \begin{align}
            x = vt
          \end{align}
          となる.よって,速さ$v$で$x$軸正方向へ運動していることがわかる.
      \end{enumerate}
    \end{myexc}

  \subsection{速度の合成則}
    次に,相対論的効果を取り込んだ速度の合成則について示す.
    $X$系は静止し,$X'$系が速さ$v$で$x$軸方向に移動しているとする.
    さらに$X'$系では粒子が速さ$u'$で$x'$軸方向に運動している.
    $X$系から見た粒子の速度$V$を求める.\refe{LorentzTransformation}において$v\to -v$とすると,
    \begin{align}
      \mqty(c\dd{t} \\ \dd{x})
      =
      \frac{1}{\sqrt{1 - (v/c)^2}}
      \mqty(
        1 & v/c\\
        v/c & 1
      )
      \mqty(c\dd{t'}\\ \dd{x'})
    \end{align}
    となる.$V=\dv{x}{t}$なので,
    \begin{align}
      V = \dv{x}{t} &= \frac{v \dd{t'} + \dd{x'}}{\dd{t'} + (v/c^2)\dd{x'}}\\
      &= \frac{v + \dv{x'}{t'}}{1 + (v/c^2)\dv{x'}{t'}}\\
      &= \frac{v + u'}{1 + vu' / c^2}
    \end{align}
    を得る.
    \begin{itembox}[l]{速度の合成}
      \begin{align}
        V = \frac{v + u'}{1 + vu'/ c^2}
      \end{align}
    \end{itembox}
    例として$u' = v = c$とすると
    \begin{align}
      V = \frac{2c}{1+c^2/c^2} = c
    \end{align}
    である.速度が合成されても光速を超えることは決してないことがわかる.
  \subsection{Lorentz収縮}
    \textbf{Lorentz収縮}(Length Contractions)について説明する.
    速さ$v$で運動している$X'$系から,$t' = 0$において,静止している$X$系の2点を見る.
    1点は原点$O:(x, t) = (0, 0)$,もう1点は$P:(x, t) = (L, t)$である.
    まず,原点$O$は$X'$系から見ると,
    \begin{align}
      \mqty(ct' \\ x')
      =
      \frac{1}{\sqrt{1 - (v/c)^2}}
      \mqty(
        1 & -v/c \\
        -v/c & 1
      )
      \mqty(0 \\ 0)
      =
      \mqty(0 \\ 0)
    \end{align}
    である.点$P$を$X'$系から見ると,
    \begin{align}
      \mqty(ct' \\ x')
      =
      \frac{1}{\sqrt{1 - (v/c)^2}}
      \mqty(
        1 & -v/c\\
        -v/c & 1
      )
      \mqty(ct \\ L)
      =
      \frac{1}{\sqrt{1 - (v/c)^2}}
      \mqty(ct - \frac{v}{c}L \\ -vt + L)
    \end{align}
    である.$t' = 0$で観測しているため,$t' = 0$を代入し
    \begin{align}
      0 &= ct - \frac{v}{c}L \\
      t &= \frac{v}{c^2}L
    \end{align}
    を得る.よって,
    \begin{align}
      x' = \frac{1}{\sqrt{1 - (v/c)^2}} \qty[-\qty(\frac{v}{c})^2 L + L] = \sqrt{1-\qty(\frac{v}{c})^2}L
    \end{align}
    である.
    これは動いている慣性系から静止系での距離($L$)を測ると縮んで見える($L'$)ことを意味している.
    \begin{itembox}[l]{Lorentz収縮}
      \begin{align}
        L' = \sqrt{1 - \qty(\frac{v}{c})^2}L
      \end{align}
    \end{itembox}
    また,ある対象に対して静止している観測者が測った距離を\textbf{固有長さ}(proper length)という.
    今回は$L$が固有長さである.
  \subsection{時間の遅れ}
    次に\textbf{時間の遅れ}(Time Dilations)について説明する.
    同様の$X$系と$X'$系を考える.
    時計が$X'$系の原点$x' = 0$に置かれていて,観測者は$X'$系においてこの時計を見ている.
    $X'$系に置かれた時計の時刻が$t'$のときの時空点$P_1$,$t' + \Delta T_0$の時空点を$P_2$とする.
    $P_1$は,
    \begin{align}
      \mqty(ct_1 \\ x_1)
      =
      \frac{1}{\sqrt{1 - (v/c)^2}}
      \mqty(
        1 & v/c \\
        v/c & 1
      )
      \mqty(ct' \\ 0)
      =
      \frac{1}{\sqrt{1 - (v/c)^2}}
      \mqty(ct' \\ vt')
    \end{align}
    $P_2$は,
    \begin{align}
      \mqty(ct_1 \\ x_1)
      =
      \frac{1}{\sqrt{1 - (v/c)^2}}
      \mqty(
        1 & v/c \\
        v/c & 1
      )
      \mqty(c(t' + \Delta T_0) \\ 0)
      =
      \frac{1}{\sqrt{1 - (v/c)^2}}
      \mqty(c(t' + \Delta T_0) \\ v(t' + \Delta T_0))
    \end{align}
    である.よって,$X$系での時間経過$\Delta T = t_2 - t_1$は,
    \begin{align}
      \Delta T = t_2 - t_1 = \frac{1}{\sqrt{1 - (v/c)^2}} \Delta T_0
    \end{align}
    である.これは$X'$系は$X$系に比べて時間の流れが遅いことを示している.
    \begin{itembox}[l]{時間の遅れ}
      \begin{align}
        \Delta T = \frac{1}{\sqrt{1 - (v/c)^2}} \Delta \tau
      \end{align}
    \end{itembox}
    観測者に対して2つの事象が同一の空間座標で起きたとき,その時間間隔$\Delta \tau$を\textbf{固有時間}(proper time)という.
  \subsection{電磁場の双対性♠}
    電磁場の双対性について説明する.マクスウェル方程式は,
    \begin{align}
      \begin{dcases}
        \div \bm{D} = \rho \\
        \div \bm{B} = 0 \\
        \curl \bm{E} = - \pdv{\bm{B}}{t} \\
        \curl \bm{H} = \pdv{\bm{D}}{t} + \bm{j}
      \end{dcases}
    \end{align}
    の4つである.
    これをLorentz変換に対して共変な形式に書き直す.
    まず,$\bm{B}$と$\bm{E}$はベクトルポテンシャル$\bm{A}$とスカラーポテンシャル$\phi$を用いて,
    \begin{align}
      \begin{dcases}
        \bm{E} = -\grad \phi - \pdv{\bm{A}}{t} \\
        \bm{B} = \curl \bm{A}
      \end{dcases}
    \end{align}
    と表される.座標を,
    \begin{align}
      (x^0, x^1, x^2, x^3) \coloneqq (ct, x, y, z)
    \end{align}
    と定義する\footnote{
      このような4次元空間を\textbf{Minkovski空間}という. 
    }.
    また4元ベクトルポテンシャルを,
    \begin{align}
      A^{\mu} &\coloneqq (\phi, c\bm{A}) \\
      A_{\mu} &\coloneqq (\phi, -c\bm{A})
    \end{align}
    と定義する.
    4元ベクトルポテンシャルを使うと電場と磁場は,
    \begin{align}
      B_1 \coloneqq B_x &= c\pdv{A_z}{y} - c\pdv{A_y}{z} \\
      &= -\pdv{A_3}{x^2}  + \pdv{A_2}{x^3} \\
      &= \partial_3 A_2 - \partial_2 A_3\\
      E_1 \coloneqq E_x = &= -\pdv{\phi}{x} - \pdv{A_x}{t} \\
      &= -\pdv{A_0}{x^1} + \pdv{A_1}{x^0} \\
      &= \partial_0 A_1 - \partial_1 A_0
    \end{align}
    のように書ける.電磁場テンソルを,
    \begin{align}
      F_{\mu\nu} \coloneqq \partial_{\mu}A_{\nu} - \partial_{\nu} A_{\mu}
    \end{align}
    を定義する.これを行列の形で表すと,
    \begin{align}
      F_{\mu\nu}=
      \mqty(
        F_{00} & F_{01} & F_{02} & F_{03}\\
        F_{10} & F_{11} & F_{12} & F_{13}\\
        F_{20} & F_{21} & F_{22} & F_{23}\\
        F_{30} & F_{31} & F_{32} & F_{33}
      )
      =
      \mqty(
        0 & E_x & E_y & E_z \\
        -E_x & 0 & -cB_z & cB_y\\
        -E_y & cB_z & 0 & -cB_x\\
        -E_z& -cB_y & cB_x & 0
      )
    \end{align}
    である.
    明らかに$F_{\mu\nu} = -F_{\nu\mu}$である.
    また電磁場テンソルの定義から,
    \begin{align}
      \partial_{\mu} F_{\nu\lambda} + \partial_{\nu} F_{\lambda\mu} + \partial_{\lambda} F_{\mu\nu} = 0
    \end{align}
    が成り立つ.実際に計算してみると
    \begin{align}
      \partial_{\mu} F_{\nu\lambda} + \partial_{\nu} F_{\lambda\mu} + \partial_{\lambda} F_{\mu\nu} &= 
      \partial_{\mu}\qty(\partial_{\nu}A_{\lambda} - \partial_{\lambda} A_{\nu}) + \partial_{\nu}\qty(\partial_{\lambda}A_{\mu} - \partial_{\mu} A_{\lambda}) + \partial_{\lambda}\qty(\partial_{\mu}A_{\nu} - \partial_{\nu} A_{\mu}) \\ 
      &= (\partial_{\mu}\partial_{\nu} A_\lambda - \partial_{\mu}\partial_{\lambda} A_{\nu}) + (\partial_{\nu}\partial_{\lambda} A_{\mu} - \partial_{\nu}\partial_{\mu} A_\lambda)
      + (\partial_{\lambda}\partial_{\mu} A_{\nu} - \partial_{\lambda}\partial_{\nu} A_{\mu}) \\
      &= 0
    \end{align}
    である.これはFaradayの電磁誘導の法則と磁束密度に関するGaussの法則を表している.
    例えば$\mu = 0$,$\nu = 1$,$\lambda = 2$とすると,
    \begin{align}
      &\partial_0 F_{12} + \partial_1 F_{20} + \partial_2 F_{01}\\
      &= -\pdv{(cB_z)}{(ct)} + \pdv{x}(-E_y) + \pdv{y}(-E_x)\\
      &= -(\grad\times \bm{E})_z - \pdv{B_z}{t} = 0
    \end{align}
    また,$\mu = 1$,$\nu = 2$,$\lambda = 3$とすると,
    \begin{align}
      &\partial_1 F_{23} + \partial_2 F_{31} + \partial_3 F_{12}\\
      &= c(\partial_xB_x + \partial_yB_y + \partial_zB_Z)\\
      &= c\div\bm{B} = 0
    \end{align}
    が得られる.
    \par
    磁場/電束密度テンソル$H_{\mu\nu}$を,
    \begin{align}
      H_{\mu\nu} =
      \mqty(
        H_{00} & H_{01} & H_{02} & H_{03} \\
        H_{10} & H_{11} & H_{12} & H_{13} \\
        H_{20} & H_{21} & H_{22} & H_{23} \\
        H_{30} & H_{31} & H_{32} & H_{33}
      )
      =
      \mqty(
        0 & cD_x & cD_y & cD_z \\
        -cD_x & 0 & -H_z & H_y\\
        -cD_y & H_z & 0 & -H_x\\
        -cD_z& -H_y & H_x & 0
      )
    \end{align}
    4次元の電流密度を$j^{\mu}$を,
    \begin{align}
      j^{\mu} \coloneqq (c\rho, \bm{j})
    \end{align}
    と定義する.
    明らかに$H_{\mu\nu} = -H_{\nu\mu}$である.
    すると,電束密度に関するGaussの法則とAmpèreの法則は,
    \begin{align}
      \partial^{\nu}H_{\nu\mu} = j_\mu
    \end{align}
    と表される.Einsteinの縮約記法を用いた.
    例えば$\mu = 0$とすると
    \begin{align}
      \partial^1H_{10} + \partial^2H_{20} + \partial^3H_{30} &= j_0\\
      \partial_xD_x + \partial_yD_y + \partial_zD_z &= \rho\\
      \div\bm{D} &= \rho
    \end{align}
    が導かれる.さらに電荷保存則,
    \begin{align}
      \pdv{\rho}{t} + \div\bm{j} &= 0 \\ 
      \Leftrightarrow \pdv{(c\rho)}{(ct)} + \div\bm{j} &= 0
    \end{align}
    を用いて,
    \begin{align}
      \partial_{\mu} j^\mu = 0
    \end{align}
    が得られる.以上をまとめるとMaxwell方程式は,
    \begin{align}
      \begin{dcases*}
        \partial_{\mu} F_{\nu\lambda} + \partial_{\nu} F_{\lambda\mu} + \partial_{\lambda} F_{\mu\nu} = 0 & Faradayの電磁誘導の法則・磁束密度に関するGaussの法則 \\
        \partial^{\nu}H_{\nu\mu} = j_\mu & 電束密度に関するGaussの法則・Ampèreの法則 \\
        \partial_{\mu} j^\mu = 0 & 電荷保存則
      \end{dcases*}
    \end{align}
    と書くことができる.これらの式はテンソルとテンソル,ベクトルとベクトルというように,Lorentz変換に対して
    同じ変換則をもつものどうしが結ばれている.よってこれらはLorentz変換に対して共変である.
    \begin{itembox}[l]{Lorentz変換に対して共変なMaxwell方程式}
      \begin{align}
        &\partial_{\mu} F_{\nu\lambda} + \partial_{\nu} F_{\lambda\mu} + \partial_{\lambda} F_{\mu\nu} = 0\\
        &\partial^{\nu}H_{\nu\mu} = j_\mu\\
        &\partial_{\mu} j^\mu = 0
      \end{align}
    \end{itembox}
  \subsection{電磁場の双対性の例題}
    \begin{myex}{}{}
      運動する電荷と電流を例に電磁場の双対性を確認してみよう.
      座標系は導線の中心を$z$軸とした円筒座標系を入れる.
      導線に$z$軸上向きに電流$I$が流れている.
      電荷$+q$が$z$軸上向きに速さ$v_0$で運動している.
      導線の中では面電荷密度$\lambda_{\pm} = \pm\lambda/2$の正(負)電荷が速さ$v$で上(下)向きに動いているとする.
      まずは静止系で考える.
      導線の中では,
      \begin{align}
        \lambda_{+} + \lambda_{-} = 0
      \end{align}
      が成り立つため,電気的に中性である.
      よって導線の周りには電場は無い.
      電流は$I = \lambda v$と表される.
      導線の周りには,
      \begin{align}
        \bm{B} = \frac{\mu_0I}{2\pi r} \bm{e}_{\phi} = \frac{\mu_0 \lambda v}{2\pi r} \bm{e}_{\phi}
      \end{align}
      の磁束密度が発生している.
      よって電荷は,
      \begin{align}
        \bm{F} = q\bm{v}\times\bm{B} = -qv_0\frac{\mu_0 \lambda v}{2\pi r}\bm{e}_r
      \end{align}
      の力を感じる.
      \par
      同様の設定を電荷とともに動く系で考える.
      非相対論的に考えると$\bm{v} = 0$であるため電荷は力を感じないことになるが,電子が力を受けて運動している実験事実に矛盾する.
      相対論的効果を考慮してこの状況を眺める必要がある.
      観測者からは導線中の正電荷は$v - v_0$で,負電荷は$v + v_0$で運動して見える.
      それぞれでLorentz収縮を計算する.
      \begin{align}
        \lambda_{+} &= \frac{1}{\sqrt{1-\qty(\cfrac{v - v_0}{c})^2}} \cdot \frac{\lambda}{2}\\
        \lambda_{-} &= \frac{1}{\sqrt{1-\qty(\cfrac{v + v_0}{c})^2}} \cdot \qty(-\frac{\lambda}{2})
      \end{align}
      明らかに,
      \begin{align}
        \lambda_{+} + \lambda_{-} \neq 0
      \end{align}
      であることがわかる.よって,導線の周りには電場が発生している.$v,v_0 \ll c$とすると
      \begin{align}
        \Delta \lambda \coloneqq \lambda_{+} + \lambda_{-} &= \frac{1}{\sqrt{1 - \qty(\cfrac{v - v_0}{c})^2}}\frac{\lambda}{2} - \frac{1}{\sqrt{1 - \qty(\cfrac{v + v_0}{c})^2}}\frac{\lambda}{2}\\
        &\simeq \qty[1 + \frac{1}{2}\qty(\frac{v - v_0}{c})^2]\frac{\lambda}{2} - \qty[1 + \frac{1}{2}\qty(\frac{v + v_0}{c})^2]\frac{\lambda}{2}\\
        &= - \frac{\lambda v_0 v}{c^2} 
      \end{align}
      と計算できて,導線は負に帯電していることがわかる.よって電荷の受ける力は,
      \begin{align}
        \bm{F} &= q\frac{\Delta \lambda}{2\pi r \epsilon_0}\bm{e}_r\\
        &= -qv_0\frac{\mu_0 \lambda v}{2\pi r}\bm{e}_r
      \end{align}
      これは先ほどの計算結果と一致している.以上の考察から,電場と磁場は観測する系によって入り混じることがわかる.
    \end{myex}
\end{document}
    \section{量子力学の修正}
      \documentclass{report}
\input{../../head.tex}
\begin{document}
SEにLorentz共変性を持たせよう.
\end{document}
\end{document}
