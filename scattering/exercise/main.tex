\documentclass{report}
\input{../../head.tex}
\begin{document}
  \begin{myexc}{Griffith Problem10.21 Neutron diffraction}{}
  結晶による中性子散乱を考える.中性子と原子核の相互作用は短距離で,
  \begin{align}
    V(\bm{r}) = \frac{2\pi\hbar^2 b}{m} \sum_{i}\delta(\bm{r} - \bm{r}_i)
  \end{align}
  と近似されるとする.ここで,$\bm{r}_i$は$i$番目の原子核の位置である.$b$はnuclear scattering lengthである.
  \begin{enumerate}
    \item 第1Born近似により散乱断面積は
          \begin{align}
            \sigma = b^2 \abs{\sum_{i} \e^{-\i \bm{q}\cdot\bm{r}_i}}^2
          \end{align}
          となることを示せ.ここで,$\bm{q}\equiv\bm{k}' - \bm{k}$,$\bm{k}$は入射波,$\bm{k}'$は散乱波とする.
    \item 原子核が間隔$a$の格子状で並んでいるとし,
          \begin{align}
            \bm{r}_i = la\bm{e}_x + ma\bm{e}_y + na\bm{e}_z
          \end{align}
          とする.ここで$l,m,n$は0から$N-1$の整数である.
          \begin{align}
            \sigma = b^2 \Pi_{i =x,y,z} \frac{\sin^2(Nq_i a/2)}{\sin^2 (q_i a/2)}
          \end{align}
          となることを示せ.
    \item \begin{align}
            \frac{1}{N}\frac{\sin^2(Nq_x a/2)}{\sin^2 (q_x a/2)}
          \end{align}
          を$N=1,5,10$について横軸を$q_xa$としたグラフを図示せよ.
  \end{enumerate}
  \tcblower
  \begin{enumerate}
    \item 散乱振幅を計算する.
          \begin{align}
            f^{(1)}(\theta) &= -\frac{1}{4\pi} \int \e^{-\i\bm{k}' \cdot \bm{r}'} \frac{2m}{\hbar^2} \qty(\frac{2\pi\hbar^2 b}{m} \sum_{i} \delta(\bm{r}' - \bm{r}'))\e^{\i \bm{k}\cdot\bm{r}'} \dd{\bm{r}'}\\
            &= -b \sum_{i} \int \e^{-\i\bm{q}\cdot \bm{r}'} \delta(\bm{r}' - \bm{r}_i) \dd{\bm{r}'}\\
            &= -b \sum_{i} \e^{-\i \bm{q}\cdot\bm{r}_i}
          \end{align}
          よって散乱断面積は
          \begin{align}
            \sigma = \abs{f^{(1)}(\theta)}^2 = b^2\abs{\sum_{i} \e^{-\i \bm{q}\cdot\bm{r}_i}}^2
          \end{align}
          である.
    \item \begin{align}
            \sum_{i} \e^{-\i \bm{q}\cdot\bm{r}_i} &= \sum_{l=0}^{N-1} \sum_{m=0}^{N-1} \sum_{n=0}^{N-1} \e^{-\i(laq_x + maq_y + naq_z)}\\
            &= \sum_{l=0}^{N-1} \e^{-\i laq_x} \sum_{m=0}^{N-1} \e^{-\i maq_y} \sum_{n=0}^{N-1} \e^{-\i naq_z}\\
            &= \Pi_{i=x,y,z} \frac{1-\e^{-\i Naq_i}}{1-\e^{-\i aq_i}}\\
            &= \Pi_{i =x,y,z} \frac{\sin(Nq_i a/2)}{\sin (q_i a/2)}
          \end{align}
          よって,
          \begin{align}
            \sigma = b^2 \Pi_{i =x,y,z} \frac{\sin^2(Nq_i a/2)}{\sin^2 (q_i a/2)}
          \end{align}
          が得られる.
    \item 散乱パターンは下図のようになる.
          \begin{figure}[H]
            \centering
            \includegraphics[width=0.8\columnwidth]{fig/neutron_diffraction.jpg}
            \caption{散乱パターン}
            \label{neutron-diffraction}
          \end{figure}
  \end{enumerate}
  \end{myexc}
  \begin{myexc}{Griffith Problem10.22 2次元散乱理論}{}
    2次元の散乱を考える.
  \begin{enumerate}
    \item 極座標ラプラシアン
    \begin{align}
      \nabla^2 = \frac{\partial^2}{\partial r} + \frac{1}{r}\frac{\partial}{\partial} + \frac{1}{r^2}\frac{\partial^2}{\partial \theta^2}
    \end{align}
    を用いてポテンシャル$V(r)$の下での波動関数を求めよ.ここで$u(r)=\sqrt{r}R(r)$が
    \begin{align}
      -\frac{\hbar^2}{2m}\frac{\r{d}^2 u}{\dd{r}^2} + \qty[V(r) + \frac{\hbar^2}{2m}\frac{(j^2 - 1/4)}{r^2}]u = Eu \label{eq-for-u}
    \end{align}
    を満たすことを用いてよい.$j$は整数である.
    \item $r$が十分大きいときの\refe{eq-for-u}を考えることにより,
    \begin{align}
      R(r) \sim \frac{\e^{\i kr}}{\sqrt{r}}
    \end{align}
    であることを示せ.ここで,$k\equiv \sqrt{2mE}/\hbar$である.
  \end{enumerate}
    \tcblower
    \begin{align}
      &\qty[-\frac{\hbar^2}{2m}\nabla^2 + V(r)]\psi(r,\theta) = E\psi(r,\theta)\\
      &\qty[\frac{\hbar^2}{2m}\qty(\frac{\partial^2}{\partial r^2} + \frac{1}{r}\frac{\partial}{\partial r} + \frac{1}{r^2}\frac{\partial^2}{\partial \theta^2}) + V(r)]\psi(r,\theta) = E\psi(r,\theta)
    \end{align}
    $\psi(r,\theta) = R(r)\Theta(\theta)$とし,上式を整理すると
    \begin{align}
      \frac{r^2}{R(r)}\qty(\frac{\partial^2}{\partial r^2}R(r)) - \frac{r}{R(r)}\qty(\frac{\partial}{\partial r}R(r)) + \frac{2mr^2}{\hbar^2}(V(r) - E) - \frac{1}{\Theta(\theta)} \frac{\partial^2}{\partial \theta^2}\Theta(\theta) = 0
    \end{align}
    となる.
    \begin{align}
      - \frac{1}{\Theta(\theta)} \frac{\partial^2}{\partial \theta^2}\Theta(\theta) = j^2
    \end{align}
    とすれば
    \begin{align}
      \Theta(\theta) = \e^{\i j\theta}
    \end{align}
    が得られる.さらに,
    \begin{align}
      \Theta(\theta) = \Theta(\theta + 2\pi)
    \end{align}
    だから$j$は整数であることがわかる.よって$r$に関する部分は
    \begin{align}
      \frac{r^2}{R(r)}\qty(\frac{\partial^2}{\partial r^2}R(r)) - \frac{r}{R(r)}\qty(\frac{\partial}{\partial r}R(r)) + \frac{2mr^2}{\hbar^2}(V(r) - E) = -j^2
    \end{align}
    である.整理すると
    \begin{align}
      -\frac{\hbar^2}{2m}\frac{\partial^2}{\partial r^2}R(r) - \frac{\hbar^2}{2mr}\frac{\partial}{\partial r}R(r) + (V(r) - E)R(r) + \frac{\hbar^2 j^2}{2mr^2}R(r) = 0
    \end{align}
    である.これは\refe{eq-for-u}に$u=\sqrt{r}R$を代入した式と同一である.よって波動関数は
    \begin{align}
      \psi(r,\theta) = R(r)\e^{\i j\theta}
    \end{align}
    と表されることがわかる.ただし,$j$は整数,$u=\sqrt{r}R$が
    \begin{align}
      -\frac{\hbar^2}{2m}\frac{\r{d}^2 u}{\dd{r}^2} + \qty[V(r) + \frac{\hbar^2}{2m}\frac{(j^2 - 1/4)}{r^2}]u = Eu
    \end{align}
    を満たすとする.
    
    \refe{eq-for-u}で$r\to\infty$とすれば
    \begin{align}
      -\frac{\hbar^2}{2m}\frac{\r{d}^2 u}{\dd{u}^2} = Eu
    \end{align}
    であるため,
    \begin{align}
      R(r) = \frac{u}{\sqrt{r}} = \frac{\e^{\i kr}}{\sqrt{r}}
    \end{align}
    であることがわかる.

    以上の結果は2次元の散乱問題の境界条件が
    \begin{align}
      \psi(r,\theta) \simeq \e^{\i kx} + f(\theta)\frac{\e^{\i kr}}{\sqrt{r}},\ \r{for}\ r \to \infty
    \end{align}
    であることを示唆している.また,2次元の部分波展開はHankel関数$H^{(1)}$を用いて,
    \begin{align}
      \psi(r,\theta) = \e^{\i kx} + \sum_{j} c_j H^{(1)}_j (kr) \e^{\i j\theta}
    \end{align}
    となる.
  \end{myexc}
  \begin{myexc}{Griffith example11.2}{}
    Fermiの黄金律を用いてポテンシャル$V(\bm{r})$に対する散乱断面積を求めよ.
    \tcblower
    初期状態と終状態はそれぞれ入射波と散乱波であるため
    \begin{align}
      \psi_i = \frac{1}{\sqrt{L^3}} \e^{\i \bm{k}\cdot\bm{r}}\\
      \psi_f = \frac{1}{\sqrt{L^3}} \e^{\i \bm{k}'\cdot\bm{r}}
    \end{align}
    と表される.ここで,規格化のために周期$L$の周期的境界条件を課した.また,この周期的境界条件により,
    \begin{align}
      \bm{k}' = \frac{2\pi}{L}(n_x \bm{e}_x + n_y \bm{e}_y + n_z \bm{e}_z)
    \end{align}
    が得られる.$n_x,n_y,n_z$は整数である.散乱体のポテンシャル$V(\bm{r})$を摂動として取り入れると
    \begin{align}
      \bra{f}\hat{V}\ket{i} = \int \psi_f^{*}V(\bm{r})\psi_i \dd{\bm{r}} = \frac{1}{L^3} \int \e^{\i(\bm{k} - \bm{k}')\cdot \bm{r}} V(\bm{r}) \dd{\bm{r}}
    \end{align}
    を得る.

    次に,状態密度を決定する.エネルギーが$[E,E+\dd{E}]$の状態は,厚さ$k$,立体角$\dd{\Omega}$の球殻の中に
    \begin{align}
      \frac{k^2 \dd{k} \dd{\Omega}}{(2\pi/L)^3}
    \end{align}
    個含まれている.よって,
    \begin{align}
      \rho(E) \dd{E} = \frac{k^2 \dd{k} \dd{\Omega}}{(2\pi/L)^3} = \qty(\frac{L}{2\pi})^3 k^2 \dv{k}{E} \dd{E} \dd{\Omega}
    \end{align}
    $E= \frac{\hbar^2 k^2}{2m}$なので,
    \begin{align}
      \rho(E) = \qty(\frac{L}{2\pi})^3 \frac{\sqrt{2m^3E}}{\hbar^3} \dd{\Omega}
    \end{align}
    が得られる.よって,Fermiの黄金律から,単位時間に立体角$\dd{\Omega}$に粒子が到達する確率は
    \begin{align}
      \omega_{i\to\dd{\Omega}} = \frac{2\pi}{\hbar} \frac{1}{L^6} \abs{\int \e^{\i(\bm{k} - \bm{k}')\cdot \bm{r}} V(\bm{r}) \dd{\bm{r}}}^2 \qty(\frac{L}{2\pi})^3 \frac{\sqrt{2m^3E_f}}{\hbar^3} \dd{\Omega}
    \end{align}
    である.さらに,
    \begin{align}
      \sigma \dd{\Omega} = \frac{(\text{単位時間に立体角$\dd{\Omega}$に粒子が到達する確率})}{(単位時間単位面積当たりに粒子が入射する確率)}
    \end{align}
    であり,分子は入射波の確率の流れである.これは
    \begin{align}
      J = \frac{1}{L^3}\frac{\hbar k}{m}
    \end{align}
    である.以上より,散乱断面積
    \begin{align}
      \sigma = \frac{\omega_{i\to\dd{\Omega}}}{J \dd{\Omega}} = \abs{-\frac{m}{2\pi\hbar^2}\int \e^{\i(\bm{k}' - \bm{k})\cdot\bm{r}}V(\bm{r})\dd{\bm{r}}}^2
    \end{align}
    が得られる.これは第1Born近似で得られた式と同一である.
  \end{myexc}
\end{document}