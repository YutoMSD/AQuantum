\documentclass{report}
\input{../../head.tex}
\begin{document}
 散乱の波動関数は
 \begin{align}
  \label{WFofSC}
  \psi(\bm{r}) = \e^{\i kz} + \int G_0 (\bm{r} - \bm{r'}) \frac{2m}{\hbar^2} V(r') \psi(\bm{r'}) \dd{\bm{r}}
 \end{align}
 と表されるのであった.この波動関数を厳密に求めることは困難であるため近似をする.まず,式(\ref{WFofSC})を簡略化して
 \begin{align}
  \label{simpleWFofSC}
  \psi(\bm{r}) = \psi_0 + \int g V \psi(\bm{r'}) \dd{\bm{r'}}
 \end{align}
 と表現する.式(\ref{simpleWFofSC})を繰り返し代入していくと
 \begin{align}
  \psi(\bm{r}) &= \psi_0 + \int g V \psi(\bm{r'}) \dd{\bm{r'}}\\
  &= \psi_0 + \int gV \qty(\psi_0 + \int g V \psi(\bm{r''}) \dd{\bm{r''}}) \dd{\bm{r'}}\\
  &= \psi_0 + \int g V \psi_0(\bm{r'}) \dd{\bm{r'}} + \iint gVgV \psi(\bm{r''}) \dd{\bm{r'}} \dd{\bm{r''}}\\
  &= \psi_0 + \int g V \psi_0(\bm{r'}) \dd{\bm{r'}} + \iint gVgV \psi_0(\bm{r''}) \dd{\bm{r'}} \dd{\bm{r''}} + \iiint gVgVgV \psi_0(\bm{r'''}) \dd{\bm{r'}} \dd{\bm{r''}} \dd{\bm{r'''}} + \cdots
 \end{align}
 を得る.これを第1項までで近似する.これは散乱の波動関数を平面波で近似することに相当する.
 つまり,
 \begin{align}
  \psi \simeq \psi_0
 \end{align}
 とする.これを\textbf{第1Born近似}という\footnote{Max Born(1882-1970)}.

 Born近似を用いて散乱振幅を求める.$\psi(\bm{r'}) = \e^{\i kz} = \e^{\i \bm{k}\cdot\bm{r'}}$だから,
 \begin{align}
  f^{(1)}(\theta) &= -\frac{1}{4\pi} \int \e^{\i \bm{k'} \cdot \bm{r'}} \frac{2m}{\hbar} V(r') \psi(\bm{r'}) \dd{\bm{r'}}\\
  &= \simeq -\frac{1}{4\pi}\frac{2m}{\hbar^2}\int\e^{-\i (\bm{k'} - \bm{k})\cdot \bm{r'}} V(r') \dd{\bm{r'}}\\
  & \equiv -\frac{1}{4\pi}\frac{2m}{\hbar^2}\int\e^{-\i \bm{q}\cdot \bm{r'}} V(r') \dd{\bm{r'}}
 \end{align}
 となる.ここで散乱による運動量変化を$\bm{q}\equiv \bm{k'} - \bm{k}$と置いた.
 つまり,散乱振幅はポテンシャル$V(r')$のFourier変換から得られることがわかる.
\end{document}