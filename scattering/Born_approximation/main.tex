\documentclass{report}
\input{../../head.tex}
\begin{document}
 散乱の波動関数は
 \begin{align}
  \label{WFofSC}
  \psi(\bm{r}) = \e^{\i kz} + \int G_0 (\bm{r} - \bm{r}') \frac{2m}{\hbar^2} V(r') \psi(\bm{r}') \dd{\bm{r}'}
 \end{align}
 と表されるのであった.この波動関数を厳密に求めることは困難であるため近似をする.まず,式(\ref{WFofSC})を簡略化して
 \begin{align}
  \label{simpleWFofSC}
  \psi(\bm{r}) = \psi_0 + \int g V \psi(\bm{r}') \dd{\bm{r}'}
 \end{align}
 と表現する.式(\ref{simpleWFofSC})を繰り返し代入していくと
 \begin{align}
  \psi(\bm{r}) &= \psi_0 + \int g V \psi(\bm{r}') \dd{\bm{r}'}\\
  &= \psi_0 + \int gV \qty(\psi_0 + \int g V \psi(\bm{r}'') \dd{\bm{r}''}) \dd{\bm{r}'}\\
  &= \psi_0 + \int g V \psi_0(\bm{r}') \dd{\bm{r}'} + \iint gVgV \psi(\bm{r}'') \dd{\bm{r}'} \dd{\bm{r}''}\\
  &= \psi_0 + \int g V \psi_0(\bm{r}') \dd{\bm{r}'} + \iint gVgV \psi_0(\bm{r}'') \dd{\bm{r}'} \dd{\bm{r}''} + \iiint gVgVgV \psi_0(\bm{r}''') \dd{\bm{r}'} \dd{\bm{r}''} \dd{\bm{r}'''} + \cdots
 \end{align}
 を得る.これを第1項までで近似する.これは散乱の波動関数を平面波で近似することに相当する.
 つまり,
 \begin{align}
  \psi \simeq \psi_0
 \end{align}
 とする.これを\textbf{第1Born近似}という\footnote{Max Born(1882-1970)}.

 Born近似を用いて散乱振幅を求める.$\psi(\bm{r}') = \e^{\i kz} = \e^{\i \bm{k}\cdot\bm{r}'}$だから,
 \begin{align}
  f^{(1)}(\theta) &= -\frac{1}{4\pi} \int \e^{\i \bm{k}' \cdot \bm{r}'} \frac{2m}{\hbar} V(r') \psi(\bm{r}') \dd{\bm{r}'}\\
  &= \simeq -\frac{1}{4\pi}\frac{2m}{\hbar^2}\int\e^{-\i (\bm{k}' - \bm{k})\cdot \bm{r}'} V(r') \dd{\bm{r}'}\\
  & \equiv -\frac{1}{4\pi}\frac{2m}{\hbar^2}\int\e^{-\i \bm{q}\cdot \bm{r}'} V(r') \dd{\bm{r}'}
 \end{align}
 となる.ここで散乱による運動量変化を$\bm{q}\equiv \bm{k}' - \bm{k}$と置いた.
 つまり,散乱振幅はポテンシャル$V(r')$のFourier変換から得られることがわかる\footnote{$f^{(n)}$は第$n$Born近似による散乱振幅を意味する.}.
 \begin{myex}{湯川ポテンシャル}{}
湯川ポテンシャル
\begin{align}
  V(r) = V_0 \frac{\e^{-\mu r}}{\mu r}
\end{align}
による散乱を考える\footnote{湯川秀樹(1907-1981)}.これは,$V(r)$の到達距離が$\frac{1}{\mu}$ほどであり,核子同士に働く力を表す.物質中では,伝導電子に遮蔽された不純物のCoulombポテンシャルを表す.
このポテンシャルの下で散乱振幅を求める.
\begin{align}
  f^{(1)}(\theta) &= -\frac{1}{4\pi} \frac{2m}{\hbar^2} \iiint \e^{-\i qr'\cos\theta'}V(r') r'^2 \sin\theta'\dd{r'}\dd{\theta'}\dd{\phi'}\\
  &= -\frac{1}{4\pi}\frac{2m}{\hbar^2} 2\pi \int_{0}^{\infty} \qty(\frac{\e^{\i qr'} - \e^{-\i qr'}}{\i qr'})V(r') r'^2 \dd{r'}\\
  &= -\frac{mV_0}{\i \hbar^2 q \mu} \int_{0}^{\infty} \qty[\e^{(-\mu + \i q)r'} - \e^{(-\mu - \i q)r'}] \dd{r'}\\
  &= - \frac{2m V_0}{\hbar^2 \mu} \frac{1}{\mu^2 + q^2}
\end{align}
よって散乱断面積は
\begin{align}
  \sigma^{(1)}(\theta) &= \abs{f^{(1)}(\theta)}^2\\
  &= \qty(\frac{2m V_0}{\hbar^2 \mu})^2 \frac{1}{(\mu^2 + q^2)^2}\\
  &= \qty(\frac{2m V_0}{\hbar^2 \mu})^2 \frac{1}{(\mu^2 + 4K^2 \sin^2\theta/2)^2}
\end{align}
である.ここで,$\bm{k'}$と$\bm{k}$のなす角が$\theta$であるため$q = 2k\sin\theta/2$であることを用いた.
 \end{myex}

 次に,Born近似の適用範囲について考える.
 散乱振幅は
 \begin{align}
  f(\theta) = -\frac{1}{4\pi} \int \e^{-\i \bm{k}\cdot \bm{r}} \frac{2m}{\hbar^2}V(r)\psi(\bm{r})\dd{\bm{r}}
 \end{align}
 であり,散乱の波動関数は
 \begin{align}
  \label{BornWF}
  \psi(\bm{r}) = \psi_0(\bm{r}) + \int g(\bm{r} - \bm{r'}) V(r') \psi_0(\bm{r'}) \dd{\bm{r'}} + \cdots
 \end{align}
 である.この波動関数を
 \begin{align}
  \psi(\bm{r}) \simeq \psi_0(\bm{r})
 \end{align}
 と近似するのが第1Born近似であった.この近似がうまくいく,つまり,
 \begin{align}
  \int \e^{-\i \bm{k}\cdot \bm{r}} \frac{2m}{\hbar^2}V(r)\psi(\bm{r})\dd{\bm{r}}
 \end{align}
 を正しく評価するには,
 $V(r) \neq 0$となる$r\simeq 0$で,
 \begin{align}
  \psi(\bm{r}) \simeq \psi_0(\bm{r})
 \end{align}
 と近似できる必要がある.

 $r \simeq 0$で式(\ref{BornWF})の第1項がそれ以外の項より十分大きければいいため,
 \begin{align}
  \abs{\psi_0(\bm{0})} \gg \abs{\int g(\bm{0} - \bm{r}') V(r') \psi_0(\bm{r}') \dd{\bm{r}'}}
 \end{align}
 これを整理すると,
 \begin{align}
  \abs{\e^{\i kz}} \gg \abs{\int -\frac{2m}{\hbar^2}\frac{\e^{\i kr'}}{4\pi r'}V(r') \e^{\i \bm{k}\cdot\bm{r}'}\dd{\bm{r}'}}\\
  1 \gg \frac{m}{2\pi \hbar^2}
 \end{align}
\end{document}