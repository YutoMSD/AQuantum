\documentclass{report}
\input{../../head.tex}
\begin{document}
  今後の議論のために\textbf{立体角}を導入する.
  立体角とは1点を中心としたときの広がり具合を表す指標である.
  2次元の場合,微小円弧と半径の比は角度にのみ依り,$r$に依らない.
  つまり,
  \begin{align}
    \frac{\dd{l_1}}{r_1} = \frac{\dd{l_2}}{r_2} = \frac{\dd{\theta}}{1}
  \end{align}
  が成り立つ.この関係から平面角を,
  \begin{align}
    \dd{\theta} \coloneqq \frac{\dd{l}}{r}
  \end{align}
  と定義する.
  これを3次元に拡張する.\reff{solid_angle}として,今考えている状況の模式図を示す.
  単位球上の面積を考えると,
  \begin{align}
    \frac{\dd{S_1}}{r_1^2} = \frac{\dd{S_2}}{r_2^2} = \frac{\dd{S}}{1}
  \end{align}
  である.ゆえに立体角を,
  \begin{align}
    \dd{\Omega} = \frac{\dd{S}}{r^2} \label{solid-angle-def}
  \end{align}
  と定義する.また,これを球座標表示に変換すると,
  \begin{align}
    \dd{\Omega} = \frac{\dd{S}}{r^2} = \frac{r^2\sin\theta\dd{\theta}\dd{\phi}}{r^2} = \sin\theta\dd{\theta}\dd{\phi}
  \end{align}
  である.
  \begin{figure}[H]
    \centering
    \includegraphics[width = 0.6\columnwidth]{fig/solid_angle.pdf}
    \caption{立体角}\label{solid_angle}
  \end{figure}
\end{document}