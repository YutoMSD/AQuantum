\documentclass{report}
\input{../../head.tex}
\begin{document}
  今後の議論のために\textbf{立体角}を導入する.
  立体角とは1点を中心としたときの広がり具合を表す指標である.
  2次元の場合,微小円弧と半径の比は角度にのみ依り,$r$に依らない.
  つまり,
  \begin{equation}
    \frac{dl_1}{r_1}=\frac{dl_2}{r_2}=\frac{d\theta}{1}
  \end{equation}
  が成り立つ.この関係から平面角$d\theta=\frac{dl}{r}$を定義する.これを3次元に拡張する.単位球上の面積を考えると,
  \begin{equation}
    \frac{dS_1}{r_1^2}=\frac{dS_2}{r_2^2}=\frac{dS}{1}
  \end{equation}
  である.ここから立体角を$d\Omega=\frac{dS}r^2$と定義する.また,これを球座標表示に変換すると,
  \begin{equation}
    d\Omega = \frac{dS}{r^2} = \frac{r^2\sin\theta d\theta d\phi}{r^2} = \sin \theta d\theta d\phi
  \end{equation}
  である.
\end{document}