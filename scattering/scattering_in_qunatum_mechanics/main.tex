\documentclass{report}
\input{../../head.tex}
\begin{document}
  散乱体が球対称ポテンシャル$V(r)$を持つとする.Schrödinger方程式は以下のようになる.
  \begin{align}
    \label{SE_Scattering}
    \qty(-\frac{\hbar^2}{2m}\nabla^2 + V(r))\psi(\bm{r}) = E \psi(\bm{r})
  \end{align}
  ここで,$V(r)$は$r \to \infty$で十分早く$V \to 0$となるとする.$z$軸に沿う入射波は平面波なので
  \begin{align}
    \psi_{\r{in}} = \e^{\i k z}\ (z \to \infty)
  \end{align}
  と表せる.また,散乱波は外向きの球面波となるので
  \begin{align}
    \psi_{\r{sc}} \simeq \frac{\e^{\i kr}}{r}\ (r \to \infty)
  \end{align}
  である.散乱問題とは,$\r \to \infty$で
  \begin{align}
    \psi(\bm{r}) = \e^{\i kz} + f(\theta) \frac{\e^{\i kr}}{r}
  \end{align}
  を満たす式(\ref{SE_Scattering})の定常解を求めることである.$f(\theta)$を散乱振幅という.上式は重要なので強調しておく.
  \begin{itembox}[l]{散乱問題の境界条件}
    \begin{equation}
      \psi(\bm{r}) = \e^{\i kz} + f(\theta) \frac{\e^{\i kr}}{r}
    \end{equation}
  \end{itembox}

  この問題を解くための準備として確率密度($\rho = \psi^{*} \psi$)の時間変化を考える.
  \begin{align}
    \frac{\partial}{\partial t}\rho &= \frac{\partial}{\partial t}(\psi^{*}\psi)\\
    &= \frac{\partial \psi^{*}}{\partial t}\psi + \psi^{*}\frac{\partial \psi}{\partial t}\\
    &= -\frac{1}{\i\hbar}\qty(-\frac{\hbar^2}{2m}\nabla^2 + V)\psi^{*}\psi + \psi^{*}\frac{1}{\i\hbar}\qty(-\frac{\hbar^2}{2m}\nabla^2 + V)\psi\\
    &= \frac{\hbar}{2m\i}\qty[(\nabla^2 \psi^{*})\psi - \psi^{*}(\nabla^2\psi)]\\
    &= -\frac{\hbar}{2m\i}\nabla \cdot (\psi^{*}\nabla \psi - \psi \nabla \psi^{*})
  \end{align}
  ここで,確率流密度を
  \begin{align}
    \bm{j} &= \frac{\hbar}{2m\i} (\psi^{*}\nabla \psi - \psi \nabla \psi^{*})\\
    &= \frac{\hbar}{m}\Im (\psi^{*}\nabla\psi)
  \end{align}
  と定義すれば,確率密度に対する連続の式
  \begin{align}
    \frac{\partial}{\partial t} \rho = -\nabla \cdot \bm{j}
  \end{align}
  を得る\footnote{
   両辺に電荷をかければ電荷保存の法則となる. 
  }.

  次に微分断面積と散乱振幅の関係を考える.
  入射波は$\psi_{\r{in}} = \e^{\i kz}$である.入射波の確率流密度は
  \begin{align}
    j_z &= \frac{\hbar}{m} \Im (\e^{-\i kz} \i k e^{\i kz}) \\
    &= \frac{\hbar k}{m}
  \end{align}
  である.散乱波は$\psi_{\r{sc}} = \frac{f(\theta)}{r}\e^{\i kr}$である.散乱波の確率流密度は
  \begin{align}
    j_r(\theta) &= \frac{\hbar}{m}\Im (\psi^{*}_{\r{sc}}\nabla\psi_{\r{sc}})\\
    &\simeq \frac{\hbar}{m}\Im (\psi^{*}_{\r{sc}}\frac{\partial}{\partial r}\psi_{\r{sc}})\\
    &= \frac{\hbar}{m} \Im \qty[\frac{f(\theta)}{r} \e^{-\i kr}\qty(\frac{\i k}{r} - \frac{1}{r^2})f(\theta)\e^{\i kr}]\\
    &\simeq \frac{\hbar}{m} \Im \qty[\frac{f(\theta)}{r} \e^{-\i kr}\frac{\i k}{r}f(\theta)\e^{\i kr}]\\
    &= \frac{\hbar}{m}\frac{k}{r^2} \abs{f(\theta)}^2\\
    &= \frac{\abs{f(\theta)}^2}{r^2}j_z
  \end{align}
  である.ここで,1行目から2行目では
  \begin{align}
    \nabla \psi_{\r{sc}} &= \qty(\frac{\partial}{\partial r}\bm{e}_r + \frac{1}{r}\frac{\partial}{\partial \theta}\bm{e}_{\theta} + \frac{1}{r \sin \theta}\frac{\partial}{\partial \phi} \bm{e}_{\phi})\psi_{\r{sc}}\\
    &\simeq \frac{\partial}{\partial r}\psi_{\r{sc}} \bm{e}_r\ (r \to \infty)
  \end{align}
  という近似を用いた.
  
  微分断面積と粒子数の関係
  \begin{align}
    \dd{N} = \sigma (\theta) N \dd{\Omega}
  \end{align}
  の両辺を$N$で割る.
  \begin{align}
    \frac{\dd{N}}{N} = \sigma (\theta) \dd{\Omega}
  \end{align}
  上式の左辺を言葉に直すと,
  \begin{align}
    \frac{\text{単位時間に位置$(r,\theta)$にある$\dd{S}$に入射する粒子数}}{\text{単位時間単位面積当たりの入射粒子数}}
  \end{align}
  である.これは
  \begin{align}
    \frac{\text{単位時間に位置$(r,\theta)$にある$\dd{S}$に粒子が入射する確率}}{\text{単位時間単位面積当たりに粒子が入射する確率}} = \frac{j_z \dd{S}}{j_z}
  \end{align}
  と等しいので
  \begin{align}
    \sigma(\theta)\dd{\Omega} = \frac{\abs{f(\theta)}^2\dd{S}}{r^2}
  \end{align}
  が得られる.よって,$\dd{\Omega} = \frac{\dd{S}}{r^2}$だから,
  \begin{align}
    \sigma(\theta) = \abs{f(\theta)}^2
  \end{align}
  という関係が成り立つ.これは,散乱振幅から微分断面積が求められることを意味する.
  \begin{itembox}[l]{散乱振幅と微分断面積の関係}
    \begin{align}
      \sigma(\theta) = \abs{f(\theta)}^2
    \end{align}
  \end{itembox}

  次に$f(\theta)$を求めるための式を作る.
  散乱のSchrödinger方程式は
  \begin{align}
    \qty(-\frac{\hbar^2}{2m}\nabla^2 + V(r))\psi(\bm{r}) = E\psi(\bm{r})
  \end{align}
  である.ここで,
  \begin{align}
    \begin{dcases}
      \kappa = \frac{\sqrt{2mE}}{\hbar}\\
      U(\bm{r}) = \frac{2m}{\hbar^2} V(\bm{r})
    \end{dcases}
  \end{align}
  とおくと,
  \begin{align}
    \label{Helmholtz align}
    \qty(\nabla^2 + \kappa^2)\psi(\bm{r}) = U(\bm{r}) \psi(\bm{r})
  \end{align}
  と表せる.式(\ref{Helmholts align})の解は,
  \begin{align}
    \qty(\nabla^2 + \kappa^2)\phi(\bm{r}) = 0
  \end{align}
  の一般解$\phi(\bm{r}) = \e^{\i k z}$と
  \begin{align}
    \label{Tokkai}
    \qty(\nabla^2 + \kappa^2)\chi(\bm{r}) = U(\bm{r}) \chi(\bm{r})
  \end{align}
  と特解$\chi(\bm{r})$の和である.
  
  では,式(\ref{Tokkai})の特解を求めよう.レシピはこうである.
  \begin{enumerate}
    \item $\qty(\nabla^2 + \kappa^2)G_0(\bm{r}) = \delta(\bm{r})$を満たすGreen関数$G_0$を求める.
    \item $\chi(\bm{r}) = \int \dd{\bm{r'}} G_0(\bm{r} - \bm{r'}) U(\bm{r'}) \phi(\bm{r'})$から特解を求める.
  \end{enumerate}
  2から特解が求められるのは,
  \begin{align}
    \qty(\nabla^2 + \kappa^2)\chi(\bm{r}) &= (\nabla^2 + \kappa^2)\int \dd{\bm{r'}} G_0(\bm{r} - \bm{r'}) U(\bm{r'}) \phi(\bm{r'})\\
    &= \int \dd{\bm{r'}} \delta(\bm{r} - \bm{r'}) U(\bm{r'}) \phi(\bm{r'})\\
    &= U(\bm{r}) \phi(\bm{r})
  \end{align}
  が成り立つためである.

  Green関数を求めよう.
  \begin{align}
    \qty(\nabla^2 + \kappa^2)G_0(\bm{r}) = \delta(\bm{r})
  \end{align}
  の両辺をFourier変換する.
  \begin{align}
    \qty(\nabla^2 + \kappa^2)\int \dd{\bm{k'}} G_0(\bm{k'}) \e^{\i\bm{k'}\cdot \bm{r}} &= \frac{1}{(2\pi)^3}\int \dd{\bm{k'}} \e^{\i\bm{k'}\cdot\bm{r}}\\
    \int \dd{\bm{k}} \qty(-\bm{k'}^2 + \kappa^2)G_0(\bm{k'}) \e^{\i \bm{k'}\cdot \bm{r}} &= \frac{1}{(2\pi)^3} \int \dd{\bm{k'}} \e^{\i \bm{k'}\cdot \bm{r}}
  \end{align}
  両辺を比較すると
  \begin{align}
    (\kappa^2 - k'^2)G_0(\bm{k'}) &= \frac{1}{(2\pi)^2}\\
    G_0(\bm{k'}) &= \frac{1}{(2\pi)^2 (\kappa^2 - k'^2)}
  \end{align}
  を得る.よってGreen関数は
  \begin{align}
    G_0(\bm{r}) = \frac{1}{(2\pi)^2}\int \dd{\bm{k'}} \frac{\e^{\i\bm{k'}\cdot\bm{r}}}{\kappa^2 - k'^2}
  \end{align}
  と表される.極座標に変換しこの積分を実行する.
  \begin{align}
    G_0(\bm{r}) &= \frac{1}{(2\pi)^2} \int \frac{\exp(\i k' r \cos \theta)}{\kappa^2 - k'^2}k'^2\sin\theta\dd{\theta}\dd{\phi}\dd{k'}\\
    &= \int \frac{1}{(2\pi)^3}2\pi \int_{0}^{\infty} k'^2 \dd{k'} \int_{0}^{\pi}\sin\theta\dd{\theta}\frac{\exp(\i k' r\cos\theta)}{\kappa^2 - k'^2}\\
    &= \frac{1}{(2\pi)^2} \int_{0}^{\infty} \frac{k'^2\dd{k'}}{\kappa^2 - k'^2}\frac{\e^{\i k'r} - \e^{-\i k'r}}{\i k'r}\\
    &= \frac{1}{4\pi^2\i r} \int_{0}^{\infty} k' \dd{k'} \frac{\e^{\i k'r} - \e^{-\i k'r}}{(\kappa - k')(\kappa + k')}\\
    &= \frac{1}{8\pi^2\i r} \int_{-\infty}^{\infty}k' \dd{k'} \frac{\e^{\i k'r} - \e^{-\i k'r}}{(\kappa - k')(\kappa + k')}\\
    &= \frac{1}{8\pi^2\i r} \int_{-\infty}^{\infty} k' \dd{k'} \qty(\frac{\e^{\i k'r}}{(\kappa - k')(\kappa + k')} - \frac{\e^{-\i k'r}}{(\kappa - k')(\kappa + k')})\\
    &= \frac{1}{8\pi^2 \i r}\qty(I_1 - I_2)
  \end{align}
  \begin{align}
    I_1 \equiv \int_{-\infty}^{\infty} \dd{k'} \frac{k' \e^{\i k'r}}{(\kappa - k')(\kappa + k')},\ I_2 \equiv \int_{-\infty}^{\infty} \dd{k'} \frac{k' \e^{-\i k'r}}{(\kappa - k')(\kappa + k')}
  \end{align}
  最後の積分を実行するにはCauchyの積分公式を用いる.
  \begin{itembox}[l]{Cauchyの積分公式}
    \begin{align}
      \oint \frac{f(z)}{(z-z_0)}\dd{z} = 2\pi \i f(z_0)
    \end{align}
  \end{itembox}
  今回の場合は極は$\kappa$と$-\kappa$である.しかし,極の避け方にはいくつかのパターンがあり,それに応じてGreen関数の計算結果は変化する.今回は図\ref{Integral}のように$-\kappa$を上に避け,$\kappa$を下に避ける.
  まず$I_1$を計算する.$I_1$は$k'$が虚部が正の値を取るときに小さな値をとる.よって,積分路は上方に閉じる形とする.この時,積分範囲内に極は$\kappa$のみになるため,
  \begin{align}
    I_1 = -\i\pi\e^{\i kr}
  \end{align}
  となる.$I_2$の積分路は下方に閉じる形とする.この時積分範囲内に含まれる極は$-\kappa$のみである.
  よって,
  \begin{align}
    I_2 = \i\pi\e^{\i kr}
  \end{align}
  したがって,Green関数は
  \begin{align}
    G_0(r) = -\frac{\e^{\i kr}}{4\pi r}
  \end{align}
  となる.これは外に広がる球面波を表す.

  \begin{figure}[H]
    \centering
    \includegraphics[width=0.7\columnwidth]{fig/IntegralGreen.pdf}
    \caption{積分経路}
    \label{Integral}
  \end{figure}

  以上の計算からSchrödinger方程式(\ref{SE_Scattering})の形式解は
  \begin{align}
    \label{keishiki}
    \psi(\bm{r}) = \e^{\i kz} - \frac{1}{4\pi} \int \frac{\e^{\i k \abs{\bm{r} - \bm{r'}}}}{\abs{\bm{r} - \bm{r'}}} \frac{2m}{\hbar^2}V(r')\psi(\bm{r'}) \dd{r'}
  \end{align}
  である.これは平面波と球面波の和となっている.

  次に式(\ref{keishiki})から散乱振幅$f(\theta)$を求める.仮定として,$V(r')$は$r'<a$でのみ$V\neq 0$とする.式(\ref{keishiki})において
  $r  \gg r'$では
  \begin{align}
    \abs{\bm{r} - \bm{r'}} &= \sqrt{r^2 - 2\bm{r} \cdot \bm{r'} + r'^2}\\
    &= r \sqrt{1 - \frac{2\bm{r'} \cdot \bm{n}}{r + \qty(\frac{r'}{r})^2}}\\
    &\simeq r - \bm{n}\cdot\bm{r'}
  \end{align}
  が成り立つ.よって,
  \begin{align}
    \e^{\i k \abs{\bm{r} - \bm{r'}}} &\simeq \e^{\i k \qty(r - \bm{n} \cdot \bm{r'})}\\
    &= \e^{\i k r} - \e^{-\i \bm{k'} \cdot \bm{r'}}
  \end{align}
  を得る.ここで,$\bm{k'} = k \bm{n}$を$z$軸と角度$\theta$をなす散乱方向の波数ベクトルとする.また,
  \begin{align}
    \frac{1}{\abs{\bm{r} - \bm{r'}}} &\simeq \frac{1}{r\qty(1 - \frac{\bm{n}\cdot\bm{r'}}{r})}\\
    &= \simeq \frac{1}{r}
  \end{align}
  である.以上より,
  \begin{align}
    \psi(\bm{r}) = \e^{\i kz} - \qty(\frac{1}{4\pi} \int \dd{\bm{r'}} \e^{- \i \bm{k'} \cdot \bm{r'}} \frac{2m}{\hbar^2}V(r')\psi(r'))\frac{\e^{\i kr}}{r}
  \end{align}
  である.したがって,散乱振幅は
  \begin{align}
    f(\theta) = -\frac{1}{4\pi} \int \dd{\bm{r'}} \e^{- \i \bm{k'} \cdot \bm{r'}} \frac{2m}{\hbar^2}V(r')\psi(r')
  \end{align}
  となる.
\end{document}
