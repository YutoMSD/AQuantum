\documentclass{report}
\input{../../head.tex}
\begin{document}
単位時間単位面積当たり$N$個の粒子を$z$軸方向に入射する.単位時間内に散乱体からの位置$(dr,\theta,\phi)$にある面積$dS$の検出器に到達する粒子数は
\begin{align}
  dN \propto N \frac{dS}{r^2} = Nd\Omega
\end{align}
を満たす.比例係数を$\sigma (\theta, \phi)$とする.
\begin{align}
  dN = \sigma(\theta, \phi) N d\Omega
\end{align}
この$\sigma(\theta ,\phi)$を\textbf{微分断面積}という.
散乱が$z$軸まわりに軸対称なとき,$\phi$依存性を取り除き$\sigma(\theta, \phi) = \sigma (\theta)$とできる.
このとき,全断面積を式(\ref{Zendanmenseki})のように定義する.
\begin{align}
  \sigma^{\r{tot}} = \int \sigma(\theta) \dd{\Omega} = 2\pi \int_{0}^{\pi} \sigma(\theta) \sin \theta \dd{\theta}
\end{align}
\end{document}