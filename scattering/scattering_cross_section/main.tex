\documentclass{report}
\input{../../head.tex}
\begin{document}
  単位時間単位面積当たり$N$個の粒子を$z$軸方向に入射する.
  \refe{scattering_detection}に散乱の様子を示す.
  単位時間内に散乱体からの位置$(r,\theta,\phi)$にある面積$\dd{S}$の検出器に到達する粒子数は,
  \begin{align}
    \dd{N} \propto N \frac{\dd{S}}{r^2} = N\dd{\Omega}
  \end{align}
  を満たす.比例係数を$\sigma (\theta, \phi)$とすると,
  \begin{align}
    \dd{N} = \sigma(\theta, \phi) N \dd{\Omega}
  \end{align}
  と書ける.
  $\sigma(\theta, \phi)$を\textbf{微分断面積}という.
  散乱が$z$軸まわりに軸対称なとき,$\phi$依存性を取り除き$\sigma(\theta, \phi) = \sigma (\theta)$とできる.
  また,全断面積を,
  \begin{align}
    \sigma^{\r{tot}} \coloneqq \int \sigma(\theta) \dd{\Omega} = 2\pi \int_{0}^{\pi} \sigma(\theta) \sin \theta \dd{\theta}\label{total-area}
  \end{align}
  のように定義する.
  \begin{figure}[H]
    \centering
    \includegraphics[width = 0.6\columnwidth]{fig/scattering_diagram.pdf}
    \caption{散乱体と検出器}\label{scattering_detection}
  \end{figure}
\end{document}