\documentclass{report}
\input{../../head.tex}
\begin{document}
  近似法の一つである部分波展開を扱う.
  まず,半古典論を用いて散乱が起こる条件から,方位量子数$l$ごとに波動関数を展開して,$l$が小さい波動関数のみ考えればよいことが分かる.
  次に,波動関数の基底展開について議論を行う.
  前節の議論より,球Bessel関数と球Neumann関数を用いて任意の関数が展開できるのであった.
  また,散乱には$\phi$依存性が無いという条件を用いれば,球Neumann関数の成分は0であると分かる.
  さらに,球Bessel関数にかかる係数のラベルは方位量子数のラベルと一致することが分かる.
  \subsection{散乱の条件}
    古典力学において角運動量$\bm{L}$は,
    \begin{align}
      \bm{L} = \bm{r} \times \bm{p} = m\bm{r} \times \bm{v}
    \end{align}
    と書けるのであった.
    球対称ポテンシャルの下では,
    \begin{align}
      \dv{\bm{L}}{t} &= m\dv{\bm{r}}{t} \times \bm{v} + m\bm{r} \times \dv{\bm{v}}{t} \\
      &= \bm{r} \times \bm{F} \\
      &= \bm{r} \times (-\grad V(r)) \\
      &= \bm{r} \times \abs{\grad V}\frac{\bm{r}}{r} \\
      &= 0
    \end{align}
    より,角運動量は保存される.
    衝突パラメータを$b$,運動量を$\bm{p}$とした粒子の角運動量は,
    \begin{align}
      \abs{\bm{L}} &= \abs{\bm{r} \times \bm{p}} \\
      &= pb
    \end{align}
    である.散乱体を半径$a$の球とすると,衝突の条件は
    \begin{align}
      b &< a \\
      L/p &< a \\
      L &< pa \label{ConditionofSC}
    \end{align}
    である.つまり,角運動量が小さい粒子のみ散乱することがわかる.
    \par
    \refe{ConditionofSC}を半古典的な散乱条件へ書き直すことを考える.
    量子力学では,角運動量の大きさ$L$は,$l = 0, 1, 2, \cdots$の値を取る方位量子数$l$を用いて$L = \hbar \sqrt{l(l + 1)}$と書けるので,
    衝突の条件は,
    \begin{align}
      L = \hbar \sqrt{l(l + 1)} < pa = \hbar ka\label{scattering-condition-of-l}
    \end{align}
    と書ける.\refe{scattering-condition-of-l}を見れば,散乱の影響を受けるのは$l$が小さいときのみであることがわかる.
    よって波動関数を,
    \begin{align}
      \psi = \phi^{(l = 0)} + \phi^{(l = 1)} + \cdots
    \end{align}
    のように異なる$l$に属する固有関数で展開し,$l$が小さい状態についてだけ散乱の影響を考える.これを\textbf{部分波展開}という.
    以下では波動関数をLegendre多項式で基底展開する手順を説明する.
    Helmholtz方程式,
    \begin{align}
      [\laplacian + \kappa^2]u(\bm{r}) = 0
    \end{align}
    を球座標系で表すと,
    \begin{align}
      \qty[\frac{1}{r^2}\pdv{r}\qty(r^2\pdv{r}) + \frac{1}{r^2 \sin \theta} \pdv{\theta}\qty(\sin\theta\pdv{\theta}) + \frac{1}{r^2 \sin^2\theta}\pdv[2]{\phi} + \kappa^2]u(r, \theta, \phi) = 0 \label{helmholtz-eq-in-sphere}
    \end{align}
    である.
  この方程式の解は動径波動関数$R(r)$と球面調和関数$Y_{l, m}(\theta, \phi)$を用いて,
  \begin{align} 
    u(r, \theta, \phi) = \sum_{l = 0}^{\infty}\sum_{m = -l}^{l}C_{lm}R(r)Y_{l, m}(\theta, \phi)
  \end{align}
  と表される.ここで,球面調和関数とLegendre陪関数,Legendre多項式は以下の関係にある.
  \begin{align}
    Y_{l,m}(\theta,\phi) &= (-1)^m \qty[\frac{(2l+1)}{4\pi}\frac{(l-m)!}{(l+m)!}]^{1/2}P_l^m(\cos\theta)\exp(\i m\phi) \\
    P_l^m(x) &= (1-x^2)^{m/2} \frac{\r{d}^m P_l(x)}{\dd{x}^m}
  \end{align}
  いま,角度$\phi$によらないとすると,$m=0$とすることで
  \begin{align}
    \psi(r,\theta) = \sum_{l=0}^{\infty} (2l+1)R_l(r)P_l(\cos\theta)
  \end{align}
  を得る.$(2l+1)$は縮退の数を表す.これを式(\ref{helmholtz-eq-in-sphere})に代入する.
  \begin{align}
    \qty[\frac{1}{r}\frac{\partial^2}{\partial r^2}r + \frac{1}{r^2 \sin \theta} \frac{\partial}{\partial \theta}\qty(\sin\theta\frac{\partial}{\partial \theta})
    + \kappa^2] \sum_{l=0}^{\infty} (2l+1)R_l(r)P_l(\cos\theta)= 0
  \end{align}
  整理すると
  \begin{align}
    \label{Helmholtz separated}
    \frac{P_l(\cos\theta)}{r}\qty[2\frac{\partial R_l(r)}{\partial r} + r \frac{\partial^2 R_l(r)}{\partial r^2}]
    + \frac{R_l(r)}{r^2 \sin \theta} \frac{\partial}{\partial \theta}\qty(\sin\theta\frac{\partial}{\partial\theta}P_l(\cos\theta)) + \kappa^2 R_l(r)P_l(\cos\theta) = 0
  \end{align}
  である.ここで,Legendre多項式は
  \begin{align}
    \frac{\r{d}}{\dd{x}}\qty[(1-x^2)\frac{\r{d}P_l(x)}{\dd{x}}] + l(l+1)P_l(x) = 0
  \end{align}
  を満たす.これを変数変換すると
  \begin{align}
    \frac{1}{\sin\theta}\frac{\r{d}}{\dd{\theta}}\qty[\sin\theta\frac{\r{d}P_l(\cos\theta)}{\dd{\theta}}] + l(l+1)P_l(\cos\theta) = 0
  \end{align}
  となる.よって,式(\ref{Helmholtz separated})は
  \begin{align}
    \frac{P_l(\cos\theta)}{r}\qty[2\frac{\partial R_l(r)}{\partial r} + r \frac{\partial^2 R_l(r)}{\partial r^2}]
    - l(l+1)\frac{R_l(r)}{r^2}P_l(\cos\theta)  + \kappa^2 R_l(r)P_l(\cos\theta) = 0
  \end{align}
  と変形される.したがって,$R_l(r)$が満たすべき式は
  \begin{align}
    \qty[\frac{\r{d}^2}{\dd{r}^2} + \frac{2}{r} \dv{}{r} + \kappa^2 - \frac{l(l+1)}{r^2}]R_l(r) = 0
  \end{align}
  である.この微分方程式の解$R_l(r)$は球面Bessel関数$j_l(\kappa r)$と球面Neumann関数$n_l(\kappa r)$の和で与えられる.よって,波動関数は
  \begin{align}
    \psi(r,\theta) = \sum_{l=0}^{\infty} (2l+1)\qty[A_l j_l(\kappa r) + B_l n_l(\kappa r)] P_l(\cos\theta)
  \end{align} 
  となる.また,球面Bessel関数と球面Neumann関数は以下の性質を持つ.
  \begin{align}
    x \to 0 \\
    j_l(x) &\to \frac{x^l}{(2l+1)!}\qty(1 - \frac{x^2}{2(2l+3)}\cdots) \\
    n_l(x) &\to -\frac{(2l -1)!!}{x^{l+1}}
  \end{align}
  平面波$\e^{\i  z} = \e^{\i kr\cos\theta}$は明らかにHelmholtz方程式を満たす.さらに,$r=0$で有限の値をもつため,
  \begin{align}
    B_l =0
  \end{align} 
  であることがわかる.以上の計算から,
  \begin{align}
    \e^{\i \kappa r\cos\theta} = \sum_{l=0}^{\infty} (2l+1)A_l j_l(\kappa r) P_l(\cos\theta)
  \end{align}
  を得る.左辺を展開すると,
  \begin{align}
    \sum_{l=0}^{\infty}\frac{(\i \kappa r \cos\theta)^l}{l!}
  \end{align}
  右辺を展開すると,
  \begin{align}
    \sum_{l=0}^{\infty} (2l+1)A_l\frac{(\kappa l)^l}{(2l+1)!!}P_l(\cos\theta)
  \end{align}
  である.また,$P_l(\cos\theta)$の$(\cos\theta)^l$の項の係数は
  \begin{align}
    \frac{1}{2^l l!}\frac{(2l)!}{l!}
  \end{align}
  である.よって,左辺と右辺を比較することで
  \begin{align}
    \frac{(\i \kappa r \cos\theta)^l}{l!} = (2l+1)A_l \frac{(2l)!}{2^l (l!)^2}\frac{(\kappa r \cos\theta)^l}{(2l+1)!!} \\
    A_l = \i^l
  \end{align}
  を得る.したがって,\textbf{Rayleighの公式}
  \begin{align}
    \e^{\i kz} = \e^{\i kr\cos\theta} = \sum_{l=0}^{\infty} (2l+1)\i^l j_l(kr)P_l(\cos\theta)
  \end{align}
  が成り立つ.以上で,平面波のLegendre多項式による表現が得られた.同様に,散乱振幅を未定係数$a_l$を用いて
  \begin{align}
    f(\theta) = \sum_{l=0}^{\infty} (2l+1)a_l P_l(\cos\theta)
  \end{align}
  と展開する.これにより球面波は
  \begin{align}
    \sum_{l=0}^{\infty} (2l+1)a_l P_l(\cos\theta) \frac{\e^{\i kr}}{r}
  \end{align}
  と展開される.Legendre多項式は完全性と直交性をもつためこの展開は妥当である.
  したがって,部分波展開した散乱の波動関数は
  \begin{align}
    \psi(\bm{r}) = \sum_{l = 0}^{\infty} (2l + 1)\i^l j_l(kr)P_l(\cos\theta) + \sum_{l = 0}^{\infty} (2l + 1)a_l P_l(\cos\theta) \frac{\e^{\i kr}}{r}
  \end{align}
  である.
  \begin{itembox}[l]{部分波展開した散乱の波動関数}
  \begin{align}
    \psi(\bm{r}) = \sum_{l = 0}^{\infty} (2l + 1)\i^l j_l(kr)P_l(\cos\theta) + \sum_{l = 0}^{\infty} (2l + 1)a_l P_l(\cos\theta) \frac{\e^{\i kr}}{r}\label{PartialWave}
  \end{align}
  \end{itembox}
  \subsection{散乱断面積と全断面積}
    部分波展開した波動関数より計算した散乱断面積は,
    \begin{align}
      \sigma(\theta) &= \abs{f(\theta)}^2 \\
      &= \sum_{l} \sum_{l'} (2l+1)(2l'+1) a_l^{*}a_{l'} P_l(\cos\theta) P_{l'}(\cos\theta)
    \end{align}
    である.全断面積は
    \begin{align}
      \sigma^{\r{tot}} &= 2\pi \int \sigma(\theta) \sin\theta \dd{\theta} \\
      &= 2\pi \sum_{l=0}^{\infty} (2l+1)^2\abs{a_l}^2\qty(\frac{2}{2l+1}) \\
      &= 4\pi \sum_{l=0}^{\infty} (2l+1) \abs{a_l}^2
    \end{align}
    である.ここで,Legendre多項式の直交性
    \begin{align}
      \int_{0}^{\pi} P_l(\cos\theta)P_{l'}(\cos\theta) \sin\theta \dd{\theta} = \frac{2}{2l+1}\delta_{l,l'}
    \end{align}
    を用いた.
    以上の議論から,部分波展開を用いた散乱問題は未定係数$a_l$を求めることに帰着する.
\end{document}