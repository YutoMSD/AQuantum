\documentclass{report}
\input{../../head.tex}
\begin{document}
  近似法の一つである部分波展開を扱う.
  まず,半古典論を用いて散乱が起こる条件から,方位量子数$l$ごとに波動関数を展開して,$l$が小さい波動関数のみ考えればよいことが分かる.
  次に,波動関数の基底展開について議論を行う.関数空間には様々な直交基底が存在するが,今回はLegendre多項式を基底に取る.
  最後に,展開に用いた未定係数を用いて,散乱断面積や全断面積をその未定係数を用いて表す.
  \subsection{散乱の条件}
    古典力学において角運動量$\bm{L}$は,
    \begin{align}
      \bm{L} = \bm{r} \times \bm{p} = m\bm{r} \times \bm{v}
    \end{align}
    と書けるのであった.
    球対称ポテンシャルの下では,
    \begin{align}
      \dv{\bm{L}}{t} &= m\dv{\bm{r}}{t} \times \bm{v} + m\bm{r} \times \dv{\bm{v}}{t} \\
      &= \bm{r} \times \bm{F} \\
      &= \bm{r} \times (-\grad V(r)) \\
      &= \bm{r} \times \abs{\grad V}\frac{\bm{r}}{r} \\
      &= 0
    \end{align}
    より,角運動量は保存される.
    衝突パラメータを$b$,運動量を$\bm{p}$とした粒子の角運動量は,
    \begin{align}
      \abs{\bm{L}} &= \abs{\bm{r} \times \bm{p}} \\
      &= pb
    \end{align}
    である.散乱体を半径$a$の球とすると,衝突の条件は
    \begin{align}
      b &< a \\
      L/p &< a \\
      L &< pa \label{ConditionofSC}
    \end{align}
    である.つまり,角運動量が小さい粒子のみ散乱することがわかる.
    \par
    \refe{ConditionofSC}を半古典的な散乱条件へ書き直すことを考える.
    量子力学では,角運動量の大きさ$L$は,$l = 0, 1, 2, \cdots$の値を取る方位量子数$l$を用いて$L = \hbar \sqrt{l(l + 1)}$と書けるので,
    衝突の条件は,
    \begin{align}
      L = \hbar \sqrt{l(l + 1)} < pa = \hbar ka\label{scattering-condition-of-l}
    \end{align}
    と書ける.\refe{scattering-condition-of-l}を見れば,散乱の影響を受けるのは$l$が小さいときのみであることがわかる.
    よって波動関数を,
    \begin{align}
      \psi = \phi^{(l = 0)} + \phi^{(l = 1)} + \cdots
    \end{align}
    のように異なる$l$に属する固有関数で展開し,$l$が小さい状態についてだけ散乱の影響を考える.これを\textbf{部分波展開}という.
  \subsection{波動関数の基底展開}
    さて,\refe{psi-r-in-sc-k}で示した散乱の波動関数,
    \begin{align}
      \psi(\bm{r}) = \e^{\i kz} + f(\theta)\frac{\e^{\i kr}}{r}
    \end{align}
    を部分波展開する\footnote{ここから\refe{PartialWave}までは飛ばしても良い.}ことを考える.
    まず,Legendre多項式の性質を調べて,直交性を知る.
    次に,Legendre多項式を用いて波動関数を展開して,展開係数$B_{ml}$が満たすべき微分方程式を導く.
    続いて,Besselの微分方程式の解であるBessel関数の表式を求めて,球Bessel関数と球Neumann関数,球Hankel関数を定義する.
    最後に,展開係数$B_{ml}$が球Bessel関数と球Neumann関数の線型結合で書けることを確かめて,線型結合の係数が波動関数を特徴づけるものだと知る.
    \subsubsection{Strum-Liouvillle演算子のHermite性}
      $a < b$として,$x\in \qty[a, b]$で定義された関数空間$V$を考える.$\rho(x)$を非負の実数関数として,$f, g\in V$に対して,
      \begin{align}
        \inner{f}{g} \coloneqq \int_{a}^{b} f^*(x)g(x)\rho(x)\dd{x}
      \end{align}
      なる内積を入れる.
      関数空間$V$上の演算子として$\mathcal{L}$を,
      \begin{align}
        \mathcal{L} \coloneqq \frac{1}{\rho(x)}\qty[\dv{x}\qty{p(x)\dv{x}} + q(x)]\label{strum-liouville-operator-def}
      \end{align}
      とする.
      \refe{strum-liouville-operator-def}なる形をした演算子をStrum-Liouvillle演算子という.
      境界条件を,$\forall f \in V$について,
      \begin{align}
        \begin{dcases}
          f(a) = f(b) \\ 
          p(a)f'(a) = p(b)f'(b)
        \end{dcases}
      \end{align}
      とすると,
      \begin{align}
        \inner{f}{\mathcal{L}g} = \inner{\mathcal{L}f}{g}\label{hermite-def}
      \end{align}
      が成立する.
      \refe{hermite-def}なる関係が成り立つ演算子$\mathcal{L}$をHermite演算子という.
      \begin{proof}
        内積の定義より,
        \begin{align}
          \inner{f}{\mathcal{L}g} &= \int_{a}^{b} f^*(x)\frac{1}{\rho(x)}\qty[\dv{x}\qty{p(x)\dv{x}g(x)} + q(x)g(x)]\rho(x)\dd{x} \\ 
          &= \int_{a}^{b} f^*(x)\qty[\dv{x}\qty{p(x)\dv{x}g(x)} + q(x)g(x)]\dd{x} \\ 
          &= \qty[f^*(x)g'(x)]_{a}^{b} - \int_{a}^{b}f'^*(x)p(x)g'(x)\dd{x} + \int_{a}^{b}f^*(x)q(x)g(x)\dd{x} \\ 
          &= \qty[f^*(x)p(x)g'(x)]_{a}^{b} - \qty[f'^*(x)p(x)g(x)]_{a}^{b} + \int_{a}^{b}g(x)\frac{1}{\rho(x)}\qty[\dv{x}\qty(p(x)\dv{x}f^*(x)) + q(x)f^*(x)]\rho(x)\dd{x} \\ 
          &= \qty[f^*(x)p(x)g'(x)]_{a}^{b} - \qty[f'^*(x)p(x)g(x)]_{a}^{b} + \inner{g}{\mathcal{L}f}^* \\ 
          &= \qty[f^*(x)p(x)g'(x)]_{a}^{b} - \qty[g(x)p(x)f'^*(x)]_{a}^{b} + \inner{\mathcal{L}f}{g} 
        \end{align}
        となる.第1項と第2項について,第1項に$f(a) = f(b)$を,第2項に$p(x)$が実数値関数であり$p(a)f'(a) = p(b)f'(b)$であることを用いると
        \begin{align}
          \qty[f^*(x)p(x)g'(x)]_{a}^{b} - \qty[f'^*(x)p(x)g(x)]_{a}^{b} = \qty{p(b)g'(b) - p(a)f(a)}f^*(a) - \qty{g(b) - g(a)}p(a)f'(a)
        \end{align}
        を得る.今度は,第1項に$p(x)$が実数値関数であり$p(a)g'(a) = p(b)g'(b)$であることを,第2項に$g(a) = g(b)$を用いれば,
        \begin{align}
          \inner{f}{\mathcal{L}g} = \inner{\mathcal{L}f}{g}
        \end{align}
        を得る.
      \end{proof}
    \subsubsection{Legendre多項式}
      \refe{strum-liouville-operator-def}において,$a = -b = 1$とする.また,
      \begin{align}
        \rho(x) &\coloneqq 1 \\ 
        p(x) &\coloneqq 1 - x^2 \\ 
        q(x) &\coloneqq -\frac{m^2}{1 - x^2},\ m\in \qty{0, 1, \cdots}
      \end{align}
      とすると,
      \begin{align}
        \mathcal{L}_m = \dv{x}\qty{(1 - x^2)\dv{x}} - \frac{m^2}{1 - x^2}
      \end{align}
      となる.
      関数の内積は,
      \begin{align}
        \inner{f}{g} = \int_{-1}^{1} f^*(x)g(x) \dd{x} \label{inner-prod-legendre}
      \end{align}
      と定義しておく.
      また,境界条件は,
      \begin{align}
        p(\pm 1)f^*(\pm 1)g'(\pm 1) = 0
      \end{align}
      とする.これより,演算子$\mathcal{L}$がHermite演算子であると確かめられる.
      たとえば,$g'(x) = p(x)$となるように$g(x)$を定めれば,$f(1) = f(-1) = 0$となる.
      また,$f(x) = 1$となるように$f(x)$を定めれば,$p(1)g'(1) = p(-1)g'(1) = 0$となるので,前節で示した境界条件を満足する.
      さて,Legendre多項式$P_n(x)$は$n$を非負整数として,
      \begin{align}
        \mathcal{L}_0P_n(x) &= -n(n + 1) P_n(x) \\ 
        \Leftrightarrow \dv{x}\qty{(1 - x^2)\dv{x}P_n(x)} &= -n(n + 1)P_n(x)\label{legendre-polynominal} 
      \end{align}
      なる$P_n(x)$のうち,$x = 0$周りで級数展開したもので,
      \begin{align}
        P_n(x) = \sum_{j = 0}^{\infty}u_jx^j\label{legendre-polynominal-u-expantion}
      \end{align}
      と書いたとき,
      \begin{align}
        u_n &= \frac{(2n)!}{2^n(n!)^2}\label{legendre-polynominal-u-expantion-n} \\ 
        u_{n + 1} &= 0\label{legendre-polynominal-u-expantion-np1}
      \end{align}
      なるものである.\refe{legendre-polynominal-u-expantion}を\refe{legendre-polynominal}に代入すると,
      \begin{align}
        \dv{x}\qty{(1 - x^2)\dv{x}\sum_{j = 0}^{\infty}u_jx^j} &= -n(n + 1)\sum_{j = 0}^{\infty}u_jx^j \\ 
        \Leftrightarrow \sum_{j = 0}^{\infty}ju_j\dv{x}\qty(x^{j - 1} - x^{j + 1}) &= -n(n + 1)\sum_{j = 0}^{\infty}u_jx^j \\ 
        \Leftrightarrow \sum_{j = 0}^{\infty}j(j - 1)u_jx^{j - 2} - \sum_{j = 0}^{\infty}j(j + 1)u_jx^j &= -n(n + 1)\sum_{j = 0}^{\infty}u_jx^j \\ 
        \Leftrightarrow \sum_{j = 0}^{\infty}j(j - 1)u_jx^{j - 2} &= \sum_{j = 0}^{\infty}u_j\qty[j(j + 1) - n(n + 1)]x^j \\ 
        \Leftrightarrow \sum_{j = 2}^{\infty}j(j - 1)u_jx^{j - 2} &= \sum_{j = 0}^{\infty}u_j\qty[j(j + 1) - n(n + 1)]x^j \\ 
        \Leftrightarrow \sum_{j = 0}^{\infty}(j + 1)(j + 2)u_{j + 2}x^{j} &= \sum_{j = 0}^{\infty}u_j\qty[j(j + 1) - n(n + 1)]x^j 
      \end{align}
      となるから,
      \begin{align}
        (j + 1)(j + 2)u_{j + 2} = \qty[j(j + 1) - n(n + 1)]u_j\label{legendre-polynominal-u-expantion-recurr}
      \end{align}
      なる漸化式が成立する.
      \refe{legendre-polynominal-u-expantion-recurr}において$j = n$を代入すると,$u_{n + 2} = 0$となる.
      また,$j = n + 1$を代入すると\refe{legendre-polynominal-u-expantion-np1}より$u_{n + 1} = 0$である.
      よって,
      \begin{align}
        0 = u_{n + 1} = u_{n + 2} = u_{n + 3} = u_{n + 4} = \cdots 
      \end{align}
      となる.また,\refe{legendre-polynominal-u-expantion-recurr}に$j = n - 2$を代入すると,
      \begin{align}
        u_{n - 2} = -\frac{n(n - 1)}{2(2n - 1)}u_n\label{legendre-polynominal-u-expantion-recurr-2}
      \end{align}
      となる.よって,\refe{legendre-polynominal-u-expantion-n}と\refe{legendre-polynominal-u-expantion-recurr-2}を用いて
      \refe{legendre-polynominal-u-expantion}を表すと,
      \begin{align}
        P_n(x) &= \frac{(2n)!}{2^n(n!)^2}\qty[x^n - \frac{n(n - 1)}{2(2n - 1)}x^{n - 2} + \frac{n(n - 1)(n - 2)(n - 3)}{2\cdot 4\cdot (2n - 1)(2n - 3)}x^{n - 4} + \cdots] \\ 
        &= \sum_{s = 0}^{\lfloor n/2 \rfloor}(-1)^s \frac{(2n - 2s)!}{2^ns!(n - s)!(n - 2s)!}x^{n - 2s}
      \end{align}
      となる.
      なお,Legendre多項式は,
      \begin{align}
        \dv[n]{x}\qty(x^2 - 1)^n &= \sum_{s = 0}^{n}\dv[n]{x}(-1)^s\mqty(n \\ k)x^{2n - 2k} \\ 
        &= \sum_{s = 0}^{\lfloor n/2 \rfloor}(-1)^s\frac{n!}{s!(n - s)!}\frac{(2n - 2s)!}{(n - 2k)!}
      \end{align}
      なる関係を用いると,
      \begin{align}
        P_n(x) &= \frac{1}{2^nn!}\dv[n]{x}\qty(x^2 - 1)^n\label{legendre-pn-easy}
      \end{align}
      となる.
    \subsubsection{Legendre陪多項式}
      Legendre陪多項式$P_n^m(x)$は$m \leq n$として,\refe{legendre-pn-easy}を用いれば,
      \begin{align}
        P_n^m(x) &\coloneqq \qty(1 - x^2)^{m/2}\dv[m]{x}P_n(x) \\ 
        &= \qty(1 - x^2)^{m/2}\dv[m]{x}\frac{1}{2^nn!}\dv[n]{x}\qty(x^2 - 1)^n \\ 
        &= \frac{1}{2^nn!}\qty(1 - x^2)^{m/2}\dv[n + m]{x}\qty(x^2 - 1)^n \label{assos-legendre-polynominal}
      \end{align}
      となる.
      Legendre陪多項式の直交性は$\mathcal{L}_m$がHermite演算子であり,その固有関数である$P_n^m(x)$が直交することより従う.
      自分自身との内積,つまり,$\inner{P_n^m(x)}{P_n^m(x)}$の値を計算する.
      \refe{assos-legendre-polynominal}を用いて,\refe{inner-prod-legendre}で示した内積の定義に従って計算すると,
      \begin{align}
        \inner{P_{n}^m(x)}{P_{n}^m(x)} &= \int_{-1}^{1}\frac{1}{2^{2n}\qty(n!)^2}\qty(1 - x^2)^m \qty{\dv[n + m]{x}\qty(x^2 - 1)^m}\qty{\dv[n + m]{x}\qty(x^2 - 1)^m} \dd{x} \\ 
        &= \frac{1}{2^{2n}\qty(n!)^2}\int_{-1}^{1}\qty(1 - x^2)^m \qty{\dv[n + m]{x}\qty(x^2 - 1)^m}\dv{x}\qty{\dv[n + m - 1]{x}\qty(x^2 - 1)^m} \dd{x} \\ 
        &= \frac{1}{2^{2n}\qty(n!)^2}\qty[\qty(1 - x^2)^m\qty{\dv[n + m]{x}\qty(x^2 - 1)^m}\qty{\dv[n + m - 1]{x}\qty(x^2 - 1)^m}]_{-1}^{1} \\ 
        &\ \ -\frac{1}{2^{2n}\qty(n!)^2} \int_{-1}^{1}\qty{\dv{x}\qty(1 - x^2)^m\qty{\dv[n + m]{x}(x^2 - 1)^n}}\qty{\dv[n + m + 1]{x}\qty(x^2 - 1)^m}\dd{x} \\ 
        &= -\frac{1}{2^{2n}\qty(n!)^2}\int_{-1}^{1}\qty{\dv{x}\qty(1 - x^2)^m\qty{\dv[n + m]{x}(x^2 - 1)^n}}\qty{\dv[n + m + 1]{x}\qty(x^2 - 1)^m} \dd{x} \\
        &= \cdots \\ 
        &= \frac{\qty(-1)^{n + m}}{2^{2n}\qty(n!)^2}\int_{-1}^{1}\qty(x^2 - 1)^n\qty{\dv[n + m]{x}\qty(1 - x^2)^m\qty{\dv[n + m]{x}\qty(x^2 - 1)^n}} \dd{x} \\ 
        &= \frac{\qty(-1)^{n + m}}{2^{2n}\qty(n!)^2}\int_{-1}^{1}\qty(x^2 - 1)^n\qty[\sum_{k = 0}^{n + m}\mqty(n + m \\ k)\qty{\dv[n + m - k]{x}\qty(1 - x^2)^m}\qty{\dv[n + m + k]{x}\qty(x^2 - 1)^n}] \dd{x}\label{use-leibniz-formula}
      \end{align}
      となる.最終行でLeibnizの公式を用いた.
      \refe{use-leibniz-formula}の和の中の$n + m - k$階微分と$n + m - k$階微分を考える.
      $\qty(1 - x^2)^m$と$(x^2 - 1)^n$の最高次数は,それぞれ$2m$と$2n$であるから,$2m \geq n + m - k$かつ$2n \geq n + m + k$なる$k$でのみ和の中は0でなくなる.
      つまり,$n - m\leq k$かつ$n - m \geq k$なる$k$は$k = n - m$のみである.
      よって,\refe{use-leibniz-formula}は,
      \begin{align}
        \inner{P_{n}^m(x)}{P_{n}^m(x)} &= \frac{\qty(-1)^{n + m}}{2^{2n}\qty(n!)^2}\int_{-1}^{1}\qty(x^2 - 1)^n\qty[\sum_{k = 0}^{n + m}\mqty(n + m \\ k)\qty{\dv[n + m - k]{x}\qty(1 - x^2)^m}\qty{\dv[n + m + k]{x}\qty(x^2 - 1)^n}] \dd{x} \\ 
        &= \frac{\qty(-1)^{n + m}}{2^{2n}\qty(n!)^2}\int_{-1}^{1}\qty(x^2 - 1)^n\qty[\mqty(n + m \\ n - m)\qty{\dv[2m]{x}\qty(1 - x^2)^m}\qty{\dv[2n]{x}\qty(x^2 - 1)^n}] \dd{x} \\ 
        &= \frac{\qty(-1)^{n + m}}{2^{2n}\qty(n!)^2}(-1)^m\qty(2m)!\qty(2n)!\frac{(n + m)!}{(n - m)!(2m)!}\int_{-1}^{1}\qty(x^2 - 1)^n\dd{x} \\ 
        &= \frac{\qty(-1)^{n + m}}{2^{2n}\qty(n!)^2}(-1)^m\qty(2m)!\qty(2n)!\frac{(n + m)!}{(n - m)!(2m)!}(-1)^n2\frac{(2n)!!}{(2n + 1)!!} \\ 
        &= \frac{2}{2n + 1}\frac{(n + m)!}{(n - m)!}
      \end{align}
      となる.
      直交性とまとめて書くと,
      \begin{align}
        \inner{P_{n'}^m(x)}{P_{n}^m(x)} = \delta_{n}^{n'}\frac{2}{2n + 1}\frac{(n + m)!}{(n - m)!}\label{pnm-norm}
      \end{align}
      となる.
    \subsubsection{Legendre陪多項式を用いた波動関数の展開}
      
    \subsubsection{Besselの微分方程式}
    \subsubsection{波動関数の展開}
    以下では波動関数をLegendre多項式で基底展開する手順を説明する.
    Helmholtz方程式,
    \begin{align}
      [\laplacian + \kappa^2]u(\bm{r}) = 0
    \end{align}
    を球座標系で表すと,
    \begin{align}
      \qty[\frac{1}{r^2}\pdv{r}\qty(r^2\pdv{r}) + \frac{1}{r^2 \sin \theta} \pdv{\theta}\qty(\sin\theta\pdv{\theta}) + \frac{1}{r^2 \sin^2\theta}\pdv[2]{\phi} + \kappa^2]u(r, \theta, \phi) = 0 \label{helmholtz-eq-in-sphere}
    \end{align}
    である.

  % この方程式の解は動径波動関数$R(r)$と球面調和関数$Y_{l, m}(\theta, \phi)$を用いて,
  % \begin{align} 
  %   u(r, \theta, \phi) = \sum_{l = 0}^{\infty}\sum_{m = -l}^{l}C_{lm}R(r)Y_{l, m}(\theta, \phi)
  % \end{align}
  % と表される.ここで,球面調和関数とLegendre陪関数,Legendre多項式は以下の関係にある.
  % \begin{align}
  %   Y_{l,m}(\theta,\phi) &= (-1)^m \qty[\frac{(2l+1)}{4\pi}\frac{(l-m)!}{(l+m)!}]^{1/2}P_l^m(\cos\theta)\exp(\i m\phi) \\
  %   P_l^m(x) &= (1-x^2)^{m/2} \frac{\r{d}^m P_l(x)}{\dd{x}^m}
  % \end{align}
  % いま,角度$\phi$によらないとすると,$m=0$とすることで
  % \begin{align}
  %   \psi(r,\theta) = \sum_{l=0}^{\infty} (2l+1)R_l(r)P_l(\cos\theta)
  % \end{align}
  % を得る.$(2l+1)$は縮退の数を表す.これを式(\ref{helmholtz-eq-in-sphere})に代入する.
  % \begin{align}
  %   \qty[\frac{1}{r}\frac{\partial^2}{\partial r^2}r + \frac{1}{r^2 \sin \theta} \frac{\partial}{\partial \theta}\qty(\sin\theta\frac{\partial}{\partial \theta})
  %   + \kappa^2] \sum_{l=0}^{\infty} (2l+1)R_l(r)P_l(\cos\theta)= 0
  % \end{align}
  % 整理すると
  % \begin{align}
  %   \label{Helmholtz separated}
  %   \frac{P_l(\cos\theta)}{r}\qty[2\frac{\partial R_l(r)}{\partial r} + r \frac{\partial^2 R_l(r)}{\partial r^2}]
  %   + \frac{R_l(r)}{r^2 \sin \theta} \frac{\partial}{\partial \theta}\qty(\sin\theta\frac{\partial}{\partial\theta}P_l(\cos\theta)) + \kappa^2 R_l(r)P_l(\cos\theta) = 0
  % \end{align}
  % である.ここで,Legendre多項式は
  % \begin{align}
  %   \frac{\r{d}}{\dd{x}}\qty[(1-x^2)\frac{\r{d}P_l(x)}{\dd{x}}] + l(l+1)P_l(x) = 0
  % \end{align}
  % を満たす.これを変数変換すると
  % \begin{align}
  %   \frac{1}{\sin\theta}\frac{\r{d}}{\dd{\theta}}\qty[\sin\theta\frac{\r{d}P_l(\cos\theta)}{\dd{\theta}}] + l(l+1)P_l(\cos\theta) = 0
  % \end{align}
  % となる.よって,式(\ref{Helmholtz separated})は
  % \begin{align}
  %   \frac{P_l(\cos\theta)}{r}\qty[2\frac{\partial R_l(r)}{\partial r} + r \frac{\partial^2 R_l(r)}{\partial r^2}]
  %   - l(l+1)\frac{R_l(r)}{r^2}P_l(\cos\theta)  + \kappa^2 R_l(r)P_l(\cos\theta) = 0
  % \end{align}
  % と変形される.したがって,$R_l(r)$が満たすべき式は
  % \begin{align}
  %   \qty[\frac{\r{d}^2}{\dd{r}^2} + \frac{2}{r} \dv{}{r} + \kappa^2 - \frac{l(l+1)}{r^2}]R_l(r) = 0
  % \end{align}
  % である.この微分方程式の解$R_l(r)$は球面Bessel関数$j_l(\kappa r)$と球面Neumann関数$n_l(\kappa r)$の和で与えられる.よって,波動関数は
  % \begin{align}
  %   \psi(r,\theta) = \sum_{l=0}^{\infty} (2l+1)\qty[A_l j_l(\kappa r) + B_l n_l(\kappa r)] P_l(\cos\theta)
  % \end{align} 
  % となる.また,球面Bessel関数と球面Neumann関数は以下の性質を持つ.
  % \begin{align}
  %   x \to 0 \\
  %   j_l(x) &\to \frac{x^l}{(2l+1)!}\qty(1 - \frac{x^2}{2(2l+3)}\cdots) \\
  %   n_l(x) &\to -\frac{(2l -1)!!}{x^{l+1}}
  % \end{align}
  % 平面波$\e^{\i  z} = \e^{\i kr\cos\theta}$は明らかにHelmholtz方程式を満たす.さらに,$r=0$で有限の値をもつため,
  % \begin{align}
  %   B_l =0
  % \end{align} 
  % であることがわかる.以上の計算から,
  % \begin{align}
  %   \e^{\i \kappa r\cos\theta} = \sum_{l=0}^{\infty} (2l+1)A_l j_l(\kappa r) P_l(\cos\theta)
  % \end{align}
  % を得る.左辺を展開すると,
  % \begin{align}
  %   \sum_{l=0}^{\infty}\frac{(\i \kappa r \cos\theta)^l}{l!}
  % \end{align}
  % 右辺を展開すると,
  % \begin{align}
  %   \sum_{l=0}^{\infty} (2l+1)A_l\frac{(\kappa l)^l}{(2l+1)!!}P_l(\cos\theta)
  % \end{align}
  % である.また,$P_l(\cos\theta)$の$(\cos\theta)^l$の項の係数は
  % \begin{align}
  %   \frac{1}{2^l l!}\frac{(2l)!}{l!}
  % \end{align}
  % である.よって,左辺と右辺を比較することで
  % \begin{align}
  %   \frac{(\i \kappa r \cos\theta)^l}{l!} = (2l+1)A_l \frac{(2l)!}{2^l (l!)^2}\frac{(\kappa r \cos\theta)^l}{(2l+1)!!} \\
  %   A_l = \i^l
  % \end{align}
  % を得る.したがって,\textbf{Rayleighの公式}
  % \begin{align}
  %   \e^{\i kz} = \e^{\i kr\cos\theta} = \sum_{l=0}^{\infty} (2l+1)\i^l j_l(kr)P_l(\cos\theta)
  % \end{align}
  % が成り立つ.以上で,平面波のLegendre多項式による表現が得られた.同様に,散乱振幅を未定係数$a_l$を用いて
  % \begin{align}
  %   f(\theta) = \sum_{l=0}^{\infty} (2l+1)a_l P_l(\cos\theta)
  % \end{align}
  % と展開する.これにより球面波は
  % \begin{align}
  %   \sum_{l=0}^{\infty} (2l+1)a_l P_l(\cos\theta) \frac{\e^{\i kr}}{r}
  % \end{align}
  % と展開される.Legendre多項式は完全性と直交性をもつためこの展開は妥当である.
  したがって,部分波展開した散乱の波動関数は
  \begin{align}
    \psi(\bm{r}) = \sum_{l = 0}^{\infty} (2l + 1)\i^l j_l(kr)P_l(\cos\theta) + \sum_{l = 0}^{\infty} (2l + 1)a_l P_l(\cos\theta) \frac{\e^{\i kr}}{r}
  \end{align}
  である.
  \begin{itembox}[l]{部分波展開した散乱の波動関数}
  \begin{align}
    \psi(\bm{r}) = \sum_{l = 0}^{\infty} (2l + 1)\i^l j_l(kr)P_l(\cos\theta) + \sum_{l = 0}^{\infty} (2l + 1)a_l P_l(\cos\theta) \frac{\e^{\i kr}}{r}\label{PartialWave}
  \end{align}
  \end{itembox}
  このとき,散乱断面積は
  \begin{align}
    \sigma(\theta) &= \abs{f(\theta)}^2 \\
    &= \sum_{l} \sum_{l'} (2l+1)(2l'+1) a_l^{*}a_{l'} P_l(\cos\theta) P_{l'}(\cos\theta)
  \end{align}
  である.全断面積は
  \begin{align}
    \sigma^{\r{tot}} &= 2\pi \int \sigma(\theta) \sin\theta \dd{\theta} \\
    &= 2\pi \sum_{l=0}^{\infty} (2l+1)^2\abs{a_l}^2\qty(\frac{2}{2l+1}) \\
    &= 4\pi \sum_{l=0}^{\infty} (2l+1) \abs{a_l}^2
  \end{align}
  である.ここで,Legendre多項式の直交性
  \begin{align}
    \int_{0}^{\pi} P_l(\cos\theta)P_{l'}(\cos\theta) \sin\theta \dd{\theta} = \frac{2}{2l+1}\delta_{l,l'}
  \end{align}
  を用いた.
  以上の議論から,部分波展開を用いた散乱問題は未定係数$a_l$を求めることに帰着する.
\end{document}