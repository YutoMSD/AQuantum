\documentclass{report}
\input{../../head.tex}
\begin{document}
  近似法の一つである部分波展開を扱う.
  まず,半古典論を用いて散乱が起こる条件から,方位量子数$l$ごとに波動関数を展開して,$l$が小さい波動関数のみ考えればよいことが分かる.
  次に,波動関数の基底展開について議論を行う.
  前節の議論より,球Bessel関数と球Neumann関数を用いて任意の関数が展開できるのであった.
  また,散乱には$\phi$依存性が無いという条件を用いれば,球Neumann関数の成分は0であると分かる.
  さらに,球Bessel関数にかかる係数のラベルは方位量子数のラベルと一致することが分かる.
  \subsection{散乱の条件}
    古典力学において角運動量$\bm{L}$は,
    \begin{align}
      \bm{L} = \bm{r} \times \bm{p} = m\bm{r} \times \bm{v}
    \end{align}
    と書けるのであった.
    球対称ポテンシャルの下では,
    \begin{align}
      \dv{\bm{L}}{t} &= m\dv{\bm{r}}{t} \times \bm{v} + m\bm{r} \times \dv{\bm{v}}{t} \\
      &= \bm{r} \times \bm{F} \\
      &= \bm{r} \times (-\grad V(r)) \\
      &= \bm{r} \times \abs{\grad V}\frac{\bm{r}}{r} \\
      &= 0
    \end{align}
    より,角運動量は保存される.
    衝突パラメータを$b$,運動量を$\bm{p}$とした粒子の角運動量は,
    \begin{align}
      \abs{\bm{L}} &= \abs{\bm{r} \times \bm{p}} \\
      &= pb
    \end{align}
    である.散乱体を半径$a$の球とすると,衝突の条件は
    \begin{align}
      b &< a \\
      L/p &< a \\
      L &< pa \label{ConditionofSC}
    \end{align}
    である.つまり,角運動量が小さい粒子のみ散乱することがわかる.
    \par
    \refe{ConditionofSC}を半古典的な散乱条件へ書き直すことを考える.
    量子力学では,角運動量の大きさ$L$は,$l = 0, 1, 2, \cdots$の値を取る方位量子数$l$を用いて$L = \hbar \sqrt{l(l + 1)}$と書ける.
    また,\refe{assos-legendre-diff-eq}と,\refe{a-condition-use-lm}の第2項以降を見ると,球面分布函数$Y_{lm}(\theta, \phi)$がラプラシアンの角度部分の固有関数に
    なっていて,ラプラシアンの角度部分は,角運動量演算子を$\hbar$で割ったものであるから角運動量の固有値が$\hbar\sqrt{l(l + 1)}$であることが分かる.
    衝突の条件は,
    \begin{align}
      L = \hbar \sqrt{l(l + 1)} < pa = \hbar ka\label{scattering-condition-of-l}
    \end{align}
    と書ける.\refe{scattering-condition-of-l}を見れば,散乱の影響を受けるのは$l$が小さいときのみであることがわかる.
    よって波動関数を,
    \begin{align}
      \psi = \psi^{(l = 0)} + \psi^{(l = 1)} + \cdots
    \end{align}
    のように異なる$l$に属する固有関数で展開し,$l$が小さい状態についてだけ散乱の影響を考える.これを\textbf{部分波展開}という.
  \subsection{部分波展開の計算}
    波動関数は\refe{u-expand-form2}より,
    \begin{align}
      \psi(r, \theta, \phi) = \sum_{m = -\infty}^{\infty}\sum_{l = \abs{m}}^{\infty}\qty[a_{lm}j_l(\kappa r) + b_{lm}y_l(\kappa r)](-1)^m\sqrt{\frac{2l + 1}{4\pi}\frac{(l - m)!}{(l + m)!}}P_l^m(x)\e^{\i m\phi}
    \end{align}
    と書けるのであった.
    ただし,角運動量演算子の固有関数とラベルを合わせるために$n \to l$とした.
    散乱において,$\phi$依存性が無いので,$m = 0$のみ和を考えればよく,
    \begin{align}
      \psi(r, \theta) = \sum_{l = 0}^{\infty}\qty[a_lj_l(\kappa r) + b_ly_l(\kappa r)]\sqrt{\frac{2l + 1}{4\pi}}P_l(x)
    \end{align} 
    となる.
    $a_l \coloneqq a_{l0}$,$b_l \coloneqq b_{l0}$である.また,$P_l^0 = P_l$であることに注意する.
    球Bessel関数と球Neumann関数は$x \to 0$で,
    \begin{align}
      j_l(r) &\sim \frac{r^l}{(2l + 1)!}\qty(1 - \frac{r^2}{2(2l + 3)}\cdots) \\
      n_l(r) &\sim -\frac{(2l - 1)!!}{r^{l + 1}}
    \end{align}
    となることが知られている.
    $r = 0$で$\psi(r, \theta)$が発散しないために,$\forall l\ b_l = 0$とする.
    まず,平面波を部分波展開すると,Rayleighの公式より,
    \begin{align}
      \e^{\i kr\cos\theta} = \sum_{l = 0}^{\infty} (2l + 1)\i^l j_l(kr)P_l(\cos\theta)
    \end{align}
    と書ける.
    \par
    球面波は未定係数$a_l$をつけて,
    \begin{align}
      \sum_{l = 0}^{\infty} (2l + 1)a_l P_l(\cos\theta) \frac{\e^{\i kr}}{r}
    \end{align}
    と展開される.
    したがって,部分波展開した散乱の波動関数は,
    \begin{align}
      \psi(\bm{r}) = \sum_{l = 0}^{\infty} (2l + 1)\i^l j_l(kr)P_l(\cos\theta) + \sum_{l = 0}^{\infty} (2l + 1)a_l P_l(\cos\theta) \frac{\e^{\i kr}}{r}
    \end{align}
    である.
    \begin{itembox}[l]{部分波展開した散乱の波動関数}
      \begin{align}
        \psi(\bm{r}) = \sum_{l = 0}^{\infty} (2l + 1)\i^l j_l(kr)P_l(\cos\theta) + \sum_{l = 0}^{\infty} (2l + 1)a_l P_l(\cos\theta) \frac{\e^{\i kr}}{r}\label{PartialWave}
      \end{align}
    \end{itembox}
    \begin{itembox}[l]{部分波展開した散乱の散乱振幅}
      \begin{align}
        f(\theta) = \sum_{l = 0}^{\infty} (2l + 1)a_l P_l(\cos\theta) \label{PartialWave-amp}
      \end{align}
    \end{itembox}
  \subsection{散乱断面積と全断面積}
    部分波展開した波動関数より計算した散乱断面積は,
    \begin{align}
      \sigma(\theta) &= \abs{f(\theta)}^2 \\
      &= \sum_{l} \sum_{l'} (2l+1)(2l'+1) a_l^{*}a_{l'} P_l(\cos\theta) P_{l'}(\cos\theta)
    \end{align}
    である.全断面積は,
    \begin{align}
      \sigma^{\r{tot}} &= 2\pi \int \sigma(\theta) \sin\theta \dd{\theta} \\
      &= 2\pi \sum_{l = 0}^{\infty} (2l + 1)^2\abs{a_l}^2\qty(\frac{2}{2l + 1}) \\
      &= 4\pi \sum_{l = 0}^{\infty} (2l + 1) \abs{a_l}^2
    \end{align}
    である.ここで,Legendre多項式の直交性,
    \begin{align}
      \int_{0}^{\pi} P_l(\cos\theta)P_{l'}(\cos\theta) \sin\theta \dd{\theta} = \frac{2}{2l+1}\delta_{l, l'}
    \end{align}
    を用いた.
    以上の議論から,部分波展開を用いた散乱問題は未定係数$a_l$を求めることに帰着する.
\end{document}