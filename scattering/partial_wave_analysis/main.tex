\documentclass{report}
\input{../../head.tex}
\begin{document}
  近似法の一つである部分波展開を扱う.


  古典力学において,角運動量は
  \begin{align}
    \bm{L} = \bm{r} \times \bm{p} = m\bm{r} \times \bm{v}
  \end{align}
  である.球対称ポテンシャルの下では,
  \begin{align}
    \dv{\bm{L}}{t} &= m\dv{\bm{r}}{t} \times \bm{v} + m\bm{r} \times \dv{\bm{v}}{t}\\
    &= \bm{r} \times \bm{F}\\
    &= \bm{r} \times (-\nabla V(r))\\
    &= \bm{r} \times \abs{\nabla V}\frac{\bm{r}}{r}\\
    &= 0
  \end{align}
  より,角運動量は保存される.
  衝突パラメータ$b$,運動量$p$の粒子の角運動量は
  \begin{align}
    \abs{\bm{L}} &= \abs{\bm{r} \times \bm{p}}\\
    &= pb
  \end{align}
  である.散乱体を半径$a$の球とすると,衝突の条件は
  \begin{align}
    &b < a\\
    &L/p < a\\
    \label{ConditionofSC}
    &L < pa
  \end{align}
  である.つまり,角運動量が小さい粒子のみ散乱することがわかる.

  式(\ref{ConditionofSC})を半古典的な散乱条件へと焼き直す.
  \begin{align}
    L = \hbar \sqrt{l(l+1)} < pa = \hbar ka
  \end{align}
  ここで,$l$は方位量子数で$l=0,1,2,\dots$の値を取る.つまり,散乱の影響を受けるのは$l$が小さいときのみであることがわかる.
  よって,波動関数を
  \begin{align}
    \psi = \phi^{l=0} + \phi^{l=1}+\cdots
  \end{align}
  のように異なる$l$の固有関数で展開し,$l$が小さい状態についてだけ散乱の影響を考える.これを\textbf{部分波展開}という.

  散乱の波動関数
  \begin{align}
    \psi(\bm{r}) = \e^{\i kz} + f(\theta)\frac{\e^{\i kr}}{r}
  \end{align}
  を部分波展開する\footnote{ここから式(\ref{PartialWave})までは飛ばしても良い.}.
  Helmholtz方程式
  \begin{align}
    [\nabla^2 + \kappa^2]\psi(\bm{r}) = 0
  \end{align}
  を球座標系で表すと
  \begin{align}
    \label{HEinSphere}
    \qty[\frac{1}{r}\frac{\partial^2}{\partial r^2}r + \frac{1}{r^2 \sin \theta} \frac{\partial}{\partial \theta}\qty(\sin\theta\frac{\partial}{\partial \theta})
    + \frac{1}{r^2 \sin^2\theta}\frac{\partial^2}{\partial \phi^2} + \kappa^2]\psi(r,\theta,\phi) = 0
  \end{align}
  である.この方程式の解は動径波動関数$R(r)$と球面調和関数$Y_{l,m}(\theta, \phi)$を用いて
  \begin{align}
    \psi(r,\theta,\phi) = \sum_{l=0}^{\infty}\sum_{m=-l}^{l}C_{lm}R(r)Y_{l,m}(\theta,\phi)
  \end{align}
  と表される.ここで,球面調和関数とLgendre陪関数,Legendre多項式は以下の関係にある.
  \begin{align}
    Y_{l,m}(\theta,\phi) &= (-1)^m \qty[\frac{(2l+1)}{4\pi}\frac{(l-m)!}{(l+m)!}]^{1/2}P_l^m(\cos\theta)\exp(\i m\phi)\\
    P_l^m(x) &= (1-x^2)^{m/2} \frac{\r{d}^m P_l(x)}{\dd{x}^m}
  \end{align}
  いま,角度$\phi$によらないとすると,$m=0$とすることで
  \begin{align}
    \psi(r,\theta) = \sum_{l=0}^{\infty} (2l+1)R_l(r)P_l(\cos\theta)
  \end{align}
  を得る.$(2l+1)$は縮退の数を表す.これを式(\ref{HEinSphere})に代入する.
  \begin{align}
    \qty[\frac{1}{r}\frac{\partial^2}{\partial r^2}r + \frac{1}{r^2 \sin \theta} \frac{\partial}{\partial \theta}\qty(\sin\theta\frac{\partial}{\partial \theta})
    + \kappa^2] \sum_{l=0}^{\infty} (2l+1)R_l(r)P_l(\cos\theta)= 0
  \end{align}
  整理すると
  \begin{align}
    \label{Helmholtz separated}
    \frac{P_l(\cos\theta)}{r}\qty[2\frac{\partial R_l(r)}{\partial r} + r \frac{\partial^2 R_l(r)}{\partial r^2}]
    + \frac{R_l(r)}{r^2 \sin \theta} \frac{\partial}{\partial \theta}\qty(\sin\theta\frac{\partial}{\partial\theta}P_l(\cos\theta)) + \kappa^2 R_l(r)P_l(\cos\theta) = 0
  \end{align}
  である.ここで,Legendre多項式は
  \begin{align}
    \frac{\r{d}}{\dd{x}}\qty[(1-x^2)\frac{\r{d}P_l(x)}{\dd{x}}] + l(l+1)P_l(x) = 0
  \end{align}
  を満たす.これを変数変換すると
  \begin{align}
    \frac{1}{\sin\theta}\frac{\r{d}}{\dd{\theta}}\qty[\sin\theta\frac{\r{d}P_l(\cos\theta)}{\dd{\theta}}] + l(l+1)P_l(\cos\theta) = 0
  \end{align}
  となる.よって,式(\ref{Helmholtz separated})は
  \begin{align}
    \frac{P_l(\cos\theta)}{r}\qty[2\frac{\partial R_l(r)}{\partial r} + r \frac{\partial^2 R_l(r)}{\partial r^2}]
    - l(l+1)\frac{R_l(r)}{r^2}P_l(\cos\theta)  + \kappa^2 R_l(r)P_l(\cos\theta) = 0
  \end{align}
  と変形される.したがって,$R_l(r)$が満たすべき式は
  \begin{align}
    \qty[\frac{\r{d}^2}{\dd{r}^2} + \frac{2}{r} \dv{}{r} + \kappa^2 - \frac{l(l+1)}{r^2}]R_l(r) = 0
  \end{align}
  である.この微分方程式の解$R_l(r)$は球面Bessel関数$j_l(\kappa r)$と球面Neumann関数$n_l(\kappa r)$の和で与えられる.よって,波動関数は
  \begin{align}
    \psi(r,\theta) = \sum_{l=0}^{\infty} (2l+1)\qty[A_l j_l(\kappa r) + B_l n_l(\kappa r)] P_l(\cos\theta)
  \end{align} 
  となる.また,球面Bessel関数と球面Neumann関数は以下の性質を持つ.
  \begin{align}
    x \to 0\\
    j_l(x) &\to \frac{x^l}{(2l+1)!}\qty(1 - \frac{x^2}{2(2l+3)}\cdots)\\
    n_l(x) &\to -\frac{(2l -1)!!}{x^{l+1}}
  \end{align}
  平面波$\e^{\i  z} = \e^{\i kr\cos\theta}$は明らかにHelmholtz方程式を満たす.さらに,$r=0$で有限の値をもつため,
  \begin{align}
    B_l =0
  \end{align} 
  であることがわかる.以上の計算から,
  \begin{align}
    \e^{\i \kappa r\cos\theta} = \sum_{l=0}^{\infty} (2l+1)A_l j_l(\kappa r) P_l(\cos\theta)
  \end{align}
  を得る.左辺を展開すると,
  \begin{align}
    \sum_{l=0}^{\infty}\frac{(\i \kappa r \cos\theta)^l}{l!}
  \end{align}
  右辺を展開すると,
  \begin{align}
    \sum_{l=0}^{\infty} (2l+1)A_l\frac{(\kappa l)^l}{(2l+1)!!}P_l(\cos\theta)
  \end{align}
  である.また,$P_l(\cos\theta)$の$(\cos\theta)^l$の項の係数は
  \begin{align}
    \frac{1}{2^l l!}\frac{(2l)!}{l!}
  \end{align}
  である.よって,左辺と右辺を比較することで
  \begin{align}
    \frac{(\i \kappa r \cos\theta)^l}{l!} = (2l+1)A_l \frac{(2l)!}{2^l (l!)^2}\frac{(\kappa r \cos\theta)^l}{(2l+1)!!}\\
    A_l = \i^l
  \end{align}
  を得る.したがって,\textbf{Rayleighの公式}
  \begin{align}
    \e^{\i kz} = \e^{\i kr\cos\theta} = \sum_{l=0}^{\infty} (2l+1)\i^l j_l(kr)P_l(\cos\theta)
  \end{align}
  が成り立つ.以上で,平面波のLegendre多項式による表現が得られた.同様に,散乱振幅を未定係数$a_l$を用いて
  \begin{align}
    f(\theta) = \sum_{l=0}^{\infty} (2l+1)a_l P_l(\cos\theta)
  \end{align}
  と展開する.これにより球面波は
  \begin{align}
    \sum_{l=0}^{\infty} (2l+1)a_l P_l(\cos\theta) \frac{\e^{\i kr}}{r}
  \end{align}
  と展開される.したがって,部分波展開した散乱の波動関数は
  \begin{align}
    \psi(\bm{r}) = \e^{\i kr\cos\theta} = \sum_{l=0}^{\infty} (2l+1)\i^l j_l(kr)P_l(\cos\theta) + \sum_{l=0}^{\infty} (2l+1)a_l P_l(\cos\theta) \frac{\e^{\i kr}}{r}
  \end{align}
  である.
  \begin{itembox}[l]{部分波展開した散乱の波動関数}
  \begin{align}
    \label{PartialWave}
    \psi(\bm{r}) = \sum_{l=0}^{\infty} (2l+1)\i^l j_l(kr)P_l(\cos\theta) + \sum_{l=0}^{\infty} (2l+1)a_l P_l(\cos\theta) \frac{\e^{\i kr}}{r}
  \end{align}
  \end{itembox}
  このとき,散乱断面積は
  \begin{align}
    \sigma(\theta) &= \abs{f(\theta)}^2\\
    &= \sum_{l} \sum_{l'} (2l+1)(2l'+1) a_l^{*}a_{l'} P_l(\cos\theta) P_{l'}(\cos\theta)
  \end{align}
  である.全断面積は
  \begin{align}
    \sigma^{\r{tot}} &= 2\pi \int \sigma(\theta) \sin\theta \dd{\theta}\\
    &= 2\pi \sum_{l=0}^{\infty} (2l+1)^2\abs{a_l}^2\qty(\frac{2}{2l+1})\\
    &= 4\pi \sum_{l=0}^{\infty} (2l+1) \abs{a_l}^2
  \end{align}
  である.ここで,Legendre多項式の直交性
  \begin{align}
    \int_{0}^{\pi} P_l(\cos\theta)P_{l'}(\cos\theta) \sin\theta \dd{\theta} = \frac{2}{2l+1}\delta_{l,l'}
  \end{align}
  を用いた.

  以上の議論から,部分波展開を用いた散乱問題は未定係数$a_l$を求めることに帰着する.

  量子力学における散乱では,散乱の前後で位相が変化する.これを\textbf{位相シフト}とよぶ.例えば,ポテンシャル
  \begin{align}
    V(x) = 
    \begin{dcases}
      0\ &(x < 0)\\
      V_0\ &(x \geq 0)
    \end{dcases}
  \end{align}
  に左から入射する粒子を考える.入射粒子のエネルギーは$E<V_0$とする.Schrödinger方程式を解くことにより波動関数は
  \begin{align}
    \psi(x) = 
    \begin{dcases}
      \label{1Dscattering}
      \e^{\i kx} + \frac{k -\i\alpha}{k + \i\alpha} \e^{-\i kx}\ &(x < 0)\\
      \frac{2k}{k + \i\alpha}\e^{-\alpha x}\ &(x \geq 0)
    \end{dcases}
  \end{align}
  となる.ただし,$k \equiv \sqrt{\frac{2mE}{\hbar^2}},\alpha \equiv \sqrt{\frac{2m(V-E)}{\hbar^2}}$とおいた.式(\ref{1Dscattering})において
  $\e^{\i kx}$は入射波を,$\frac{k -\i\alpha}{k + \i\alpha} \e^{-\i kx}$は反射波を表している.反射波は
  \begin{align}
    k + \i\alpha = \sqrt{k^2 + \alpha^2} \e^{\i\delta_k}
  \end{align}
  とすると
  \begin{align}
    \e^{-2 \i \delta_k}\e^{-\i kx}
  \end{align}
  と表される.よって,波動関数は
  \begin{align}
    \psi(x) = \e^{\i kx} + \e^{-2 \i \delta_k}\e^{-\i kx}
  \end{align}
  となる.この$\delta_k$は位相シフトと呼ばれ,量子力学的散乱を特徴づけるパラメータである.

  これを3次元に拡張する.式(\ref{PartialWave})において$r\to\infty$とすると,$j_l(kr)\to \frac{\e^{\i(kr - \frac{l\pi}{2})} - \e^{-\i(kr - \frac{l\pi}{2})}}{2\i kr}$であるため,
  \begin{align}
    \psi(\bm{r}) = \sum_{l=0}^{\infty} \frac{2l+1}{2\i kr}\qty[\i^l \qty(\e^{\i(kr - \frac{l\pi}{2})} - \e^{-\i(kr - \frac{l\pi}{2})}) + (2\i k) a_l \e^{\i kr}]P_l(\cos\theta)
  \end{align}
  を得る.さらに
  \begin{align}
    \i^l \qty(\e^{\i(kr - \frac{l\pi}{2})} - \e^{-\i(kr - \frac{l\pi}{2})}) &= \i^l \e^{-\i\frac{l\pi}{2}}\qty(\e^{\i kr} - \e^{-\i kr} e^{\i l\pi})\\
    &= 1\cdot\qty(\e^{\i kr} - (-1)^l) \e^{-\i kr}
  \end{align}
  と直せるため,波動関数として
  \begin{align}
    \label{SCPartialWave}
    \psi(\bm{r}) = \sum_{l=0}^{\infty} \qty[(1 + 2\i ka_l)\e^{\i kr} - (-1)^l \e^{-\i kr}]P_l(\cos\theta)
  \end{align}
  を得る.第1項は外向き球面波,第2項は内向き球面波を表す.この散乱は全反射であるため入射波と反射波の振幅は等しい.つまり,
  \begin{align}
    \abs{1 + 2\i ka_l} = 1
  \end{align}
  が成り立つ.よって,散乱による位相のずれを$\delta_l$とおくと
  \begin{align}
    1 + 2\i ka_l = \e^{2\i \delta_l}
  \end{align}
  である.この式から$a_l$を求めると
  \begin{align}
    a_l &= \frac{1}{2\i k}(\e^{\i \delta_l} - 1)\\
    &= \frac{1}{k}\e^{\i \delta_l} \sin \delta_l
  \end{align}
  を得る.さらに全断面積と散乱振幅を求めることができる.
  \begin{align}
    \sigma^{\r{tot}} &= 4\pi \sum_{l=0}^{\infty} (2l+1)\abs{a_l}^2\\
    &= \frac{4\pi}{k^2} \sum_{l=0}^{\infty} (2l+1)\sin^2\delta_l\\
    f(\theta) &= \sum_{l=0}^{\infty} (2l+1)a_lP_l(\cos\theta)\\
    &= \frac{1}{k} \sum_{l=0}^{\infty}(2l+1)\e^{\i\delta_l}\sin\delta_lP_l(\cos\theta)
  \end{align}
  ここで,$\theta = 0$のとき,
  \begin{align}
    f(0) = \frac{1}{k} \sum_{l=0}^{\infty} (2l+1) (\cos\delta_l + \i \sin\delta_l)\sin\delta_l P_l(1)
  \end{align}
  が成り立つ.したがって以下の\textbf{光学定理}が成り立つ.
  \begin{itembox}[l]{光学定理}
    \begin{align}
      \sigma^{\r{tot}} = \frac{4\pi}{k} \Im f(0)
    \end{align}
  \end{itembox}
  これは全断面積が前方散乱の散乱振幅からわかることを示している\footnote{
   砂川,散乱の量子論,「光学定理は,前方散乱によって,入射波の強度が減少した分だけ,四方に散乱されるという,まことに当然なことを述べているのである.」 
  }.
  \begin{myex}{半径$a$の剛体球による散乱}{}
    散乱体のポテンシャルを
    \begin{align}
      V(r) =
      \begin{dcases}
         \infty\ &(r \leq a)\\
        0\ &(r \geq a)
      \end{dcases}
    \end{align}
    とする.低エネルギー散乱($ka \ll 1$)とする.このとき,散乱の影響を受けるのはほぼ$l=0$のみである.
    式(\ref{SCPartialWave})において$l=0$とすることで
    \begin{align}
      \psi(\bm{r}) = \frac{1}{2\i kr}\qty[(1 + 2\i k a_0)\e^{\i kr} - \e^{-\i kr}] P_0(\cos\theta)
    \end{align}
    を得る.位相シフトを考慮し$\e^{2\i \delta_0} = 1 + 2\i ka_0$とおく.境界条件より
    \begin{align}
      \psi(r=a) = 0
    \end{align}
    であるから
    \begin{align}
      \e^{2\i \delta_0}\e^{\i ka} - \e^{-\i ka} = 0
    \end{align}
    が成り立つ.よって位相シフトは
    \begin{align}
      \delta_0 = -ka
    \end{align}
    である.全断面積は
    \begin{align}
      \sigma^{\r{tot}} &= \frac{4\pi}{k^2}\sin^2\delta_0\\
      &\simeq \frac{4\pi}{k^2}\delta_0\\
      &= 4\pi a^2
    \end{align}
    である.これは古典力学における剛体級の散乱$\sigma^{\r{tot}}=\pi a^2$の4倍の値である.波長$\lambda=\frac{2\pi}{k}$が散乱体の半径$a$より十分に大きいため,
    回折によって剛体球を取り囲み,球の表面積を疑似的に増加させたためと説明できる.
  \end{myex}
\end{document}