\documentclass{report}
\input{../../head.tex}
\begin{document}
  一般に,散乱の波動関数は,
  \begin{align}
    \psi(\bm{r}) = \e^{\i kz} + \int G_0 (\bm{r} - \bm{r'}) \frac{2m}{\hbar^2} V(\bm{r'}) \psi(\bm{r'}) \dd{\bm{r'}}\label{wave-funtion-of-scattering}
  \end{align}
  と表されるのであった.
  この波動関数を厳密に求めることは困難であるため近似を考える.
  まず,\refe{wave-funtion-of-scattering}を簡略化して,
  \begin{align}
    \psi(\bm{r}) = \psi_0 + \int g V \psi(\bm{r}') \dd{\bm{r}'}\label{simple-wave-funtion-of-scattering}
  \end{align}
  と表現する.
  ただし,
  \begin{align}
    g(\bm{r}) \coloneqq G_0(\bm{r})\frac{2m}{\hbar^2}
  \end{align}
  である.
  \refe{simple-wave-funtion-of-scattering}を再帰的に代入すると,
  \begin{align}
    \psi(\bm{r}) &= \psi_0 + \int g V \psi(\bm{r'}) \dd{\bm{r'}} \\
    &= \psi_0 + \int gV \qty(\psi_0 + \int g V \psi(\bm{r''}) \dd{\bm{r''}}) \dd{\bm{r'}} \\
    &= \psi_0 + \int g V \psi_0(\bm{r'}) \dd{\bm{r'}} + \iint gVgV \psi(\bm{r''}) \dd{\bm{r'}} \dd{\bm{r''}} \\
    &= \psi_0 + \int g V \psi_0(\bm{r'}) \dd{\bm{r'}} + \iint gVgV \psi_0(\bm{r''}) \dd{\bm{r'}} \dd{\bm{r''}} + \iiint gVgVgV \psi_0(\bm{r'''}) \dd{\bm{r'}} \dd{\bm{r''}} \dd{\bm{r'''}} + \cdots\label{dozens-of-gv}
  \end{align}
  を得る.
  \refe{dozens-of-gv}を第1項までで近似して,残りの項を捨てる.
  これは散乱の波動関数を平面波で近似することに相当する.
  つまり,
  \begin{align}
    \psi \simeq \psi_0
  \end{align}
  とする.これを\textbf{第1 Born近似}という\footnote{Max Born(1882-1970)}\footnote{砂川,散乱の量子論,
  「第1 Born近似がとくによく利用される理由は,何といってもその簡単さにある.したがって,ある散乱問題を手がけたとき,だれもが最初に試してみるのが,この近似である.
  そして思わしい結果がえられないとき,他の近似法を考えるのである.」}\footnote{2023年度期末試験第3問(4).}.
  \par
  Born近似を用いて散乱振幅を求める.$\bm{k} \coloneqq k\bm{e}_z$,$\bm{r} \coloneqq z\bm{e}_z$とすると,
  $\psi(\bm{r}') = \e^{\i \bm{k}\cdot\bm{r}'}$となるから,
  \begin{align}
    f^{(1)}(\theta) &= -\frac{1}{4\pi} \int \e^{\i \bm{k'}\cdot\bm{r'}} \frac{2m}{\hbar^2} V(\bm{r'}) \psi(\bm{r'}) \dd{\bm{r'}} \\
    &= -\frac{1}{4\pi}\frac{2m}{\hbar^2}\int\e^{-\i (\bm{k'} - \bm{k})\cdot\bm{r}'} V(\bm{r'}) \dd{\bm{r}'} \\
    &= -\frac{1}{4\pi}\frac{2m}{\hbar^2}\int\e^{-\i \bm{q} \cdot \bm{r}'} V(\bm{r'}) \dd{\bm{r}'}\label{potential-fourier-transform}
  \end{align}
  となる.ただし,散乱による運動量変化に対応する物理量を$\bm{q} \coloneqq \bm{k'} - \bm{k}$と定義した.
  \refe{potential-fourier-transform}を見ると,散乱振幅はポテンシャル$V(r)$のFourier変換から得られることがわかる\footnote{
    $f^{(n)}$は第$n$ Born近似による散乱振幅を意味する.
  }.
  また,球対称ポテンシャルのとき\refe{potential-fourier-transform}は簡略化できて,$V(\bm{r'}) \to V(r')$としてよいから,
  \begin{align}
    f^{(1)}(\theta) &= -\frac{1}{4\pi}\frac{2m}{\hbar^2}\int\e^{-\i \bm{q}\cdot \bm{r}'} V(r') \dd{\bm{r}'}\\
    &= -\frac{1}{4\pi}\frac{2m}{\hbar^2} \int_{\phi' = 0}^{2\pi}\dd{\phi'} \int_{\theta' = 0}^{\pi}\dd{\theta'}\int_{r' = 0}^{\infty}r'^2 \sin\theta' \dd{r'} \e^{-\i qr' \cos\theta'} V(r') \\
    &= -\frac{1}{4\pi}\frac{2m}{\hbar^2} 2\pi \int_{r' = 0}^{\infty}V(r')r'^2\int_{\theta' = 0}^{\pi} \e^{-\i qr' \cos\theta'} \sin\theta' \dd{r'}\dd{\theta'} \\ 
    &= -\frac{m}{\hbar^2} \int_{0}^{\infty} r'^2V(r') \frac{\e^{\i qr'} - \e^{-\i qr'}}{\i qr'} \dd{r'} \\ 
    &= -\frac{m}{\hbar^2} \int_{0}^{\infty} rV(r) \frac{\e^{\i qr} - \e^{-\i qr}}{\i q} \dd{r} \\ 
    &= -\frac{2m}{\hbar^2 q} \int_{0}^{\infty} rV(r)\sin\qty(qr) \dd{r}
  \end{align}
  となる.
  \begin{itembox}[l]{球対称ポテンシャルの散乱振幅}
    \begin{align}
      f^{(1)}(\theta) &= -\frac{m}{\hbar^2} \int_{0}^{\infty} rV(r) \frac{\e^{\i qr} - \e^{-\i qr}}{\i q} \dd{r}\label{SCamp1st-exp} \\ 
      &= -\frac{2m}{\hbar^2 q} \int_{0}^{\infty} rV(r)\sin\qty(qr) \dd{r}\label{SCamp1st-sin}
    \end{align}
  \end{itembox}
  \begin{myex}{湯川ポテンシャル}{}
    球対称ポテンシャル$V(r)$が,湯川ポテンシャルの場合を考える.
    湯川ポテンシャル,
    \begin{align}
      V(r) = V_0 \frac{\e^{-\mu r}}{\mu r}
    \end{align}
    による散乱を考える\footnote{湯川秀樹(1907-1981)}.$V(r)$は到達距離が$\mu^{-1}$ほどであり,核子同士に働く力を表す.
    物質中では,伝導電子に遮蔽された不純物のCoulombポテンシャルを表す.
    $\mu = 1,2$及びCoulombポテンシャルのグラフを\reff{yukawa-potential-graph}に示す.
    このポテンシャルの下で散乱振幅$f^{(1)}(\theta)$と散乱断面積$\sigma^{(1)}(\theta)$を求めよ.
    \tcblower
    散乱振幅は\refe{SCamp1st-exp}より,
    \begin{align}
      f^{(1)}(\theta) &= -\frac{m}{\hbar^2} \int_{0}^{\infty} rV(r) \frac{\e^{\i qr} - \e^{-\i qr}}{\i q} \dd{r} \\ 
      &= -\frac{m}{\i \hbar^2 q \mu} \int_{0}^{\infty} rV_0\frac{e^{-\mu r}}{\mu r} \frac{\e^{\i qr} - \e^{-\i qr}}{\i q}\dd{r} \\
      &= -\frac{mV_0}{\i \hbar^2 q \mu} \int_{0}^{\infty} \qty[\exp\qty{(-\mu + \i q)r} - \exp\qty{(-\mu - \i q)r}] \dd{r} \\
      &= - \frac{2m V_0}{\hbar^2 \mu} \frac{1}{\mu^2 + q^2}
    \end{align}
    \par
    散乱振幅は,
    \begin{align}
      \sigma^{(1)}(\theta) &= \abs{f^{(1)}(\theta)}^2\\
      &= \qty(\frac{2m V_0}{\hbar^2 \mu})^2 \frac{1}{(\mu^2 + q^2)^2}\\
      &= \qty(\frac{2m V_0}{\hbar^2 \mu})^2 \frac{1}{\qty[\mu^2 + 4k^2 \sin^2\qty(\theta/2)]^2}
    \end{align}
    である.なお,$\bm{k'}$と$\bm{k}$のなす角が$\theta$であるので余弦定理より,$q = 2k\sin\theta/2$であることを用いた.
    \begin{figure}[H]
      \centering
      \includegraphics[width = 0.5\columnwidth]{fig/yukawa_potential.jpg}
      \caption{湯川ポテンシャルとCoulombポテンシャルの比較}\label{yukawa-potential-graph}
    \end{figure}
  \end{myex}
  \begin{myex}{Rutherford散乱(Griffith Example 10.6)}{}
    湯川ポテンシャルで$V_0 = \frac{q_1q_2}{4\pi \epsilon_0}$,$\mu = 0$とするとCoulombポテンシャル
    \begin{align}
      V(r) = \frac{q_1q_2}{4\pi\epsilon_0}
    \end{align}
    と一致する.\refe{SCamp1st-sin}に代入して散乱振幅を求める.
    \begin{align}
      f(\theta) &= - \frac{2m}{\hbar^2q} \int_{0}^{\infty} r \frac{q_1q_2}{4\pi\epsilon_0 r} \sin\qty(qr) \dd{r}\\
      &= -\frac{2m}{\hbar^2q}\frac{q_1q_2}{4\pi\epsilon_0}\int_{0}^{\infty} \sin\qty(qr) \dd{r} \\
      &=  -\frac{2m}{\hbar^2q}\frac{q_1q_2}{4\pi\epsilon_0}\qty(-\frac{1}{q}\qty[\cos\qty(qr)]_0^{\infty})\\
      &\simeq - \frac{mq_1q_1}{2\pi\epsilon_0\hbar^2q^2}\\
      &= - \frac{mq_1q_1}{8\pi\epsilon_0\hbar^2 \sin^2 \theta/2}
    \end{align}
  \end{myex}
\end{document}