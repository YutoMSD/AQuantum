\documentclass{report}
\input{../../head.tex}
\begin{document}
 散乱の波動関数は
 \begin{align}
  \label{WFofSC}
  \psi(\bm{r}) = \e^{\i kz} + \int G_0 (\bm{r} - \bm{r}') \frac{2m}{\hbar^2} V(r') \psi(\bm{r}') \dd{\bm{r}'}
 \end{align}
 と表されるのであった.この波動関数を厳密に求めることは困難であるため近似をする.まず,式(\ref{WFofSC})を簡略化して
 \begin{align}
  \label{simpleWFofSC}
  \psi(\bm{r}) = \psi_0 + \int g V \psi(\bm{r}') \dd{\bm{r}'}
 \end{align}
 と表現する.式(\ref{simpleWFofSC})を繰り返し代入していくと
 \begin{align}
  \psi(\bm{r}) &= \psi_0 + \int g V \psi(\bm{r}') \dd{\bm{r}'}\\
  &= \psi_0 + \int gV \qty(\psi_0 + \int g V \psi(\bm{r}'') \dd{\bm{r}''}) \dd{\bm{r}'}\\
  &= \psi_0 + \int g V \psi_0(\bm{r}') \dd{\bm{r}'} + \iint gVgV \psi(\bm{r}'') \dd{\bm{r}'} \dd{\bm{r}''}\\
  &= \psi_0 + \int g V \psi_0(\bm{r}') \dd{\bm{r}'} + \iint gVgV \psi_0(\bm{r}'') \dd{\bm{r}'} \dd{\bm{r}''} + \iiint gVgVgV \psi_0(\bm{r}''') \dd{\bm{r}'} \dd{\bm{r}''} \dd{\bm{r}'''} + \cdots
 \end{align}
 を得る.これを第1項までで近似する.これは散乱の波動関数を平面波で近似することに相当する.
 つまり,
 \begin{align}
  \psi \simeq \psi_0
 \end{align}
 とする.これを\textbf{第1Born近似}という\footnote{Max Born(1882-1970)}\footnote{砂川,散乱の量子論,
 「第1Born近似がとくによく利用される理由は,何といってもその簡単さにある.したがって,ある散乱問題を手がけたとき,だれもが最初に試してみるのが,この近似である.
 そして思わしい結果がえられないとき,他の近似法を考えるのである.」}.

 Born近似を用いて散乱振幅を求める.$\psi(\bm{r}') = \e^{\i kz} = \e^{\i \bm{k}\cdot\bm{r}'}$だから,
 \begin{align}
  f^{(1)}(\theta) &= -\frac{1}{4\pi} \int \e^{\i \bm{k}' \cdot \bm{r}'} \frac{2m}{\hbar} V(r') \psi(\bm{r}') \dd{\bm{r}'}\\
  & \simeq -\frac{1}{4\pi}\frac{2m}{\hbar^2}\int\e^{-\i (\bm{k}' - \bm{k})\cdot \bm{r}'} V(r') \dd{\bm{r}'}\\
  & \equiv -\frac{1}{4\pi}\frac{2m}{\hbar^2}\int\e^{-\i \bm{q}\cdot \bm{r}'} V(r') \dd{\bm{r}'}
 \end{align}
 となる.ここで散乱による運動量変化を$\bm{q}\equiv \bm{k}' - \bm{k}$と置いた.
 つまり,散乱振幅はポテンシャル$V(r')$のFourier変換から得られることがわかる\footnote{$f^{(n)}$は第$n$Born近似による散乱振幅を意味する.}.
 また,球対称ポテンシャルのときこれは簡略化でき,
 \begin{align}
  f^{(1)}(\theta) &= -\frac{1}{4\pi}\frac{2m}{\hbar^2}\int\e^{-\i \bm{q}\cdot \bm{r}'} V(r') \dd{\bm{r}'}\\
  &= -\frac{1}{4\pi}\frac{2m}{\hbar^2} 2\pi \iint \e^{-\i qr' \cos\theta'} V(r') r'^2 \sin\theta' \dd{\theta'} \dd{r'}\\
  &= -\frac{m}{\hbar^2} \int r'^2 \dd{r'} V(r') \qty[\frac{\e^{-\i qr' \cos\theta}}{\i qr'}]^{\cos\theta = -1}_{\cos\theta =1}\\
  &= -\frac{m}{\hbar^2} \int r'^2 \dd{r'} \frac{2\i \sin qr'}{\i qr'}V(r')\\
  &= -\frac{2m}{\hbar^2 q} \int rV(r)\sin qr \dd{r}
 \end{align}
 となる.
 \begin{itembox}[l]{球対称ポテンシャルの散乱振幅}
  \begin{align}
    \label{SCamp1st}
    f^{(1)}(\theta) =  -\frac{2m}{\hbar^2 q} \int_{0}^{\infty} rV(r)\sin qr \dd{r}
  \end{align}
 \end{itembox} 

 \begin{myex}{湯川ポテンシャル}{}
湯川ポテンシャル
\begin{align}
  V(r) = V_0 \frac{\e^{-\mu r}}{\mu r}
\end{align}
による散乱を考える\footnote{湯川秀樹(1907-1981)}.これは,$V(r)$の到達距離が$\frac{1}{\mu}$ほどであり,核子同士に働く力を表す.物質中では,伝導電子に遮蔽された不純物のCoulombポテンシャルを表す.
$\mu = 1,2$及びCoulombポテンシャルのグラフを\reff{yukawa-potential-graph}に示す.
このポテンシャルの下で散乱振幅を求める.
\begin{align}
  f^{(1)}(\theta) &= -\frac{1}{4\pi} \frac{2m}{\hbar^2} \iiint \e^{-\i qr'\cos\theta'}V(r') r'^2 \sin\theta'\dd{r'}\dd{\theta'}\dd{\phi'}\\
  &= -\frac{1}{4\pi}\frac{2m}{\hbar^2} 2\pi \int_{0}^{\infty} \qty(\frac{\e^{\i qr'} - \e^{-\i qr'}}{\i qr'})V(r') r'^2 \dd{r'}\\
  &= -\frac{mV_0}{\i \hbar^2 q \mu} \int_{0}^{\infty} \qty[\e^{(-\mu + \i q)r'} - \e^{(-\mu - \i q)r'}] \dd{r'}\\
  &= - \frac{2m V_0}{\hbar^2 \mu} \frac{1}{\mu^2 + q^2}
\end{align}
よって散乱断面積は
\begin{align}
  \sigma^{(1)}(\theta) &= \abs{f^{(1)}(\theta)}^2\\
  &= \qty(\frac{2m V_0}{\hbar^2 \mu})^2 \frac{1}{(\mu^2 + q^2)^2}\\
  &= \qty(\frac{2m V_0}{\hbar^2 \mu})^2 \frac{1}{(\mu^2 + 4K^2 \sin^2\theta/2)^2}
\end{align}
である.ここで,$\bm{k'}$と$\bm{k}$のなす角が$\theta$であるため$q = 2k\sin\theta/2$であることを用いた.

\begin{figure}[H]
  \centering
  \includegraphics[width = 0.6\columnwidth]{fig/yukawa_potential.jpg}
  \caption{湯川ポテンシャルとCoulombポテンシャルの比較}\label{yukawa-potential-graph}
\end{figure}

 \end{myex}

\begin{myex}{Rutherford散乱(Griffith Example 10.6)}{}
  湯川ポテンシャルで$V_0 = \frac{q_1q_2}{4\pi \epsilon_0}$,$\mu = 0$とするとCoulombポテンシャル
  \begin{align}
    V(r) = \frac{q_1q_2}{4\pi\epsilon_0}
  \end{align}
  と一致する.式(\ref{SCamp1st})に代入して散乱振幅を求める.
  \begin{align}
    f(\theta) &= - \frac{2m}{\hbar^2q} \int_{0}^{\infty} r \frac{q_1q_2}{4\pi\epsilon_0 r} \sin qr \dd{r}\\
    &= -\frac{2m}{\hbar^2q}\frac{q_1q_2}{4\pi\epsilon_0}\int_{0}^{\infty} \sin qr \dd{r} \\
    &=  -\frac{2m}{\hbar^2q}\frac{q_1q_2}{4\pi\epsilon_0}\qty(-\frac{1}{q}\qty[\cos qr]_0^{\infty})\\
    &\simeq - \frac{mq_1q_1}{2\pi\epsilon_0\hbar^2q^2}\\
    &= - \frac{mq_1q_1}{8\pi\epsilon_0\hbar^2 \sin^2 \theta/2}
  \end{align}
\end{myex}


 次に,Born近似の適用範囲について考える.
 散乱振幅は
 \begin{align}
  f(\theta) = -\frac{1}{4\pi} \int \e^{-\i \bm{k}\cdot \bm{r}} \frac{2m}{\hbar^2}V(r)\psi(\bm{r})\dd{\bm{r}}
 \end{align}
 であり,散乱の波動関数は
 \begin{align}
  \label{BornWF}
  \psi(\bm{r}) = \psi_0(\bm{r}) + \int g(\bm{r} - \bm{r'}) V(r') \psi_0(\bm{r'}) \dd{\bm{r'}} + \cdots
 \end{align}
 である.この波動関数を
 \begin{align}
  \psi(\bm{r}) \simeq \psi_0(\bm{r})
 \end{align}
 と近似するのが第1Born近似であった.この近似がうまくいく,つまり,
 \begin{align}
  \int \e^{-\i \bm{k}\cdot \bm{r}} \frac{2m}{\hbar^2}V(r)\psi(\bm{r})\dd{\bm{r}}
 \end{align}
 を正しく評価するには,
 $V(r) \neq 0$となる$r\simeq 0$で,
 \begin{align}
  \psi(\bm{r}) \simeq \psi_0(\bm{r})
 \end{align}
 と近似できる必要がある.

 $r \simeq 0$で式(\ref{BornWF})の第1項がそれ以外の項より十分大きければいいため,
 \begin{align}
  \abs{\psi_0(\bm{0})} \gg \abs{\int g(\bm{0} - \bm{r}') V(r') \psi_0(\bm{r}') \dd{\bm{r}'}}
 \end{align}
 これを整理すると,
 \begin{align}
  \abs{\e^{\i kz}} \gg \abs{\int -\frac{2m}{\hbar^2}\frac{\e^{\i kr'}}{4\pi r'}V(r') \e^{\i \bm{k}\cdot\bm{r}'}\dd{\bm{r}'}}\\
  1 \gg \frac{m}{2\pi \hbar^2}\abs{\int \frac{\e^{\i kr'}}{r'}V(r')\e^{\i\bm{k}\cdot\bm{r}'}\dd{\bm{r}'}}
  \label{BornCondition}
 \end{align}
 を得る.これが第1Born近似が有効であるための条件じゃ.

 \begin{myex}{Born近似の適用条件}{}
  ポテンシャル
  \begin{align}
    V(r) = \begin{dcases}
      -V_0\ &(r \leq a)\\
      0\ &(r \geq a)
    \end{dcases}
  \end{align}
  による散乱を考える.
  Born近似の適用条件(\ref{BornCondition})より,
  \begin{align}
    1 &\gg \frac{m}{2\pi \hbar^2}\abs{\int \frac{\e^{\i kr'}}{r'}V_0\e^{\i\bm{k}\cdot\bm{r}'}\dd{\bm{r}'}}\\
    &= \frac{mV_0}{2\pi \hbar^2}\abs{\int \e^{\i kr'}\e^{\i kr'\cos \theta'}r'\sin\theta' \dd{r'}\dd{\theta'}\dd{\phi'}}\\
    &= \frac{mV_0}{2\hbar^2 k^2}\abs{\e^{2\i ka} - 1 - 2\i ka}
  \end{align}
  を得る\footnote{「試験に出そうな計算.」}.これを低エネルギー散乱と高エネルギー散乱に場合分けして見積もる.

  (i)低エネルギー散乱($ka \ll 1$)
  \begin{align}
    \abs{\e^{2\i ka} - 1 - 2\i ka} &\simeq \abs{\qty(1 + 2\i ka + \frac{1}{2}\qty(2 \i ka)^2) - 1 - 2\i ka}\\
    &= 2k^2 a^2
  \end{align}
  より,適用条件は
  \begin{align}
    1 \gg \frac{mV_0}{2 \hbar^2 k^2}2k^2 a^2 = \frac{mV_0 a^2}{\hbar^2}
  \end{align}
  である.つまり,ポテンシャルの大きさ$V_0$または半径$a$が小さい時に近似が有効であることがわかる.

  (ii)高エネルギー散乱($ka \gg 1$)
  \begin{align}
    \abs{\e^{2\i ka} - 1 - 2\i ka} \simeq 2ka
  \end{align}
  より,適用条件は
  \begin{align}
    1 \gg \frac{mV_0 a}{\hbar^2 k}
  \end{align}
  である.$k\to\infty$に対して近似が成立することがわかる.つまり,Born近似は高エネルギー粒子に対して常によい近似であることがわかる.
 \end{myex}

\end{document}