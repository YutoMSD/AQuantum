\documentclass{report}
\input{../../head.tex}
\begin{document}
  Helmholtz方程式
  \begin{align}
    [\nabla^2 + \kappa^2]\psi(\bm{r}) = 0
  \end{align}
  を球座標系で表すと
  \begin{align}
    \label{HEinSphere}
    \qty[\frac{1}{r}\frac{\partial^2}{\partial r^2}r + \frac{1}{r^2 \sin \theta} \frac{\partial}{\partial \theta}\qty(\sin\theta\frac{\partial}{\partial \theta})
    + \frac{1}{r^2 \sin^2\theta}\frac{\partial^2}{\partial \phi^2} + \kappa^2]\psi(r,\theta,\phi) = 0
  \end{align}
  である.この方程式の解は動径波動関数$R(r)$と球面調和関数$Y_{l,m}(\theta, \phi)$を用いて
  \begin{align}
    \psi(r,\theta,\phi) = \sum_{l=0}^{\infty}\sum_{m=-l}^{l}C_{lm}R(r)Y_{l,m}(\theta,\phi)
  \end{align}
  と表される.ここで,球面調和関数とLgendre陪関数,Legendre多項式は以下の関係にある.
  \begin{align}
    Y_{l,m}(\theta,\phi) &= (-1)^m \qty[\frac{(2l+1)}{4\pi}\frac{(l-m)!}{(l+m)!}]^{1/2}P_l^m(\cos\theta)\exp(\i m\phi)\\
    P_l^m(x) &= (1-x^2)^{m/2} \frac{\r{d}^m P_l(x)}{\dd{x}^m}
  \end{align}
  いま,角度$\phi$によらないとすると,$m=0$とすることで
  \begin{align}
    \psi(r,\theta) = \sum_{l=0}^{\infty} (2l+1)R_l(r)P_l(\cos\theta)
  \end{align}
  を得る.$(2l+1)$は縮退の数を表す.これを式(\ref{HEinSphere})に代入する.
  \begin{align}
    \qty[\frac{1}{r}\frac{\partial^2}{\partial r^2}r + \frac{1}{r^2 \sin \theta} \frac{\partial}{\partial \theta}\qty(\sin\theta\frac{\partial}{\partial \theta})
    + \kappa^2] \sum_{l=0}^{\infty} (2l+1)R_l(r)P_l(\cos\theta)= 0
  \end{align}
  整理すると
  \begin{align}
    \frac{P_l(\cos\theta)}{r}\qty[2\frac{\partial R_l(r)}{\partial r} + r \frac{\partial^2 R_l(r)}{\partial r^2}]
    + \frac{R_l(r)}{r^2 \sin \theta} \frac{\partial}{\partial \theta}\qty(\sin\theta\frac{\partial}{\partial\theta}P_l(\cos\theta)) + \kappa^2 R_l(r)P_l(\cos\theta) = 0
  \end{align}
  である.ここで,Legendre多項式は
  \begin{align}
    \frac{\r{d}}{\dd{x}}\qty[(1-x^2)\frac{\r{d}P_l(x)}{\dd{x}}] + l(l+1)P_l(x) = 0
  \end{align}
  を満たす.これを変数変換すると
  \begin{align}
    \frac{1}{\sin\theta}\frac{\r{d}}{\dd{\theta}}\qty[\sin\theta\frac{\r{d}P_l(\cos\theta)}{\dd{\theta}}] + l(l+1)P_l(\cos\theta) = 0
  \end{align}
  となる.よって,
  \begin{align}
    \frac{P_l(\cos\theta)}{r}\qty[2\frac{\partial R_l(r)}{\partial r} + r \frac{\partial^2 R_l(r)}{\partial r^2}]
    - l(l+1)\frac{R_l(r)}{r^2}P_l(\cos\theta)  + \kappa^2 R_l(r)P_l(\cos\theta) = 0
  \end{align}
  を得る.したがって,$R_l(r)$が満たすべき式は
  \begin{align}
    \qty[\frac{\r{d}^2}{\dd{r}^2} + \frac{2}{r} \dv{}{r} + \kappa^2 - \frac{l(l+1)}{r^2}]R_l(r) = 0
  \end{align}
  である.この微分方程式の解$R_l(r)$は球面Bessel関数$j_l(\kappa r)$と球面Neumann関数$n_l(\kappa r)$の和で与えられる.よって,波動関数は
  \begin{align}
    \psi(r,\theta) = \sum_{l=0}^{\infty} (2l+1)\qty[A_l j_l(\kappa r) + B_l n_l(\kappa r)] P_l(\cos\theta)
  \end{align} 
  となる.また,球面Bessel関数と球面Neumann関数は以下の性質を持つ.
  \begin{align}
    x \to 0\\
    j_l(x) &\to \frac{x^l}{(2l+1)!}\qty(1 - \frac{x^2}{2(2l+3)}\cdots)\\
    n_l(x) &\to -\frac{(2l -1)!!}{x^{l+1}}
  \end{align}
  平面波$\e^{\i  z} = \e^{\i kr\cos\theta}$は$r=0$で有限の値をもつため,
  \begin{align}
    B_l =0
  \end{align} 
  であることがわかる.以上の計算から,
  \begin{align}
    \e^{\i \kappa r\cos\theta} = \sum_{l=0}^{\infty} (2l+1)A_l j_l(\kappa r) P_l(\cos\theta)
  \end{align}
  を得る.左辺を展開すると,
  \begin{align}
    \sum_{l=0}^{\infty}\frac{(\i \kappa r \cos\theta)^l}{l!}
  \end{align}
  右辺を展開すると,
  \begin{align}
    \sum_{l=0}^{\infty} (2l+1)A_l\frac{(\kappa l)^l}{(2l+1)!!}P_l(\cos\theta)
  \end{align}
  である.また,$P_l(\cos\theta)$の$(\cos\theta)^l$の項の係数は
  \begin{align}
    \frac{1}{2^l l!}\frac{(2l)!}{l!}
  \end{align}
  である.よって,左辺と右辺を比較することで
  \begin{align}
    \frac{(\i \kappa r \cos\theta)^l}{l!} &= (2l+1)A_l \frac{(2l)!}{2^l (l!)^2}\frac{(\kappa r \cos\theta)^l}{(2l+1)!!}\\
    A_l &= \i^l
  \end{align}
  を得る.したがって,\textbf{Rayleighの公式}
  \begin{align}
    \e^{\i kz} = \e^{\i kr\cos\theta} = \sum_{l=0}^{\infty} (2l+1)\i^l j_l(kr)P_l(\cos\theta)
  \end{align}
  が成り立つ.
\end{document}