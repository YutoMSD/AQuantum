\documentclass{report}
\input{../../head.tex}
\begin{document}
  $z$軸に沿って粒子を単位時間単位面積当たり$n$個入射する.
  \begin{figure}[H]
    \centering
    \includegraphics[width=0.7\columnwidth]{fig/scattering_cm.pdf}
    \caption{古典力学における散乱}
    \label{sc-in-cm-image}
  \end{figure}
  $z$軸から距離$b$ (衝突パラメータ),角度$\dd{\phi}$,面積$\dd{S'}$のスリットを単位時間当たりに通過する粒子数は,
  \begin{align}
    n\dd{S'} = n \dd{\phi} (b \dd{b})
  \end{align}
  を満たす.
  また,単位時間に検出器に到達する粒子数は微分断面積の定義から,
  \begin{align}
    \dd{N} = \sigma(\theta) n \dd{\Omega}
  \end{align}
  である.
  古典力学ではスリットを通過した粒子は,必ず検出器で検出されるので,
  \begin{align}
    n\dd{S'} &= \dd{N} \\ 
    \Leftrightarrow \sigma (\theta) n \dd{\Omega} &= n \dd{\phi} (b \dd{\phi})
  \end{align}
  を得る.よって,微分断面積は,
  \begin{align}
    \sigma(\theta) = \frac{1}{\sin \theta} b \abs{\frac{\dd{b}}{\dd{\theta}}}
  \end{align}
  と表される.
  \begin{myex}{剛体球}{}
    散乱体を半径$a$の剛体球のポテンシャル$V(r)$を,
    \begin{align}
      V(r) = 
      \begin{dcases}
        \infty & r < a \\
        0 & r > a
      \end{dcases}
    \end{align}
    とする.
    衝突パラメータを$b$として,粒子が散乱体の角度$\phi$の位置で散乱し,その散乱角を$\theta$とする.
    これらのパラメータは,
    \begin{align}
      \begin{dcases}
        2\phi + \theta = \pi\\
        b = a\sin \phi
      \end{dcases}
    \end{align}
    を満たすため,
    \begin{align}
      b = a\cos\qty(\frac{\theta}{2})
    \end{align}
    を得る.よって,微分断面積は,
    \begin{align}
      \sigma(\theta) &= \frac{1}{\sin \theta} b \abs{\dv{b}{\theta}} \\
      &= \frac{a\cos\qty(\frac{\theta}{2})}{\sin\theta}\abs{-\frac{a}{2}\sin\qty(\frac{\theta}{2})} \\ 
      &= \frac{a^2}{4}
    \end{align}
    となる.$\theta$に依存しない等方散乱であることがわかる.また,全断面積は,
    \begin{align}
      \sigma^{\r{tot}} = \int \sigma(\theta) \dd{\Omega} = \pi a^2
    \end{align}
    である.剛体球の断面積と一致する.
  \end{myex}
\end{document}