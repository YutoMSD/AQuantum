\documentclass{report}
\input{../../head.tex}
\begin{document}
  Born近似の適用範囲について考える.
  \refe{scattering-amp-general}より,一般に散乱振幅は,
  \begin{align}
    f(\theta) = -\frac{1}{4\pi} \int \e^{-\i\bm{k'}\cdot\bm{r}} \frac{2m}{\hbar^2}V(r)\psi(\bm{r})\dd{\bm{r}}
  \end{align}
  と書けるのであった.また\refe{dozens-of-gv}より,散乱の波動関数は,
  \begin{align}
    \psi(\bm{r}) = \psi_0(\bm{r}) + \int g(\bm{r} - \bm{r'}) V(r') \psi_0(\bm{r'}) \dd{\bm{r'}} + \cdots \label{BornWF}
  \end{align}
  であった.第1 Born近似とは,波動関数$\psi(\bm{r})$を,
  \begin{align}
    \psi(\bm{r}) \simeq \psi_0(\bm{r})
  \end{align}
  と近似するのものであった.第1 Born近似が十分良い評価である,つまり,
  \begin{align}
    \int \e^{-\i \bm{k'}\cdot \bm{r}} \frac{2m}{\hbar^2}V(r)\psi(\bm{r})\dd{\bm{r}}
  \end{align}
  を正しく評価できるには,
  $V(r) \neq 0$なる$r\simeq 0$で,
  \begin{align}
    \psi(\bm{r}) \simeq \psi_0(\bm{r})
  \end{align}
  と近似できる必要がある.
  $r \simeq 0$で\refe{BornWF}の第1項がそれ以外の項より十分大きければ良いから,
  \begin{align}
    \abs{\psi_0(0)} \gg \abs{\int g(0 - \bm{r}) V(r) \psi_0(\bm{r}) \dd{\bm{r}}}
  \end{align}
  となればよい.
  整理すると,
  \begin{align}
    \abs{\e^{\i kz}} \gg \abs{\int -\frac{2m}{\hbar^2}\frac{\e^{\i kr}}{4\pi r}V(r) \e^{\i \bm{k '}\cdot\bm{r}}\dd{\bm{r}}} \\
    1 \gg \frac{m}{2\pi \hbar^2}\abs{\int \frac{\e^{\i kr}}{r}V(r)\e^{\i\bm{k '}\cdot\bm{r}}\dd{\bm{r}}}\label{BornCondition}
  \end{align}
  を得る.これが第1 Born近似が有効であるための条件である.
  \begin{myex}{第1 Born近似の適用条件}{}
    球対称ポテンシャル$V(r)$,
    \begin{align}
      V(r) = \begin{dcases}
        -V_0 &(r \leq a)\\
        0 &(r \geq a)
      \end{dcases}
    \end{align}
    による散乱を考える.
    Born近似の適用条件である\refe{BornCondition}を用いて,満たすべき条件とその物理的意味を述べよ.
    \tcblower
    \refe{BornCondition}よりBorn近似が成立する条件は,% kappaとkのごちゃをなおす.
    \begin{align}
      1 &\gg \frac{m}{2\pi \hbar^2}\abs{\int \frac{\e^{\i kr'}}{r'}V(r')\e^{\i\bm{k '}\cdot\bm{r'}}\dd{\bm{r'}}} \\
      &= \frac{m}{2\pi \hbar^2}\abs{\int \frac{\e^{\i kr'}}{r'}V(r')\e^{\i\bm{k'}\cdot\bm{r'}}\dd{\bm{r'}}} \\
      &= \frac{m}{2\pi \hbar^2}\abs{\int_{\phi' = 0}^{2\pi}\dd{\phi'}\int_{\theta' = 0}^{\pi}\dd{\theta'}\int_{r' = 0}^{\infty}\dd{r'}V(r')\e^{\i\bm{k'}\cdot\bm{r'}}r'^2\sin\theta'} \\ 
      &= \frac{mV_0}{2\pi \hbar^2}\abs{\int_{\phi' = 0}^{2\pi}\dd{\phi'}\int_{\theta' = 0}^{\pi}\dd{\theta'}\int_{r' = 0}^{a}\dd{r'} \frac{\e^{\i kr'}}{r'}\e^{\i kr'\cos\theta'}r'^2\sin\theta'} \\ 
      &= \frac{mV_0}{\hbar^2}\abs{\int_{r' = 0}^{a}r'\e^{\i kr'}\dd{r'}\int_{\theta' = 0}^{\pi}\e^{\i kr'\cos\theta'}\sin\theta'\dd{\theta'}} \\
      &= \frac{mV_0}{\hbar^2}\abs{\int_{r' = 0}^{a}r'\e^{\i kr'}\frac{\e^{\i kr'} - \e^{-\i kr'}}{\i kr'}\dd{r'}} \\
      &= \frac{mV_0}{\hbar^2k}\abs{\int_{r' = 0}^{a}\qty(\e^{2\i kr'} - 1)\dd{r'}} \\
      &= \frac{mV_0}{2\hbar^2 k^2}\abs{\e^{2\i ka} - 1 - 2\i ka}\label{born-limit-ball}
    \end{align}
    を得る\footnote{授業中の教員の発言: 「試験に出そうな計算.」}.これを低エネルギー散乱と高エネルギー散乱に場合分けして見積もる.
    \refe{born-limit-ball}の絶対値の中では,$ka$がまとめて1つの変数のようになっていることに注意する.
    また,\refe{kappa-def}で定めた$\kappa$の定義と,$\kappa \to k$と書き換えたことより,
    \begin{align}
      ka = \frac{\sqrt{2mE}}{\hbar}a
    \end{align}
    であるから,$ka$はエネルギーの$1/2$乗に比例することに注意する.
    \begin{enumerate}
      \item 低エネルギー散乱($ka \ll 1$)\par
        \refe{born-limit-ball}の絶対値の部分を近似すると,
        \begin{align}
          \abs{\e^{2\i ka} - 1 - 2\i ka} &= \abs{\qty(1 + 2\i ka + \frac{1}{2}\qty(2 \i ka)^2 + \cdots) - 1 - 2\i ka}\\
          &\simeq \abs{\qty(1 + 2\i ka + \frac{1}{2}\qty(2 \i ka)^2) - 1 - 2\i ka} \\
          &= 2k^2 a^2
        \end{align}
        となるから,Born近似の適用条件は
        \begin{align}
          1 \gg \frac{mV_0}{2 \hbar^2 k^2}2k^2 a^2 = \frac{mV_0 a^2}{\hbar^2}\label{born-low-energy}
        \end{align}
        である.
        \refe{born-low-energy}において,$m$と$\hbar$は定数であるから,
        ポテンシャルの大きさ$V_0$または半径$a$が小さい時に近似が有効であることがわかる.
      \item 高エネルギー散乱($ka \gg 1$)\par
        \refe{born-limit-ball}の絶対値の部分を近似すると,$ka \gg 1$の領域では,$\abs{\e^{2\i ka}} \simeq \abs{-1} \gg \abs{2\i ka}$であるから,
        \begin{align}
          \abs{\e^{2\i ka} - 1 - 2\i ka} \simeq 2ka
        \end{align}
        となる.
        よってBorn近似の適用条件は,
        \begin{align}
          1 \gg \frac{mV_0 a}{\hbar^2 k} = \frac{mV_0a}{\hbar\sqrt{2mE}}\label{born-high-energy}
        \end{align}
        である.\refe{born-high-energy}を見ると,$k\to\infty$に対して近似が成立することがわかる.
        つまり,Born近似は高エネルギー粒子に対して常によい近似であることがわかる.
    \end{enumerate}
  \end{myex}
\end{document}