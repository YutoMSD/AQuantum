\documentclass{report}
\input{../head.tex}
\begin{document}
  この章では解析力学から量子力学への接続を駆け足で書く.
  座標と運動量に依存するある物理量$\omega(x_i,p_i)$を考える.$\omega$の時間微分は
  \begin{align}
    \dv{\omega}{t} = \frac{\partial\omega}{\partial x_i}\dot{x}_i + \frac{\partial \omega}{\partial p_i}\dot{p}_i \label{time-derivative-omega}
  \end{align}
  である.ここで正準方程式
  \begin{align}
    \begin{dcases}
      \dot{x}_i = \frac{\partial H}{\partial p_i}\\
      \dot{p}_i = -\frac{\partial H}{\partial x_i}
    \end{dcases}
  \end{align}
  より,\refe{time-derivative-omega}の右辺は
  \begin{align}
    \frac{\partial\omega}{\partial x_i}\dot{x}_i + \frac{\partial \omega}{\partial p_i}\dot{p}_i
    =
    \frac{\partial\omega}{\partial x_i}\frac{\partial H}{\partial p_i} - \frac{\partial \omega}{\partial p_i}\frac{\partial H}{\partial x_i}
  \end{align}
  と書ける.上式の右辺を
  \begin{align}
    \qty{\omega,H} \equiv   \frac{\partial\omega}{\partial x_i}\frac{\partial H}{\partial p_i} - \frac{\partial \omega}{\partial p_i}\frac{\partial H}{\partial x_i}
  \end{align}
  とおき,これを\textbf{Poisson括弧}という.つまり,$\omega$の時間微分は
  \begin{align}
    \dv{\omega}{t} = \qty{\omega,H} \label{poisson-h}
  \end{align}
  と書ける.Poisson括弧は一般の物理量$\lambda$で
  \begin{align}
    \qty{\omega,\lambda} \equiv   \frac{\partial\omega}{\partial x_i}\frac{\partial \lambda}{\partial p_i} - \frac{\partial \omega}{\partial p_i}\frac{\partial \lambda}{\partial x_i}
  \end{align}
  と定義される.

  $\omega$の時間発展
  \begin{align}
    \omega(t) = \omega(t+\epsilon) = \omega(t) + \epsilon\dv{\omega}{t}
  \end{align}
  を考える.これは\refe{poisson-h}を用いると,
  \begin{align}
    \omega(t+\epsilon) = \omega(t) + \epsilon\qty{\omega,H}
  \end{align}
  と書くことができる.つまり,$\omega$の時間発展による変化量$\delta\omega=\epsilon\dv{\omega}{t}$はPoisson括弧により,
  \begin{align}
    \delta\omega = \epsilon\qty{\omega,H} \label{time-evolution-by-poisson}
  \end{align}
  と表される.\refe{time-evolution-by-poisson}から,\textbf{$\omega$の時間発展が$\omega$と$H$のPoisson括弧をとることにより生み出される}ことがわかる.
  もし$\omega$が保存量であれば
  \begin{align}
    \qty{\omega,H}=0
  \end{align}
  である.

  この概念を一般化する.物理量$\omega$が何らかの変換により,
  \begin{align}
    \omega \to \omega + \delta\omega = \omega + \epsilon\qty[    \frac{\partial\omega}{\partial x_i}\delta x_i + \frac{\partial \omega}{\partial p_i}\delta p_i
    ]
  \end{align}
  と変化したとする.ここである関数$Q(x_i,p_i)$が$\delta x_i,\delta p_i$を生み出したとする.つまり,
  \begin{align}
    \delta x_i &= \frac{\partial Q}{\partial p_i}\\
    \delta p_i &= -\frac{\partial Q}{\partial x_i}
  \end{align}
  とすると,
  $\omega$の変化は,
  \begin{align}
    \omega + \epsilon\qty{\omega,Q}
  \end{align}
  と書ける.$Q$はPoisson括弧をとることにより変換を生み出したとみなせるので,$Q$は\textbf{Generator}と呼ばれる.
  \begin{myex}{}{}
  $\omega=x_i$,$Q=p_j$とする.
  \begin{align}
    \qty{x_i, p_j} &= \frac{\partial\x_i}{\partial x_i}\frac{\partial p_j}{\partial p_i} + \frac{\partial \x_i}{\partial p_i}\frac{\partial p_i}{\partial x_i}\\
    &= \delta_{ji}
  \end{align}
  よって,
  \begin{align}
    x_i \to x_i + \delta x_i = x_i + \epsilon \delta_{ji}
  \end{align}
  したがって,$j$方向の運動量は$j$方向への並進操作を生成する.
  \end{myex}
  まとめると,$H$のPoisson括弧は時間発展,$p$のPoisson括弧は並進操作を生み出す.同様に角運動量$l$のPoisson括弧が回転操作に対応することも
  確認できる.また,$x$と$p$の間には
  \begin{align}
    \qty{x,p} = 1 \label{xp-poisson}
  \end{align}
  が成り立つ.

  量子力学の世界に入る.$p$は並進操作を生むので,運動量演算子$\hat{p}$が並進操作のgeneratorと考える.
  並進操作の演算子を
  \begin{align}
    \e^{-\i\frac{a\hat{p}}{\hbar}}\ket{x} = \ket{x+a}
  \end{align}
  と定義する.$a$が微小量$\epsilon$のとき,
  \begin{align}
    \qty(\hat{I} - \frac{\i}{\hbar}\epsilon\hat{p}) \ket{x} = \ket{x+\epsilon}
  \end{align}
  と書ける.これを用いて座標空間での$\hat{p}$の形を導出する.ここで,座標空間では
  \begin{align}
    \hat{x}\ket{\psi} \to x\psi(x)
  \end{align}
  である.

  一般の量子状態$\ket{\psi}$に$\ket{x+\epsilon}$を作用させて,
  \begin{align}
    \bra{x+\epsilon}\ket{\psi} &= \bra{x}\qty(\hat{I} + \frac{\i}{\hbar}\epsilon\hat{p})\ket{\psi}\\
    &=\braket{x}{\psi} + \frac{\i}{\hbar}\epsilon\bra{x}\hat{p}\ket{\psi}\\
    &= \braket{x}{\psi} + \frac{\i}{\hbar} \epsilon\int\dd{x'}\bra{x}\hat{p}\ket{x'}\bra{x'}\ket{\psi}
  \end{align}
  を得る.
  また,$\braket{x+\epsilon}{\psi}$は波動関数$\psi(x)$を用いて,
  \begin{align}
    \braket{x+\epsilon}{\psi} &= \psi(x+\epsilon)\\
    &= \psi(x) + \epsilon\dv{\psi}{x}\\
    &= \psi(x) + \epsilon \int \dd{x'}\delta(x-x')\dv{\psi(x')}{x'}\\
    &= \psi(x) - \epsilon\int\dd{x'}\dv{\delta(x-x')}{x'}\psi(x')
  \end{align}
  とも書ける.これら2式を見比べると,
  \begin{align}
    \bra{x}\hat{p}\ket{x'} = -\i\hbar\dv{\delta(x-x')}{x}
  \end{align}
  が得られる.よって,
  \begin{align}
    \bra{x}\hat{p}\ket{\psi} = -\i\hbar\dv{\psi}{x}
  \end{align}
  である.したがって,
  \begin{align}
    \hat{p} \to -\i\hbar\frac{\partial}{\partial x}
  \end{align}
  とできる.さらに,$\hat{x}$と$\hat{p}$の間には交換関係
  \begin{align}
    [\hat{x},\hat{p}] = \i\hbar
  \end{align}
  が成り立つ.これは解析力学でのPoisson括弧\refe{xp-poisson}の$\i\hbar$倍である.
\end{document}
