\documentclass{report}
\input{../head.tex}
\begin{document}
  \begin{itembox}[l]{変分法の基本原理}
  任意の状態ベクトル$\ket{\psi}$に対して$\ket{\psi}$でのエネルギー関数$E(\psi)$について,
    \begin{align*}
      E(\psi) = \cfrac{\mel**{\psi}{\hat{H}}{\psi}}{\braket{\psi}} \geq E_0 \label{minimum}
    \end{align*}
  なる不等式が成り立つ.
  ただし$E_0$は$\hat{H}$の固有エネルギーの中で最低のものである.
  \end{itembox}
  \begin{itembox}[l]{1次摂動によるエネルギー補正}
    \begin{align*}
      E_n^{(1)} = \ev**{\hat{V}}{n^{(0)}}
    \end{align*}
  \end{itembox}
  \begin{itembox}[l]{1次摂動による固有ベクトル補正}
    \begin{align*}
      \ket{n^{(1)}} = \sum_{m\neq n}\cfrac{\mel**{m^{(0)}}{\hat{V}}{n^{(0)}}}{E_n^{(0)} - E_m^{(0)}}\ket{m^{(0)}}
    \end{align*}
  \end{itembox}
  \begin{itembox}[l]{2次摂動によるエネルギー補正}
    \begin{align*}
      E_n^{(2)} = \sum_{m\neq n}\cfrac{\abs{\mel**{m^{(0)}}{\hat{V}}{n^{(0)}}}^2}{E_n^{(0)} - E_m^{(0)}}
    \end{align*}
  \end{itembox}
  \begin{itembox}[l]{縮退がある場合の摂動論による1次エネルギー補正}
    \begin{align*}
      E_n^{(1)} = \frac{1}{2}\qty[(V_{aa} + V_{bb}\pm\sqrt{(V_{aa} - V_{bb})^2 + 4\abs{V_{ab}}^2})]
    \end{align*}
  \end{itembox}
  \begin{itembox}[l]{相互作用表示}
    \begin{align*}
      \ket{\psi(t)}_{\r{I}} \coloneqq \exp\qty(\i\frac{\hat{H}^{(0)}}{\hbar}t)\ket{\psi(t)}
    \end{align*}
  \end{itembox}
  \begin{itembox}[l]{朝永・Schwinger方程式}
    \begin{align*}
      \begin{dcases}
        \i\hbar\dv{t}\ket{\psi(t)}_{\r{I}} = \hat{V}_{\r{I}}(t)\ket{\psi(t)}_{\r{I}} \\
        \hat{V}_{\r{I}}(t) = \exp\qty(\i\frac{\hat{H}^{(0)}}{\hbar}t)\hat{V}(t)\exp\qty(-\i\frac{\hat{H}^{(0)}}{\hbar}t)
      \end{dcases}
    \end{align*}
  \end{itembox}
  \begin{itembox}[l]{非定常摂動量子系の時間発展}
    \begin{align*}
      \i\hbar\dv{t}c_m(t) &= \sum_{n}c_n(t)V_{mn}(t)\e^{\i\omega_{mn}t}
    \end{align*}
  \end{itembox}
  \begin{itembox}[l]{共鳴条件}
    \begin{align*}
      \omega = \omega_{21} = \frac{E_2 - E_1}{\hbar}
    \end{align*}
  \end{itembox}
  \begin{itembox}[l]{$\ket{\psi(t_0)}_{\r{I}} = \ket{i}$の時間発展}
    \begin{align*}
      \begin{dcases}
        \ket{\psi(t)}_{\r{I}}& = \qty(1 + c_{i, i}^{(1)}(t))\ket{i} + \sum_{n\neq i}c_{n, i}^{(1)}(t)\ket{n} \\
        c_{n, i}^{(1)}(t) &= -\frac{\i}{\hbar}\int_{t_0}^{t}V_{n, i}\e^{i\omega_{ni}t}\dd{t}
      \end{dcases}
    \end{align*}
  \end{itembox}
  \begin{itembox}[l]{Fermiの黄金律}
    \begin{align*}
      \omega_{i\to f} = \frac{2\pi}{\hbar}\abs{\mel**{f}{\hat{V}}{i}}^2\delta(E_f - E_i)
    \end{align*}
  \end{itembox}
  \begin{itembox}[l]{電磁場中の電子のハミルトニアン}
    \begin{align*}
      H = \frac{1}{2m}\qty(\bm{p} + e\bm{A})^2 - e\phi
    \end{align*}
  \end{itembox}
  \begin{itembox}[l]{局所ゲージ変換に対して不変なSchrödinger方程式}
    \begin{align*}
      \i\hbar\pdv{t}\psi = \qty[\frac{1}{2m}\qty(\hat{\bm{p}} + e\hat{\bm{A}})^2 - e\phi]\psi
    \end{align*}
  \end{itembox}
  \begin{itembox}[l]{散乱問題の境界条件}
    \begin{align*}
      \psi(\bm{r}) = \e^{\i kz} + f(\theta) \frac{\e^{\i kr}}{r}
    \end{align*}
  \end{itembox}
  \begin{itembox}[l]{散乱振幅と微分断面積の関係}
    \begin{align*}
      \sigma(\theta) = \abs{f(\theta)}^2
    \end{align*}
  \end{itembox}
  \begin{itembox}[l]{球対称ポテンシャルの散乱振幅}
    \begin{align*}
      f^{(1)}(\theta) =  -\frac{2m}{\hbar^2 q} \int_{0}^{\infty} rV(r)\sin qr \dd{r}
    \end{align*}
  \end{itembox}
  \begin{itembox}[l]{部分波展開した散乱の波動関数}
    \begin{align*}
      \psi(\bm{r}) = \sum_{l=0}^{\infty} (2l+1)\i^l j_l(kr)P_l(\cos\theta) + \sum_{l=0}^{\infty} (2l+1)a_l P_l(\cos\theta) \frac{\e^{\i kr}}{r}
    \end{align*}
  \end{itembox}
  \begin{itembox}[l]{光学定理}
    \begin{align*}
      \sigma^{\r{tot}} = \frac{4\pi}{k} \Im f(0)
    \end{align*}
  \end{itembox}
  \begin{itembox}[l]{特殊相対性原理}
    あらゆる慣性系で同じ物理法則が成り立つ.
  \end{itembox}
  \begin{itembox}[l]{光速度不変の原理}
    あらゆる慣性形で真空中の光の速さは同一である.
  \end{itembox}
  \begin{itembox}[l]{Lorentz変換}
    \begin{align*}
      \begin{pmatrix}
        ct' \\ x'
      \end{pmatrix}
      =
      \frac{1}{\sqrt{1 - (v/c)^2}}
      \begin{pmatrix}
        1 & -v/c\\
        -v/c & 1
      \end{pmatrix}
      \begin{pmatrix}
        ct\\x
      \end{pmatrix}
    \end{align*}
  \end{itembox}
  \begin{itembox}[l]{速度の合成}
    \begin{align*}
      V = \frac{v + u'}{1 + \frac{vu'}{c^2}}
    \end{align*}
  \end{itembox}
  \begin{itembox}[l]{Lorentz収縮}
    \begin{align*}
      L' = \sqrt{1-\qty(\frac{v}{c})^2}L
    \end{align*}
  \end{itembox}
  \begin{itembox}[l]{時間の遅れ}
    \begin{align*}
      \Delta T = \frac{1}{\sqrt{1 - (v/c)^2}} \Delta \tau
    \end{align*}
  \end{itembox}
  \begin{itembox}[l]{Lorentz変換に対して共変なMaxwell方程式}
    \begin{align*}
      &\partial_\mu F_{\nu\lambda} + \partial_\nu F_{\lambda\mu} + \partial_\lambda F_{\mu\nu} = 0\\
      &\partial^{\nu}H_{\nu\mu}=j_\mu\\
      &\partial_\mu j^\mu = 0
    \end{align*}
  \end{itembox}
  \begin{itembox}[l]{Klein-Gordon方程式}
    \begin{align*}
      \qty(\frac{1}{c^2}\frac{\partial^2}{\partial t^2} - \grad^2 + \qty(\frac{mc}{\hbar})^2)\psi = 0
    \end{align*}
  \end{itembox}
  \begin{itembox}[l]{Dirac方程式}
    \begin{align*}
      \qty(\i\hbar \frac{\partial}{\partial t} + \i \hbar c \bm{\alpha}\cdot \grad - \beta mc^2)\psi = 0
    \end{align*}
  \end{itembox}
\end{document}