\documentclass{report}
\input{../../head.tex}
\begin{document}
  角運動量$\bm{L}$はSchrödinger方程式においては保存されるが,Dirac方程式では保存されない.本節ではこれを確認し,次節でスピンが自然に導入される.
  \subsection{Schrödinger方程式における角運動量}
    角運動量演算子は,
    \begin{align}
      \hat{\bm{L}} &= \hat{\bm{r}} \times \hat{\bm{p}}\\
      &= \begin{pmatrix}
        \hat{y}\hat{p}_z - \hat{z}\hat{p}_y\\
        \hat{z}\hat{p}_x - \hat{x}\hat{p}_z\\
        \hat{x}\hat{p}_y - \hat{y}\hat{p}_x
      \end{pmatrix}
    \end{align}
    である.
    自由粒子のSchrödinger方程式,
    \begin{align}
      \i \hbar \pdv{t}\psi &= \hat{H} \psi\\
      \hat{H} &= \frac{1}{2m}\qty(\hat{p}_x^2 + \hat{p}_y^2 + \hat{p}_z^2)
    \end{align}
    において,$\hat{H}$と$\hat{L}_x$の交換関係を計算すると,
    \begin{align}
      \qty[\hat{H},\hat{L}_x] &= \frac{1}{2m}\qty[\hat{p}_x^2 + \hat{p}_y^2 + \hat{p}_z^2, \hat{L}_x]\\
      &= \frac{1}{2m}\qty(\qty[\hat{p}_x^2, \hat{L}_x] + \qty[\hat{p}_y^2, \hat{L}_y] + \qty[\hat{p}_z^2, \hat{L}_x])\\
      &=\frac{1}{2m}\qty(\hat{p}_x\qty[\hat{p}_x,\hat{L}_x] + \qty[\hat{p}_x,\hat{L}_x]\hat{p}_x + \hat{p}_y\qty[\hat{p}_y,\hat{L}_x] + \qty[\hat{p}_y,\hat{L}_x]\hat{p}_y
      + \hat{p}_ \qty[\hat{p}_z,\hat{L}_x] + \qty[\hat{p}_z,\hat{L}_x]\hat{p}_z)\\
      &= \frac{1}{2m}\qty(0 + 0 -\i\hbar\hat{p}_y\hat{p}_z - \i\hbar\hat{p}_z\hat{p}_y + \i\hbar\hat{p}_z\hat{p}_y + \i\hbar\hat{p}_y\hat{p}_z)\\
      &= 0
    \end{align}
    となる.$\hat{L}_y$と$\hat{L}_z$も同様であり,よって,
    \begin{align}
      \qty[\hat{H},\hat{\bm{L}}] = 0
    \end{align}
    であることがわかる\footnote{
      上記の計算には
      \begin{align}
        [\hat{L}_i,\hat{p}_j] &= [\epsilon_{ijk}\hat{x}_j\hat{p}_k, \hat{p}_j]\\
        &=\epsilon_{ijk}\qty(\hat{x}_j[\hat{p}_k,\hat{p}_j] + [\hat{x}_j, \hat{p}_j]\hat{p}_k)\\
        &= \i\hbar\epsilon_{ijk}\hat{p}_k
      \end{align}
    を用いた.
    }.つまり,Schrödinger方程式において角運動量は保存量である.
  \subsection{Dirac方程式における角運動量}
    角運動量をDirac方程式で計算する.
    ハミルトニアン$\hat{H}$は,
    \begin{align}
      \hat{H} = c \bm{\alpha} \cdot \hat{\bm{p}} + \beta mc^2
    \end{align}
    と書けるので,$\hat{H}$と$\hat{L}_x$の交換関係は,
    \begin{align}
      \qty[\hat{H}, \hat{L}_x] &= \qty[c \bm{\alpha} \cdot \hat{\bm{p}} + \beta mc^2, \hat{L}_x]\\
      &= c\qty(\qty[\alpha_x \hat{p}_x, \hat{L}_x] + \qty[\alpha_y \hat{p}_y, \hat{L}_x] + \qty[\alpha_z \hat{p}_z, \hat{L}_x]) + \qty[\beta mc^2, \hat{L}_x]\\
      &= c\qty(\alpha_x\qty[\hat{p}_x, \hat{L}_x] + \alpha_y\qty[\hat{p}_y, \hat{L}_x] + \alpha_z\qty[\hat{p}_z, \hat{L}_x])\\
      &= -\i\hbar c \qty(\alpha_y \hat{p}_z - \alpha_z \hat{p}_y)
    \end{align}
    が得られる.同様に,
    \begin{align}
      \qty[\hat{H}, \hat{L}_y] &= -\i\hbar c \qty(\alpha_z \hat{p}_x - \alpha_x \hat{p}_z)\\
      \qty[\hat{H}, \hat{L}_z] &= -\i\hbar c \qty(\alpha_x \hat{p}_y - \alpha_y \hat{p}_x)
    \end{align}
    が得られる.よって,
    \begin{align}
      \qty[\hat{H}, \hat{\bm{L}}] \neq 0
    \end{align}
    であり,Dirac方程式において角運動量が保存量ではないことがわかる.
    しかし,空間の等方性を考えると,全角運動量が保存されないのは不自然である.
    言い換えれば,Dirac方程式において角運動量が保存量ではないことは\textbf{$\bm{L}$以外の角運動量が存在することを示唆している.}
  \end{document}