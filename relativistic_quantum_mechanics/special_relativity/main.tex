\documentclass{report}
\input{../../head.tex}
\begin{document}
  特殊相対性理論(Special Relativity)は次の2つの事柄を原理とする.
  \begin{itembox}[l]{特殊相対性原理}
    あらゆる慣性系で同じ物理法則が成り立つ.
  \end{itembox}
  \begin{itembox}[l]{光速度不変の原理}
    あらゆる慣性形で真空中の光の速さは同一である.
  \end{itembox}
  \subsection{Lorentz変換}
    特殊相対性原理と光速度不変の原理の下で成り立つ座標変換の法則(Lorentz変換)を導く.
    まず,慣性系$X$系の原点$O$と$X'$系の原点$O'$が$t = t' = 0$で一致している.$t = t' = 0$で光が原点($O = O'$)を通過したとする.
    $X$系の空間座標を$(x, y, z)$,$X'$系の空間座標を$(x', y', z')$とすると,光速度不変の原理より,
    \begin{align}
      \frac{\sqrt{x^2 + y^2 + z^2}}{t} = \frac{\sqrt{x'^2 + y'^2 + z'^2}}{t'} = c
    \end{align}
    が成り立つ.上式から\textbf{世界長さ}(spacetime interval)
    \begin{align}
      s^2 \coloneqq x^2 + y^2 + z^2 - (ct)^2
    \end{align}
    が不変量であることが導かれる.
    \par
    次に,世界長さ不変性から,慣性系間の座標変換の法則である\textbf{Lorentz変換}(Lorentz Transformation)を導く
    慣性系$X'$が$x$軸正の方向に速さ$v$で移動しているとする.このとき$y=y',z=z'$である.わかりやすいように$T=\i t$とおく.
    世界長さは座標系に依らないから,
    \begin{align}
      x^2 - (ct)^2 &= x'^2 - (ct')^2\\
      x^2 + (cT)^2 &= x'^2 + (cT)^2
    \end{align}
    が成り立つ.これが回転座標変換と類似していることから,
    \begin{align}
      \mqty(cT' \\ x') 
      = \mqty(
        \cos\theta & -\sin\theta\\
        \sin\theta & \cos\theta
      )
      \mqty(cT \\ x)
    \end{align}
    と書ける.表示を$t$に戻すと,
    \begin{align}
      \mqty(ct' \\ x')
      =
      \mqty(
        \cos\theta & \i\sin\theta\\
        \i\sin\theta & \cos\theta
      )
      \mqty(ct\\ x)
    \end{align}
    である.さらに,$\theta = \i\phi$とすると
    \begin{align}
      \mqty(ct' \\ x')
      =
      \mqty(
        \cosh\phi & -\sinh\phi\\
        -\sinh\phi & \cosh\phi
      )
      \mqty(ct\\ x)
    \end{align}
    となる.よって,
    \begin{align}
      x' = (-\sinh\phi)ct + (\cosh\phi)x
    \end{align}
    を得る.$X$系において時刻$t$が経過したとする.$X$系から見る$X'$系の原点の位置は$x=vt$である.一方,$X'$系から見ると$x'=0$である.
    よって上式から,
    \begin{align}
      0 = (-\sinh\phi)ct + (\cosh\phi)vt
    \end{align}
    が成り立つ.よって,
    \begin{align}
      \frac{v}{c} = \frac{\sinh\phi}{\cosh\phi} = \tanh\phi
    \end{align}
    である.以上より,
    \begin{align}
      \begin{dcases}
        \sinh\phi = \frac{v/c}{\sqrt{1- (v/c)^2}}\\
        \cosh\phi = \frac{1}{\sqrt{1- (v/c)^2}}
      \end{dcases}
    \end{align}
    であることがわかる.したがって,
    \begin{itembox}[l]{Lorentz変換}
      \begin{align}
        \mqty(ct' \\ x')
        =
        \frac{1}{\sqrt{1 - (v/c)^2}}
        \mqty(
          1 & -v/c\\
          -v/c & 1
        )
        \mqty(ct\\ x)\label{LorentzTransformation}
      \end{align}
    \end{itembox}
    を得る.Lorentz変換
    \begin{align}
      x' = \frac{x-vt}{\sqrt{1-(v/c)^2}}
    \end{align}
    において$v \ll c$とすると
    \begin{align}
      x' = x - vt
    \end{align}
    となる.これはGalilei変換と一致している.

    \begin{myexc}{Lorentz変換の問題}{}
      \begin{enumerate}
        \item Lorentz変換\refe{LorentzTransformation}において,$x^2 - (ct)^2$が不変であることを確認せよ.
        \item $X'$系の原点が$X$系から見ると速さ$v$で$x$軸正方向へ運動していることを示せ.
      \end{enumerate}
      \tcblower
      \begin{enumerate}
        \item \refe{LorentzTransformation}より,
          \begin{align}
            x'^2 - (ct')^2 &= \frac{x^2 -2xvt + v^2t^2}{1 - (v/c)^2} - \frac{c^2t^2 - 2xvt + (v/c)^2x^2}{1 - (v/c)^2}\\
            &= \frac{[1-(v/c)^2]x^2 - [1 - (v/c)^2](ct)^2}{1 - (v/c)^2}\\
            &= x^2 - (ct)^2
          \end{align}
        \item $X'$系の原点は
          \begin{align}
            x' = \frac{-vt + x}{\sqrt{1-(v/c)^2}} = 0
          \end{align}
          より,$X$系から見ると
          \begin{align}
            x = vt
          \end{align}
          となる.よって,速さ$v$で$x$軸正方向へ運動していることがわかる.
      \end{enumerate}
    \end{myexc}

  \subsection{速度の合成則}
    次に,相対論的効果を取り込んだ速度の合成則について示す.
    $X$系は静止し,$X'$系が速さ$v$で$x$軸方向に移動しているとする.
    さらに$X'$系では粒子が速さ$u'$で$x'$軸方向に運動している.
    $X$系から見た粒子の速度$V$を求める.\refe{LorentzTransformation}において$v\to -v$とすると,
    \begin{align}
      \mqty(c\dd{t} \\ \dd{x})
      =
      \frac{1}{\sqrt{1 - (v/c)^2}}
      \mqty(
        1 & v/c\\
        v/c & 1
      )
      \mqty(c\dd{t'}\\ \dd{x'})
    \end{align}
    となる.$V=\dv{x}{t}$なので,
    \begin{align}
      V = \dv{x}{t} &= \frac{v \dd{t'} + \dd{x'}}{\dd{t'} + (v/c^2)\dd{x'}}\\
      &= \frac{v + \dv{x'}{t'}}{1 + (v/c^2)\dv{x'}{t'}}\\
      &= \frac{v + u'}{1 + vu' / c^2}
    \end{align}
    を得る.
    \begin{itembox}[l]{速度の合成}
      \begin{align}
        V = \frac{v + u'}{1 + vu'/ c^2}
      \end{align}
    \end{itembox}
    例として$u' = v = c$とすると
    \begin{align}
      V = \frac{2c}{1+c^2/c^2} = c
    \end{align}
    である.速度が合成されても光速を超えることは決してないことがわかる.
  \subsection{Lorentz収縮}
    \textbf{Lorentz収縮}(Length Contractions)について説明する.
    速さ$v$で運動している$X'$系から,$t' = 0$において,静止している$X$系の2点を見る.
    1点は原点$O:(x, t) = (0, 0)$,もう1点は$P:(x, t) = (L, t)$である.
    まず,原点$O$は$X'$系から見ると,
    \begin{align}
      \mqty(ct' \\ x')
      =
      \frac{1}{\sqrt{1 - (v/c)^2}}
      \mqty(
        1 & -v/c \\
        -v/c & 1
      )
      \mqty(0 \\ 0)
      =
      \mqty(0 \\ 0)
    \end{align}
    である.点$P$を$X'$系から見ると,
    \begin{align}
      \mqty(ct' \\ x')
      =
      \frac{1}{\sqrt{1 - (v/c)^2}}
      \mqty(
        1 & -v/c\\
        -v/c & 1
      )
      \mqty(ct \\ L)
      =
      \frac{1}{\sqrt{1 - (v/c)^2}}
      \mqty(ct - \frac{v}{c}L \\ -vt + L)
    \end{align}
    である.$t' = 0$で観測しているため,$t' = 0$を代入し
    \begin{align}
      0 &= ct - \frac{v}{c}L \\
      t &= \frac{v}{c^2}L
    \end{align}
    を得る.よって,
    \begin{align}
      x' = \frac{1}{\sqrt{1 - (v/c)^2}} \qty[-\qty(\frac{v}{c})^2 L + L] = \sqrt{1-\qty(\frac{v}{c})^2}L
    \end{align}
    である.
    これは動いている慣性系から静止系での距離($L$)を測ると縮んで見える($L'$)ことを意味している.
    \begin{itembox}[l]{Lorentz収縮}
      \begin{align}
        L' = \sqrt{1 - \qty(\frac{v}{c})^2}L
      \end{align}
    \end{itembox}
    また,ある対象に対して静止している観測者が測った距離を\textbf{固有長さ}(proper length)という.
    今回は$L$が固有長さである.
  \subsection{時間の遅れ}
    次に\textbf{時間の遅れ}(Time Dilations)について説明する.
    同様の$X$系と$X'$系を考える.
    時計が$X'$系の原点$x' = 0$に置かれていて,観測者は$X'$系においてこの時計を見ている.
    $X'$系に置かれた時計の時刻が$t'$のときの時空点$P_1$,$t' + \Delta T_0$の時空点を$P_2$とする.
    $P_1$は,
    \begin{align}
      \mqty(ct_1 \\ x_1)
      =
      \frac{1}{\sqrt{1 - (v/c)^2}}
      \mqty(
        1 & v/c \\
        v/c & 1
      )
      \mqty(ct' \\ 0)
      =
      \frac{1}{\sqrt{1 - (v/c)^2}}
      \mqty(ct' \\ vt')
    \end{align}
    $P_2$は,
    \begin{align}
      \mqty(ct_1 \\ x_1)
      =
      \frac{1}{\sqrt{1 - (v/c)^2}}
      \mqty(
        1 & v/c \\
        v/c & 1
      )
      \mqty(c(t' + \Delta T_0) \\ 0)
      =
      \frac{1}{\sqrt{1 - (v/c)^2}}
      \mqty(c(t' + \Delta T_0) \\ v(t' + \Delta T_0))
    \end{align}
    である.よって,$X$系での時間経過$\Delta T = t_2 - t_1$は,
    \begin{align}
      \Delta T = t_2 - t_1 = \frac{1}{\sqrt{1 - (v/c)^2}} \Delta T_0
    \end{align}
    である.これは$X'$系は$X$系に比べて時間の流れが遅いことを示している.
    \begin{itembox}[l]{時間の遅れ}
      \begin{align}
        \Delta T = \frac{1}{\sqrt{1 - (v/c)^2}} \Delta \tau
      \end{align}
    \end{itembox}
    観測者に対して2つの事象が同一の空間座標で起きたとき,その時間間隔$\Delta \tau$を\textbf{固有時間}(proper time)という.
  \subsection{電磁場の双対性♠}
    電磁場の双対性について説明する.マクスウェル方程式は,
    \begin{align}
      \begin{dcases}
        \div \bm{D} = \rho \\
        \div \bm{B} = 0 \\
        \curl \bm{E} = - \pdv{\bm{B}}{t} \\
        \curl \bm{H} = \pdv{\bm{D}}{t} + \bm{j}
      \end{dcases}
    \end{align}
    の4つである.
    これをLorentz変換に対して共変な形式に書き直す.
    まず,$\bm{B}$と$\bm{E}$はベクトルポテンシャル$\bm{A}$とスカラーポテンシャル$\phi$を用いて,
    \begin{align}
      \begin{dcases}
        \bm{E} = -\grad \phi - \pdv{\bm{A}}{t} \\
        \bm{B} = \curl \bm{A}
      \end{dcases}
    \end{align}
    と表される.座標を,
    \begin{align}
      (x^0, x^1, x^2, x^3) \coloneqq (ct, x, y, z)
    \end{align}
    と定義する\footnote{
      このような4次元空間を\textbf{Minkovski空間}という. 
    }.
    また4元ベクトルポテンシャルを,
    \begin{align}
      A^{\mu} &\coloneqq (\phi, c\bm{A}) \\
      A_{\mu} &\coloneqq (\phi, -c\bm{A})
    \end{align}
    と定義する.
    4元ベクトルポテンシャルを使うと電場と磁場は,
    \begin{align}
      B_1 \coloneqq B_x &= c\pdv{A_z}{y} - c\pdv{A_y}{z} \\
      &= -\pdv{A_3}{x^2}  + \pdv{A_2}{x^3} \\
      &= \partial_3 A_2 - \partial_2 A_3\\
      E_1 \coloneqq E_x = &= -\pdv{\phi}{x} - \pdv{A_x}{t} \\
      &= -\pdv{A_0}{x^1} + \pdv{A_1}{x^0} \\
      &= \partial_0 A_1 - \partial_1 A_0
    \end{align}
    のように書ける.電磁場テンソルを,
    \begin{align}
      F_{\mu\nu} \coloneqq \partial_{\mu}A_{\nu} - \partial_{\nu} A_{\mu}
    \end{align}
    を定義する.これを行列の形で表すと,
    \begin{align}
      F_{\mu\nu}=
      \mqty(
        F_{00} & F_{01} & F_{02} & F_{03}\\
        F_{10} & F_{11} & F_{12} & F_{13}\\
        F_{20} & F_{21} & F_{22} & F_{23}\\
        F_{30} & F_{31} & F_{32} & F_{33}
      )
      =
      \mqty(
        0 & E_x & E_y & E_z \\
        -E_x & 0 & -cB_z & cB_y\\
        -E_y & cB_z & 0 & -cB_x\\
        -E_z& -cB_y & cB_x & 0
      )
    \end{align}
    である.
    明らかに$F_{\mu\nu} = -F_{\nu\mu}$である.
    また電磁場テンソルの定義から,
    \begin{align}
      \partial_{\mu} F_{\nu\lambda} + \partial_{\nu} F_{\lambda\mu} + \partial_{\lambda} F_{\mu\nu} = 0
    \end{align}
    が成り立つ.実際に計算してみると
    \begin{align}
      \partial_{\mu} F_{\nu\lambda} + \partial_{\nu} F_{\lambda\mu} + \partial_{\lambda} F_{\mu\nu} &= 
      \partial_{\mu}\qty(\partial_{\nu}A_{\lambda} - \partial_{\lambda} A_{\nu}) + \partial_{\nu}\qty(\partial_{\lambda}A_{\mu} - \partial_{\mu} A_{\lambda}) + \partial_{\lambda}\qty(\partial_{\mu}A_{\nu} - \partial_{\nu} A_{\mu}) \\ 
      &= (\partial_{\mu}\partial_{\nu} A_\lambda - \partial_{\mu}\partial_{\lambda} A_{\nu}) + (\partial_{\nu}\partial_{\lambda} A_{\mu} - \partial_{\nu}\partial_{\mu} A_\lambda)
      + (\partial_{\lambda}\partial_{\mu} A_{\nu} - \partial_{\lambda}\partial_{\nu} A_{\mu}) \\
      &= 0
    \end{align}
    である.これはFaradayの電磁誘導の法則と磁束密度に関するGaussの法則を表している.
    例えば$\mu = 0$,$\nu = 1$,$\lambda = 2$とすると,
    \begin{align}
      &\partial_0 F_{12} + \partial_1 F_{20} + \partial_2 F_{01}\\
      &= -\pdv{(cB_z)}{(ct)} + \pdv{x}(-E_y) + \pdv{y}(-E_x)\\
      &= -(\grad\times \bm{E})_z - \pdv{B_z}{t} = 0
    \end{align}
    また,$\mu = 1$,$\nu = 2$,$\lambda = 3$とすると,
    \begin{align}
      &\partial_1 F_{23} + \partial_2 F_{31} + \partial_3 F_{12}\\
      &= c(\partial_xB_x + \partial_yB_y + \partial_zB_Z)\\
      &= c\div\bm{B} = 0
    \end{align}
    が得られる.
    \par
    磁場/電束密度テンソル$H_{\mu\nu}$を,
    \begin{align}
      H_{\mu\nu} =
      \mqty(
        H_{00} & H_{01} & H_{02} & H_{03} \\
        H_{10} & H_{11} & H_{12} & H_{13} \\
        H_{20} & H_{21} & H_{22} & H_{23} \\
        H_{30} & H_{31} & H_{32} & H_{33}
      )
      =
      \mqty(
        0 & cD_x & cD_y & cD_z \\
        -cD_x & 0 & -H_z & H_y\\
        -cD_y & H_z & 0 & -H_x\\
        -cD_z& -H_y & H_x & 0
      )
    \end{align}
    4次元の電流密度を$j^{\mu}$を,
    \begin{align}
      j^{\mu} \coloneqq (c\rho, \bm{j})
    \end{align}
    と定義する.
    明らかに$H_{\mu\nu} = -H_{\nu\mu}$である.
    すると,電束密度に関するGaussの法則とAmpèreの法則は,
    \begin{align}
      \partial^{\nu}H_{\nu\mu} = j_\mu
    \end{align}
    と表される.Einsteinの縮約記法を用いた.
    例えば$\mu = 0$とすると
    \begin{align}
      \partial^1H_{10} + \partial^2H_{20} + \partial^3H_{30} &= j_0\\
      \partial_xD_x + \partial_yD_y + \partial_zD_z &= \rho\\
      \div\bm{D} &= \rho
    \end{align}
    が導かれる.さらに電荷保存則,
    \begin{align}
      \pdv{\rho}{t} + \div\bm{j} &= 0 \\ 
      \Leftrightarrow \pdv{(c\rho)}{(ct)} + \div\bm{j} &= 0
    \end{align}
    を用いて,
    \begin{align}
      \partial_{\mu} j^\mu = 0
    \end{align}
    が得られる.以上をまとめるとMaxwell方程式は,
    \begin{align}
      \begin{dcases*}
        \partial_{\mu} F_{\nu\lambda} + \partial_{\nu} F_{\lambda\mu} + \partial_{\lambda} F_{\mu\nu} = 0 & Faradayの電磁誘導の法則・磁束密度に関するGaussの法則 \\
        \partial^{\nu}H_{\nu\mu} = j_\mu & 電束密度に関するGaussの法則・Ampèreの法則 \\
        \partial_{\mu} j^\mu = 0 & 電荷保存則
      \end{dcases*}
    \end{align}
    と書くことができる.これらの式はテンソルとテンソル,ベクトルとベクトルというように,Lorentz変換に対して
    同じ変換則をもつものどうしが結ばれている.よってこれらはLorentz変換に対して共変である.
    \begin{itembox}[l]{Lorentz変換に対して共変なMaxwell方程式}
      \begin{align}
        &\partial_{\mu} F_{\nu\lambda} + \partial_{\nu} F_{\lambda\mu} + \partial_{\lambda} F_{\mu\nu} = 0\\
        &\partial^{\nu}H_{\nu\mu} = j_\mu\\
        &\partial_{\mu} j^\mu = 0
      \end{align}
    \end{itembox}
  \subsection{電磁場の双対性の例題}
    \begin{myex}{}{}
      運動する電荷と電流を例に電磁場の双対性を確認してみよう.
      座標系は導線の中心を$z$軸とした円筒座標系を入れる.
      導線に$z$軸上向きに電流$I$が流れている.
      電荷$+q$が$z$軸上向きに速さ$v_0$で運動している.
      導線の中では面電荷密度$\lambda_{\pm} = \pm\lambda/2$の正(負)電荷が速さ$v$で上(下)向きに動いているとする.
      まずは静止系で考える.
      導線の中では,
      \begin{align}
        \lambda_{+} + \lambda_{-} = 0
      \end{align}
      が成り立つため,電気的に中性である.
      よって導線の周りには電場は無い.
      電流は$I = \lambda v$と表される.
      導線の周りには,
      \begin{align}
        \bm{B} = \frac{\mu_0I}{2\pi r} \bm{e}_{\phi} = \frac{\mu_0 \lambda v}{2\pi r} \bm{e}_{\phi}
      \end{align}
      の磁束密度が発生している.
      よって電荷は,
      \begin{align}
        \bm{F} = q\bm{v}\times\bm{B} = -qv_0\frac{\mu_0 \lambda v}{2\pi r}\bm{e}_r
      \end{align}
      の力を感じる.
      \par
      同様の設定を電荷とともに動く系で考える.
      非相対論的に考えると$\bm{v} = 0$であるため電荷は力を感じないことになるが,電子が力を受けて運動している実験事実に矛盾する.
      相対論的効果を考慮してこの状況を眺める必要がある.
      観測者からは導線中の正電荷は$v - v_0$で,負電荷は$v + v_0$で運動して見える.
      それぞれでLorentz収縮を計算する.
      \begin{align}
        \lambda_{+} &= \frac{1}{\sqrt{1-\qty(\cfrac{v - v_0}{c})^2}} \cdot \frac{\lambda}{2}\\
        \lambda_{-} &= \frac{1}{\sqrt{1-\qty(\cfrac{v + v_0}{c})^2}} \cdot \qty(-\frac{\lambda}{2})
      \end{align}
      明らかに,
      \begin{align}
        \lambda_{+} + \lambda_{-} \neq 0
      \end{align}
      であることがわかる.よって,導線の周りには電場が発生している.$v,v_0 \ll c$とすると
      \begin{align}
        \Delta \lambda \coloneqq \lambda_{+} + \lambda_{-} &= \frac{1}{\sqrt{1 - \qty(\cfrac{v - v_0}{c})^2}}\frac{\lambda}{2} - \frac{1}{\sqrt{1 - \qty(\cfrac{v + v_0}{c})^2}}\frac{\lambda}{2}\\
        &\simeq \qty[1 + \frac{1}{2}\qty(\frac{v - v_0}{c})^2]\frac{\lambda}{2} - \qty[1 + \frac{1}{2}\qty(\frac{v + v_0}{c})^2]\frac{\lambda}{2}\\
        &= - \frac{\lambda v_0 v}{c^2} 
      \end{align}
      と計算できて,導線は負に帯電していることがわかる.よって電荷の受ける力は,
      \begin{align}
        \bm{F} &= q\frac{\Delta \lambda}{2\pi r \epsilon_0}\bm{e}_r\\
        &= -qv_0\frac{\mu_0 \lambda v}{2\pi r}\bm{e}_r
      \end{align}
      これは先ほどの計算結果と一致している.以上の考察から,電場と磁場は観測する系によって入り混じることがわかる.
    \end{myex}
\end{document}