\documentclass{report}
\input{../../head.tex}
\begin{document}
  全角運動量以外の保存量として\textbf{ヘリシティ}がある.
  \begin{align}
    \bm{\bm{S}} = 
    \mqty(
      \bm{\sigma} & 0\\
      0 & \bm{\sigma}
    )
  \end{align}
  として,ヘリシティは,
  \begin{align}
    h \coloneqq \frac{\bm{\bm{S}} \cdot \bm{p}}{\abs{\bm{p}}}
  \end{align}
  と定義される.
  これが保存量であることは,
  \begin{align}
    \qty(\bm{\alpha} \cdot \hat{\bm{p}}) \hat{h} 
    &= \frac{1}{p}\mqty(
      0 & \bm{\sigma} \cdot \hat{\bm{p}}\\
      \bm{\sigma} \cdot \hat{\bm{p}} & 0
    )\mqty(
      \bm{\sigma} \cdot \hat{\bm{p}} & 0\\
      0 & \bm{\sigma} \cdot \hat{\bm{p}}
    ) \\
    &= \frac{1}{p}\mqty(
      0 & (\bm{\sigma} \cdot \hat{\bm{p}})(\bm{\sigma} \cdot \hat{\bm{p}})\\
      (\bm{\sigma} \cdot \hat{\bm{p}})(\bm{\sigma} \cdot \hat{\bm{p}}) & 0
    ) \\
    &= \frac{1}{p}\mqty(
      0 & \hat{\bm{p}}^2\\
      \hat{\bm{p}}^2 & 0
    ) \\
    &= \hat{h}(\bm{\alpha}\cdot \hat{\bm{p}})
  \end{align}
  であるから,
  \begin{align}
    \qty[\bm{\alpha} \cdot \hat{\bm{p}}, \hat{h}] = 0\\
    \qty[\beta, \hat{h}] = 0
  \end{align}
  となることより,
  \begin{align}
    \qty[\hat{H}, \hat{h}] = 0
  \end{align}
  したがって,
  Dirac方程式においてヘリシティは保存量である.
\end{document}