\documentclass{report}
\input{../../head.tex}
\begin{document}
最後に,電磁場中の電子を考え,Dirac方程式の非相対論的極限($\abs{\bm{p}} \ll c$)がSchrödinger方程式であることを示す.
Dirac方程式は
\begin{align}
  \i\hbar\frac{\partial}{\partial t}\psi = (c \bm{\alpha}\cdot \bm{p} + \beta mc^2)\psi
\end{align}
である.ここに電磁場を次のように導入する.
\begin{align}
  \begin{dcases}
    \bm{p} \to \bm{p} + e\bm{A} (e>0)\\
    \phi = 0
  \end{dcases}
\end{align}
これを用いると
\begin{align}
  \i\hbar\frac{\partial}{\partial t}\psi = (c \bm{\alpha}\cdot (\bm{p} + e\bm{A}) + \beta mc^2)\psi \label{dirac-eq-in-emfield}
\end{align}
である.ここで,解の形を
\begin{align}
  \psi = \begin{pmatrix}
    \psi_1\\\psi_2\\\psi_3\\\psi_4
  \end{pmatrix}
  =
  \begin{pmatrix}
    \psi_a\\
    \psi_b
  \end{pmatrix}
\end{align}
とする.これを\refe{dirac-eq-in-emfield}に代入する.
\begin{align}
  \i\hbar\frac{\partial}{\partial t}
  \begin{pmatrix}
    \psi_a\\
    \psi_b
  \end{pmatrix}
  &=
  \begin{pmatrix}
    0 & c\bm{\sigma} \cdot (\bm{p} + e \bm{A})\\
    c\bm{\sigma} \cdot (\bm{p} + e \bm{A}) & 0
  \end{pmatrix}
  \begin{pmatrix}
    \psi_a\\
    \psi_b
  \end{pmatrix}
  +
  \begin{pmatrix}
    mc^2 & 0\\
    0 & -mc^2
  \end{pmatrix}
  \begin{pmatrix}
    \psi_a\\
    \psi_b
  \end{pmatrix}\\
  &=
  c\bm{\sigma}\cdot(\bm{p} + e\bm{A})
  \begin{pmatrix}
    \psi_b\\
    \psi_a
  \end{pmatrix}
  +mc^2
  \begin{pmatrix}
    \psi_a\\
    -\psi_b
  \end{pmatrix}
\end{align}
また,粒子のエネルギーを静止質量エネルギー($mc^2$)と非相対論的エネルギーの項($\epsilon_{\r{NR}}$)に分け,
\begin{align}
  \epsilon \sim mc^2 + \epsilon_{\r{NR}}
\end{align}
波動関数の時間発展を次のように記述する.
\begin{align}
  \begin{pmatrix}
    \psi_a\\
    \psi_b
  \end{pmatrix}
  =
  \begin{pmatrix}
    \psi_a^0\\
    \psi_b^0
  \end{pmatrix}
  \e^{-\i \frac{mc^2}{\hbar}t}
\end{align}
この表式を用いると\refe{dirac-eq-in-emfield}の左辺は
\begin{align}
  \i\hbar \frac{\partial}{\partial t}
  \begin{pmatrix}
    \psi_a\\
    \psi_b
  \end{pmatrix}
  =
  \i\hbar\qty[
    \frac{\partial}{\partial t}
    \begin{pmatrix}
      \psi_a^0\\
      \psi_b^0
    \end{pmatrix}
  ]
  \e^{-\i\frac{mc^2}{\hbar}t}
  + mc^2
  \begin{pmatrix}
    \psi_a^0\\
    \psi_b^0
  \end{pmatrix}
  \e^{-\i\frac{mc^2}{\hbar}t}
\end{align}
となる.これより,
\begin{align}
  \i\hbar\qty[
    \frac{\partial}{\partial t}
    \begin{pmatrix}
      \psi_a^0\\
      \psi_b^0
    \end{pmatrix}
  ]
  \e^{-\i\frac{mc^2}{\hbar}t}
  + mc^2
  \begin{pmatrix}
    \psi_a^0\\
    \psi_b^0
  \end{pmatrix}
  \e^{-\i\frac{mc^2}{\hbar}t}
  &=
  c\bm{\sigma}\cdot(\bm{p} + e\bm{A})
  \begin{pmatrix}
    \psi_b^0\\
    \psi_a^0
  \end{pmatrix}
  +mc^2
  \begin{pmatrix}
    \psi_a\\
    -\psi_b
  \end{pmatrix}\\
  \i\hbar\frac{\partial}{\partial t}
  \begin{pmatrix}
    \psi_a^0\\
    \psi_b^0
  \end{pmatrix}
  &=
  c\bm{\sigma}\cdot(\bm{p} + e\bm{A})
  \begin{pmatrix}
    \psi_b^0\\
    \psi_a^0
  \end{pmatrix}
  -2mc^2
  \begin{pmatrix}
    0\\
    \psi_b^0
  \end{pmatrix}
\end{align}
が得られる.上式の2行目に着目する.両辺を$mc^2$で割ると
\begin{align}
  \frac{1}{mc^2}\qty(\i\hbar\frac{\partial}{\partial t}\psi_b^0) = \frac{\bm{\sigma}\cdot(\bm{p} + e\bm{A})}{mc}\psi^0_a - 2 \psi^0_b
\end{align}
である.ここで左辺は
\begin{align}
  \frac{1}{mc^2}\qty(\i\hbar\frac{\partial}{\partial t}\e^{-\i\frac{\epsilon_{\r{NR}}}{\hbar} t}) \propto \frac{\epsilon_{\r{NR}}}{mc^2}
\end{align}
である.
また,
\begin{align}
  \epsilon_{\r{NR}} = \frac{\bm{p}^2}{2m}
\end{align}
だから非相対論的極限($\abs{\bm{p}} \ll c$)で左辺は0となる.よって
\begin{align}
  \psi_b^0 = \frac{\bm{\sigma}\cdot(\bm{p} + e\bm{A})}{2mc}\psi^0_a\\
\end{align}
を得る.これは粒子解と反粒子解を結ぶ式である.この式から,反粒子の成分$\psi^0_b$は粒子の成分$\psi^0_a$より$v/c$のオーダーで小さく,非相対論的極限では重要ではないことがわかる.
これを用いると1行目は
\begin{align}
  \i\hbar\frac{\partial}{\partial t}\psi^0_a &= c\bm{\sigma} \cdot (\bm{p} + e\bm{A})\psi_b^0\\
  &= \frac{[\bm{\sigma}\cdot(\bm{p} + e\bm{A})][\bm{\sigma}\cdot(\bm{p} + e\bm{A})]}{2m}\psi^0_a
\end{align}
と書き直される.次に$[\bm{\sigma}\cdot(\bm{p} + e\bm{A})][\bm{\sigma}\cdot(\bm{p} + e\bm{A})]$を計算する.
まず,Pauli行列には次の性質がある.
\begin{align}
  (\bm{\sigma}\cdot\hat{\bm{a}})(\bm{\sigma}\cdot\hat{\bm{b}}) &= (\sum_{i} \sigma_i \hat{a}_i) (\sum_{j} \sigma_j \hat{a}_j)\\
  &= \sum_{i} \sum_{j} \sigma_{i} \sigma_{j} \hat{a}_i \hat{b}_j\\
  &= \sum_{i} \sum_{j} \frac{1}{2}(\qty{\sigma_i, \sigma_j} + [\sigma_i, \sigma_j]) \hat{a}_i \hat{b}_j\\
  &= \sum_{i} \sum_{j} \frac{1}{2}(2 \delta_{i,j} + 2\i \sum_{k} \epsilon_{ijk} \sigma_k) \hat{a}_i \hat{b}_j\\
  &= \sum_{i} \hat{a}_i \hat{b}_i + \i \sigma_k \sum_{i} \sum_{j} \sum_{k} \epsilon_{ijk} \hat{a}_i \hat{b}_j\\
  &= \hat{\bm{a}}\cdot\hat{\bm{b}} + \i \sum_{k} \sigma_k (\hat{\bm{a}} \times \hat{\bm{b}})_k\\
  &= \hat{\bm{a}}\cdot\hat{\bm{b}} + \i \bm{\sigma} \cdot (\hat{\bm{a}} \times \hat{\bm{b}}) \label{pauli-inner-product}
\end{align}
ここで$\epsilon_{ijk}$はLevi-Civitaの完全反対称テンソルで
\begin{align}
  \epsilon_{ijk} = 
  \begin{dcases}
    1\ &(i,j,k) = (x,y,z),(y,z,x),(z,x,y)\\
    -1\ &(i,j,k) = (z,y,x),(y,x,z),(x,z,y)\\
    0\ &\text{otherwise}
  \end{dcases}
\end{align}
である.これを使えば外積は
\begin{align}
  (\bm{a} \times \bm{b})_k = \sum_{i} \sum_{j} \epsilon_{ijk} a_i b_j
\end{align}
と表される.また,$\qty{\hat{A},\hat{B}}$は反交換関係で
\begin{align}
  \qty{\hat{A},\hat{B}} = \hat{A}\hat{B} + \hat{B}\hat{A}
\end{align}
という演算子である.
\refe{paili-inner-product}を用いると
\begin{align}
  [\bm{\sigma}\cdot(\bm{p} + e\bm{A})][\bm{\sigma}\cdot(\bm{p} + e\bm{A})] &= (\hat{\bm{p}} + e\bm{A})^2 + \i \bm{\sigma} \cdot [(\hat{\bm{p}} + e\bm{A}) \times (\hat{\bm{p}} + e\bm{A})]\\
  &= (\hat{\bm{p}} + e\bm{A})^2 + \i e \bm{\sigma} \cdot [(\bm{A} \times \hat{\bm{p}}) + (\hat{\bm{p}} \times e\bm{A})]\\
\end{align}
と変形される.ここで,
\begin{align}
  [(\bm{A} \times \hat{\bm{p}}) + (\hat{\bm{p}} \times e\bm{A})]\psi &= (\bm{A} \times \hat{\bm{p}})\psi + (\hat{\bm{p}} \times \bm{A})\psi\\
  &= -\i \hbar \qty[(\bm{A} \times \grad) \psi + (\grad \times \bm{A}) \psi]\\
  &= -\i\hbar \qty[\bm{A} \times (\grad \psi) + (\grad\psi)\times \bm{A} + (\grad \times \bm{A}) \psi]\\
  %&= \bm{A} \times (\hat{\bm{p}}\psi) + (\hat{\bm{p}}\psi) \times \bm{A} + (\hat{\bm{p}} \times \bm{A})\psi\\
  %&= (\hat{\bm{p}} \times \bm{A}) \psi\\
  &= -\i\hbar \qty[-(\grad\psi)\times \bm{A} + (\grad\psi)\times \bm{A} + (\grad \times \bm{A}) \psi]\\
  &= -\i\hbar(\grad \times \bm{A}) \psi
\end{align}
が成り立つ.以上より,
\begin{align}
  [\bm{\sigma}\cdot(\bm{p} + e\bm{A})][\bm{\sigma}\cdot(\bm{p} + e\bm{A})] &= (\hat{\bm{p}} + e\bm{A})^2 + e\hbar \bm{\sigma}\cdot(\grad \times \bm{A})\\
  &= (\hat{\bm{p}} + e\bm{A})^2 + e\hbar \bm{\sigma} \cdot \bm{B}
\end{align}
したがって,
\begin{align}
  \i\hbar \frac{\partial}{\partial t} \psi_a^0 = \qty[\frac{1}{2m} (\hat{\bm{p}} + e\bm{A})^2 + \frac{e\hbar}{2m}\bm{\sigma}\cdot\bm{B}] \psi_a^0
\end{align}
が得られる.右辺第2項$\frac{e\hbar}{2m}\bm{\sigma}\cdot\bm{B}$はZeeman相互作用によるエネルギーを表す.これは磁場中の自由粒子のSchrödinger方程式と一致している.つまり,
電磁場中の自由粒子のDirac方程式の非相対論的極限は,電磁場中の自由粒子のSchrödinger方程式と一致する.
\end{document}