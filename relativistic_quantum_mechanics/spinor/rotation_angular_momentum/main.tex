\documentclass{report}
\input{../../../head.tex}
\begin{document}
  $z$方向の角運動量は
  \begin{align}
    L_z = xp_y - y p_x
  \end{align}
  で定義される.量子力学では演算子を用いて,
  \begin{align}
    \hat{L}_z = -\i\hbar(x\partial_y - y\partial_x)
  \end{align}
  と表される.上式を極座標に変換すると,
  \begin{align}
    \hat{L}_z = -\i\hbar\frac{\partial}{\partial \phi}
  \end{align}
  となる.ここで,$\epsilon_z$を微小量とし,
  \begin{align}
    \hat{U}(\epsilon_z) = \e^{\i \epsilon_z\hat{L}_z/\hbar} \label{rotation-operator-of-z}
  \end{align}
  を定義する.波動関数$\psi(r,\theta,\phi)$に\refe{rotation-operator-of-z}を作用させると,
  \begin{align}
    \hat{U}(\epsilon_z)\psi(r,\theta,\phi) &\simeq \qty(\hat{I} - \i\epsilon_z\frac{\hat{L}_z}{\hbar})\psi(r,\theta,\phi)\\
    &= \qty(1 - \epsilon_z \frac{\partial}{\partial \phi})\psi(r,\theta,\phi)\\
    &= \psi(r,\theta,\phi - \epsilon_z)
  \end{align}
  が得られる.よって,角運動量演算子から回転操作をつくることができることがわかる.
  同様に,$\hat{L}_x$は$x$軸まわりの回転を,$\hat{L}_y$は$y$軸まわりの回転を生む.

  次に,角運動量の交換関係
  \begin{align}
    [\hat{L}_i, \hat{L}_j] = \i\hbar\epsilon_{ijk} \hat{L}_k \label{commutation-of-am}
  \end{align}
  が回転操作を定義することを確認する.以下の図\ref{rotation-image}のような状況を考える.$\epsilon_x,\epsilon_y$は微小量とする.
  一方のルートでは$x$軸中心に$\epsilon_x$回転,$y$軸中心に$\epsilon_y$回転させた後,$x$軸中心に$-\epsilon_x$回転,$y$軸中心に$-\epsilon_y$回転させている.
  これらの操作を加えた後の状況は,もう一方のルート,$z$軸中心に$\theta_z$回転させたときと一致している.これを数式的に書く.
  あるベクトル$\bm{V}$があるとする.これを$x$軸,$y$軸中心で回転させる行列はそれぞれ
  \begin{align}
    \epsilon_x R_x &=
    \begin{pmatrix}
      1 & 0 & 0\\
      0 & \cos\epsilon_x & -\sin\epsilon_x\\
      0 & \sin\epsilon_x & \cos\epsilon_x
    \end{pmatrix}
    \simeq
    \begin{pmatrix}
      1 & 0 & 0\\
      0 & 1 - \frac{1}{2}\epsilon_x^2 & -\epsilon_x\\
      0 & \epsilon_x & 1 - \frac{1}{2}\epsilon_x^2
    \end{pmatrix}\\
    \epsilon_y R_y &=
    \begin{pmatrix}
      \cos\epsilon_y & 0 & 0\\
      0 & 1 & 0\\
      -\sin\epsilon_y & 0 & \cos\epsilon_y
    \end{pmatrix}
    \simeq
    \begin{pmatrix}
      1 - \frac{1}{2}\epsilon_y^2 & 0 & \epsilon_y\\
      0 & 1 & 0\\
      -\epsilon_y & 0 & 1 - \frac{1}{2}\epsilon_y^2
    \end{pmatrix}
  \end{align}
  である.これを使うと,
  \begin{align}
    \epsilon_xR_x \to \epsilon_y R_y \to -\epsilon_x R_x \to -\epsilon_yR_y
  \end{align}
  という操作は
  \begin{align}
    &\begin{pmatrix}
      1 - \frac{1}{2}\epsilon_y^2 & 0 & -\epsilon_y\\
      0 & 1 & 0\\
      \epsilon_y & 0 & 1 - \frac{1}{2}\epsilon_y^2
    \end{pmatrix}
    \begin{pmatrix}
      1 & 0 & 0\\
      0 & 1 - \frac{1}{2}\epsilon_x^2 & \epsilon_x\\
      0 & -\epsilon_x & 1 - \frac{1}{2}\epsilon_x^2
    \end{pmatrix}
    \begin{pmatrix}
      1 - \frac{1}{2}\epsilon_y^2 & 0 & \epsilon_y\\
      0 & 1 & 0\\
      -\epsilon_y & 0 & 1 - \frac{1}{2}\epsilon_y^2
    \end{pmatrix}
    \begin{pmatrix}
      1 & 0 & 0\\
      0 & 1 - \frac{1}{2}\epsilon_x^2 & -\epsilon_x\\
      0 & \epsilon_x & 1 - \frac{1}{2}\epsilon_x^2
    \end{pmatrix}\\
    &\simeq
    \begin{pmatrix}
      1 & \epsilon_x\epsilon_y & 0\\
      -\epsilon_x\epsilon_y & 1 & 0\\
      0 & 0 & 1
    \end{pmatrix}\\
    &= - \epsilon_x\epsilon_y R_z
  \end{align}
  と計算できる.よって,前述の図を用いた議論は数式的に保証されている.つまり,
  \begin{align}
    \epsilon_xR_x \to \epsilon_y R_y \to -\epsilon_x R_x \to -\epsilon_yR_y = -\epsilon_x\epsilon_y R_z
  \end{align}
  ということである.この関係からも\refe{commutation-of-am}が回転操作を定義していることがわかるだろう.
  さらに,\refe{rotation-operator-of-z}より,$z$軸まわりに$\epsilon_z$回転させるユニタリ演算子$\hat{U}[\epsilon_zR_z]$は
  \begin{align}
    \hat{U}[\epsilon_zR_z] = \e^{-\i\epsilon_z\hat{L}_z/\hbar} \simeq \hat{I} - \i\frac{\epsilon_z}{\hbar}\hat{L}_z + \qty(\frac{-\i}{\hbar})^2\epsilon_z^2 \hat{L}_z^2 + \cdots
  \end{align}
  と表される.$x$.$y$に関しても同様である.これらを用いて,
  \begin{align}
    \epsilon_xR_x \to \epsilon_y R_y \to -\epsilon_x R_x \to -\epsilon_yR_y = -\epsilon_x\epsilon_y R_z
  \end{align}
  を角運動量演算子で表すと
  \begin{align}
    \qty(\hat{I} + \i\frac{\epsilon_y}{\hbar}\hat{L}_y + \qty(\frac{\i}{\hbar})^2\epsilon_y^2 \hat{L}_y^2 + \cdots) \qty(\hat{I} + \i\frac{\epsilon_x}{\hbar}\hat{L}_x + \qty(\frac{\i}{\hbar})^2\epsilon_x^2 \hat{L}_x^2 + \cdots) \qty(\hat{I} - \i\frac{\epsilon_y}{\hbar}\hat{L}_y + \qty(\frac{-\i}{\hbar})^2\epsilon_y^2 \hat{L}_y^2 + \cdots)\\ \qty(\hat{I} - \i\frac{\epsilon_x}{\hbar}\hat{L}_x + \qty(\frac{-\i}{\hbar})^2\epsilon_x^2 \hat{L}_x^2 + \cdots)
    =\qty(\hat{I} + \frac{\i}{\hbar}\epsilon_x\epsilon_y \hat{L}_z + \cdots)
  \end{align}
  と書ける.上式の左辺を整理すると
  \begin{align}
    \hat{I} - \qty(\frac{\i}{\hbar})^2\epsilon_x\epsilon_y[\hat{L}_x, \hat{L}_y]
  \end{align}
  である.よって,
  \begin{align}
    \qty[\hat{L}_x, \hat{L}_y] = \i\hbar\hat{L}_z
  \end{align}
  が成立する.角運動量演算子の交換関係$\qty[\hat{L}_x, \hat{L}_y] = \i\hbar\hat{L}_z$は$z$軸まわりの回転操作を意味していたのである.逆に,交換関係\refe{rotation-operator-of-z}を満たす
  演算子$\hat{\bm{L}}$で
  \begin{align}
    \hat{R}(\phi\bm{n}) = \exp\qty(\frac{\i}{\hbar}\phi\bm{n}\cdot\bm{L})
  \end{align}
  というユニタリ演算子を作れば,それは$\bm{n}$軸まわりの角度$\phi$の回転操作を意味するのである.注目すべきは,以上の議論で$\hat{\bm{L}}$の行列の次元や表現を指定していないことである.

  \begin{figure}[H]
    \centering
    \includegraphics[width=0.8\columnwidth]{fig/rotation.pdf}
    \caption{回転操作}
    \label{rotation-image}
  \end{figure}
\end{document}