\documentclass{report}
\input{../../head.tex}
\begin{document}
  全角運動量が保存するために,
  \begin{align}
    [\hat{H}, \hat{\bm{L}} + \hat{\bm{S}}] = 0
  \end{align}
  を満たす$\bm{S}$がDirac方程式に含まれていると考える.
  スピン演算子を次のように導入する.
  \begin{align}
    \hat{S}_x &= \frac{\hbar}{2}
    \begin{pmatrix}
      \sigma_x & 0\\
      0& \sigma_x
    \end{pmatrix}\\
    \hat{S}_y &= \frac{\hbar}{2}
    \begin{pmatrix}
      \sigma_y & 0\\
      0& \sigma_y
    \end{pmatrix}\\
    \hat{S}_z &= \frac{\hbar}{2}
    \begin{pmatrix}
      \sigma_z & 0\\
      0& \sigma_z
    \end{pmatrix}
  \end{align}
  $\sigma_i\ (i=x,y,z)$はPauli行列である.このようにスピン演算子を導入すると,これらは交換関係
  \begin{align}
    [\alpha_i, \hat{S}_j] &= \i \hbar \epsilon_{ijk} \alpha_k\\
    [\beta, \hat{S}_i] &= 0
  \end{align}
  を満たすため
  \begin{align}
    [\hat{H}, \hat{S}_x] &= \i\hbar c(\alpha_y \hat{p}_z - \alpha_z \hat{p}_y)\\
    [\hat{H}, \hat{S}_y] &= \i\hbar c(\alpha_z \hat{p}_x - \alpha_x \hat{p}_z)\\
    [\hat{H}, \hat{S}_x] &= \i\hbar c(\alpha_x \hat{p}_y - \alpha_y \hat{p}_x)
  \end{align}
  でが得られる.これらは
  \begin{align}
    [\hat{H}, \hat{S}_i] = - [\hat{H}, \hat{L}_i]
  \end{align}
  であるため,
  \begin{align}
    [\hat{H}, \hat{L}_x + \hat{S}_x] = [\hat{H}, \hat{L}_y + \hat{S}_y] = [\hat{H}, \hat{L}_z + \hat{S}_z] = 0
  \end{align}
  が成り立つ.よって,
  Dirac方程式において
  \begin{align}
    [\hat{H}, \hat{\bm{L}} + \hat{\bm{S}}] = 0
  \end{align}
  であり,\textbf{全角運動量}$\bm{J} = \bm{L} + \bm{S}$が保存されることがわかる.ここで,$\bm{S} = \frac{\hbar}{2}\bm{\sigma}$は\textbf{スピン角運動量}である.
  まとめると,非相対論的量子論では軌道角運動量$\bm{L}$が保存量であった.しかし,相対論を取り入れたDirac方程式ではスピン角運動量$\bm{S}$も含めた全角運動量$\bm{J}$が保存量なのである\footnote{
    これを実験的に確かめたのが\textbf{Einstein-de Haas効果}や\textbf{Barnett効果}である.前者は,強磁性体の磁化の向きがそろった時,つまりスピン角運動量が変化したとき,強磁性体全体が力学的回転をするというものであり,Einsteinが生涯で
    行った唯一の実験と言われている.後者は,逆に,力学的回転により磁化の向きがそろうというものである.これらは\textbf{磁気回転効果}(gyromagnetic effect)と言われている.磁気回転効果は常磁性体,核スピン,液体金属流体,常磁性金属薄膜,強磁性金属薄膜,クォーク・グルオンプラズマでも観測されている.
  }.

  以上の議論では,スピンの存在が自然に導入された.この議論に用いた要請は
  \begin{itembox}[l]{スピンの存在のために用いた要請}
  \begin{enumerate}
    \item Lorentz共変性
    \item 状態の時間発展が時間の1階微分で表されること
  \end{enumerate}
  \end{itembox}
  である.1はSchrödinger方程式と相対論の矛盾を解決するために用いた.2は波動関数の確率解釈を可能にするために用いた.以上の要請からDirac方程式が
  導かれ,その式にはスピンの存在が内包されていた.

  ここで,Dirac方程式の平面波解
  \begin{align}
    \psi_{\uparrow}^{+}
    =
    \e^{\i(kz-\omega t)}
    \begin{pmatrix}
      1\\0\\0\\0
    \end{pmatrix}\ 
    \psi_{\downarrow}^{+}
    =
    \e^{\i(kz-\omega t)}
    \begin{pmatrix}
      0\\1\\0\\0
    \end{pmatrix}\ 
    \psi_{\uparrow}^{-}
    =
    \e^{\i(kz-\omega t)}
    \begin{pmatrix}
      0\\0\\1\\0
    \end{pmatrix}\ 
    \psi_{\downarrow}^{-}
    =
    \e^{\i(kz-\omega t)}
    \begin{pmatrix}
      0\\0\\0\\1
    \end{pmatrix}
  \end{align}
  に$\hat{S}_z$を作用させてみると,
  \begin{align}
    \begin{dcases}
      \hat{S}_z \psi^{+}_{\uparrow} &= \frac{\hbar}{2}\psi^{+}_{\uparrow}\\
      \hat{S}_z \psi^{+}_{\downarrow} &= -\frac{\hbar}{2}\psi^{+}_{\downarrow}
    \end{dcases}\\
    \begin{dcases}
      \hat{S}_z \psi^{-}_{\uparrow} &= \frac{\hbar}{2}\psi^{-}_{\uparrow}\\
      \hat{S}_z \psi^{-}_{\downarrow} &= -\frac{\hbar}{2}\psi^{-}_{\downarrow}
    \end{dcases}
  \end{align}
  が成り立つ.よって,平面波解は$\hat{S}_z$の固有状態であることがわかる.

  全角運動量以外の保存量として\textbf{ヘリシティ}がある\footnote{授業では触れていない.}.
  \begin{align}
    \bm{\Sigma} = 
    \begin{pmatrix}
      \bm{\sigma} & 0\\
      0 & \bm{\sigma}
    \end{pmatrix}
  \end{align}
  として,ヘリシティは
  \begin{align}
    h = \frac{\bm{\Sigma} \cdot \bm{p}}{\abs{\bm{p}}}
  \end{align}
  と定義される.
  これが保存量であることは以下のように確かめられる.
  \begin{align}
    (\bm{\alpha} \cdot \hat{\bm{p}}) \hat{h} &= \frac{1}{p}
    \begin{pmatrix}
      0 & \bm{\sigma} \cdot \hat{\bm{p}}\\
      \bm{\sigma} \cdot \hat{\bm{p}} & 0
    \end{pmatrix}
    \begin{pmatrix}
      \bm{\sigma} \cdot \hat{\bm{p}} & 0\\
      0 & \bm{\sigma} \cdot \hat{\bm{p}}
    \end{pmatrix}\\
    &= \frac{1}{p}
    \begin{pmatrix}
      0 & (\bm{\sigma} \cdot \hat{\bm{p}})(\bm{\sigma} \cdot \hat{\bm{p}})\\
      (\bm{\sigma} \cdot \hat{\bm{p}})(\bm{\sigma} \cdot \hat{\bm{p}}) & 0
    \end{pmatrix}\\
    &=\frac{1}{p}
    \begin{pmatrix}
      0 & \hat{\bm{p}}^2\\
      \hat{\bm{p}}^2 & 0
    \end{pmatrix}\\
    &= \hat{h}(\bm{\alpha}\cdot \hat{\bm{p}})
  \end{align}
  よって
  \begin{align}
    [\bm{\alpha} \cdot \hat{\bm{p}}, \hat{h}] = 0\\
    [\beta, \hat{h}] = 0
  \end{align}
  したがって,
  \begin{align}
    [\hat{H}, \hat{h}] = 0
  \end{align}
  ヘリシティは保存量である.
\end{document}