\documentclass{report}
\input{../../head.tex}
\begin{document}
  全角運動量が保存するために,
  \begin{align}
    \qty[\hat{H}, \hat{\bm{L}} + \hat{\bm{S}}] = 0
  \end{align}
  なる物理量として,スピン$\hat{\bm{S}}$が存在すると考える.
  スピン演算子を,
  \begin{align}
    \hat{S}_x &\coloneqq \frac{\hbar}{2}
    \mqty(
      \sigma_x & 0\\
      0 & \sigma_x
    ) \\
    \hat{S}_y &\coloneqq \frac{\hbar}{2}
    \mqty(
      \sigma_y & 0\\
      0 & \sigma_y
    ) \\
    \hat{S}_z &\coloneqq \frac{\hbar}{2}
    \mqty(
      \sigma_z & 0\\
      0 & \sigma_z
    )
  \end{align}
  ただし,$\sigma_i\ (i = x, y, z)$はPauli行列である.
  このようにスピン演算子を導入すると,これらは$\alpha_i$や$\beta$との交換関係において,
  \begin{align}
    \qty[\alpha_i, \hat{S}_j] &= \i \hbar \epsilon_{ijk} \alpha_k\\
    \qty[\beta, \hat{S}_i] &= 0
  \end{align}
  を満たすため,
  \begin{align}
    \qty[\hat{H}, \hat{S}_x] &= \i\hbar c\qty(\alpha_y \hat{p}_z - \alpha_z \hat{p}_y)\\
    \qty[\hat{H}, \hat{S}_y] &= \i\hbar c\qty(\alpha_z \hat{p}_x - \alpha_x \hat{p}_z)\\
    \qty[\hat{H}, \hat{S}_x] &= \i\hbar c\qty(\alpha_x \hat{p}_y - \alpha_y \hat{p}_x)
  \end{align}
  が得られる.$\hat{\bm{S}}$の定義より,
  \begin{align}
    \qty[\hat{H}, \hat{S}_i] = - \qty[\hat{H}, \hat{L}_i]
  \end{align}
  であるため,
  \begin{align}
    \qty[\hat{H}, \hat{L}_x + \hat{S}_x] = \qty[\hat{H}, \hat{L}_y + \hat{S}_y] = \qty[\hat{H}, \hat{L}_z + \hat{S}_z] = 0
  \end{align}
  が成り立つ.よって,
  Dirac方程式において
  \begin{align}
    \qty[\hat{H}, \hat{\bm{L}} + \hat{\bm{S}}] = 0
  \end{align}
  であり,\textbf{全角運動量}$\bm{J} \coloneqq \bm{L} + \bm{S}$が保存されることがわかる. 
  ただし,$\bm{S} = \frac{\hbar}{2}\bm{\sigma}$は\textbf{スピン角運動量}である.
  まとめると,非相対論的量子論では軌道角運動量$\bm{L}$が保存量であった.
  しかし,相対論を取り入れたDirac方程式ではスピン角運動量$\bm{S}$も含めた全角運動量$\bm{J}$が保存量なのである\footnote{
    これを実験的に確かめたのが\textbf{Einstein - de Haas効果}や\textbf{Barnett効果}である.
    前者は,強磁性体の磁化の向きがそろった時,つまりスピン角運動量が変化したとき,強磁性体全体が力学的回転をするというものであり,Einsteinが生涯で行った唯一の実験と言われている.
    後者は,逆に,力学的回転により磁化の向きがそろうというものである.これらは\textbf{磁気回転効果}(gyromagnetic effect)と言われている.
    磁気回転効果は常磁性体,核スピン,液体金属流体,常磁性金属薄膜,強磁性金属薄膜,クォーク・グルオンプラズマでも観測されている.
  }.
  以上の議論では,スピンの存在が自然に導入された.この議論に用いた要請は,
  \begin{itembox}[l]{スピンの存在のために用いた要請}
    \begin{enumerate}
      \item Lorentz共変性
      \item 状態の時間発展が時間の1階微分で表されること
    \end{enumerate}
  \end{itembox}
  である.
  1. はSchrödinger方程式と相対論の矛盾を解決するために用いた.
  2. は波動関数の確率解釈を可能にするために用いた.
  以上の要請からDirac方程式が導かれ,その式にはスピンの存在が内包されていた.
  \par
  \refe{dirac-plane-particle-up}から\refe{dirac-plane-antiparticle-down}で表したDirac方程式の平面波解と,$\hat{S}_z$の関係を調べる.
  平面波解に$\hat{S}_z$を作用させてみると,
  \begin{align}
    \begin{dcases}
      \hat{S}_z \psi^{+}_{\uparrow} = \frac{\hbar}{2}\psi^{+}_{\uparrow}\\
      \hat{S}_z \psi^{+}_{\downarrow} = -\frac{\hbar}{2}\psi^{+}_{\downarrow}
    \end{dcases}\\
    \begin{dcases}
      \hat{S}_z \psi^{-}_{\uparrow} = \frac{\hbar}{2}\psi^{-}_{\uparrow}\\
      \hat{S}_z \psi^{-}_{\downarrow} = -\frac{\hbar}{2}\psi^{-}_{\downarrow}
    \end{dcases}
  \end{align}
  が成り立つ.よって,平面波解は$\hat{S}_z$の固有状態であることがわかる.

  全角運動量以外の保存量として\textbf{ヘリシティ}がある\footnote{授業では触れていない.}.
  \begin{align}
    \bm{\bm{S}} = 
    \mqty(
      \bm{\sigma} & 0\\
      0 & \bm{\sigma}
    )
  \end{align}
  として,ヘリシティは
  \begin{align}
    h = \frac{\bm{\bm{S}} \cdot \bm{p}}{\abs{\bm{p}}}
  \end{align}
  と定義される.
  これが保存量であることは以下のように確かめられる.
  \begin{align}
    (\bm{\alpha} \cdot \hat{\bm{p}}) \hat{h} &= \frac{1}{p}
    \mqty(
      0 & \bm{\sigma} \cdot \hat{\bm{p}}\\
      \bm{\sigma} \cdot \hat{\bm{p}} & 0
    )
    \mqty(
      \bm{\sigma} \cdot \hat{\bm{p}} & 0\\
      0 & \bm{\sigma} \cdot \hat{\bm{p}}
    )\\
    &= \frac{1}{p}
    \mqty(
      0 & (\bm{\sigma} \cdot \hat{\bm{p}})(\bm{\sigma} \cdot \hat{\bm{p}})\\
      (\bm{\sigma} \cdot \hat{\bm{p}})(\bm{\sigma} \cdot \hat{\bm{p}}) & 0
    )\\
    &=\frac{1}{p}
    \mqty(
      0 & \hat{\bm{p}}^2\\
      \hat{\bm{p}}^2 & 0
    )\\
    &= \hat{h}(\bm{\alpha}\cdot \hat{\bm{p}})
  \end{align}
  よって
  \begin{align}
    [\bm{\alpha} \cdot \hat{\bm{p}}, \hat{h}] = 0\\
    [\beta, \hat{h}] = 0
  \end{align}
  したがって,
  \begin{align}
    [\hat{H}, \hat{h}] = 0
  \end{align}
  ヘリシティは保存量である.
\end{document}