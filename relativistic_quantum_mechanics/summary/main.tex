\documentclass{report}
\input{../../head.tex}
\begin{document}
第3章の流れをまとめる.

特殊相対性原理と光速度不変の原理を要請することで特殊相対性理論が導かれた.この理論は,観測者のとる慣性系によって時間や空間の間隔が変化することを主張する.
特殊相対性理論により,時間軸と空間座標は同等であることがわかった.

しかし,上記の主張は今まで用いてきたSchrödinger方程式と矛盾する.なぜなら,Schrödinger方程式は空間の1階微分と時間の2階微分を含むからである.
よって,Schrödinger方程式の修正が必要になった.相対論的量子論の基礎方程式の候補となったのがKlein-Gordon方程式である.これはハミルトニアンに静止質量エネルギーを加えることで得られた.
この方程式は時間と空間を同等に扱っているが,波動関数の確率解釈が成り立たないという問題を抱えていた.この問題は時間の2階微分に起因する.そこで,時間と空間の1階微分から構成された方程式が
必要になった.それがDirac方程式である.Dirac方程式は特殊相対論との矛盾を解決し,かつ,波動関数の確率解釈も可能にした.

Dirac方程式は2つ新たな概念を自然に含んでいた.1つが反粒子の存在である.Dirac方程式はスピノルという,4成分をもつ量の方程式であり,そこには正エネルギー解と負エネルギー解が存在する.
Diracはこの負エネルギー解を反粒子解と解釈することで説明した.もう1つは,スピンの存在である.Schrödinger方程式では保存量であった軌道角運動量$\bm{L}$がDirac方程式では保存量ではなくなった.
そこで,角運動量保存を成り立たせるために導入されたのがスピン角運動量$\bm{S}$である.相対論を取り込んだDirac方程式では全角運動量$\bm{J} = \bm{L} + \bm{S}$が保存量なのである.
\end{document}