\documentclass{report}
\input{../../head.tex}
\begin{document}
角運動量$\bm{L}$はSchrödinger方程式においては保存されるが,Dirac方程式では保存されない.本節ではこれを確認し,次節でスピンが自然に導入される.

角運動量演算子は
\begin{align}
  \hat{\bm{L}} &= \hat{\bm{r}} \times \hat{\bm{p}}\\
  &= \begin{pmatrix}
    \hat{y}\hat{p}_z - \hat{z}\hat{p}_y\\
    \hat{z}\hat{p}_x - \hat{x}\hat{p}_z\\
    \hat{x}\hat{p}_y - \hat{y}\hat{p}_x
  \end{pmatrix}
\end{align}
である.これは自由粒子のSchrödinger方程式
\begin{align}
  \i \hbar \frac{\partial}{\partial t}\psi &= \hat{H} \psi\\
  \hat{H} &= \frac{1}{2m}\qty(\hat{p}_x^2 + \hat{p}_y^2 + \hat{p}_z^2)
\end{align}
においては,$\hat{\bm{L}}$とハミルトニアンが交換するため保存する.実際に計算してみる.
\begin{align}
  [\hat{H},\hat{L}_x] &= \frac{1}{2m}[\hat{p}_x^2 + \hat{p}_y^2 + \hat{p}_z^2, \hat{L}_x]\\
  &= \frac{1}{2m}\qty([\hat{p}_x^2, \hat{L}_x] + [\hat{p}_y^2, \hat{L}_y] + [\hat{p}_z^2, \hat{L}_x])\\
  &=\frac{1}{2m}\qty(\hat{p}_x[\hat{p}_x,\hat{L}_x] + [\hat{p}_x,\hat{L}_x]\hat{p}_x + \hat{p}_y[\hat{p}_y,\hat{L}_x] + [\hat{p}_y,\hat{L}_x]\hat{p}_y
  + \hat{p}_z[\hat{p}_z,\hat{L}_x] + [\hat{p}_z,\hat{L}_x]\hat{p}_z)\\
  &= \frac{1}{2m}\qty(0 + 0 -\i\hbar\hat{p}_y\hat{p}_z - \i\hbar\hat{p}_z\hat{p}_y + \i\hbar\hat{p}_z\hat{p}_y + \i\hbar\hat{p}_y\hat{p}_z)\\
  &= 0
\end{align}
$\hat{L}_y$と$\hat{L}_z$も同様であり,よって,
\begin{align}
  [\hat{H},\hat{\bm{L}}] = 0
\end{align}
であることがわかる\footnote{
  上記の計算には
  \begin{align}
    [\hat{L}_i,\hat{p}_j] &= [\epsilon_{ijk}\hat{x}_j\hat{p}_k, \hat{p}_j]\\
    &=\epsilon_{ijk}\qty(\hat{x}_j[\hat{p}_k,\hat{p}_j] + [\hat{x}_j, \hat{p}_j]\hat{p}_k)\\
    &= \i\hbar\epsilon_{ijk}\hat{p}_k
  \end{align}
を用いた.
}.つまり,角運動量は保存量である.

これをDirac方程式で計算する.
\begin{align}
  \hat{H} = c \bm{\alpha} \cdot \hat{\bm{p}} + \beta mc^2
\end{align}
を使うと,
\begin{align}
  [\hat{H}, \hat{L}_x] &= [c \bm{\alpha} \cdot \hat{\bm{p}} + \beta mc^2, \hat{L}_x]\\
  &= c\qty([\alpha_x \hat{p}_x, \hat{L}_x] + [\alpha_y \hat{p}_y, \hat{L}_x] + [\alpha_z \hat{p}_z, \hat{L}_x]) + [\beta mc^2, \hat{L}_x]\\
  &= c\qty(\alpha_x[\hat{p}_x, \hat{L}_x] + \alpha_y[\hat{p}_y, \hat{L}_x] + \alpha_z[\hat{p}_z, \hat{L}_x])\\
  &= -\i\hbar c (\alpha_y \hat{p}_z - \alpha_z \hat{p}_y)
\end{align}
が得られる.同様に,
\begin{align}
  [\hat{H}, \hat{L}_y] &= -\i\hbar c (\alpha_z \hat{p}_x - \alpha_x \hat{p}_z)\\
  [\hat{H}, \hat{L}_z] &= -\i\hbar c (\alpha_x \hat{p}_y - \alpha_y \hat{p}_x)
\end{align}
が得られる.よって,
\begin{align}
  [\hat{H}, \hat{\bm{L}}] \neq 0
\end{align}
であり,角運動量が保存量ではないことがわかる.しかし,対称性から,全角運動量が保存されないのは不自然である.このことは\textbf{$\bm{L}$以外の角運動量が存在することを示唆している.}
\end{document}