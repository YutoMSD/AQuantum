\documentclass{report}
\input{../../head.tex}
\begin{document}
  全角運動量が保存するために,
  \begin{align}
    [\hat{H}, \hat{\bm{L}} + \hat{\bm{S}}] = 0
  \end{align}
  を満たす$\bm{S}$がDirac方程式に含まれていると考える.
  スピン演算子を次のように導入する.
  \begin{align}
    \hat{S}_x &= \frac{\hbar}{2}
    \begin{pmatrix}
      \sigma_x & 0\\
      0& \sigma_x
    \end{pmatrix}\\
    \hat{S}_y &= \frac{\hbar}{2}
    \begin{pmatrix}
      \sigma_y & 0\\
      0& \sigma_y
    \end{pmatrix}\\
    \hat{S}_z &= \frac{\hbar}{2}
    \begin{pmatrix}
      \sigma_z & 0\\
      0& \sigma_z
    \end{pmatrix}
  \end{align}
  $\sigma_i\ (i=x,y,z)$はPauli行列である.このようにスピン演算子を導入すると,これらは交換関係
  \begin{align}
    [\alpha_i, \hat{S}_j] &= \i \hbar \epsilon_{ijk} \alpha_k\\
    [\beta, \hat{S}_i] &= 0
  \end{align}
  を満たすため
  \begin{align}
    [\hat{H}, \hat{S}_x] &= \i\hbar c(\alpha_y \hat{p}_z - \alpha_z \hat{p}_y)\\
    [\hat{H}, \hat{S}_y] &= \i\hbar c(\alpha_z \hat{p}_x - \alpha_x \hat{p}_z)\\
    [\hat{H}, \hat{S}_x] &= \i\hbar c(\alpha_x \hat{p}_y - \alpha_y \hat{p}_x)
  \end{align}
  であり,
  \begin{align}
    [\hat{H}, \hat{L}_x + \hat{S}_x] = [\hat{H}, \hat{L}_y + \hat{S}_y] = [\hat{H}, \hat{L}_z + \hat{S}_z] = 0
  \end{align}
  が成り立つ.よって,
  Dirac方程式において
  \begin{align}
    [\hat{H}, \hat{\bm{L}} + \hat{\bm{S}}] = 0
  \end{align}
  であり,\textbf{全角運動量}$\bm{J} = \bm{L} + \bm{S}$が保存が保存される.ここで,$\bm{S} = \frac{\hbar}{2}\bm{\sigma}$は\textbf{スピン角運動量}である.

  以上の議論では,スピンの存在が自然に導入された.この議論に用いた要請は
  \begin{itembox}[l]{スピンの存在のために用いた要請}
  \begin{enumerate}
    \item Lorentz共変性
    \item 状態の時間発展が時間の1階微分で表されること
  \end{enumerate}
  \end{itembox}
  である.1はSchrödinger方程式と相対論の矛盾を解決するために用いた.2は波動関数の確率解釈を可能にするために用いた.以上の要請からDirac方程式が
  導かれ,その式にはスピンの存在が内包されていた.

  Dirac方程式の平面波解
  \begin{align}
    \psi_{\uparrow}^{+}
    =
    \e^{\i(kz-\omega t)}
    \begin{pmatrix}
      1\\0\\0\\0
    \end{pmatrix}\ 
    \psi_{\downarrow}^{+}
    =
    \e^{\i(kz-\omega t)}
    \begin{pmatrix}
      0\\1\\0\\0
    \end{pmatrix}\ 
    \psi_{\uparrow}^{-}
    =
    \e^{\i(kz-\omega t)}
    \begin{pmatrix}
      0\\0\\1\\0
    \end{pmatrix}\ 
    \psi_{\downarrow}^{-}
    =
    \e^{\i(kz-\omega t)}
    \begin{pmatrix}
      0\\0\\0\\1
    \end{pmatrix}
  \end{align}
  に$\hat{S}_z$を作用させるてみると,
  \begin{align}
    \begin{dcases}
      \hat{S}_z \psi^{+}_{\uparrow} &= \frac{\hbar}{2}\psi^{+}_{\uparrow}\\
      \hat{S}_z \psi^{+}_{\downarrow} &= -\frac{\hbar}{2}\psi^{+}_{\downarrow}
    \end{dcases}\\
    \begin{dcases}
      \hat{S}_z \psi^{-}_{\uparrow} &= \frac{\hbar}{2}\psi^{-}_{\uparrow}\\
      \hat{S}_z \psi^{-}_{\downarrow} &= -\frac{\hbar}{2}\psi^{-}_{\downarrow}
    \end{dcases}
  \end{align}
  が成り立つ.つまり,平面波解は$\hat{S}_z$の固有状態であることがわかる.
\end{document}