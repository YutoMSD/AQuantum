\documentclass{report}
\input{../../head.tex}
\begin{document}
最後に,電磁場中の電子を考え,Dirac方程式の非相対論的極限($\abs{\bm{p}} \ll c$)がSchrödinger方程式であることを示す.
Dirac方程式は
\begin{align}
  \i\hbar\frac{\partial}{\partial t}\psi = (c \bm{\alpha}\cdot \bm{p} + \beta mc^2)\psi
\end{align}
である.ここに電磁場を次のように導入する.
\begin{align}
  \begin{dcases}
    \bm{p} \to \bm{p} + e\bm{A} (e>0)\\
    \phi = 0
  \end{dcases}
\end{align}
これを用いると
\begin{align}
  \i\hbar\frac{\partial}{\partial t}\psi = (c \bm{\alpha}\cdot (\bm{p} + e\bm{A}) + \beta mc^2)\psi \label{dirac-eq-in-emfield}
\end{align}
である.ここで,解の形を
\begin{align}
  \psi = \begin{pmatrix}
    \psi_1\\\psi_2\\\psi_3\\\psi_4
  \end{pmatrix}
  =
  \begin{pmatrix}
    \psi_a\\
    \psi_b
  \end{pmatrix}
\end{align}
とする.これを\refe{dirac-eq-in-emfield}に代入する.
\begin{align}
  \i\hbar\frac{\partial}{\partial t}
  \begin{pmatrix}
    \psi_a\\
    \psi_b
  \end{pmatrix}
  &=
  \begin{pmatrix}
    0 & c\bm{\sigma} \cdot (\bm{p} + e \bm{A})\\
    c\bm{\sigma} \cdot (\bm{p} + e \bm{A}) & 0
  \end{pmatrix}
  \begin{pmatrix}
    \psi_a\\
    \psi_b
  \end{pmatrix}
  +
  \begin{pmatrix}
    mc^2 & 0\\
    0 & -mc^2
  \end{pmatrix}
  \begin{pmatrix}
    \psi_a\\
    \psi_b
  \end{pmatrix}\\
  &=
  c\bm{\sigma}\cdot(\bm{p} + e\bm{A})
  \begin{pmatrix}
    \psi_b\\
    \psi_a
  \end{pmatrix}
  +mc^2
  \begin{pmatrix}
    \psi_a\\
    -\psi_b
  \end{pmatrix}
\end{align}
また,粒子のエネルギーを静止質量エネルギー($mc^2$)と非相対論的エネルギーの項($\epsilon_{\r{NR}}$)に分け,
\begin{align}
  \epsilon \sim mc^2 + \epsilon_{\r{NR}}
\end{align}
波動関数の時間発展を次のように記述する.
\begin{align}
  \begin{pmatrix}
    \psi_a\\
    \psi_b
  \end{pmatrix}
  =
  \begin{pmatrix}
    \psi_a^0\\
    \psi_b^0
  \end{pmatrix}
  \e^{-\i \frac{mc^2}{\hbar}t}
\end{align}
この表式を用いると\refe{dirac-eq-in-emfield}の左辺は
\begin{align}
  \i\hbar \frac{\partial}{\partial t}
  \begin{pmatrix}
    \psi_a\\
    \psi_b
  \end{pmatrix}
  =
  \i\hbar\qty[
    \frac{\partial}{\partial t}
    \begin{pmatrix}
      \psi_a^0\\
      \psi_b^0
    \end{pmatrix}
  ]
  \e^{-\i\frac{mc^2}{\hbar}t}
  + mc^2
  \begin{pmatrix}
    \psi_a^0\\
    \psi_b^0
  \end{pmatrix}
  \e^{-\i\frac{mc^2}{\hbar}t}
\end{align}
となる.これより,
\begin{align}
  \i\hbar\qty[
    \frac{\partial}{\partial t}
    \begin{pmatrix}
      \psi_a^0\\
      \psi_b^0
    \end{pmatrix}
  ]
  \e^{-\i\frac{mc^2}{\hbar}t}
  + mc^2
  \begin{pmatrix}
    \psi_a^0\\
    \psi_b^0
  \end{pmatrix}
  \e^{-\i\frac{mc^2}{\hbar}t}
  &=
  c\bm{\sigma}\cdot(\bm{p} + e\bm{A})
  \begin{pmatrix}
    \psi_b\\
    \psi_a
  \end{pmatrix}
  +mc^2
  \begin{pmatrix}
    \psi_a\\
    -\psi_b
  \end{pmatrix}\\
  \i\hbar\frac{\partial}{\partial t}
  \begin{pmatrix}
    \psi_a^0\\
    \psi_b^0
  \end{pmatrix}
  &=
  c\bm{\sigma}\cdot(\bm{p} + e\bm{A})
  \begin{pmatrix}
    \psi_b^0\\
    \psi_a^0
  \end{pmatrix}
  -2mc^2
  \begin{pmatrix}
    0\\
    \psi_b^0
  \end{pmatrix}
\end{align}
が得られる.上式の2行目に着目する.両辺を$mc^2$で割ると
\begin{align}
  \frac{1}{mc^2}\qty(\i\hbar\frac{\partial}{\partial t}\psi_b^0) = \frac{\bm{\sigma}\cdot(\bm{p} + e\bm{A})}{mc}\psi^0_a - 2 \psi^0_b
\end{align}
である.ここで左辺は
\begin{align}
  \frac{1}{mc^2}\qty(\i\hbar\frac{\partial}{\partial t}\e^{-\i\frac{\epsilon_{\r{NR}}}{\hbar}t}) \propto \frac{\epsilon_{\r{NR}}}
\end{align}
\end{document}