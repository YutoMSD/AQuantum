\documentclass{report}
\input{../../head.tex}
\begin{document}
  特殊相対性理論(Special Relativity)は次の2つの事柄を原理とする.
  \begin{itembox}[l]{特殊相対性原理}
    あらゆる慣性系で同じ物理法則が成り立つ.
  \end{itembox}

  \begin{itembox}[l]{光速度不変の原理}
    あらゆる慣性形で真空中の光の速さは同一である.
  \end{itembox}

  この原理の下で成り立つ座標変換の法則(Lorentz変換)を導く.
  まず,慣性系$X$系の原点$O$とと$X'$系の原点$O'$が$t=t'=0$で一致している.$t=t'=0$で光が原点($O=O'$)を通過したとする.
  $X$系の空間座標を$(x,y,z)$,$X'$系の空間座標を$(x',y',z')$とすると,光速度不変の原理より,
  \begin{align}
    \frac{\sqrt{x^2 + y^2 + z^2}}{t} = \frac{\sqrt{x'^2 + y'^2 + z'^2}}{t'} = c
  \end{align}
  が成り立つ.上式から\textbf{世界長さ}(spacetime interval)
  \begin{align}
    s^2 = x^2 + y^2 + z^2 - (ct)^2
  \end{align}
  が不変量であることが導かれる.

  慣性系$X'$が$x$軸正の方向に速さ$v$で移動している..このとき$y=y',z=z'$である.わかりやすいように$T=it$とおく.光速不変より,
  \begin{align}
    x^2 - (ct)^2 &= x'^2 - (ct')^2\\
    x^2 + (cT)^2 &= x'^2 + (cT)^2
  \end{align}
  が成り立つ.これが回転座標変換と類似していることから,
  \begin{align}
    \begin{pmatrix}
      cT' \\ x'
    \end{pmatrix}
    =
    \begin{pmatrix}
      \cos\theta & -\sin\theta\\
      \sin\theta & \cos\theta
    \end{pmatrix}
    \begin{pmatrix}
      cT\\x
    \end{pmatrix}
  \end{align}
  と置く.表示を$t$に戻すと
  \begin{align}
    \begin{pmatrix}
      ct' \\ x'
    \end{pmatrix}
    =
    \begin{pmatrix}
      \cos\theta & \i\sin\theta\\
      \i\sin\theta & \cos\theta
    \end{pmatrix}
    \begin{pmatrix}
      ct\\x
    \end{pmatrix}
  \end{align}
  である.さらに,$\theta = \i\phi$とすると
  \begin{align}
    \begin{pmatrix}
      ct' \\ x'
    \end{pmatrix}
    =
    \begin{pmatrix}
      \cosh\phi & -\sinh\phi\\
      -\sinh\phi & \cosh\phi
    \end{pmatrix}
    \begin{pmatrix}
      ct\\x
    \end{pmatrix}
  \end{align}
  となる.よって,
  \begin{align}
    x' = (-\sinh\phi)ct + (\cosh\phi)x
  \end{align}
  を得る.$X$系において時刻$t$が経過したとする.$X$系から見るお$X'$系の原点の位置は$x=vt$である.一方,$X'$系から見ると$x'=0$である.
  よって上式から
  \begin{align}
    0 = (-\sinh\phi)ct + (\cosh\phi)vt
  \end{align}
  が成り立つ.よって,これを変形すると
  \begin{align}
    \frac{v}{c} = \frac{\sinh\phi}{\cosh\phi} = \tanh\phi
  \end{align}
  である.以上より,
  \begin{align}
    \begin{dcases}
      \sinh\phi = \frac{v/c}{\sqrt{1- (v/c)^2}}\\
      \cosh\phi = \frac{1}{\sqrt{1- (v/c)^2}}
    \end{dcases}
  \end{align}
  であることがわかる.したがって,
  \begin{itembox}[l]{Lorentz変換}
    \begin{equation}
      \label{LorentzTransformation}
      \begin{pmatrix}
        ct' \\ x'
      \end{pmatrix}
      =
      \frac{1}{\sqrt{1 - (v/c)^2}}
      \begin{pmatrix}
        1 & -v/c\\
        -v/c & 1
      \end{pmatrix}
      \begin{pmatrix}
        ct\\x
      \end{pmatrix}
    \end{equation}
  \end{itembox}
  を得る.Lorentz変換
  \begin{align}
    x' = \frac{x-vt}{\sqrt{1-(v/c)^2}}
  \end{align}
  において$v \ll c$とすると
  \begin{align}
    x' = x - vt
  \end{align}
  となる.これはGalilei変換と一致している.

  次に,相対論的効果を取り込んだ速度の合成則について示す.状況は,$X$系は静止し,$X'$系が速さ$v$で$x$軸方向に移動しているとする.さらに$X'$系では粒子が速さ$u'$で$x'$軸方向に運動している.
  $X$系から見た粒子の速度$V$を求める.式(\ref{LorentzTransformation})において$v\to -v$とすると
  \begin{align}
    \begin{pmatrix}
      c\dd{t} \\ \dd{x}
    \end{pmatrix}
    =
    \frac{1}{\sqrt{1 - (v/c)^2}}
    \begin{pmatrix}
      1 & v/c\\
      v/c & 1
    \end{pmatrix}
    \begin{pmatrix}
      c\dd{t'}\\\dd{x'}
    \end{pmatrix}
  \end{align}
  となる.$V=\dv{x}{t}$なので
  \begin{align}
    V = \dv{x}{t} &= \frac{v \dd{t'} + \dd{x'}}{\dd{t'} + (v/c^2)\dd{x'}}\\
    &= \frac{v + \dv{x'}{t'}}{1 + (v/c^2)\dv{x'}{t'}}\\
    &= \frac{v + u'}{1 + \frac{vu'}{c^2}}
  \end{align}
  を得る.
  \begin{itembox}[l]{速度の合成}
    \begin{equation}
      V = \frac{v + u'}{1 + \frac{vu'}{c^2}}
    \end{equation}
  \end{itembox}
  例として$u' = v = c$とすると
  \begin{align}
    V = \frac{2c}{1+c^2/c^2} = c
  \end{align}
  である.速度が合成されても光速を超えることは決してないことがわかる.

  \textbf{Lorentz収縮}について説明する.速さ$v$で運動している$X'$系から,$t'=0$において,静止している$X$系の2点を見る.1点は原点$O:(x,t)=(0,0)$
  もう1点は$P:(x,t)=(L,t)$とする.まず,原点$O$は$X'$系から見ると
  \begin{equation}
    \begin{pmatrix}
      ct' \\ x'
    \end{pmatrix}
    =
    \frac{1}{\sqrt{1 - (v/c)^2}}
    \begin{pmatrix}
      1 & -v/c\\
      -v/c & 1
    \end{pmatrix}
    \begin{pmatrix}
      0\\0
    \end{pmatrix}
    =
    \begin{pmatrix}
      0\\0
    \end{pmatrix}
  \end{equation}
  である.点$P$を$X'$系から見ると,
  \begin{equation}
    \begin{pmatrix}
      ct' \\ x'
    \end{pmatrix}
    =
    \frac{1}{\sqrt{1 - (v/c)^2}}
    \begin{pmatrix}
      1 & -v/c\\
      -v/c & 1
    \end{pmatrix}
    \begin{pmatrix}
      ct\\L
    \end{pmatrix}
    =
    \frac{1}{\sqrt{1 - (v/c)^2}}
    \begin{pmatrix}
      ct - \frac{v}{c}L\\-vt + L
    \end{pmatrix}
  \end{equation}
  である.$t'=0$で観測しているため,$t'=0$を代入し
  \begin{align}
    0 = ct - \frac{v}{c}L\\
    t = \frac{v}{c^2}L
  \end{align}
  を得る.よって,
  \begin{align}
    x' = \frac{1}{\sqrt{1 - (v/c)^2}} \qty(-\qty(\frac{v}{c})^2 L + L) = \sqrt{1-\qty(\frac{v}{c})^2}L
  \end{align}
  である.これは動いている慣性系から静止系での距離($L$)を測る($L'$)と縮んで見えることを意味している.
  \begin{itembox}[l]{Lorentz収縮}
    \begin{equation}
      L' = \sqrt{1-\qty(\frac{v}{c})^2}L
    \end{equation}
  \end{itembox}
  また,その対象に対して静止している観測者が測った距離を\textbf{固有長さ}(proper length)という.今回は$L$が固有長さである.

  次に\textbf{時間の遅れ}について説明する.同様の$X$系と$X'$系を考える.時計が$X'$系の原点$x'=0$に置かれており,観測者は$X'$系においてこの時計を見ている.
  $X'$系に置かれた時計の時刻が$t'$のときの時空点$P_1$,$t'+\Delta T_0$の時空点を$P_2$とする.$P_1$は,
  \begin{equation}
    \begin{pmatrix}
      ct_1 \\ x_1
    \end{pmatrix}
    =
    \frac{1}{\sqrt{1 - (v/c)^2}}
    \begin{pmatrix}
      1 & v/c\\
      v/c & 1
    \end{pmatrix}
    \begin{pmatrix}
      ct'\\0
    \end{pmatrix}
    =
    \frac{1}{\sqrt{1 - (v/c)^2}}
    \begin{pmatrix}
      ct'\\vt'
    \end{pmatrix}
  \end{equation}
  $P_2$は,
  \begin{equation}
    \begin{pmatrix}
      ct_1 \\ x_1
    \end{pmatrix}
    =
    \frac{1}{\sqrt{1 - (v/c)^2}}
    \begin{pmatrix}
      1 & v/c\\
      v/c & 1
    \end{pmatrix}
    \begin{pmatrix}
      c(t' + \Delta T_0)\\0
    \end{pmatrix}
    =
    \frac{1}{\sqrt{1 - (v/c)^2}}
    \begin{pmatrix}
      c(t' + \Delta T_0)\\v(t' + \Delta T_0)
    \end{pmatrix}
  \end{equation}
  である.よって,$X$系での時間経過$\Delta T = t_2 - t_1$は,
  \begin{align}
    \Delta T = t_2 - t_1 = \frac{1}{\sqrt{1 - (v/c)^2}} \Delta T_0
  \end{align}
  である.これは$X'$系は$X$系に比べて時間の流れが遅いことを示している.
  \begin{itembox}[l]{時間の遅れ}
    \begin{equation}
      \Delta T = \frac{1}{\sqrt{1 - (v/c)^2}} \Delta \tau
    \end{equation}
  \end{itembox}
  観測者に対して2つの事象が同一の空間座標で起きたとき,その時間間隔$\Delta \tau$を\textbf{固有時間}(proper time)という.
\end{document}