\documentclass{report}
\input{../../head.tex}
\begin{document}
  特殊相対性理論(Special Relativity)は次の2つの事柄を原理とする.
  \begin{itembox}[l]{特殊相対性原理}
    あらゆる慣性系で同じ物理法則が成り立つ.
  \end{itembox}

  \begin{itembox}[l]{光速度不変の原理}
    あらゆる慣性形で真空中の光の速さは同一である.
  \end{itembox}

  この原理の下で成り立つ座標変換の法則(Lorentz変換)を導く.
  まず,慣性系$X$系の原点$O$とと$X'$系の原点$O'$が$t=t'=0$で一致している.$t=t'=0$で光が原点($O=O'$)を通過したとする.
  $X$系の空間座標を$(x,y,z)$,$X'$系の空間座標を$(x',y',z')$とすると,光速度不変の原理より,
  \begin{align}
    \frac{\sqrt{x^2 + y^2 + z^2}}{t} = \frac{\sqrt{x'^2 + y'^2 + z'^2}}{t'} = c
  \end{align}
  が成り立つ.上式から\textbf{世界長さ}(spacetime interval)
  \begin{align}
    s^2 = x^2 + y^2 + z^2 - (ct)^2
  \end{align}
  が不変量であることが導かれる.

  慣性系$X'$が$x$軸正の方向に速さ$v$で移動している..このとき$y=y',z=z'$である.わかりやすいように$T=it$とおく.光速不変より,
  \begin{align}
    x^2 - (ct)^2 &= x'^2 - (ct')^2\\
    x^2 + (cT)^2 &= x'^2 + (cT)^2
  \end{align}
  が成り立つ.これが回転座標変換と類似していることから,
  \begin{align}
    \begin{pmatrix}
      cT' \\ x'
    \end{pmatrix}
    =
    \begin{pmatrix}
      \cos\theta & -\sin\theta\\
      \sin\theta & \cos\theta
    \end{pmatrix}
    \begin{pmatrix}
      cT\\x
    \end{pmatrix}
  \end{align}
  と置く.表示を$t$に戻すと
  \begin{align}
    \begin{pmatrix}
      ct' \\ x'
    \end{pmatrix}
    =
    \begin{pmatrix}
      \cos\theta & \i\sin\theta\\
      \i\sin\theta & \cos\theta
    \end{pmatrix}
    \begin{pmatrix}
      ct\\x
    \end{pmatrix}
  \end{align}
  である.さらに,$\theta = \i\phi$とすると
  \begin{align}
    \begin{pmatrix}
      ct' \\ x'
    \end{pmatrix}
    =
    \begin{pmatrix}
      \cosh\phi & -\sinh\phi\\
      -\sinh\phi & \cosh\phi
    \end{pmatrix}
    \begin{pmatrix}
      ct\\x
    \end{pmatrix}
  \end{align}
  となる.よって,
  \begin{align}
    x' = (-\sinh\phi)ct + (\cosh\phi)x
  \end{align}
  を得る.$X$系において時刻$t$が経過したとする.$X$系から見るお$X'$系の原点の位置は$x=vt$である.一方,$X'$系から見ると$x'=0$である.
  よって上式から
  \begin{align}
    0 = (-\sinh\phi)ct + (\cosh\phi)vt
  \end{align}
  が成り立つ.よって,これを変形すると
  \begin{align}
    \frac{v}{c} = \frac{\sinh\phi}{\cosh\phi} = \tanh\phi
  \end{align}
  である.以上より,
  \begin{align}
    \begin{dcases}
      \sinh\phi = \frac{v/c}{\sqrt{1- (v/c)^2}}\\
      \cosh\phi = \frac{1}{\sqrt{1- (v/c)^2}}
    \end{dcases}
  \end{align}
  であることがわかる.したがって,
  \begin{itembox}[l]{Lorentz変換}
    \begin{equation}
      \label{LorentzTransformation}
      \begin{pmatrix}
        ct' \\ x'
      \end{pmatrix}
      =
      \frac{1}{\sqrt{1 - (v/c)^2}}
      \begin{pmatrix}
        1 & -v/c\\
        -v/c & 1
      \end{pmatrix}
      \begin{pmatrix}
        ct\\x
      \end{pmatrix}
    \end{equation}
  \end{itembox}
  を得る.Lorentz変換
  \begin{align}
    x' = \frac{x-vt}{\sqrt{1-(v/c)^2}}
  \end{align}
  において$v \ll c$とすると
  \begin{align}
    x' = x - vt
  \end{align}
  となる.これはGalilei変換と一致している.

  次に,相対論的効果を取り込んだ速度の合成則について示す.状況は,$X$系は静止し,$X'$系が速さ$v$で$x$軸方向に移動しているとする.さらに$X'$系では粒子が速さ$u'$で$x'$軸方向に運動している.
  $X$系から見た粒子の速度$V$を求める.式(\ref{LorentzTransformation})において$v\to -v$とすると
  \begin{align}
    \begin{pmatrix}
      c\dd{t} \\ \dd{x}
    \end{pmatrix}
    =
    \frac{1}{\sqrt{1 - (v/c)^2}}
    \begin{pmatrix}
      1 & v/c\\
      v/c & 1
    \end{pmatrix}
    \begin{pmatrix}
      c\dd{t'}\\\dd{x'}
    \end{pmatrix}
  \end{align}
  となる.$V=\dv{x}{t}$なので
  \begin{align}
    V = \dv{x}{t} &= \frac{v \dd{t'} + \dd{x'}}{\dd{t'} + (v/c^2)\dd{x'}}\\
    &= \frac{v + \dv{x'}{t'}}{1 + (v/c^2)\dv{x'}{t'}}\\
    &= \frac{v + u'}{1 + \frac{vu'}{c^2}}
  \end{align}
  を得る.
  \begin{itembox}[l]{速度の合成}
    \begin{equation}
      V = \frac{v + u'}{1 + \frac{vu'}{c^2}}
    \end{equation}
  \end{itembox}
  例として$u' = v = c$とすると
  \begin{align}
    V = \frac{2c}{1+c^2/c^2} = c
  \end{align}
  である.速度が合成されても光速を超えることは決してないことがわかる.

  \textbf{Lorentz収縮}について説明する.速さ$v$で運動している$X'$系から,$t'=0$において,静止している$X$系の2点を見る.1点は原点$O:(x,t)=(0,0)$
  もう1点は$P:(x,t)=(L,t)$とする.まず,原点$O$は$X'$系から見ると
  \begin{equation}
    \begin{pmatrix}
      ct' \\ x'
    \end{pmatrix}
    =
    \frac{1}{\sqrt{1 - (v/c)^2}}
    \begin{pmatrix}
      1 & -v/c\\
      -v/c & 1
    \end{pmatrix}
    \begin{pmatrix}
      0\\0
    \end{pmatrix}
    =
    \begin{pmatrix}
      0\\0
    \end{pmatrix}
  \end{equation}
  である.点$P$を$X'$系から見ると,
  \begin{equation}
    \begin{pmatrix}
      ct' \\ x'
    \end{pmatrix}
    =
    \frac{1}{\sqrt{1 - (v/c)^2}}
    \begin{pmatrix}
      1 & -v/c\\
      -v/c & 1
    \end{pmatrix}
    \begin{pmatrix}
      ct\\L
    \end{pmatrix}
    =
    \frac{1}{\sqrt{1 - (v/c)^2}}
    \begin{pmatrix}
      ct - \frac{v}{c}L\\-vt + L
    \end{pmatrix}
  \end{equation}
  である.$t'=0$で観測しているため,$t'=0$を代入し
  \begin{align}
    0 = ct - \frac{v}{c}L\\
    t = \frac{v}{c^2}L
  \end{align}
  を得る.よって,
  \begin{align}
    x' = \frac{1}{\sqrt{1 - (v/c)^2}} \qty(-\qty(\frac{v}{c})^2 L + L) = \sqrt{1-\qty(\frac{v}{c})^2}L
  \end{align}
  である.これは動いている慣性系から静止系での距離($L$)を測る($L'$)と縮んで見えることを意味している.
  \begin{itembox}[l]{Lorentz収縮}
    \begin{equation}
      L' = \sqrt{1-\qty(\frac{v}{c})^2}L
    \end{equation}
  \end{itembox}
  また,その対象に対して静止している観測者が測った距離を\textbf{固有長さ}(proper length)という.今回は$L$が固有長さである.

  次に\textbf{時間の遅れ}について説明する.同様の$X$系と$X'$系を考える.時計が$X'$系の原点$x'=0$に置かれており,観測者は$X'$系においてこの時計を見ている.
  $X'$系に置かれた時計の時刻が$t'$のときの時空点$P_1$,$t'+\Delta T_0$の時空点を$P_2$とする.$P_1$は,
  \begin{equation}
    \begin{pmatrix}
      ct_1 \\ x_1
    \end{pmatrix}
    =
    \frac{1}{\sqrt{1 - (v/c)^2}}
    \begin{pmatrix}
      1 & v/c\\
      v/c & 1
    \end{pmatrix}
    \begin{pmatrix}
      ct'\\0
    \end{pmatrix}
    =
    \frac{1}{\sqrt{1 - (v/c)^2}}
    \begin{pmatrix}
      ct'\\vt'
    \end{pmatrix}
  \end{equation}
  $P_2$は,
  \begin{equation}
    \begin{pmatrix}
      ct_1 \\ x_1
    \end{pmatrix}
    =
    \frac{1}{\sqrt{1 - (v/c)^2}}
    \begin{pmatrix}
      1 & v/c\\
      v/c & 1
    \end{pmatrix}
    \begin{pmatrix}
      c(t' + \Delta T_0)\\0
    \end{pmatrix}
    =
    \frac{1}{\sqrt{1 - (v/c)^2}}
    \begin{pmatrix}
      c(t' + \Delta T_0)\\v(t' + \Delta T_0)
    \end{pmatrix}
  \end{equation}
  である.よって,$X$系での時間経過$\Delta T = t_2 - t_1$は,
  \begin{align}
    \Delta T = t_2 - t_1 = \frac{1}{\sqrt{1 - (v/c)^2}} \Delta T_0
  \end{align}
  である.これは$X'$系は$X$系に比べて時間の流れが遅いことを示している.
  \begin{itembox}[l]{時間の遅れ}
    \begin{equation}
      \Delta T = \frac{1}{\sqrt{1 - (v/c)^2}} \Delta \tau
    \end{equation}
  \end{itembox}
  観測者に対して2つの事象が同一の空間座標で起きたとき,その時間間隔$\Delta \tau$を\textbf{固有時間}(proper time)という.

  電磁場の双対性について説明する.マクスウェル方程式は3次元では次の4つの式である.
  \begin{align}
    \begin{dcases}
      \nabla \cdot \bm{D} = \rho \\
      \nabla \cdot \bm{B} = 0\\
      \nabla \times \bm{E} = - \frac{\partial \bm{B}}{\partial t}\\
      \nabla \times \bm{H} = \frac{\partial \bm{D}}{\partial t} + \bm{j}
    \end{dcases}
  \end{align}
  これをLorentz変換に対して共変な形式に書き直す.まず,$\bm{B}$と$bm{E}$はベクトルポテンシャル$\bm{A}$とスカラーポテンシャル$\phi$を
  用いて
  \begin{align}
    \begin{dcases}
      \bm{E} = -\nabla \phi - \frac{\partial \bm{A}}{\partial t}\\
      \bm{B} = \nabla \times \bm{A}
    \end{dcases}
  \end{align}
  と表される.ここで,座標を
  \begin{align}
    (x^0, x^1, x^2, x^3) = (ct, x, y, z)
  \end{align}
  と表し,
  4元ベクトルポテンシャルを
  \begin{align}
    A^{\mu} &= (\phi, c\bm{A})\\
    A_{\mu} &= (\phi, -c\bm{A})
  \end{align}
  と定義する.この4元ベクトルポテンシャルを使うと電場と磁場は
  \begin{align}
    B_x &= c \frac{\partial A_z}{\partial y} - c \frac{\partial A_y}{\partial z}\\
    &= -\frac{\partial A_3}{\partial x^2} + \frac{\partial A_2}{\partial x^3}\\
    &= \partial_3 A_2 - \partial_2 A_3\\
    E_x &= -\frac{\partial \phi}{\partial x} - \frac{\partial A_x}{\partial t}\\
    &= -\frac{\partial A_0}{\partial x^1} + \frac{\partial A_1}{\partial x^0}\\
    &= \partial_0 A_1 - \partial_1 A_0
  \end{align}
  のように書ける.ここで,電磁場テンソル
  \begin{align}
    F_{\mu\nu} = \partial_{\mu}A_\nu - \partial_\nu A_\mu
  \end{align}
  を定義する.これを行列の形で表すと
  \begin{align}
    F_{\mu\nu}=
    \begin{pmatrix}
      F_{00} & F_{01} & F_{02} & F_{03}\\
      F_{10} & F_{11} & F_{12} & F_{13}\\
      F_{20} & F_{21} & F_{22} & F_{23}\\
      F_{30} & F_{31} & F_{32} & F_{33}
    \end{pmatrix}
    =
    \begin{pmatrix}
      0 & E_x & E_y & E_z \\
      -E_x & 0 & -cB_z & cB_y\\
      -E_y & cB_z & 0 & -cB_x\\
      -E_z& -cB_y & cB_x & 0
    \end{pmatrix}
  \end{align}
  である.明らかに$F_{\mu\nu} = F_{\nu\mu}$である.また,この定義から
  \begin{align}
    \partial_\mu F_{\nu\lambda} + \partial_\nu F_{\lambda\mu} + \partial_\lambda F_{\mu\nu} = 0
  \end{align}
  が成り立つ.実際に計算してみると
  \begin{align}
    &\partial_\mu F_{\nu\lambda} + \partial_\nu F_{\lambda\mu} + \partial_\lambda F_{\mu\nu} \\
    &= (\partial_\mu\partial_\nu A_\lambda - \partial_\mu\partial_\lambda A_\nu) + (\partial_\nu\partial_\lambda A_\mu - \partial_\nu\partial_\mu A_\lambda)
    + (\partial_\lambda\partial_\mu A_\nu - \partial_\lambda\partial_\nu A_\mu)\\
    &= 0
  \end{align}
  である.これはFaradayの電磁誘導の法則と磁束密度に関するGaussの法則を表している.
  例えば$\mu = 0,\nu = 1, \lambda =2$とすると
  \begin{align}
    &\partial_0 F_{12} + \partial_1 F_{20} + \partial_2 F_{01}\\
    &= -\frac{\partial (cB_z)}{\partial (ct)} + \frac{\partial}{\partial x}(-E_y) + \frac{\partial}{\partial y}(-E_x)\\
    &= -(\nabla\times \bm{E})_z - \frac{\partial B_z}{\partial t} = 0
  \end{align}
  $\mu=1,\nu=2,\lambda=3$とすると
  \begin{align}
    &\partial_1 F_{23} + \partial_2 F_{31} + \partial_3 F_{12}\\
    &= c(\partial_xB_x + \partial_yB_y + \partial_zB_Z)\\
    &= c\nabla\cdot\bm{B} = 0
  \end{align}
  が得られる.さらに,磁場/電束密度テンソル$H_{\mu\nu}$を
  \begin{align}
    H_{\mu\nu}=
    \begin{pmatrix}
      H_{00} & H_{01} & H_{02} & H_{03}\\
      H_{10} & H_{11} & H_{12} & H_{13}\\
      H_{20} & H_{21} & H_{22} & H_{23}\\
      H_{30} & H_{31} & H_{32} & H_{33}
    \end{pmatrix}
    =
    \begin{pmatrix}
      0 & cD_x & cD_y & cD_z \\
      -cD_x & 0 & -H_z & H_y\\
      -cD_y & H_z & 0 & -H_x\\
      -cD_z& -H_y & H_x & 0
    \end{pmatrix}
  \end{align}
  4次元の電流密度を$j^{\mu}$を
  \begin{align}
    j^{\mu} = (c\rho, \bm{j})
  \end{align}
  と定義する.明らかに$H_{\mu\nu} = -H_{\nu\mu}$である.
  すると,電束密度に関するGaussの法則とAmpèreの法則は次の式で表される.
  \begin{align}
    \partial^{\nu}H_{\nu\mu}=j_\mu
  \end{align}
  例えば$\mu=0$とすると
  \begin{align}
    \partial^1H_{10}+\partial^2H_{20} + \partial^3H_{30} &= j_0\\
    \partial_xD_x + \partial_yD_y + \partial_zD_z = \rho\\
    \nabla\cdot\bm{D} = \rho
  \end{align}
  が導かれる.さらに,電荷保存則
  \begin{align}
    \frac{\partial \rho}{\partial t} + \nabla\cdot\bm{j} = 0
  \end{align}
  から
  \begin{align}
    \frac{\partial (c\rho)}{\partial (ct)} + \nabla\cdot\bm{j} = 0
  \end{align}
  よって
  \begin{align}
    \partial_\mu j^\mu = 0
  \end{align}
  が得られる.以上をまとめるとMaxwell方程式は
  \begin{align}
    \begin{dcases}
      \partial_\mu F_{\nu\lambda} + \partial_\nu F_{\lambda\mu} + \partial_\lambda F_{\mu\nu} = 0\\
      \partial^{\nu}H_{\nu\mu}=j_\mu\\
      \partial_\mu j^\mu = 0
    \end{dcases}
  \end{align}
  と書くことができる.これらの式はテンソルとテンソル,ベクトルとベクトルというように,Lorentz変換に対して
  同じ変換則をもつものどうしが結ばれている.よってこれらはLorentz変換に対して共変である.
  \begin{itembox}[l]{Lorentz変換に対して共変なMaxwell方程式}
    \begin{align}
      &\partial_\mu F_{\nu\lambda} + \partial_\nu F_{\lambda\mu} + \partial_\lambda F_{\mu\nu} = 0\\
      &\partial^{\nu}H_{\nu\mu}=j_\mu\\
      &\partial_\mu j^\mu = 0
    \end{align}
  \end{itembox}

  運動する電荷と電流を例に電磁場の双対性を確認してみよう.導線には$z$軸上向きに電流$I$が流れている.電荷$+q$が$z$軸上向きに速さ$v_0$で運動している.
  導線の中では面電荷密度$\lambda_{\pm} = \pm\lambda/2$の正(負)電荷が速さ$v$で上(下)向きに動いているとする.
  まずは静止系で考える.導線の中では
  \begin{align}
    \lambda_{+} + \lambda_{-} = 0
  \end{align}
  が成り立つため,電気的に中性である.よってm導線の周りには電場は無い.電流は$I=\lambda v$と表される.
  導線の周りには
  \begin{align}
    \bm{B} = \frac{\mu_0I}{2\pi r} \bm{e}_{\phi} = \frac{\mu_0 \lambda v}{2\pi r} \bm{e}_{\phi}
  \end{align}
  の磁束密度が発生している.よって電荷は
  \begin{align}
    \bm{F} = q\bm{v}\times\bm{B} = -qv_0\frac{\mu_0 \lambda v}{2\pi r}\bm{e}_r
  \end{align}
  の力を感じる.

  これを電荷とともに動く系で考える.非相対論的に感がると$\bm{v}=0$であるため電荷は力を感じない.しかし,特殊相対性原理よりこれはありえない.
  相対論的効果を考慮してこの状況を眺める必要がある.
  観測者からは導線中の正電荷は$v-v_0$で,負電荷は$v+v_0$で運動して見える.それぞれでLorentz収縮を計算する.
  \begin{align}
    \lambda_{+} &= \frac{1}{\sqrt{1-\qty(\frac{v-v_0}{c})^2}}\frac{\lambda}{2}\\
    \lambda_{-} &= \frac{1}{\sqrt{1-\qty(\frac{v+v_0}{c})^2}}\qty(-\frac{\lambda}{2})
  \end{align}
  明らかに
  \begin{align}
    \lambda_{+} + \lambda_{-} \neq 0
  \end{align}
  であることがわかる.よって,導線の周りには電場が発生している.$v,v_0 \ll c$とすると
  \begin{align}
    \lambda_{+} + \lambda_{-} &= \frac{1}{\sqrt{1-\qty(\frac{v-v_0}{c})^2}}\frac{\lambda}{2} - \frac{1}{\sqrt{1-\qty(\frac{v+v_0}{c})^2}}\frac{\lambda}{2}\\
    &\simeq \qty(1 + \frac{1}{2}\qty(\frac{v-v_0}{c})^2)\frac{\lambda}{2} - \qty(1 + \frac{1}{2}\qty(\frac{v+v_0}{c})^2)\frac{\lambda}{2}\\
    &= - \frac{\lambda v_0 v}{c^2}
    &\equiv \Delta \lambda
  \end{align}
  と計算でき,導線は負に帯電していることがわかる.よって,電荷の受ける力は
  \begin{align}
    \bm{F} &= q\frac{\Delta \lambda}{2\pi r \epsilon_0}\bm{e}_r\\
    &=-qv_0\frac{\mu_0 \lambda v}{2\pi r}\bm{e}_r
  \end{align}
  これは先ほどの計算結果と一致している.以上の考察から,電場と磁場は観測する系によって入り混じることがわかる.
\end{document}