\documentclass{report}
\input{../../head.tex}
\begin{document}
  特殊相対性理論(Special Relativity)は次の2つの事柄を原理とする.
  \begin{itembox}[l]{特殊相対性原理}
    あらゆる慣性系で同じ物理法則が成り立つ.
  \end{itembox}

  \begin{itembox}[l]{光速度不変の原理}
    あらゆる慣性形で進級中の光の速さは同一である.
  \end{itembox}

  この原理の下で成り立つ座標変換の法則(Lorentz変換)を導く.
  まず,慣性系$X$系の原点$O$とと$X'$系の原点$O'$が$t=t'=0$で一致している.$t=t'=0$で光が原点($O=O'$)を通過したとする.
  $X$系の空間座標を$(x,y,z)$,$X'$系の空間座標を$(x',y',z')$とすると,光速度不変の原理より,
  \begin{align}
    \frac{\sqrt{x^2 + y^2 + z^2}}{t} = \frac{\sqrt{x'^2 + y'^2 + z'^2}}{t'} = c
  \end{align}
  が成り立つ.上式から\textbf{世界長さ}
  \begin{align}
    s^2 = x^2 + y^2 + z^2 - (ct)^2
  \end{align}
  が不変量であることが導かれる.

  慣性系$X'$が$x$軸正の方向に速さ$v$で移動している..このとき$y=y',z=z'$である.わかりやすいように$T=it$とおく.光速不変より,
  \begin{align}
    x^2 - (ct)^2 &= x'^2 - (ct')^2\\
    x^2 + (cT)^2 &= x'^2 + (cT)^2
  \end{align}
  が成り立つ.これが回転座標変換と類似していることから,
  \begin{align}
    \begin{pmatrix}
      cT' \\ x'
    \end{pmatrix}
    =
    \begin{pmatrix}
      \cos\theta & -\sin\theta\\
      \sin\theta & \cos\theta
    \end{pmatrix}
    \begin{pmatrix}
      cT\\x
    \end{pmatrix}
  \end{align}
  と置く.表示を$t$に戻すと
  \begin{align}
    \begin{pmatrix}
      ct' \\ x'
    \end{pmatrix}
    =
    \begin{pmatrix}
      \cos\theta & \i\sin\theta\\
      \i\sin\theta & \cos\theta
    \end{pmatrix}
    \begin{pmatrix}
      ct\\x
    \end{pmatrix}
  \end{align}
  である.さらに,$\theta = \i\phi$とすると
  \begin{align}
    \begin{pmatrix}
      ct' \\ x'
    \end{pmatrix}
    =
    \begin{pmatrix}
      \cosh\phi & -\sinh\phi\\
      -\sinh\phi & \cosh\phi
    \end{pmatrix}
    \begin{pmatrix}
      ct\\x
    \end{pmatrix}
  \end{align}
  となる.よって,
  \begin{align}
    x' = (-\sinh\phi)ct + (\cosh\phi)x
  \end{align}
  を得る.$X$系において時刻$t$が経過したとする.$X$系から見るお$X'$系の原点の位置は$x=vt$である.一方,$X'$系から見ると$x'=0$である.
  よって上式から
  \begin{align}
    0 = (-\sinh\phi)ct + (\cosh\phi)vt
  \end{align}
  が成り立つ.よって,これを変形すると
  \begin{align}
    \frac{v}{c} = \frac{\sinh\phi}{\cosh\phi} = \tanh\phi
  \end{align}
  である.以上より,
  \begin{align}
    \begin{dcases}
      \sinh\phi = \frac{v/c}{\sqrt{1- (v/c)^2}}\\
      \cosh\phi = \frac{1}{\sqrt{1- (v/c)^2}}
    \end{dcases}
  \end{align}
  であることがわかる.したがって,
  \begin{itembox}[l]{Lorent変換}
    \begin{equation}
      \begin{pmatrix}
        ct' \\ x'
      \end{pmatrix}
      =
      \frac{1}{\sqrt{1 - (v/c)^2}}
      \begin{pmatrix}
        1 & -v/c\\
        -v/c & 1
      \end{pmatrix}
      \begin{pmatrix}
        ct\\x
      \end{pmatrix}
    \end{equation}
  \end{itembox}
  を得る.
\end{document}